\classheader{2019-01-23}{Algebra Basics and Some Applications}

\section{Groups}

What is a group?

A group is a set $G$ together with an operation $\cdot : G\times G \rightarrow G$ such that:

\begin{itemize}
	\item There is an identity element $e$ such that $eg = ge = g$ for all $g\in G$
	\item $\cdot$ is associative.  That is, $g(hk) = (gh)k$ for all $g,h,k\in G$
	\item For any $g\in G$, there exists a unique $g^{-1}$ such that $gg^{-1}=g^{-1}g= e$
\end{itemize}


\begin{definition}
	{A group is called \textbf{abelian} or \textbf{commutative} if $gh=gh$ for all $g,h\in G$.}
\end{definition}

\begin{definition}
	A group $G$ is \textbf{finite} if $|G|$ is finite. It is \textbf{infinite} otherwise.
\end{definition}

Some examples of groups:
\begin{enumerate}
	\item The integers $\Z$ with the operation $+$ is an abelian group
	\item The integers modulo $n$ ($\Z_n$) is a finite abelian group under addition modulo $n$
	\item $\Z_p^*$, the elements of $\Z_n$ which are relatively prime to $p$ is a finite abelian group under multiplication modulo $p$. If $p$ is prime, its elements are $\{1,2,\dots,p-1\}$.
	\item The set of permutations of $n$ letters is the \textit{symmetric group} $S_n$ under the operation of composition. It has $n!$ elements and is non-abelian (unless $n=1$ or $n=2$).
	\item Sets of square invertible matrices (of a fixed size) under matrix multiplication are non-abelian, sometimes infinite groups.
	\item The group of symmetries of a regular polygon with $n$ vertices is the dihedral group $D_{2n}$ and it is non-abelian (it's the semidirect product of $\Z_2$ with $\Z_n$)
\end{enumerate}


\begin{definition}
	A \textbf{subgroup} $H$ of a group $G$ is a set of elements $H\subset G$ which themselves form a group with respect to the operation of $G$.  The identity element $e\in G$ is the identity in $H$. We denote this as $H<G$
\end{definition}



We can prove our first theorem about groups!
\begin{theorem}[Lagrange's Theorem]
	If $G$ is a finite group and $H<G$, then $|H|$ divides $|G|$.
\end{theorem}

\begin{proof}
	We begin by constructing an equivalence relation $\sim$ on $G$ where $g\sim h$ if $gh^{-1}\in H$.  We can easily verify this is a proper equivalence relation.  It is reflexive because $gg^{-1}=e\in H$.  It is symmetric because $gh^{-1}\in H$ implies $(gh^{-1})^{-1}\in H$, but this is equal to $hg^{-1}$.  Transitivity follows from the cancellation property.  If $gh^{-1}\in H$ and $hk^{-1}\in H$, then their product is also in $H$. The product is $gh^{-1}hk^{-1} = gk^{-1}$, which shows transitivity.
	
	This equivalence relation partitions the group into disjoint subsets (called cosets). The set of things equivalent to $a$ (written $[a]$) is exactly the set $Ha$, which is the set of things which are the product of something in $H$ with $a$.  Then, we can see that $|Ha|=|H|$, since for distinct elements $h_1,h_2\in H$, $h_1a\neq h_2a$, meaning that each element in $Ha$ can be written as the product of a unique element in $H$ with $a$.  Thus, $|G| = |H|\text{number of equivalence classes}$ and $|H|$ divides $|G|$.  This number of equivalence classes is called the \textit{index of $H$ in $G$}.
\end{proof}

This has a corollary:
\begin{corollary}
	If $H\subsetneq G$, $|H|$ is at most half of $|G|$, since $|H|$ is a proper factor of $|G|$ (or 1...).
\end{corollary}

We can use this to do a randomized thingy!

\begin{example}
	\textbf{Freivald's Matrix Multiplication Checker} is an algorithm which does the following.  Suppose, for the sake of ease of exposition, everything here is done over $\F_2$.  It takes in 3 $n\times n$ matrices $A,B,C$ such that we are trying to verify $AB\overset{?}{=}C$.  We can do this efficiently by taking a uniformly random binary vector $v$ and checking that $ABv = Cv$.  If the multiplication was correct, the test will always pass.  If there is an error in $C$, what is the probability we detect it? Call $v$ a \textit{witness} if $ABv\neq Cv$ and a \textit{non-witness} otherwise.  We can observe that the set of non-witnesses forms a group under componentwise addition (mod 2), since for two non-witnesses $v_1,v_2$, then $(AB-C)v_1 = (AB-C)v_2 = 0$, so $(AB-C)(v_1+v_2)=0$ (closure), and the inverse of a non-witness is a non-witness since everything here is self-inverse.\footnote{This is true in general, too.}  Since this is a subgroup of the group of all binary vectors, the set of non-witnesses has set at most half of $2^n$ as long as a witness exists.  This means that if we pick a random $v$, with probability at least $.5$ we can detect an error.
	
	To see that a witness must exist, consider again $AB-C$, which is non-zero if $AB\neq C$.  Let $ij$ be a non-zero entry in $AB-C$ and let $v$ be a non-witness. Then $(AB-C)v=0$ and the $i$th position of $(AB-C)v$ is the $i$th row of $AB-C$ times $v$.  Therefore, the vector which is all zeroes except a $1$ in the $j$th position constitutes a witness.
	
	The test is efficient because we can compute $A(Bv)$ in $n^2$ time rather than $(AB)v$ which takes more.  
\end{example}

The idea of this, and many other applications, is that the set of bad examples forms a group and a proper subgroup of the set of all examples, so there aren't too many of them.

\begin{definition}
	A mapping $\varphi:G\rightarrow H$ from a group $G$ to a group $H$ is a \textbf{homomorphism} if $\varphi(xy) = \varphi(x)\varphi(y)$ for all $x,y\in G$, where the operation on the left is with respect to $G$ and on the right with respect to $H$.
	
	In particular $\varphi(e_G) = e_H$ and $\varphi(x^{-1}) = \varphi(x)^{-1}$.
\end{definition}

\begin{definition}
	The \textbf{kernel} of a homomorphism $\varphi$ is the set $K\subset G$ such that $\varphi(k)=e$ for all $k\in K$.
\end{definition}

\begin{example}
	Take $\varphi:\Z_2\rightarrow Z_7$ where $x\mapsto  x{\mod{7}}$.  The kernel is the set of all multiples of 7 in  $\Z$, i.e. $7\Z$.
\end{example}

\begin{lemma}
	{The kernel of a homomorphism $K$ forms a subgroup of $G$.}  By definition, $e\in K$, and if $x,y\in K$, then $\varphi(x)=\varphi(y)=e$ and so $\varphi(ab) = \varphi(a)\varphi(b)=e$.
\end{lemma}

\begin{definition}
Let $G$ be an abelian group and $K$ a subgroup. Then the quotient $G/K$ is a group whose elements are in correspondence with the equivalence classes (cosets) of $K$ as defined in the proof of Lagrange's theorem.
\end{definition}


\begin{example}
	Take $G=\Z$ and $K=7\Z$. Then the elements of $G/K$ are equivalence classes of $[0],[1],\dots,[6]$ which is isomorphic to $\Z_7$, where we mapped each element to its remainder modulo 7.
\end{example}

\begin{theorem}[First Isomorphism Theorem]
	If $\varphi$ is a surjective homomorphism from $G$ to $H$ with kernel $K$, then $G/K \iso H$.
\end{theorem}

We can apply this in proving the Chinese Remainder Theorem:

\begin{theorem}[Chinese Remainder Theorem]
	Let $m_1,m_2$ be two coprime numbers. Then $\Z_{m_1m_2}\iso \Z_{m_1}\times \Z_{m_2}$.
\end{theorem}


\begin{proof}
	First, it's clear that  $\Z_{m_1}\times \Z_{m_2}$ is a group, where everything just happens componentwise.  We'll prove this by showing that there is a homomorphism  $\varphi: \Z_{m_1m_2}\rightarrow \Z_{m_1}\times \Z_{m_2}$ with trivial kernel.  We check that $x\mapsto(x\mod{m_1},x\mod{m_2})$ suffices.  It's a homomorphism by checking it across the factors, and $0\mapsto(0,0)$ is the entirety of the kernel.
\end{proof}

The Chinese Remainder Theorem is a special case of the following statement:

\begin{lemma}
	Let $\varphi: G\rightarrow H$ be a surjective homomorphism with trivial kernel.  Then $G\iso H$.
\end{lemma}

Here's an application to communication complexity:

\begin{example}[The Man on the Moon Problem]
Let Alice be on the Moon and Bob on Earth.  Alice has a string $x$ of length $n$ and Bob has a copy $y$, and they want to make sure that they have the same string. We want a communication protocol to check $x\overset{?}{=}y$.  We can assume without loss that $y$ is also of length $n$, since checking this only requires $\log{n}$ bits of communication.  We'll assume as well that the communication channel is reliable.

We can have a dumb algorithm where Alice sends all of $x$ to Bob, which requires $n$ bits of communication.  If the potential corruption is adversarial, then this is the best deterministic procedure.  To see this, observe that Alice and Bob are trying to compute a function together which can be viewed as mapping $X\times Y\rightarrow \{0,1\}$, where $|X|=|Y|=2^n$.  We can write this down in a table which looks like the $2^n\times 2^n$ identity.  Suppose further, without loss, that the protocol is alternating.  That is, the transcript of the communication will have things sent by Alice in odd positions and Bob in even positions.  

If for any two pairs $(x^i,y^i)$ and $(x^j,y^j)$ we demand that the transcripts differ, in order for the $2^n$ matching pairs to have different transcripts, we need to communicate $n$ bits.  To see this, suppose that $(x^i,y^i)$ and $(x^j,y^j)$ do have the same transcript, but represent different pairs of matching strings.  We'll show that this implies that the transcripts of $(x^i,y^j)$ and $(x^i,y^j)$ are identical, leading Alice and Bob to reach the incorrect decision.  We can prove this by induction.  Alice has $x^i$ and sends the first bit accordingly.  Bob has $y^j$ and sees this first bit, which is consistent with the case where he has $y^j$ and Alice has $x^j$, so he continues accordingly. Alice proceeds, given that everything she sees is consistent with Bob having $y^i$.

We can give a randomized algorithm which does better.  A witness here will be a prime $p$ such that $x\mod p\neq y \mod p$.  First, note that if two primes $p_1$ and $p_2$ both divide a number $x-y$, then $p_1p_2$ does as well.  Since the product of $n$ distinct primes is at least $2^n$, since $x$, $y$, and $x-y$ are all at most $2^n$, then there are at most $n$ primes which divide $x-y$ (i.e. are non-witnesses).  Alice can write down the first $2n$ primes and picks one of them $p$ at random.  She can then send Bob $p$ $x\mod p$.  Since there are at most $n$ non-witnesses, the probability that $p$ is a non-witness is at most $.5$.  It takes $\log{n}$ bits to communicate $p$ and another $\log{n}$ bits to send $x\mod p$.
\end{example}




\begin{definition}
	Let $G$ be a group and $g,h,\dots\in G$.  We denote by $\langle g,h,\dots \rangle$ the \textbf{group generated by $g,h,\dots$}.  That is, the set of all elements of $G$ which can be written as a finite product of $g$.  Equivalently, $\langle g,h,\dots \rangle$ is the smallest subset of $G$ which is closed under the operation and contains the \textbf{generators} $g,h,\dots$.  We sometimes replace the elements with a set $S\subset G$ and write $\langle S \rangle$.
\end{definition}


\begin{definition}
	Thee \textbf{order} of an element $g$ is the size of the group generated by $g$.  Equivalently, it is the smallest number $k$ such that $g^k=e$.
\end{definition}

\begin{example}
	In $\Z_7^*$, the order of $2$ is 3, since $2\times 2 = 4$, $4\times 2=2^3=1$.
\end{example}

\begin{definition}
	A group is \textbf{cyclic} if it can be generated by a single element.
\end{definition}

\begin{lemma}
	If $G$ is a group with  $|G|=n$, then $G$ can be generated by at most $\log{n}$ generators.
\end{lemma}
\begin{proof}
	We proceed inductively. Let $|G|\geq 2$ and choose $a_1\neq e$ as the first generator.  We look at $|\langle a_1 \rangle |\geq 2$.  Choose some $a_2\notin \langle a_1 \rangle $.  This is a strict supergroup of $\langle a_1 \rangle $, so $|\langle a_1,a_2 \rangle | \geq 2|\langle a_1 \rangle |$.  Since adding each generator at least doubles the size of the resultant group, we only need $\log{n}$ generators to reach order $n$.
\end{proof}

\begin{example}
 The \textit{group isomorphism problem} asks (algorithmically) whether two group $G$ and $H$ are isomorphic.  One way to present a group to a computer is to list out the $n^2$ entries in the multiplication table of a group, so we can phrase the problem as asking whether the table for $G$ is a permutation of the table of $H$.
 
 There is an $O(n^{\log{n}})$ algorithm for this which works by finding a set of $\log{n}$ generators for $G$ and match against every set of the same size in $H$ to check if there is an isomorphic mapping among the generators. 
\end{example}


A consequence of this and Lagrange's theorem is Fermat's Little Theorem.

\begin{theorem}[Fermat's Little Theorem]

If $p$ is a prime then $a^{p-1} \cong 1\mod{p}$ for all $a$ relatively prime to $p$.	

\end{theorem}
\begin{proof}
	Consider $\Z_p^*$, which has size $p-1$.  Then the order of $\langle a \rangle $ is $\ell$, which divides $p-1$.  Thus $a^{p-1} = (a^{\ell})^q = e$.
\end{proof}

\begin{example}
This was proposed as a (wrong) primality test.  To test if $n$ is prime, pick $a$ at random from $1,2,\dots, n$ and compute $a^{n-1}\mod n$.  If this is equal to 1, say that $n$ is prime. Otherwise, say that it isn't.  This is good because it will always be 1 if $n$ is prime.  This algorithm actually works on most inputs, but not all.  The ones that fail are called the \textit{Carmichael numbers}.  These are composite numbers which satisfy Fermat's Little Theorem.
\end{example}

\begin{example}
	Given $a\in\Z_p^*$, how can we find $a^{-1}$.  We know that $a$ is relatively prime to $p$, so we know that $\gcd(a,p)=1$.  To compute a GCD, we can use the \textit{extended Euclid's algorithm}, which takes in an $x$ and $y$ and produces the GCD $d$ as well as the two integers $q_1$ and $q_2$ such that $xq_1+yq_2=d$.  Using this, we can get the $q_1,q_2$ such that $aq_1+pq_2 = 1$.  Since modulo $p$, $pq_1=0$, $q_1=a^{-1}$.
\end{example}
