\renewcommand{\exc}[1]{\subsubsection*{Exercise 4.#1}}

\classheader{: Arc-Transitive Graphs}

Recall that an \textit{arc} in a graph is an ordered pair of adjacent vertices.  Thus a graph is \textit{arc-transitive} if its automorphism group acts transitively on the set of arcs.  This is stronger than vertex- or edge-transitivity, so we can prove some interesting things in this setting which apply to these weaker notions.  We begin by building up to Tutte's results on cubic arc-transitive graphs.  Then, we consider some famous examples of arc-transitive graphs.

\section*{Arc-Transitive Graphs}
\definition{An \textbf{$\boldsymbol{s}$-arc} in a graph is a sequence of $s$ vertices such that consecutive vertices are adjacent and $v_{n-1}\neq v_{n+1}$.  We are permitted to use the same vertex twice in general, just not in second-adjacent positions.  A graph is \textbf{$\boldsymbol{s}$-arc-transitive} if its automorphism group acts transitively on $s$-arcs.  This property is inductive, in that if a graph is $s$-arc-transitive, it is also $(s-1)$-arc-transitive.  A $0$-arc-transitive graph is vertex-transitive. A $1$-arc-transitive graph is sometimes called \textbf{symmetric}.}

A cycle on $n$ vertices is $s$ arc-transitive for all $s$.\footnote{The comment in the book is one of the best things I've ever read in a math book: "...which only goes to show that truth and utility are different concepts."}  More interestingly, the cube $Q$ is $2$-arc-transitive but not $3$-arc-transitive, as the three arcs which form three sides of a $4$-cycle can't be mapped to three arcs which do not form three sides of a $4$-cycle.

A graph $X$ is $s$-arc transitive if it has a group $G$ of automorphisms such that $G$ is transitive, and the stabilizer $G_u$ of a vertex $u$ acts transitively on $s$-arcs starting at $u$.

\begin{lemma}
	The graphs $J(v,k,i)$ are at least arc transitive.
\end{lemma}
\begin{proof}
	Consider the vertex $\{1,2,3,\dots,k\}$.  The stabilizer of this vertex contains $Sym(k)\times Sym(v-k)$.  Clearly any two $k$-sets meeting this initial vertex in an $i$-set can be mapped to each other by this group.
\end{proof}


\begin{lemma}
	The graphs $J(2k+1,k,0)$ are at least $2$-arc transitive.
\end{lemma}
\begin{proof}
	Each edge in this graph can be labeled by the single element not in either of the sets it is incident to.  Thus a $2$-arc is an ordered pair of distinct elements.  Suppose we want to map the arc $[a,b]$ onto an arc $[x,y]$.  Then the permutation $(ax)(by)$ does the job.
\end{proof}

\definition{The \textbf{girth} of a graph is the length of the shortest cycle in it.}

\begin{lemma}[Tutte]
	If $X$ is an $s$-arc transitive graph with valency at least $3$ and girth $g$, then $g\geq 2s-2$.
\end{lemma}
\begin{proof}
	
	
	Let's assume $s\geq 3$, otherwise the conclusion is meaningless.  By our assumption, $X$ contains a cycle of length $g$ as well as a path of length $g$ whose endpoints are not adjacent.  This is a $g$-arc with adjacent end vertices and a $g$-arc with nonadjacent end vertices.  No automorphism can map one to the other, so $s<g$.  Since $X$ contains cycles of length $g$, and since these contain $s$-arcs, any $s$-arc must lie in some cycle of length $g$.  Let $\alpha=v_1,v_2,\dots,v_s$ be an $s$-arc.  Since $v_{s-1}$ has valency at least $3$, it is adjacent to some vertex $w\neq v_s,v_{s-2}$, and since the girth of $X$ is at least $s$, $w$ doesn't lie in $\alpha$, otherwise we could have constructed a shorter cycle.  Thus if we replace $v_s$ with $w$, we get a second $s$-arc $\beta$ which intersects $\alpha$ in an $(s-1)$-arc.  Since $\beta$ lies in a $g$-cycle as well, we get a pair of cycles of length $g$ which share at least $s-1$ edges in common.
	
	Deleting these $s-1$ edges from the graph, the graph must still contain a cycle of length at most $2g-2s+2$, so $2g-s2+2\geq g$, and the result follows from algebraic manipulation.
	
	
	
	
\end{proof}	
	
A natural question to ask is what we can say about the $s$-arc transitive graphs with girth equal to $2s-2$.  It follows from the next lemma that these graphs are what we will later refer to as \textit{generalized polygons}.

\begin{lemma}[Tutte]
	If $X$ is $s$-arc transitive with girth $2s-2$, it is bipartite with diameter $s-1$.
\end{lemma}
\begin{proof}
	First, if $X$ has girth $2s-2$, then any $s$-arc lies in at most one cycle of length $2s-2$, and so if $X$ is $s$-arc transitive, every $s$-arc lies in a unique cycle of length $2s-2$.  Thus $X$ has diameter at least $s-1$, which is the distance between opposite points in such a cycle.  Let $u$ be a vertex of $X$ and suppose, for the sake of contradiction, that $v$ is a vertex at distance $s$ from $u$.  Then there is an $s$-arc joining $u$ and $v$, which lies in a cycle of length $2s-2$.  Since the diameter of such a cycle is $s-1$, we have a contradiction.  Thus the diameter of $X$ is at most $s-1$, so it is equal to this quantity.
	
	If $X$ is not bipartite, it contains an odd cycle of minimal length $C$.  Because the diameter of $X$ is $s-1$, the cycle must have length $2s-1$.  Let $u$ be a vertex in $C$ and let $v,v'$ be the two vertices at distance $s-1$ from $u$.  Then we can form an $s$-arc $u,\dots,v,v'$ which must lie in a cycle $C'$ of length $2s-2$.  The vertices of $C$ and $C'$ not internal to this $s$-arc form a cycle of length less than $2s-2$, which contradicts our earlier result.
\end{proof}

In a little bit, we'll use this result to show that $s$-arc transitive graphs with girth $2s-2$ are distance transitive.

\ifdraft

\input{../../zach_private_repo/alggraphth_exc/ex4}
\fi