%---------------------------------------------------------------------%
%  LaTeX Course Notes Template                                        %
%                                                                     %
%  Copyright (C) 2012 Zev Chonoles                                    %
%  zevchonoles@gmail.com                                              %
%  http://math.uchicago.edu/~chonoles/                                %
%                                                                     %
%  Please leave this information in the source code as                %
%  attribution if you choose to edit or redistribute this file.       %
%                                                                     %
%  This work is licensed under the Creative Commons Attribution-      %
%  ShareAlike 3.0 Unported License. To view a copy of this license,   %
%  visit http://creativecommons.org/licenses/by-sa/3.0/.              %
%                                                                     %
%---------------------------------------------------------------------%

\documentclass[11pt]{article}






%----------%
%  Basics  %
%----------%


%  Specfies basic information.
%  In the metadata section of the preamble, you can specify the subject and a list of keywords for the PDF.
%
\newcommand{\coursetitle}{Algebraic Graph Theory}
\newcommand{\lecturer}{}
\newcommand{\notetaker}{Zach Schutzman}
\newcommand{\notetakersemail}{ianzach+notes@seas.upenn.edu}
\newcommand{\courseterm}{Fall 2017}
\newcommand{\institution}{University of Pennsylvania}


%  array provides more column styles for the tabular and array environments.
%  (http://ctan.org/pkg/array)
%
%  parskip sets block paragraphs as the default, instead of indentation.
%  (http://www.ctan.org/pkg/parskip)
%
\usepackage[margin=1in]{geometry}
\usepackage{amsmath,amssymb,amsthm,amsfonts,array,parskip}


%  Allows equation, align, gather, etc. environments to split across pages.
\allowdisplaybreaks


%  Sets date formatting to the ISO 8601 standard, YYYY-MM-DD.
\usepackage{datetime} \renewcommand{\dateseparator}{-} \yyyymmdddate






%---------%
%  Fonts  %
%---------%


%  Defines \cal for standard calligraphy, \eucal for Euler calligraphy, and \frak for Fraktur.
\usepackage{eucal}  \let\eucal\mathcal  \let\cal\CMcal  \renewcommand{\frak}{\mathfrak}


%  Removes ligatures (e.g. the connection ordinarily made between the two f's in "differentiable").
\usepackage{microtype} \DisableLigatures{encoding=*,family=*}


%  Removes extra space after periods.




%-------------------------------%
%  Environments and Sectioning  %
%-------------------------------%


%  Defines some standard theorem environments, in both numbered and non-numbered versions. The numbering of each enviroment will be reset for each lecture.
\newcounter{lecture}       \setcounter{lecture}{0}
\newcounter{tN}[lecture]   \newcounter{dN}[lecture]
\newcounter{lN}[lecture]   \newcounter{rN}[lecture]
\newcounter{cN}[lecture]   \newcounter{eN}[lecture]
\newcounter{pN}[lecture]
\newcounter{clN}[lecture]

\newtheorem*{theorem}{Theorem}          \newtheorem{theorem-N}[tN]{Theorem}
\newtheorem*{lemma}{Lemma}              \newtheorem{lemma-N}[lN]{Lemma}
\newtheorem*{corollary}{Corollary}      \newtheorem{corollary-N}[cN]{Corollary}
\newtheorem*{proposition}{Proposition}  \newtheorem{proposition-N}[pN]{Proposition}

\theoremstyle{definition}
\newtheorem*{definition}{Definition}    \newtheorem{definition-N}[dN]{Definition}
\newtheorem*{remark}{Remark}            \newtheorem{remark-N}[rN]{Remark}
\newtheorem*{example}{Example}          \newtheorem{example-N}[eN]{Example}


\newtheorem*{claim}{Claim}    \newtheorem{claim-N}[clN]{Claim}


%  Modifies the spacing above theorem environments, which is messed up when using the parskip package.
%  (http://tex.stackexchange.com/questions/22119)
%
\makeatletter \def\thm@space@setup{\thm@preskip=\parskip \thm@postskip=0pt} \makeatother


%  Modifies the spacing above the proof environment.
%  (http://tex.stackexchange.com/questions/49801)
%
\makeatletter \renewenvironment{proof}[1][\proofname]{\pushQED{\qed}\normalfont
	\partopsep=\z@skip \topsep=\z@skip \trivlist \item[\hskip\labelsep\itshape #1\@addpunct{.}]
	\ignorespaces}{\popQED\endtrivlist\@endpefalse} \makeatother


%  Removes extra space before and after section headings.
\usepackage[compact]{titlesec}






%-------------------------%
%  Pictures and Diagrams  %
%-------------------------%


%  Allows for the use of colors.
%  (http://www.ctan.org/pkg/xcolor)
%
\usepackage[usenames,dvipsnames]{xcolor}
\definecolor{myred}{rgb}{0.9,0.2,0.2}
\definecolor{mygreen}{rgb}{0.2,0.6,0.2}
\definecolor{myblue}{rgb}{0.2,0.2,0.8}


%  graphicx provides advanced graphics options.
%  (http://ctan.org/pkg/graphicx)
%
\usepackage{graphicx}


%  tikz is for drawing all sorts of pictures and diagrams.
%  tikz-cd makes creating commutative diagrams in tikz a bit easier.
%  (http://www.ctan.org/pkg/pgf)
%  (http://www.ctan.org/pkg/tikz-cd)
%
\usepackage{tikz}
\usepackage{tikz-cd}
\usepackage{pgf,pgfplots}
\usetikzlibrary{arrows,calc,decorations,decorations.markings,fadings,positioning,patterns,shapes}
\tikzset{>=latex}
\tikzstyle{mypoint}=[inner sep=0pt,outer sep=0pt,minimum size=5pt,fill,circle]


\definecolor{ttqqqq}{rgb}{0.2,0.,0.}
\definecolor{ffffff}{rgb}{1.,1.,1.}
%\usetikzlibrary{external}
%\tikzexternalize



%------------------------%
%  Commands and Symbols  %
%------------------------%


%  Creates commands by running over a comma-separated list. For example,
%
%     \forcsvlist{\define{\newcommand}{\textbf}{bold}}{A,B}
%
%  would create
%
%     \newcommand{\boldA}{\textbf{A}}    \newcommand{\boldB}{\textbf{B}}
%
%  (http://tex.stackexchange.com/a/5776/20882)
%
\usepackage{etoolbox}
\newcommand{\define}[4]{\expandafter#1\csname#3#4\endcsname{#2{#4}}}
\forcsvlist{\define{\DeclareMathOperator}{}{}}{im,coker,rad,nil,Ann,Ass,codim,Spec,mSpec,diam,ord,Supp,supp,disc,Ob,vol,rank,Sym,Alt,Ind}
\forcsvlist{\define{\newcommand}{\mathrm}{}}{Hom,Mor,id,GL,SL,SO,SU,U,M,Mat,Ext,Tor,Res,Cor,Inf,End,Irr,Aut,Gal,lcm,tr,sign,triv,diag,Map,op,ev,act,alg,sep,unr,nr,ab}

%  Creates commands for some names of categories in the sans-serif font.
\forcsvlist{\define{\newcommand}{\mathsf}{}}{Set,Grp,Ab,CRing,Mod,Vect,Cat,Top,PreSh,Sh,Sch,Nat,Fun,Diff}

%  Creates commands for some blackboard bold letters.
\forcsvlist{\define{\newcommand}{\mathbb}{}}{N,Z,Q,R,C,F,G,T,A,B,D}


%  Saves the section symbol, paragraph symbol, Hungarian accent, and Scandanavian O in the macros \SS, \PP, \HH, and \OO, then redefines \S, \P, \H, and \O to be the corresponding blackboard bold letters.
%
\let\SS\S  \let\PP\P  \let\HH\H  \let\OO\O
\forcsvlist{\define{\renewcommand}{\mathbb}{}}{S,P,H,O}


%  latexsym defines some alternative versions of amssymb symbols.
%  (http://www.bakoma-tex.com/doc/latex/base/latexsym.pdf)
%
\usepackage{latexsym}


%  Defines a copyright symbol that is a bit nicer than the built-in one.
\newcommand{\mycopyrightsymbol}{\raisebox{-0.3ex}{\tikz{\node[inner sep=0pt,outer sep=0pt] at (0,0) {\textsc{c}};\draw (0,0) circle (0.18);}}}


%  Defines commands for real and complex projective space.
\newcommand{\RP}{\mathbb{R}\mathrm{P}}  \newcommand{\CP}{\mathbb{C}\mathrm{P}}


%  Defines a bordered matrix with square bracket delimiters instead of parentheses.
%  (http://tex.stackexchange.com/questions/55054)
%
\let\bbordermatrix\bordermatrix
\patchcmd{\bbordermatrix}{8.75}{4.75}{}{}
\patchcmd{\bbordermatrix}{\left(}{\left[}{}{}
\patchcmd{\bbordermatrix}{\right)}{\right]}{}{}


%  Calls one of the mathabx font families so that it is possible to use its symbols without making a global change.
%  (http://www.ctan.org/pkg/mathabx)
%  (http://tex.stackexchange.com/questions/14386)
%
\DeclareFontFamily{U}{mathb}{\hyphenchar\font45}
\DeclareFontShape{U}{mathb}{m}{n}{<5> <6> <7> <8> <9> <10> gen * mathb
	<10.95> mathb10 <12> <14.4> <17.28> <20.74> <24.88> mathb12}{}
\DeclareSymbolFont{mathb}{U}{mathb}{m}{n}


%  Defines circular arrows.
\DeclareMathSymbol{\lcirclearrow}{0}{mathb}{'366}
\DeclareMathSymbol{\rcirclearrow}{0}{mathb}{'367}
\newcommand{\leftcirclearrow}{\mathrel{\ensuremath{\raisebox{0.1ex}{\scalebox{0.9}{\rotatebox[origin=c]{90}{$\lcirclearrow$}}}}}}
\newcommand{\rightcirclearrow}{\mathrel{\ensuremath{\raisebox{0.1ex}{\scalebox{0.9}{\rotatebox[origin=c]{270}{$\rcirclearrow$}}}}}}


%  Gives semantic names for some common math symbols.
\newcommand{\iso}{\cong}
\newcommand{\htop}{\sim}
\newcommand{\htopequiv}{\simeq}
\newcommand{\cupprod}{\mathbin{\smallsmile}}
\newcommand{\capprod}{\mathbin{\smallfrown}}
\newcommand{\wedgesum}{\mathbin{\vee}}
\newcommand{\boundary}{\partial}
\renewcommand{\emptyset}{\varnothing}
\newcommand{\characteristic}{\mathrm{char}}
\newcommand{\symdiff}{\mathbin{\vartriangle}}
\newcommand{\convolute}{\mathbin{\ast}}
\newcommand{\actson}{\rightcirclearrow}
\newcommand{\actedonby}{\leftcirclearrow}
\newcommand{\directsum}{\oplus}
\newcommand{\bigdirectsum}{\bigoplus}
\newcommand{\tensor}{\otimes}
\newcommand{\bigtensor}{\bigotimes}
\newcommand{\free}{\mathbin{\ast}}
\newcommand{\bigfree}{\mathop{\ensuremath{\raisebox{-0.7ex}{\scalebox{2.3}{$\ast$}}}}}
\renewcommand{\complement}[1]{{#1}^{\mathsf{c}}}
\newcommand{\transpose}[1]{{#1}^{\textsf{T}}}
\newcommand{\union}{\cup}
\newcommand{\intersect}{\cap}
\newcommand{\transverse}{\mathrel{\raisebox{1.1ex}{$-$}\mathllap{\pitchfork\hspace{0.22mm}}}}



\def\multiset#1#2{\ensuremath{\left(\kern-.3em\left(\genfrac{}{}{0pt}{}{#1}{#2}\right)\kern-.3em\right)}}



%-----------------------------------%
%  Things Specific to Course Notes  %
%-----------------------------------%


%  Formatting for the table of contents. The first line allows for multi-column environments, the second line removes the heading "Contents".
\usepackage{multicol} \setlength{\columnsep}{3cm}
\makeatletter \renewcommand\tableofcontents{\@starttoc{toc}} \makeatother


%  Sets the page style.
\usepackage{fancyhdr}
\pagestyle{fancy}
\renewcommand{\headrulewidth}{0pt}
\renewcommand{\footrulewidth}{0.5pt}
\setlength{\headheight}{14pt}
\lfoot{\parbox[t]{1in}{\centering Last edited\\ \today}}
\cfoot{\parbox[t]{3in}{\centering \coursetitle}}
\rfoot{\parbox[t]{0.9in}{\centering Page \thepage\\ Chapter \arabic{lecture}}}


%  Sets the inputs for \maketitle.
\author{%Lectures by \lecturer\\ 
	Notes by \notetaker}
\title{\coursetitle}
\date{\institution, \courseterm}


%  Defines headings for each day's notes.
\newcommand{\classheader}[1]{\stepcounter{lecture}\newpage\section*{Chapter \arabic{lecture} #1}
	\phantomsection \addcontentsline{toc}{section}{Chapter \arabic{lecture} #1}}


%---------------------------------------%
%  Miscellaneous Additions to Template  %
%---------------------------------------%

% http://tex.stackexchange.com/questions/18359
\pgfplotsset{compat=newest}

\newcommand{\Cinfty}{\ensuremath{C^{\infty}}}
\newcommand{\Crit}{\mathrm{Crit}}
\usepackage{mathtools}
\newcommand{\Or}{\mathrm{Or}}
\renewcommand{\Re}{\mathrm{Re}}
\renewcommand{\Im}{\mathrm{Im}}
\usepackage{mathrsfs}
\newtheorem*{examples}{Examples}
\newtheorem*{exercise}{Exercise}
\usepackage{pdfpages}
\newcommand{\Lie}{\mathrm{Lie}}
\newcommand{\Diffeo}{\mathrm{Diffeo}}

\newcommand{\connection}{\nabla}
\newcommand{\new}{\mathrm{new}}


\newcommand{\review}{{\huge\color{myred}{$\star$}}}


%---------------------------%
%  Hyperlinks and Metadata  %
%---------------------------%
%
% (this section must come last!)


%  hyperref enables for the creation of hyperlinks, and also specifies the metadata of the PDF file.
%  hyperxmp allows more metadata to be specified.
%  (http://www.ctan.org/pkg/hyperref)
%  (http://www.ctan.org/pkg/hyperxmp)
%  (http://tex.stackexchange.com/questions/41461)
%
\usepackage{hyperref}
\usepackage{hyperxmp}
\hypersetup{
	pdfauthor={\notetaker},
	pdftitle={\coursetitle},
	pdfproducer={LaTeX},
	%pdfcopyright={Copyright (C) \the\year\ \notetaker. This work is licensed under a Creative Commons Attribution-ShareAlike 3.0 Unported License. All attribution should be to \lecturer\ as the lecturer, and to \notetaker\ as the person taking these notes.},
	pdfsubject={differential topology},
	pdfkeywords={},
	%pdflicenseurl={http://creativecommons.org/licenses/by-sa/3.0/},
	colorlinks=true,
	linkcolor=myred,
	citecolor=mygreen,
	urlcolor=myblue,
	linktoc=page,
	pdfstartview=FitH
}






%------------%
%  Document  %
%------------%


\begin{document}
	
	
	%  The command
	%
	%  \thispagepdflabel{text}
	%
	%  sets the PDF page number (*not* the internal LaTeX page number) to be "text". This does not have to be a numeral; it could be a word, e.g. "Title". This lets one avoid the issue of having the PDF's page numbering not aligning with the page numbering LaTeX used in the document.
	%
	%  (http://tex.stackexchange.com/questions/85558)
	
	
	%  Title
	%
	\maketitle
	\thispdfpagelabel{Title}
	\thispagestyle{empty}
	\setcounter{page}{-1}
	\vspace{0.3in}
	
	
	
	%  Table of Contents
	%
	\begin{center}
		\begin{minipage}[t]{0.9\textwidth}
			\begin{multicols}{2}
				\tableofcontents
			\end{multicols}
		\end{minipage}
	\end{center}
	
	
	
	\newpage
	\thispdfpagelabel{-}
	\thispagestyle{empty}
	
	
	
	%  Introduction
	%
	\section*{Introduction}
	Taking notes and working through proofs is the best way for me to teach myself advanced mathematics.  Typing (and thoroughly backing up) notes is the best way to make sure they are preserved and readable well into the future.  As such, these notes are from my process of working through \textit{Algebraic Graph Theory} by Chris Godsil and Gordon Royle.
	
	I am taking these notes under the assumption that the reader has a familiarity with the basic notions of graph theory and algebra.  I omit elementary definitions and proofs from both domains.  I may go back and fill some of these in if there comes a need or demand for it, but for now, they will be skipped.
	
	My notation differs slightly from that used by Godsil and Royle, and is slightly more consistent with conventions from computer science and algorithmic graph theory at the expense of diverging from algebraic convention.
	
	These notes are being written intermittently, as Algebraic Graph Theory is (currently) not my main research focus. I am using the editor TeXstudio.  The template for these notes was created by Zev Chonoles and is made available (and being used here) under a Creative Commons License. 
	
		I am responsible for all faults in this document, mathematical or otherwise; any merits of the material here should be credited to the authors and those mathematicians they reference.
	
	Please email any corrections or suggestions to \expandafter\href{mailto:\notetakersemail}{\texttt{\notetakersemail}}.
	
	%\medskip
	%
	%\section*{Acknowledgments}
	%
	%Thank you to all of my fellow students who sent me suggestions and corrections, and who lent me their own notes from days I was absent. My notes are much improved due to your help.
	
	
	%%  Copyright
	%%
	%\section*{Copyright}
	%Copyright \mycopyrightsymbol\ 2012 \notetaker.
	%
	%This work is licensed under a Creative Commons Attribution-ShareAlike 3.0 Unported License. This means you are welcome to do essentially anything with this work, including editing, %adapting, distributing, and making commercial use of it, as long as you
	%\begin{itemize}
	%\item include an attribution of \lecturer\ as the lecturer of the course these notes are based on, and \notetaker\ as the person taking the notes,
	%\item do so in a way that does not suggest either of us endorses you or your use of this work, and
	%\item if you alter, transform, or build upon this work, you must apply to your work the same, or similar, license to this one.
	%\end{itemize}
	%More details are available at \href{https://creativecommons.org/licenses/by-sa/3.0/deed.en\_US}{\texttt{https://creativecommons.org/licenses/by-sa/3.0/deed.en\_US}}.
	
	\newpage
	
	
	%  Make a separate file for each lecture, for example, using a naming scheme like this:
	%
	%  lecture1.tex, lecture2.tex, ...
	%
	%  and keep them in the same folder as this main file. By doing it this way (instead of keeping all the notes in the main file), if you're only working on the notes for one lecture, you can easily comment out the lines corresponding to the other lectures.
	%
	\renewcommand{\exc}[1]{\subsubsection*{Exercise 1.#1}}

\classheader{: Graphs}


\subsection*{What is Algebraic Graph Theory?}

\textit{Algebraic graph theory} (abbreviated \textbf{AGT} here \footnote{Not to be confused with \textit{algorithmic game theory}, an area of mathematics and computer science much closer to my primary research interests...} is the subject which explores the relationship between algebra, which broadly studies the properties of abstract mathematical structures, and graph theory, which broadly studies a very particular kind of concrete mathematical structure.  Among these subjects are graph groups and morphisms, spectral graph theory, graph cuts and flows, colorings, and knots.






\section*{Definitions and Fundamentals}

If $X$ is a graph, we let $V(X)$ and $E(X)$ denote the vertex set and edge set of $X$, respectively, using $A(X)$ for the arc set of $X$ in settings with directed graphs.  Unless otherwise specified, we assume all graphs are undirected.  If vertices $u,v\in V(X)$, then we write $(u,v)\in E(X)$ to represent the edge between $u$ and $v$ (or $(u,v)\in A(X)$ to denote the arc \textit{from} $u$ \textit{to} $v$).

\definition{Two graphs $X$ and $Y$ are \textbf{isomorphic} if there exists a bijective function $\phi:V(X)\rightarrow V(Y)$ such that $(\phi(u),\phi(v))\in E(Y)$ if and only if $(u,v)\in E(X)$.}

\definition{A \textbf{subgraph} $Y$ of a graph $X$ is a graph such that $V(Y)\subset V(X)$ and $E(Y)\subset E(X)$.  An \textbf{induced subgraph} is one such that $E(Y)$ consists exactly of the edges $(u,v)$ in $X$ such that $u$ and $v$ are both in $V(Y)$.  That is, an induced subgraph is one which can be realized by deleting vertices from $X$ and removing only those edges incident to those removed vertices.  A \textbf{spanning subgraph} is one such that $V(Y)=V(X)$.}

\definition{A \textbf{cycle} is a subgraph such that every vertex has degree $2$.  A \textbf{tree} is a graph such that no subgraph is a cycle.  A \textbf{spanning tree} is a spanning subgraph with no cycles.}

\definition{A set of vertices which induce an empty (edge-free) subgraph is called an \textbf{independent set}.  A set of vertices which induces a complete graph is called a \textbf{clique}.  The largest independent set and clique in a graph $X$ are denoted $\alpha(X)$ and $\omega(X)$, respectively.}

These values $\alpha(X)$ and $\omega(X)$ will come back later.

\definition{A \textbf{connected component} of a graph is a collection of vertices such that there exist a path between all pairs.}

While adjacency in a graph is not an equivalence relation (it's not transitive), membership in connected components is, hence a graph can be partitioned into disjoint connected components.

\section*{Graph Automorphisms}


\definition{An \textbf{automorphism} of a graph $X$ is an isomorphism $X\rightarrow X$.}

The set of automorphisms of a graph form a group. The identity function is clearly an automorphism, and if $g$ is an automorphism, then its inverse, $g^{-1}$ is as well.  We can also compose automorphisms to get another automorphism, and this inherits associativity from function composition.  By Cayley's theorem, we can think about $Aut(X)$ as being a subgroup of $Sym(V(X))$, the symmetric group on the vertices of $X$. We'll write $Sym(n)$ for $n=|V(X)|$ to denote the symmetric group on $n$ elements in place of $Sym(V(X))$. 

In general, it is difficult to determine whether two graphs are isomoprhic (this is a well-known NP problem) or whether a graph has a nontrivial automorphism.  However, some cases are easy.  For a complete graph $K_n$, $Aut(X)=Sym(n)$, and the same holds for an empty graph on $n$ vertices.

If $v$ is a vertex and $g$ a group element, we denote $v^g$ the action of $g$ on $v$.  If $g\in Aut(X)$ and $Y$ is a subgraph of $X$, then we denote $Y^g = \{x^g|x\in V(Y)\}$.  Then we have that $E(Y^g) = \{(u^g,v^g)|(u,v)\in E(Y)\}$.  The graphs $Y$ and $Y^g$ are isomorphic, and $Y^g$ is a subgraph of $X$.

\definition{The \textbf{valency} of a vertex $x$ is the number of neighbors of $x$ in the graph $X$.  We can talk about the maximum and minimum valencies over all vertices of a graph.}

\lemma{If $x$ is a vertex of a graph $X$ and $g$ is an automorphism of $X$, then the vertex $y=x^g$, has the same valency as $x$.}

\begin{proof}
	Let $N(x)$ be the subgraph of $X$ induced by $x$ and its neighbors.  Then $N(x)^g\iso N(x^g)\iso N(y)$, so $N(x)\iso N(y)$ as subgraphs of $X$, so they have the same number of vertices.  Thus the valencies of $x$ and $y$ are equal.
\end{proof}

\corollary{An automorphism of a graph necessarily permutes vertices of the same valency.}

\definition{A graph where every vertex has valency $k$ is called $\boldsymbol{k}$\textbf{-regular}.}

\definition{The \textbf{distance} between vertices $x$ and $y$ is the length of the shortest path in $X$ between $x$ and $y$, denoted $d_X(x,y)$ or $d(x,y)$ if it is clear which graph we are talking about.}

\lemma{If $g$ is an automorphism of a graph $X$, then $d_X(x,y)=d_X(x^g,y^g)$ for all pairs of vertices.}

\begin{proof}
	If they are the same vertex, the distance $d(x,y)=d(x^g,y^g)=0$ is trivially preserved.  If $d(x,y)=1$, then $x$ and $y$ are adjacent, so their images $x^g$ and $y^g$ must be adjacent as well, by definition of graph isomorphism.
	
	Suppose, for the sake of contradiction that $d(x,y)\lneq d(x^g,y^g)$.  Then there is some path $x,r_2,r_3,\dots r_{n-1},y$ such that $r_{n-1}^g$ is not adjacent to $y^g$.  But this is impossible, as automorphism preserves adjacency, and $r_{n-1}$ is adjacent to $y$.  A symmetric argument on $g^{-1}$ gives the case where $d(x,y)\gneq d(x^g,y^g)$.
	
	
	
\end{proof}

\definition{The \textbf{complement} of a graph $X$, denoted $\overline{X}$, is the graph such that $V(\overline{X})=V(X)$ and $E(\overline{X}) = \{(u,v)|(u,v)\notin E(X)\}$.  That is, the complement of a graph is the one which has an edge between two vertices if and only if the original graph does not.}

\lemma{$Aut(X)=Aut(\overline{X})$.}

\begin{proof}
	Since automorphisms preserve adjacency, they also preserve non-adjacency.  Thus $x^g$ is not adjacent to $y^g$ if and only if $x$ and $y$ are not adjacent.  Therefore, $g\in Aut(\overline{X})$.
\end{proof}

We'll quickly note that automorphisms of directed graphs also preserve the direction of the arcs.


\section*{Graph Homomorphisms}
\definition{A \textbf{graph homomorphism} is a function $\phi:V(X)\rightarrow V(Y)$ such that if $u$ and $v$ are adjacent in $X$, they are adjacent in $Y$.}

We'll quickly contrast this to isomorphisms, which preserves adjacency in both directions, whereas a homomorphism only requires that adjacent vertices in $X$ are still adjacent in $Y$ under $\phi$.  Every isomorphism is a homomorphism, but not every homomorphism is an isomorphism.

\definition{A graph is \textbf{bipartite} if there exists a partition of $V(X)$ into disjoint sets $A$ and $B$ such that every edge has one end in $A$ and the other in $B$. Analogously, we can define $\boldsymbol{k}$\textbf{-partite} graphs as being those which admit a partition into $k$ components such that no edge has both endpoints in the same component.}

If a graph is bipartite, there exists a homomorphism $X\rightarrow K_2$ where the image of each component is one of the vertices in $K_2$.  Similarly, there is a homomorphism from a $k$-partite graph onto $K_k$.

This leads to the notion of \textit{proper colorings}.

\definition{A \textbf{proper coloring} is a map from $V(X)$ to a finite set of colors such that for any edge $(u,v)\in E(X)$, $u$ and $v$ are assigned different colors.}

\definition{The \textbf{chromatic number} of a graph, denoted $\chi(X)$ is the minimum number $k$ such that $X$ can be properly $k$-colored.}

Nonempty bipartite graphs have chromatic number $2$.  Complete graphs $K_n$ have chromatic number $n$.

Let's observe that the set of vertices assigned some particular color, called a \textit{color class}, forms an independent set in $X$.

\lemma{The chromatic number of a graph $\chi(X)$ is the minimum number $r$ such that there exists a homomorphism from $X$ to $K_r$.}


\begin{proof}
	Suppose $\phi:V(X)\rightarrow V(Y)$ is a homomorphism.  For $y\in V(Y)$, define $\phi^{-1}(y)$ to be $\phi^{-1}(y)=\{x\in V(X)|\phi(x)=y\}$, the set of elements in $V(X)$ which map to $Y$ under $\phi$.  As $y$ is not adjacent to itself, $\phi^{-1}(y)$ is an independent set.  Hence if $Y$ has $r$ vertices, each of the $r$ sets is independent and forms a color class of an $r$-coloring, so $X$ can be properly $r$-colored.  Conversely, suppose that $X$ can be properly $r$-colored.  Then there exists a homomorphism onto $K_r$ which sends each color class to a unique vertex.
\end{proof}

\definition{A \textbf{retraction} is a homomorphism $\phi$ from $X$ to $Y$ where $Y$ is a subgraph of $X$ such that the restriction of $X$ to $Y$ is the identity map.}

If $X$ is a graph with a $k$-clique, then any $k$-coloring of $X$ determines a retraction of $X$ onto the clique.

When we think about directed graphs, we will also stipulate that homomorphisms preserve the directions of arcs.

\definition{An \textbf{endomorphism} of a graph is a homomorphism from a graph to itself.  The set of endomorphisms, $End(X)$, forms a monoid. An automorphism is a special case of an endomorphism, so $Aut(X)$ is a submonoid of $End(X)$.}


\section*{Circulant Graphs}

Let's give a more particular definition of a \textit{cycle} in a graph.  We can think of a cycle of $n$ vertices as a set $C_n=\{0,1,2,\dots n-1\}$ of vertices such that $i$ and $j$ are adjacent if and only if $j-i\equiv\pm 1\mod n$.

Let's look at some automorphisms of the cycle.  The set of permutations which map $i$ to $i+1$ (and $n-1$ to $0$) forms a subset of $Aut(C_n)$.  By composition, we can realize an entire copy of the cyclic group on $n$ elements ($\mathbb{Z}_n$) in this way.  Also, the permutation $h$ which sends $i$ to $-i\mod n$ is an element of $Aut(C_n)$.  We have that $h(0)=0$ but the cyclic group is fixed point-free, so this automorphism isn't contained in that subgroup.  Also, $h=h^{-1}$, so there are two cosets induced by this element, and the order of $Aut(C_n)$ is at least $2n$.  (In fact, it's equal to $2n$, but we can't quite prove that yet...)

The cycles are a subclass of the \textit{circulant graphs}.  If $C\subset \mathbb{Z}_n{\setminus}0$, then we can construct the directed graph $X=X(\mathbb{Z}_n,C)$ through the following process.  First, let $V(X)$ be the elements of $\mathbb{Z}_n$ and let $(i,j)\in A(X)$ if and only if $j-i\in C$.  This graph $X(\mathbb{Z}_m,C)$ is called a \textit{circulant of order $n$} and $C$ is its \textit{connection set}.  If $C$ itself is also closed under additive inverses (modulo $n$), then $(i,j)$ is an arc in $X$ if and only if $(j,i)$ is, so we can view the graph as being undirected.  In this case, the map which sends $i$ to $-i$ is an automorphism, and the map which sends $i$ to $i+1$ is always an automorphism of a circulant graph, so the automorphism group of a circulant graph with an inverse-closed connection set is at least $2n$.  We can think of the ordinary cycle on $n$ vertices as being $X(\mathbb{Z}_n,\{-1,1\})$.  The complete graph is a circulant graph with connection set $\mathbb{Z}_n$, and an empty graph is one with empty connection set.  Since these graphs have automorphism groups with order $n!$, we clearly have examples of circulant graphs with orders much larger than $2n$.

\section*{Johnson Graphs}

Now we consider another family of graphs, denoted $J(v,k,i)$ for positive integers $v\geq k\geq i$.  Let $\Omega$ be some fixed set of size $v$.  The vertices of $J(v,k,i)$ are the subsets of $\Omega$ with size $k$, and two vertices are adjacent if and only if their corresponding sets have intersection size $i$.  Thus $J(v,k,i)$ has $\binom{v}{k}$ vertices, and it is a regular graph in which each vertex has valency $\binom{k}{i}\binom{v-k}{k-i}$.  We'll assume $v\geq 2k$.

\begin{lemma}
{The function which maps a set of size $k$ to its complement in $\Omega$ is an isomorphism between the graphs $J(v,k,i)$ and $J(v,v-k,v-2k+i)$.}
\end{lemma}
\begin{proof}
	The proof of this is just a DeMorgan's Laws definition-chase.  
	
	If $|A|=|B|=k$, then $|\overline{A}|=|\overline{B}|=v-k$.
	
	If $A$ and $B$ are adjacent, then $|A\cap B|=i$, so $|\overline{A}\cap\overline{B}| = |\overline{A\cup B}| = v-2k+i$.
	
	Therefore, if we define a map by mapping a set to its complement and adjacency occurs if and only if the intersection of the sets is size $v-2k+i$, this is indeed an automorphism, as $A$ and $B$ are adjacent if and only if $\overline{A}$ and $\overline{B}$ are adjacent, and set complements is an obvious bijection between the vertex sets.
	
	
	
\end{proof}

A graph is called a \textit{Johnson graph} if it is isomorphic to $J(v,k,k-1)$.  The \textit{Kneser graphs} are isomorphic to $J(v,k,0)$.  As an example, the Petersen graph, which we will study later, is $J(5,2,0)$ and is therefore a Kneser graph.

\begin{lemma}
	{If $v\geq k \geq i$, then $Aut(J(v,k,i))$ contains a subgroup isomorphic to $Sym(v)$.}
\end{lemma}

\begin{proof}
	Let $g$ be a permutation of $\Omega$ and $S\subset \Omega$, and let $S^g$ denote the image of $S$ under $g$.  Any such $g$ also determines a permutation of the subsets $S$ of size $k$.  In particular, if $S$ and $T$ are of size $k$, then $|S\cap T|=|S^g\cap T^g|$, so $g$ is an automorphism of $J(v,k,i)$.
\end{proof}

We note that $Aut(J(v,k,i))$ acts on a set of size $\binom{v}{k}$, so when this quantity is not equal to $v$, it's not equal to $Sym(v)$, but it is \textit{usually} isomorphic, which is often not an easy thing to prove.


\section*{Line Graphs}

\definition{If $X$ is a graph, the \textbf{line graph} of $X$, denoted $L(X)$ is the graph where the vertices of $L(X)$ correspond to edges of $X$ and two vertices in $L(X)$ are adjacent if and only if the corresponding edges in $X$ are incident to the same vertex.}

As examples, the star $K_{1,n}$ (one hub with $n$ `spokes') has line graph $K_n$, as all $n$ edges in the star are incident to the center vertex.  The path graph on $n$ vertices $P_n$ has $L(P_n)=P_{n-1}$.  The cycle $C_n$ is isomorphic to its own line graph.

\begin{lemma}
{If $X$ is regular with valency $k$, then $L(X)$ is regular with valency $2k-2$.}
\end{lemma}
\begin{proof}
	Each vertex has degree $k$, so when we translate each edge into a vertex, for each original vertex, we get a $k$-clique, but each of these new vertices belongs to two such cliques.  Thus each vertex has $k-1$ adjacent vertices in each of the cliques it belongs to, thus a total valency of $2k-2$.
\end{proof}


\theorem{A graph is the line graph of some other graph if and only if there exists a partition of its vertex set into cliques such that each vertex belongs to at most two cliques.}

\begin{proof}
	
	To see that the condition is necessary, observe that the process of constructing a line graph necessarily turns the neighborhood of each vertex into a clique, and since an edge connects two vertices, each new vertex belongs to at most two such cliques.
	
	To see that it is sufficient, we will construct a graph from a line graph which decomposes into cliques in this way.  Let $S_1,S_2,\dots,S_k$ be the cliques, and let $v_1,v_2,\dots,v_m$ be the vertices (if there are any) which are in exactly one $S_i$.  The vertex set of our graph will be $S_1,\dots,S_k,\{v_1\},\dots \{v_m\}$ with an edge between sets if and only their intersection is nonempty.  It is clear that the line graph of this graph is our original graph, and we are done.
	
\end{proof}

Observe that if $X$ and $Y$ are isomorphic, then $L(X)$ and $L(Y)$ are isomorphic, but the converse isn't true, as $K_3$ and $K_{1,3}$ have the same line graphs.

\begin{lemma}
	
{If $X$ and $Y$ are graphs with minimum valency at least $4$, then $X\iso Y$ if and only if $L(X)\iso L(Y)$.}
\end{lemma}
\begin{proof}
	Let $C$ be a clique in $L(X)$ with $|C|=c<4$.  The vertices in $C$ correspond to a set of $c$ edges in $X$, all of which are incident to a common vertex $x$.  Thus, there is a bijection between vertices of $X$ and maximal cliques in $L(X)$ which maps adjacent vertices in $X$ to pairs of cliques in $L(X)$ which share exactly one vertex.  We can similarly construct an analogous bijection between $Y$ and $L(Y)$.  Let $f:X\rightarrow L(X)$ and $g:Y\rightarrow L(Y)$ be these functions.
	
	If we assume $X\iso Y$ by $\phi$, then we want to show that $L(X)\iso L(Y)$ by demonstrating that $g\circ\phi\circ f^{-1}:L(X)\rightarrow L(Y)$ is an isomorphism.  It suffices to show that the image of a $k$-clique under this composite function is a $k$-clique in $L(Y)$.  But this is obvious.  $f^{-1}$ takes a maximal $k$-clique to a set of $k$ edges in $X$ incident to some vertex $x$, which has valency $k$.  Then $\phi(x)=y$ is some vertex in $Y$ with valency $k$, so $g$ sends this neighborhood to a maximal $k$-clique.
	
	The other direction has an identical proof, except that we show that vertices in $X$ and $Y$ with equal valency are mapped to each other.
\end{proof}

\theorem{A graph is a line graph if and only if each induced graph on at most six vertices is also a line graph.}

\begin{proof}
	
	This is an alternative phrasing of Beineke's Theorem.  I'll fill in a proof later.
	
\end{proof}

\corollary{The set of graphs which are not line graphs but every induced subgraph is a line graph is finite and, in fact, of size nine.}



\definition{A bipartite graph is \textbf{semiregular} if it has a proper $2$-coloring such that all vertices of the same color have the same valency.  As an example, the complete bipartite graphs $K_{m,n}$ (a set of $m$ vertices connected to each of a set of $n$ vertices) are semiregular.}

\begin{lemma}
{If the line graph of a graph is regular, then the graph itself is regular or a semiregular bipartite graph.}
\end{lemma}
\begin{proof}
	Suppose $L(X)$ is $k$-regular.  If $u$ and $v$ are adjacent in $X$, then their valencies sum to $k+2$, so all neighbors of $u$ have the same valency, so if two vertices share a neighbor, they have identical valencies.  This only occurs in graphs which are regular or bipartite and semiregular, as if it contains an odd cycle, it must have two adjacent vertices with the same valencies, and bipartite graphs have no odd cycles.
\end{proof}


\section*{Planar Graphs}

\definition{A graph is called \textbf{planar} if it can be drawn (in the plane) without crossing edges.  More precisely, a graph is planar if there exists a function which maps each vertex to a unique point in $\mathbb{R}^2$ and each edge to a non-self-intersecting curve with endpoints equal to the image of the vertices it's incident to such that no two such curves intersect. Such a function is called a \textbf{planar embedding}.}

\definition{A \textbf{plane graph} is a planar graph together with a planar embedding.}

The edges of a plane graph divide the plane into disjoint regions called \textit{faces}.  All but one (the \textit{external} or \textit{infinite}) face is bounded.  The \textit{length} of a face is the number of edges bounding it.

\theorem[Euler]{If $v-e+f=2$, where $v,e,f$ are the number of vertices, edges, and faces of a plane graph, respectively.}

\begin{proof}
	
	The proof proceeds by strong induction on the number of edges. Observe that a tree on $v$ vertices is a planar graph with $v-1$ edges and $1$ face.  If a planar graph is not a tree, it contains a cycle.  Removing an edge in this cycle (which does not disconnect the graph) merges two faces, which preserves the quantity $v-e+f$.  Since a tree is a graph without cycles, and this process eventually transforms a graph into a tree, but since a tree satisfies $v-e+f=2$, this quantity must be preserved at all steps of the process, hence it is true for the original graph.
	
	
\end{proof}

\definition{A \textbf{maximal planar graph} is one in which adding an edge between any two vertices which are not already adjacent makes the graph non-planar.  If a planar graph has an embedding where the length of some face is greater than 3, we can add edges interior to this face in without violating planarity.  Thus any maximal planar graph must have every face be of length 3, called a \textbf{planar triangulation}.}

In a triangulation, each edge is incident to two faces, so we have $3f=2e$.  Then by Euler's theorem, $e=3n-6$.  Any planar graph with $3n-6$ edges must be maximal and a planar triangulation.

A planar graph may have multiple distinct embeddings, and they don't necessarily preserve the lengths of the faces (although it must preserve the \textit{number} of faces).  It is a result in topological graph theory that a $3$-connected planar graph has a (topologically) unique planar embedding.

Given a plane graph $X$, we can construct its dual $X^*$, where each face of $X$ becomes a vertex of $X^*$ with edges between vertices in $X^*$ if and only if there is an edge separating the corresponding faces in $X$.  Sometimes this gives rise to multiple edges between vertices, but we'll be sure to only worry about that if we have to.

The dual of a planar graph is connected, so if $X$ is not connected, $(X^*)^*$ is not isomorphic to $X$, but this is true if $X$ is connected.

We can generalize the notion of planar embeddings to embeddings in any surface.  The  dual is defined analogously in these topological spaces.  The real projective plane $\RP^2$ is a non-orientable surface which looks like the closed disk with an antipodal identification along the boundary.  The graph $K_6$ is not planar, but it does have an embedding in $\RP^2$ (which is triangular!), and its dual in this space is cubic, and turns out to be the Petersen graph.

The torus is an orientable surface, which looks like the surface of a donut.  We can represent it as a rectangle with opposite edges identified.  The graph $K_7$ is not planar, but there is an embedding on the torus (which is also triangular!), and its dual is the Heawood graph.


\ifdraft
\input{../../zach_private_repo/alggraphth_exc/ex1}
\fi

	\classheader{}



	\end{document}
	
