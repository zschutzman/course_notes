%---------------------------------------------------------------------%
%  LaTeX Course Notes Template                                        %
%                                                                     %
%  Copyright (C) 2012 Zev Chonoles                                    %
%  zevchonoles@gmail.com                                              %
%  http://math.uchicago.edu/~chonoles/                                %
%                                                                     %
%  Please leave this information in the source code as                %
%  attribution if you choose to edit or redistribute this file.       %
%                                                                     %
%  This work is licensed under the Creative Commons Attribution-      %
%  ShareAlike 3.0 Unported License. To view a copy of this license,   %
%  visit http://creativecommons.org/licenses/by-sa/3.0/.              %
%                                                                     %
%---------------------------------------------------------------------%
\newif\ifdraft \draftfalse


\documentclass[11pt]{article}






%----------%
%  Basics  %
%----------%


%  Specfies basic information.
%  In the metadata section of the preamble, you can specify the subject and a list of keywords for the PDF.
%
\newcommand{\coursetitle}{Algebraic Graph Theory}
\newcommand{\lecturer}{}
\newcommand{\notetaker}{Zach Schutzman}
\newcommand{\notetakersemail}{ianzach+notes@seas.upenn.edu}
\newcommand{\courseterm}{Fall 2017}
\newcommand{\institution}{University of Pennsylvania}


%  array provides more column styles for the tabular and array environments.
%  (http://ctan.org/pkg/array)
%
%  parskip sets block paragraphs as the default, instead of indentation.
%  (http://www.ctan.org/pkg/parskip)
%
\usepackage[margin=1in]{geometry}
\usepackage{amsmath,amssymb,amsthm,amsfonts,array,parskip,comment}


%  Allows equation, align, gather, etc. environments to split across pages.
\allowdisplaybreaks


%  Sets date formatting to the ISO 8601 standard, YYYY-MM-DD.
\usepackage{datetime} \renewcommand{\dateseparator}{-} \yyyymmdddate






%---------%
%  Fonts  %
%---------%


%  Defines \cal for standard calligraphy, \eucal for Euler calligraphy, and \frak for Fraktur.
\usepackage{eucal}  \let\eucal\mathcal  \let\cal\CMcal  \renewcommand{\frak}{\mathfrak}


%  Removes ligatures (e.g. the connection ordinarily made between the two f's in "differentiable").
\usepackage{microtype} \DisableLigatures{encoding=*,family=*}


%  Removes extra space after periods.




%-------------------------------%
%  Environments and Sectioning  %
%-------------------------------%


%  Defines some standard theorem environments, in both numbered and non-numbered versions. The numbering of each enviroment will be reset for each lecture.
\newcounter{lecture}       \setcounter{lecture}{0}
\newcounter{tN}[lecture]   \newcounter{dN}[lecture]
\newcounter{lN}[lecture]   \newcounter{rN}[lecture]
\newcounter{cN}[lecture]   \newcounter{eN}[lecture]
\newcounter{pN}[lecture]
\newcounter{clN}[lecture]

\newtheorem*{theorem}{Theorem}          \newtheorem{theorem-N}[tN]{Theorem}
\newtheorem*{lemma}{Lemma}              \newtheorem{lemma-N}[lN]{Lemma}
\newtheorem*{corollary}{Corollary}      \newtheorem{corollary-N}[cN]{Corollary}
\newtheorem*{proposition}{Proposition}  \newtheorem{proposition-N}[pN]{Proposition}

\theoremstyle{definition}
\newtheorem*{definition}{Definition}    \newtheorem{definition-N}[dN]{Definition}
\newtheorem*{remark}{Remark}            \newtheorem{remark-N}[rN]{Remark}
\newtheorem*{example}{Example}          \newtheorem{example-N}[eN]{Example}


\newtheorem*{claim}{Claim}    \newtheorem{claim-N}[clN]{Claim}


%  Modifies the spacing above theorem environments, which is messed up when using the parskip package.
%  (http://tex.stackexchange.com/questions/22119)
%
\makeatletter \def\thm@space@setup{\thm@preskip=\parskip \thm@postskip=0pt} \makeatother


%  Modifies the spacing above the proof environment.
%  (http://tex.stackexchange.com/questions/49801)
%
\makeatletter \renewenvironment{proof}[1][\proofname]{\pushQED{\qed}\normalfont
	\partopsep=\z@skip \topsep=\z@skip \trivlist \item[\hskip\labelsep\itshape #1\@addpunct{.}]
	\ignorespaces}{\popQED\endtrivlist\@endpefalse} \makeatother


%  Removes extra space before and after section headings.
\usepackage[compact]{titlesec}






%-------------------------%
%  Pictures and Diagrams  %
%-------------------------%


%  Allows for the use of colors.
%  (http://www.ctan.org/pkg/xcolor)
%
\usepackage[usenames,dvipsnames]{xcolor}
\definecolor{myred}{rgb}{0.9,0.2,0.2}
\definecolor{mygreen}{rgb}{0.2,0.6,0.2}
\definecolor{myblue}{rgb}{0.2,0.2,0.8}


%  graphicx provides advanced graphics options.
%  (http://ctan.org/pkg/graphicx)
%
\usepackage{graphicx}


%  tikz is for drawing all sorts of pictures and diagrams.
%  tikz-cd makes creating commutative diagrams in tikz a bit easier.
%  (http://www.ctan.org/pkg/pgf)
%  (http://www.ctan.org/pkg/tikz-cd)
%
\usepackage{tikz}
\usepackage{tikz-cd}
\usepackage{pgf,pgfplots}
\usetikzlibrary{arrows,calc,decorations,decorations.markings,fadings,positioning,patterns,shapes}
\tikzset{>=latex}
\tikzstyle{mypoint}=[inner sep=0pt,outer sep=0pt,minimum size=5pt,fill,circle]


\definecolor{ttqqqq}{rgb}{0.2,0.,0.}
\definecolor{ffffff}{rgb}{1.,1.,1.}
%\usetikzlibrary{external}
%\tikzexternalize



%------------------------%
%  Commands and Symbols  %
%------------------------%


%  Creates commands by running over a comma-separated list. For example,
%
%     \forcsvlist{\define{\newcommand}{\textbf}{bold}}{A,B}
%
%  would create
%
%     \newcommand{\boldA}{\textbf{A}}    \newcommand{\boldB}{\textbf{B}}
%
%  (http://tex.stackexchange.com/a/5776/20882)
%
\usepackage{etoolbox}
\newcommand{\define}[4]{\expandafter#1\csname#3#4\endcsname{#2{#4}}}
\forcsvlist{\define{\DeclareMathOperator}{}{}}{im,coker,rad,nil,Ann,Ass,codim,Spec,mSpec,diam,ord,Supp,supp,disc,Ob,vol,rank,Sym,Alt,Ind}
\forcsvlist{\define{\newcommand}{\mathrm}{}}{Hom,Mor,id,GL,SL,SO,SU,U,M,Mat,Ext,Tor,Res,Cor,Inf,End,Irr,Aut,Gal,lcm,tr,sign,triv,diag,Map,op,ev,act,alg,sep,unr,nr,ab}

%  Creates commands for some names of categories in the sans-serif font.
\forcsvlist{\define{\newcommand}{\mathsf}{}}{Set,Grp,Ab,CRing,Mod,Vect,Cat,Top,PreSh,Sh,Sch,Nat,Fun,Diff}

%  Creates commands for some blackboard bold letters.
\forcsvlist{\define{\newcommand}{\mathbb}{}}{N,Z,Q,R,C,F,G,T,A,B,D}


%  Saves the section symbol, paragraph symbol, Hungarian accent, and Scandanavian O in the macros \SS, \PP, \HH, and \OO, then redefines \S, \P, \H, and \O to be the corresponding blackboard bold letters.
%
\let\SS\S  \let\PP\P  \let\HH\H  \let\OO\O
\forcsvlist{\define{\renewcommand}{\mathbb}{}}{S,P,H,O}


%  latexsym defines some alternative versions of amssymb symbols.
%  (http://www.bakoma-tex.com/doc/latex/base/latexsym.pdf)
%
\usepackage{latexsym}


%  Defines a copyright symbol that is a bit nicer than the built-in one.
\newcommand{\mycopyrightsymbol}{\raisebox{-0.3ex}{\tikz{\node[inner sep=0pt,outer sep=0pt] at (0,0) {\textsc{c}};\draw (0,0) circle (0.18);}}}


%  Defines commands for real and complex projective space.
\newcommand{\RP}{\mathbb{R}\mathrm{P}}  \newcommand{\CP}{\mathbb{C}\mathrm{P}}


%  Defines a bordered matrix with square bracket delimiters instead of parentheses.
%  (http://tex.stackexchange.com/questions/55054)
%
\let\bbordermatrix\bordermatrix
\patchcmd{\bbordermatrix}{8.75}{4.75}{}{}
\patchcmd{\bbordermatrix}{\left(}{\left[}{}{}
\patchcmd{\bbordermatrix}{\right)}{\right]}{}{}


%  Calls one of the mathabx font families so that it is possible to use its symbols without making a global change.
%  (http://www.ctan.org/pkg/mathabx)
%  (http://tex.stackexchange.com/questions/14386)
%
\DeclareFontFamily{U}{mathb}{\hyphenchar\font45}
\DeclareFontShape{U}{mathb}{m}{n}{<5> <6> <7> <8> <9> <10> gen * mathb
	<10.95> mathb10 <12> <14.4> <17.28> <20.74> <24.88> mathb12}{}
\DeclareSymbolFont{mathb}{U}{mathb}{m}{n}


%  Defines circular arrows.
\DeclareMathSymbol{\lcirclearrow}{0}{mathb}{'366}
\DeclareMathSymbol{\rcirclearrow}{0}{mathb}{'367}
\newcommand{\leftcirclearrow}{\mathrel{\ensuremath{\raisebox{0.1ex}{\scalebox{0.9}{\rotatebox[origin=c]{90}{$\lcirclearrow$}}}}}}
\newcommand{\rightcirclearrow}{\mathrel{\ensuremath{\raisebox{0.1ex}{\scalebox{0.9}{\rotatebox[origin=c]{270}{$\rcirclearrow$}}}}}}


%  Gives semantic names for some common math symbols.
\newcommand{\iso}{\cong}
\newcommand{\htop}{\sim}
\newcommand{\htopequiv}{\simeq}
\newcommand{\cupprod}{\mathbin{\smallsmile}}
\newcommand{\capprod}{\mathbin{\smallfrown}}
\newcommand{\wedgesum}{\mathbin{\vee}}
\newcommand{\boundary}{\partial}
\renewcommand{\emptyset}{\varnothing}
\newcommand{\characteristic}{\mathrm{char}}
\newcommand{\symdiff}{\mathbin{\vartriangle}}
\newcommand{\convolute}{\mathbin{\ast}}
\newcommand{\actson}{\rightcirclearrow}
\newcommand{\actedonby}{\leftcirclearrow}
\newcommand{\directsum}{\oplus}
\newcommand{\bigdirectsum}{\bigoplus}
\newcommand{\tensor}{\otimes}
\newcommand{\bigtensor}{\bigotimes}
\newcommand{\free}{\mathbin{\ast}}
\newcommand{\bigfree}{\mathop{\ensuremath{\raisebox{-0.7ex}{\scalebox{2.3}{$\ast$}}}}}
\renewcommand{\complement}[1]{{#1}^{\mathsf{c}}}
\newcommand{\transpose}[1]{{#1}^{\textsf{T}}}
\newcommand{\union}{\cup}
\newcommand{\intersect}{\cap}
\newcommand{\transverse}{\mathrel{\raisebox{1.1ex}{$-$}\mathllap{\pitchfork\hspace{0.22mm}}}}



\def\multiset#1#2{\ensuremath{\left(\kern-.3em\left(\genfrac{}{}{0pt}{}{#1}{#2}\right)\kern-.3em\right)}}



%-----------------------------------%
%  Things Specific to Course Notes  %
%-----------------------------------%


%  Formatting for the table of contents. The first line allows for multi-column environments, the second line removes the heading "Contents".
\usepackage{multicol} \setlength{\columnsep}{3cm}
\makeatletter \renewcommand\tableofcontents{\@starttoc{toc}} \makeatother


%  Sets the page style.
\usepackage{fancyhdr}
\pagestyle{fancy}
\renewcommand{\headrulewidth}{0pt}
\renewcommand{\footrulewidth}{0.5pt}
\setlength{\headheight}{14pt}
\lfoot{\parbox[t]{1in}{\centering Last edited\\ \today}}
\cfoot{\parbox[t]{3in}{\centering \coursetitle}}
\rfoot{\parbox[t]{0.9in}{\centering Page \thepage\\ Chapter \arabic{lecture}}}


%  Sets the inputs for \maketitle.
\author{%Lectures by \lecturer\\ 
	Notes by \notetaker}
\title{\coursetitle}
\date{\institution, \courseterm}


%  Defines headings for each day's notes.
\newcommand{\classheader}[1]{\stepcounter{lecture}\newpage\section*{Chapter \arabic{lecture} #1}
	\phantomsection \addcontentsline{toc}{section}{Chapter \arabic{lecture} #1}}


%---------------------------------------%
%  Miscellaneous Additions to Template  %
%---------------------------------------%

% http://tex.stackexchange.com/questions/18359
\pgfplotsset{compat=newest}

\newcommand{\Cinfty}{\ensuremath{C^{\infty}}}
\newcommand{\Crit}{\mathrm{Crit}}
\usepackage{mathtools}
\newcommand{\Or}{\mathrm{Or}}
\renewcommand{\Re}{\mathrm{Re}}
\renewcommand{\Im}{\mathrm{Im}}
\usepackage{mathrsfs}
\newtheorem*{examples}{Examples}
\newtheorem*{exercise}{Exercise}
\usepackage{pdfpages}
\newcommand{\Lie}{\mathrm{Lie}}
\newcommand{\Diffeo}{\mathrm{Diffeo}}

\newcommand{\connection}{\nabla}
\newcommand{\new}{\mathrm{new}}


\newcommand{\review}{{\huge\color{myred}{$\star$}}}


%---------------------------%
%  Hyperlinks and Metadata  %
%---------------------------%
%
% (this section must come last!)


%  hyperref enables for the creation of hyperlinks, and also specifies the metadata of the PDF file.
%  hyperxmp allows more metadata to be specified.
%  (http://www.ctan.org/pkg/hyperref)
%  (http://www.ctan.org/pkg/hyperxmp)
%  (http://tex.stackexchange.com/questions/41461)
%
\usepackage{hyperref}
\usepackage{hyperxmp}
\hypersetup{
	pdfauthor={\notetaker},
	pdftitle={\coursetitle},
	pdfproducer={LaTeX},
	%pdfcopyright={Copyright (C) \the\year\ \notetaker. This work is licensed under a Creative Commons Attribution-ShareAlike 3.0 Unported License. All attribution should be to \lecturer\ as the lecturer, and to \notetaker\ as the person taking these notes.},
	pdfsubject={differential topology},
	pdfkeywords={},
	%pdflicenseurl={http://creativecommons.org/licenses/by-sa/3.0/},
	colorlinks=true,
	linkcolor=myred,
	citecolor=mygreen,
	urlcolor=myblue,
	linktoc=page,
	pdfstartview=FitH
}



\newcommand{\thrm}[1]{\theorem{#1}}



\makeatletter
\renewcommand\subsubsection{\@startsection{subsubsection}{3}{\z@}%
	{-3.25ex\@plus -1ex \@minus -.2ex}%
	{-1.5ex \@plus -.2ex}% Formerly 1.5ex \@plus .2ex
	{\normalfont\normalsize\bfseries}}
\makeatother

\newcommand{\exc}[1]{\subsubsection*{Exercise \hwnumber.#1}}
%------------%
%  Document  %
%------------%


\begin{document}
	
	
	%  The command
	%
	%  \thispagepdflabel{text}
	%
	%  sets the PDF page number (*not* the internal LaTeX page number) to be "text". This does not have to be a numeral; it could be a word, e.g. "Title". This lets one avoid the issue of having the PDF's page numbering not aligning with the page numbering LaTeX used in the document.
	%
	%  (http://tex.stackexchange.com/questions/85558)
	
	
	%  Title
	%
	\maketitle
	\thispdfpagelabel{Title}
	\thispagestyle{empty}
	\setcounter{page}{-1}
	\vspace{0.3in}
	
	
	
	%  Table of Contents
	%
	\begin{center}
		\begin{minipage}[t]{0.9\textwidth}
			\begin{multicols}{2}
				\tableofcontents
			\end{multicols}
		\end{minipage}
	\end{center}
	
	
	
	\newpage
	\thispdfpagelabel{-}
	\thispagestyle{empty}
	
	
	
	%  Introduction
	%
	\section*{Introduction}
	Taking notes and working through proofs is the best way for me to teach myself advanced mathematics.  Typing (and thoroughly backing up) notes is the best way to make sure they are preserved and readable well into the future.  As such, these notes are from my process of working through \textit{Algebraic Graph Theory} by Chris Godsil and Gordon Royle.
	
	I am taking these notes under the assumption that the reader has a familiarity with the basic notions of graph theory and algebra.  I omit elementary definitions and proofs from both domains.  I may go back and fill some of these in if there comes a need or demand for it, but for now, they will be skipped.
	
	My notation differs slightly from that used by Godsil and Royle, and is slightly more consistent with conventions from computer science and algorithmic graph theory at the expense of diverging from algebraic convention.
	
	These notes are being written intermittently, as Algebraic Graph Theory is (currently) not my main research focus. I am using the editor TeXstudio.  The template for these notes was created by Zev Chonoles and is made available (and being used here) under a Creative Commons License. 
	
		I am responsible for all faults in this document, mathematical or otherwise; any merits of the material here should be credited to the authors and those mathematicians they reference.
	
	Please email any corrections or suggestions to \expandafter\href{mailto:\notetakersemail}{\texttt{\notetakersemail}}.
	
	%\medskip
	%
	%\section*{Acknowledgments}
	%
	%Thank you to all of my fellow students who sent me suggestions and corrections, and who lent me their own notes from days I was absent. My notes are much improved due to your help.
	
	
	%%  Copyright
	%%
	%\section*{Copyright}
	%Copyright \mycopyrightsymbol\ 2012 \notetaker.
	%
	%This work is licensed under a Creative Commons Attribution-ShareAlike 3.0 Unported License. This means you are welcome to do essentially anything with this work, including editing, %adapting, distributing, and making commercial use of it, as long as you
	%\begin{itemize}
	%\item include an attribution of \lecturer\ as the lecturer of the course these notes are based on, and \notetaker\ as the person taking the notes,
	%\item do so in a way that does not suggest either of us endorses you or your use of this work, and
	%\item if you alter, transform, or build upon this work, you must apply to your work the same, or similar, license to this one.
	%\end{itemize}
	%More details are available at \href{https://creativecommons.org/licenses/by-sa/3.0/deed.en\_US}{\texttt{https://creativecommons.org/licenses/by-sa/3.0/deed.en\_US}}.
	
	\newpage
	
	
	%  Make a separate file for each lecture, for example, using a naming scheme like this:
	%
	%  lecture1.tex, lecture2.tex, ...
	%
	%  and keep them in the same folder as this main file. By doing it this way (instead of keeping all the notes in the main file), if you're only working on the notes for one lecture, you can easily comment out the lines corresponding to the other lectures.
	%
	\renewcommand{\exc}[1]{\subsubsection*{Exercise 1.#1}}

\classheader{: Graphs}


\subsection*{What is Algebraic Graph Theory?}

\textit{Algebraic graph theory} (abbreviated \textbf{AGT} here \footnote{Not to be confused with \textit{algorithmic game theory}, an area of mathematics and computer science much closer to my primary research interests...} is the subject which explores the relationship between algebra, which broadly studies the properties of abstract mathematical structures, and graph theory, which broadly studies a very particular kind of concrete mathematical structure.  Among these subjects are graph groups and morphisms, spectral graph theory, graph cuts and flows, colorings, and knots.






\section*{Definitions and Fundamentals}

If $X$ is a graph, we let $V(X)$ and $E(X)$ denote the vertex set and edge set of $X$, respectively, using $A(X)$ for the arc set of $X$ in settings with directed graphs.  Unless otherwise specified, we assume all graphs are undirected.  If vertices $u,v\in V(X)$, then we write $(u,v)\in E(X)$ to represent the edge between $u$ and $v$ (or $(u,v)\in A(X)$ to denote the arc \textit{from} $u$ \textit{to} $v$).

\definition{Two graphs $X$ and $Y$ are \textbf{isomorphic} if there exists a bijective function $\phi:V(X)\rightarrow V(Y)$ such that $(\phi(u),\phi(v))\in E(Y)$ if and only if $(u,v)\in E(X)$.}

\definition{A \textbf{subgraph} $Y$ of a graph $X$ is a graph such that $V(Y)\subset V(X)$ and $E(Y)\subset E(X)$.  An \textbf{induced subgraph} is one such that $E(Y)$ consists exactly of the edges $(u,v)$ in $X$ such that $u$ and $v$ are both in $V(Y)$.  That is, an induced subgraph is one which can be realized by deleting vertices from $X$ and removing only those edges incident to those removed vertices.  A \textbf{spanning subgraph} is one such that $V(Y)=V(X)$.}

\definition{A \textbf{cycle} is a subgraph such that every vertex has degree $2$.  A \textbf{tree} is a graph such that no subgraph is a cycle.  A \textbf{spanning tree} is a spanning subgraph with no cycles.}

\definition{A set of vertices which induce an empty (edge-free) subgraph is called an \textbf{independent set}.  A set of vertices which induces a complete graph is called a \textbf{clique}.  The largest independent set and clique in a graph $X$ are denoted $\alpha(X)$ and $\omega(X)$, respectively.}

These values $\alpha(X)$ and $\omega(X)$ will come back later.

\definition{A \textbf{connected component} of a graph is a collection of vertices such that there exist a path between all pairs.}

While adjacency in a graph is not an equivalence relation (it's not transitive), membership in connected components is, hence a graph can be partitioned into disjoint connected components.

\section*{Graph Automorphisms}


\definition{An \textbf{automorphism} of a graph $X$ is an isomorphism $X\rightarrow X$.}

The set of automorphisms of a graph form a group. The identity function is clearly an automorphism, and if $g$ is an automorphism, then its inverse, $g^{-1}$ is as well.  We can also compose automorphisms to get another automorphism, and this inherits associativity from function composition.  By Cayley's theorem, we can think about $Aut(X)$ as being a subgroup of $Sym(V(X))$, the symmetric group on the vertices of $X$. We'll write $Sym(n)$ for $n=|V(X)|$ to denote the symmetric group on $n$ elements in place of $Sym(V(X))$. 

In general, it is difficult to determine whether two graphs are isomoprhic (this is a well-known NP problem) or whether a graph has a nontrivial automorphism.  However, some cases are easy.  For a complete graph $K_n$, $Aut(X)=Sym(n)$, and the same holds for an empty graph on $n$ vertices.

If $v$ is a vertex and $g$ a group element, we denote $v^g$ the action of $g$ on $v$.  If $g\in Aut(X)$ and $Y$ is a subgraph of $X$, then we denote $Y^g = \{x^g|x\in V(Y)\}$.  Then we have that $E(Y^g) = \{(u^g,v^g)|(u,v)\in E(Y)\}$.  The graphs $Y$ and $Y^g$ are isomorphic, and $Y^g$ is a subgraph of $X$.

\definition{The \textbf{valency} of a vertex $x$ is the number of neighbors of $x$ in the graph $X$.  We can talk about the maximum and minimum valencies over all vertices of a graph.}

\lemma{If $x$ is a vertex of a graph $X$ and $g$ is an automorphism of $X$, then the vertex $y=x^g$, has the same valency as $x$.}

\begin{proof}
	Let $N(x)$ be the subgraph of $X$ induced by $x$ and its neighbors.  Then $N(x)^g\iso N(x^g)\iso N(y)$, so $N(x)\iso N(y)$ as subgraphs of $X$, so they have the same number of vertices.  Thus the valencies of $x$ and $y$ are equal.
\end{proof}

\corollary{An automorphism of a graph necessarily permutes vertices of the same valency.}

\definition{A graph where every vertex has valency $k$ is called $\boldsymbol{k}$\textbf{-regular}.}

\definition{The \textbf{distance} between vertices $x$ and $y$ is the length of the shortest path in $X$ between $x$ and $y$, denoted $d_X(x,y)$ or $d(x,y)$ if it is clear which graph we are talking about.}

\lemma{If $g$ is an automorphism of a graph $X$, then $d_X(x,y)=d_X(x^g,y^g)$ for all pairs of vertices.}

\begin{proof}
	If they are the same vertex, the distance $d(x,y)=d(x^g,y^g)=0$ is trivially preserved.  If $d(x,y)=1$, then $x$ and $y$ are adjacent, so their images $x^g$ and $y^g$ must be adjacent as well, by definition of graph isomorphism.
	
	Suppose, for the sake of contradiction that $d(x,y)\lneq d(x^g,y^g)$.  Then there is some path $x,r_2,r_3,\dots r_{n-1},y$ such that $r_{n-1}^g$ is not adjacent to $y^g$.  But this is impossible, as automorphism preserves adjacency, and $r_{n-1}$ is adjacent to $y$.  A symmetric argument on $g^{-1}$ gives the case where $d(x,y)\gneq d(x^g,y^g)$.
	
	
	
\end{proof}

\definition{The \textbf{complement} of a graph $X$, denoted $\overline{X}$, is the graph such that $V(\overline{X})=V(X)$ and $E(\overline{X}) = \{(u,v)|(u,v)\notin E(X)\}$.  That is, the complement of a graph is the one which has an edge between two vertices if and only if the original graph does not.}

\lemma{$Aut(X)=Aut(\overline{X})$.}

\begin{proof}
	Since automorphisms preserve adjacency, they also preserve non-adjacency.  Thus $x^g$ is not adjacent to $y^g$ if and only if $x$ and $y$ are not adjacent.  Therefore, $g\in Aut(\overline{X})$.
\end{proof}

We'll quickly note that automorphisms of directed graphs also preserve the direction of the arcs.


\section*{Graph Homomorphisms}
\definition{A \textbf{graph homomorphism} is a function $\phi:V(X)\rightarrow V(Y)$ such that if $u$ and $v$ are adjacent in $X$, they are adjacent in $Y$.}

We'll quickly contrast this to isomorphisms, which preserves adjacency in both directions, whereas a homomorphism only requires that adjacent vertices in $X$ are still adjacent in $Y$ under $\phi$.  Every isomorphism is a homomorphism, but not every homomorphism is an isomorphism.

\definition{A graph is \textbf{bipartite} if there exists a partition of $V(X)$ into disjoint sets $A$ and $B$ such that every edge has one end in $A$ and the other in $B$. Analogously, we can define $\boldsymbol{k}$\textbf{-partite} graphs as being those which admit a partition into $k$ components such that no edge has both endpoints in the same component.}

If a graph is bipartite, there exists a homomorphism $X\rightarrow K_2$ where the image of each component is one of the vertices in $K_2$.  Similarly, there is a homomorphism from a $k$-partite graph onto $K_k$.

This leads to the notion of \textit{proper colorings}.

\definition{A \textbf{proper coloring} is a map from $V(X)$ to a finite set of colors such that for any edge $(u,v)\in E(X)$, $u$ and $v$ are assigned different colors.}

\definition{The \textbf{chromatic number} of a graph, denoted $\chi(X)$ is the minimum number $k$ such that $X$ can be properly $k$-colored.}

Nonempty bipartite graphs have chromatic number $2$.  Complete graphs $K_n$ have chromatic number $n$.

Let's observe that the set of vertices assigned some particular color, called a \textit{color class}, forms an independent set in $X$.

\lemma{The chromatic number of a graph $\chi(X)$ is the minimum number $r$ such that there exists a homomorphism from $X$ to $K_r$.}


\begin{proof}
	Suppose $\phi:V(X)\rightarrow V(Y)$ is a homomorphism.  For $y\in V(Y)$, define $\phi^{-1}(y)$ to be $\phi^{-1}(y)=\{x\in V(X)|\phi(x)=y\}$, the set of elements in $V(X)$ which map to $Y$ under $\phi$.  As $y$ is not adjacent to itself, $\phi^{-1}(y)$ is an independent set.  Hence if $Y$ has $r$ vertices, each of the $r$ sets is independent and forms a color class of an $r$-coloring, so $X$ can be properly $r$-colored.  Conversely, suppose that $X$ can be properly $r$-colored.  Then there exists a homomorphism onto $K_r$ which sends each color class to a unique vertex.
\end{proof}

\definition{A \textbf{retraction} is a homomorphism $\phi$ from $X$ to $Y$ where $Y$ is a subgraph of $X$ such that the restriction of $X$ to $Y$ is the identity map.}

If $X$ is a graph with a $k$-clique, then any $k$-coloring of $X$ determines a retraction of $X$ onto the clique.

When we think about directed graphs, we will also stipulate that homomorphisms preserve the directions of arcs.

\definition{An \textbf{endomorphism} of a graph is a homomorphism from a graph to itself.  The set of endomorphisms, $End(X)$, forms a monoid. An automorphism is a special case of an endomorphism, so $Aut(X)$ is a submonoid of $End(X)$.}


\section*{Circulant Graphs}

Let's give a more particular definition of a \textit{cycle} in a graph.  We can think of a cycle of $n$ vertices as a set $C_n=\{0,1,2,\dots n-1\}$ of vertices such that $i$ and $j$ are adjacent if and only if $j-i\equiv\pm 1\mod n$.

Let's look at some automorphisms of the cycle.  The set of permutations which map $i$ to $i+1$ (and $n-1$ to $0$) forms a subset of $Aut(C_n)$.  By composition, we can realize an entire copy of the cyclic group on $n$ elements ($\mathbb{Z}_n$) in this way.  Also, the permutation $h$ which sends $i$ to $-i\mod n$ is an element of $Aut(C_n)$.  We have that $h(0)=0$ but the cyclic group is fixed point-free, so this automorphism isn't contained in that subgroup.  Also, $h=h^{-1}$, so there are two cosets induced by this element, and the order of $Aut(C_n)$ is at least $2n$.  (In fact, it's equal to $2n$, but we can't quite prove that yet...)

The cycles are a subclass of the \textit{circulant graphs}.  If $C\subset \mathbb{Z}_n{\setminus}0$, then we can construct the directed graph $X=X(\mathbb{Z}_n,C)$ through the following process.  First, let $V(X)$ be the elements of $\mathbb{Z}_n$ and let $(i,j)\in A(X)$ if and only if $j-i\in C$.  This graph $X(\mathbb{Z}_m,C)$ is called a \textit{circulant of order $n$} and $C$ is its \textit{connection set}.  If $C$ itself is also closed under additive inverses (modulo $n$), then $(i,j)$ is an arc in $X$ if and only if $(j,i)$ is, so we can view the graph as being undirected.  In this case, the map which sends $i$ to $-i$ is an automorphism, and the map which sends $i$ to $i+1$ is always an automorphism of a circulant graph, so the automorphism group of a circulant graph with an inverse-closed connection set is at least $2n$.  We can think of the ordinary cycle on $n$ vertices as being $X(\mathbb{Z}_n,\{-1,1\})$.  The complete graph is a circulant graph with connection set $\mathbb{Z}_n$, and an empty graph is one with empty connection set.  Since these graphs have automorphism groups with order $n!$, we clearly have examples of circulant graphs with orders much larger than $2n$.

\section*{Johnson Graphs}

Now we consider another family of graphs, denoted $J(v,k,i)$ for positive integers $v\geq k\geq i$.  Let $\Omega$ be some fixed set of size $v$.  The vertices of $J(v,k,i)$ are the subsets of $\Omega$ with size $k$, and two vertices are adjacent if and only if their corresponding sets have intersection size $i$.  Thus $J(v,k,i)$ has $\binom{v}{k}$ vertices, and it is a regular graph in which each vertex has valency $\binom{k}{i}\binom{v-k}{k-i}$.  We'll assume $v\geq 2k$.

\begin{lemma}
{The function which maps a set of size $k$ to its complement in $\Omega$ is an isomorphism between the graphs $J(v,k,i)$ and $J(v,v-k,v-2k+i)$.}
\end{lemma}
\begin{proof}
	The proof of this is just a DeMorgan's Laws definition-chase.  
	
	If $|A|=|B|=k$, then $|\overline{A}|=|\overline{B}|=v-k$.
	
	If $A$ and $B$ are adjacent, then $|A\cap B|=i$, so $|\overline{A}\cap\overline{B}| = |\overline{A\cup B}| = v-2k+i$.
	
	Therefore, if we define a map by mapping a set to its complement and adjacency occurs if and only if the intersection of the sets is size $v-2k+i$, this is indeed an automorphism, as $A$ and $B$ are adjacent if and only if $\overline{A}$ and $\overline{B}$ are adjacent, and set complements is an obvious bijection between the vertex sets.
	
	
	
\end{proof}

A graph is called a \textit{Johnson graph} if it is isomorphic to $J(v,k,k-1)$.  The \textit{Kneser graphs} are isomorphic to $J(v,k,0)$.  As an example, the Petersen graph, which we will study later, is $J(5,2,0)$ and is therefore a Kneser graph.

\begin{lemma}
	{If $v\geq k \geq i$, then $Aut(J(v,k,i))$ contains a subgroup isomorphic to $Sym(v)$.}
\end{lemma}

\begin{proof}
	Let $g$ be a permutation of $\Omega$ and $S\subset \Omega$, and let $S^g$ denote the image of $S$ under $g$.  Any such $g$ also determines a permutation of the subsets $S$ of size $k$.  In particular, if $S$ and $T$ are of size $k$, then $|S\cap T|=|S^g\cap T^g|$, so $g$ is an automorphism of $J(v,k,i)$.
\end{proof}

We note that $Aut(J(v,k,i))$ acts on a set of size $\binom{v}{k}$, so when this quantity is not equal to $v$, it's not equal to $Sym(v)$, but it is \textit{usually} isomorphic, which is often not an easy thing to prove.


\section*{Line Graphs}

\definition{If $X$ is a graph, the \textbf{line graph} of $X$, denoted $L(X)$ is the graph where the vertices of $L(X)$ correspond to edges of $X$ and two vertices in $L(X)$ are adjacent if and only if the corresponding edges in $X$ are incident to the same vertex.}

As examples, the star $K_{1,n}$ (one hub with $n$ `spokes') has line graph $K_n$, as all $n$ edges in the star are incident to the center vertex.  The path graph on $n$ vertices $P_n$ has $L(P_n)=P_{n-1}$.  The cycle $C_n$ is isomorphic to its own line graph.

\begin{lemma}
{If $X$ is regular with valency $k$, then $L(X)$ is regular with valency $2k-2$.}
\end{lemma}
\begin{proof}
	Each vertex has degree $k$, so when we translate each edge into a vertex, for each original vertex, we get a $k$-clique, but each of these new vertices belongs to two such cliques.  Thus each vertex has $k-1$ adjacent vertices in each of the cliques it belongs to, thus a total valency of $2k-2$.
\end{proof}


\theorem{A graph is the line graph of some other graph if and only if there exists a partition of its vertex set into cliques such that each vertex belongs to at most two cliques.}

\begin{proof}
	
	To see that the condition is necessary, observe that the process of constructing a line graph necessarily turns the neighborhood of each vertex into a clique, and since an edge connects two vertices, each new vertex belongs to at most two such cliques.
	
	To see that it is sufficient, we will construct a graph from a line graph which decomposes into cliques in this way.  Let $S_1,S_2,\dots,S_k$ be the cliques, and let $v_1,v_2,\dots,v_m$ be the vertices (if there are any) which are in exactly one $S_i$.  The vertex set of our graph will be $S_1,\dots,S_k,\{v_1\},\dots \{v_m\}$ with an edge between sets if and only their intersection is nonempty.  It is clear that the line graph of this graph is our original graph, and we are done.
	
\end{proof}

Observe that if $X$ and $Y$ are isomorphic, then $L(X)$ and $L(Y)$ are isomorphic, but the converse isn't true, as $K_3$ and $K_{1,3}$ have the same line graphs.

\begin{lemma}
	
{If $X$ and $Y$ are graphs with minimum valency at least $4$, then $X\iso Y$ if and only if $L(X)\iso L(Y)$.}
\end{lemma}
\begin{proof}
	Let $C$ be a clique in $L(X)$ with $|C|=c<4$.  The vertices in $C$ correspond to a set of $c$ edges in $X$, all of which are incident to a common vertex $x$.  Thus, there is a bijection between vertices of $X$ and maximal cliques in $L(X)$ which maps adjacent vertices in $X$ to pairs of cliques in $L(X)$ which share exactly one vertex.  We can similarly construct an analogous bijection between $Y$ and $L(Y)$.  Let $f:X\rightarrow L(X)$ and $g:Y\rightarrow L(Y)$ be these functions.
	
	If we assume $X\iso Y$ by $\phi$, then we want to show that $L(X)\iso L(Y)$ by demonstrating that $g\circ\phi\circ f^{-1}:L(X)\rightarrow L(Y)$ is an isomorphism.  It suffices to show that the image of a $k$-clique under this composite function is a $k$-clique in $L(Y)$.  But this is obvious.  $f^{-1}$ takes a maximal $k$-clique to a set of $k$ edges in $X$ incident to some vertex $x$, which has valency $k$.  Then $\phi(x)=y$ is some vertex in $Y$ with valency $k$, so $g$ sends this neighborhood to a maximal $k$-clique.
	
	The other direction has an identical proof, except that we show that vertices in $X$ and $Y$ with equal valency are mapped to each other.
\end{proof}

\theorem{A graph is a line graph if and only if each induced graph on at most six vertices is also a line graph.}

\begin{proof}
	
	This is an alternative phrasing of Beineke's Theorem.  I'll fill in a proof later.
	
\end{proof}

\corollary{The set of graphs which are not line graphs but every induced subgraph is a line graph is finite and, in fact, of size nine.}



\definition{A bipartite graph is \textbf{semiregular} if it has a proper $2$-coloring such that all vertices of the same color have the same valency.  As an example, the complete bipartite graphs $K_{m,n}$ (a set of $m$ vertices connected to each of a set of $n$ vertices) are semiregular.}

\begin{lemma}
{If the line graph of a graph is regular, then the graph itself is regular or a semiregular bipartite graph.}
\end{lemma}
\begin{proof}
	Suppose $L(X)$ is $k$-regular.  If $u$ and $v$ are adjacent in $X$, then their valencies sum to $k+2$, so all neighbors of $u$ have the same valency, so if two vertices share a neighbor, they have identical valencies.  This only occurs in graphs which are regular or bipartite and semiregular, as if it contains an odd cycle, it must have two adjacent vertices with the same valencies, and bipartite graphs have no odd cycles.
\end{proof}


\section*{Planar Graphs}

\definition{A graph is called \textbf{planar} if it can be drawn (in the plane) without crossing edges.  More precisely, a graph is planar if there exists a function which maps each vertex to a unique point in $\mathbb{R}^2$ and each edge to a non-self-intersecting curve with endpoints equal to the image of the vertices it's incident to such that no two such curves intersect. Such a function is called a \textbf{planar embedding}.}

\definition{A \textbf{plane graph} is a planar graph together with a planar embedding.}

The edges of a plane graph divide the plane into disjoint regions called \textit{faces}.  All but one (the \textit{external} or \textit{infinite}) face is bounded.  The \textit{length} of a face is the number of edges bounding it.

\theorem[Euler]{If $v-e+f=2$, where $v,e,f$ are the number of vertices, edges, and faces of a plane graph, respectively.}

\begin{proof}
	
	The proof proceeds by strong induction on the number of edges. Observe that a tree on $v$ vertices is a planar graph with $v-1$ edges and $1$ face.  If a planar graph is not a tree, it contains a cycle.  Removing an edge in this cycle (which does not disconnect the graph) merges two faces, which preserves the quantity $v-e+f$.  Since a tree is a graph without cycles, and this process eventually transforms a graph into a tree, but since a tree satisfies $v-e+f=2$, this quantity must be preserved at all steps of the process, hence it is true for the original graph.
	
	
\end{proof}

\definition{A \textbf{maximal planar graph} is one in which adding an edge between any two vertices which are not already adjacent makes the graph non-planar.  If a planar graph has an embedding where the length of some face is greater than 3, we can add edges interior to this face in without violating planarity.  Thus any maximal planar graph must have every face be of length 3, called a \textbf{planar triangulation}.}

In a triangulation, each edge is incident to two faces, so we have $3f=2e$.  Then by Euler's theorem, $e=3n-6$.  Any planar graph with $3n-6$ edges must be maximal and a planar triangulation.

A planar graph may have multiple distinct embeddings, and they don't necessarily preserve the lengths of the faces (although it must preserve the \textit{number} of faces).  It is a result in topological graph theory that a $3$-connected planar graph has a (topologically) unique planar embedding.

Given a plane graph $X$, we can construct its dual $X^*$, where each face of $X$ becomes a vertex of $X^*$ with edges between vertices in $X^*$ if and only if there is an edge separating the corresponding faces in $X$.  Sometimes this gives rise to multiple edges between vertices, but we'll be sure to only worry about that if we have to.

The dual of a planar graph is connected, so if $X$ is not connected, $(X^*)^*$ is not isomorphic to $X$, but this is true if $X$ is connected.

We can generalize the notion of planar embeddings to embeddings in any surface.  The  dual is defined analogously in these topological spaces.  The real projective plane $\RP^2$ is a non-orientable surface which looks like the closed disk with an antipodal identification along the boundary.  The graph $K_6$ is not planar, but it does have an embedding in $\RP^2$ (which is triangular!), and its dual in this space is cubic, and turns out to be the Petersen graph.

The torus is an orientable surface, which looks like the surface of a donut.  We can represent it as a rectangle with opposite edges identified.  The graph $K_7$ is not planar, but there is an embedding on the torus (which is also triangular!), and its dual is the Heawood graph.


\ifdraft
\input{../../zach_private_repo/alggraphth_exc/ex1}
\fi

	\classheader{}


	\renewcommand{\exc}[1]{\subsubsection*{Exercise 3.#1}}

\classheader{: Transitive Graphs}



Now we can really start bringing together groups and graphs.  We'll study graphs whose automorphism group acts transitively on the vertices.  That is, for any pair of vertices $x$ and $y$, there is some group element which sends $x$ to $y$.  Such graphs are necessarily regular, and one challenge is finding properties of vertex transitive graphs which do not hold for all regular graphs.  We'll see that, in general, transitive graphs are more strongly connected than regular graphs.  Cayley graphs are an important class of vertex transitive graphs, and we'll see a bit of them in this chapter.



\section*{Vertex-Transitive Graphs}

\definition{A graph $X$ is \textbf{vertex-transitive} (or just \textbf{transitive}) if its automorphism group acts transitively on its vertex set $V(X)$.}

One family of transitive graphs are the $k$-cubes $Q_k$.  We can think about these combinatorially by thinking of each vertex as one of the $2^k$ binary strings (or tuples) and two vertices are adjacent if and only if their corresponding strings differ in exactly one position.  The cube $Q_3$ is usually just called `the cube', and we have seen this object already, when looking at systems of imprimitivity.

\begin{lemma}
The $k$-cube, $Q_k$, is vertex transitive.
\end{lemma}

\begin{proof}
If $v$ is a fixed binary tuple, then the mapping $\rho_v:x\mapsto x+v$ where we do binary addition placewise permutes the vertices of $Q_k$.  This is an automorphism because the tuples $x$ and $y$ differ in exactly one position if and only if $x+v$ and $y+v$ differ in exactly one position (we flip the same bits in each).  This group $H$ acts transitively on the vertices because for any two vertices $x$ and $y$, we can send $x$ to $y$ with $\rho_{y-x}$. There are $2^k$ such permutations. 

Note that $H$ is \textit{not} the full automorphism group.  Any permutation of the $k$ coordinate positions is also an automorphism of $Q_k$, and there are $k!$ of these, and they form a subgroup $K$ isomorphic to $Sym(k)$.  By standard results in group theory, the group $HK$ is a subgroup of $Aut(Q_k)$ and the size of $HK$ is
$$|HK|=\frac{|H||K|}{|H\cap K|}$$
It is clear that the intersection of these groups is the identity permutation, so we have that $|Aut(Q_k)|$ is at least $2^kk!$. 
\end{proof}

Another family of vertex transitive graphs are the circulants, as any vertex can be sent to any other by using the appropriate element of the cyclic subgroup of its automorphism group. Both the cubes and the circulants are part of a construction which produces many (not all) vertex-transitive graphs.

\definition{Let $G$ be a group and $C$ a subset of $G$ which is closed under taking inverses and does not contain the identity.  Then the \textbf{Cayley graph} $X(G,C)$ is the graph with vertex set $G$ and edge set $E(X(G,C))=\{(g,h)|hg^{-1}\in C\}$.  That is, there is an edge between group elements $g$ and $h$ if there is an element $a$ of $C$ such that $h=ag$.  If $C$ is an arbitrary subset of $G$, then we can create a directed graph in this way, but if $C$ is inverse-closed, then this directed graph has arcs in both directions and reduces to the previous construction.}

Many results for Cayley graphs hold for this general directed case, but we will be explicit when we are using this construction rather than the canonical one.

\thrm{The Cayley graph $X(G,C)$ is vertex-transitive.}


\begin{proof}


For each $g\in G$, the mapping $\rho_g:x\mapsto xg$ is a permutation of the elements of $G$, and it is an automorphism of $X(G,C)$, as 
$$(yg)(xg)^{-1} = ygg^{-1}x^{-1}=yx^{-1}$$
so $xg$ is adjacent to $yg$ if and only if $x$ is adjacent to $y$.  The permutations $\rho_g$ are a subgroup of the automorphism group of $X(G,C)$ which acts transitively because for any vertices $g$ and $h$, $\rho_{g^{-1}h}$ sends $g$ to $h$.

\end{proof}

The $k$-cube is a Cayley graph for the group $(\mathbb{Z}_2)^k$ and the circulant on $n$ vertices is a Cayley graph for $\mathbb{Z}_n$.  Most small vertex-transitive families of graphs are Cayley graphs, but there are may such families which are not Cayley graphs.  In particular, the graphs $J(v,k,i)$ are vertex transitive because we can pick an element of $Sym(v)$ to map any $k$-set to any other, but they are not Cayley graphs in general.  We can prove this for one counterexample and        move on from there:

\begin{lemma}
The Petersen graph is not a Cayley graph.
\end{lemma}

\begin{proof}

The Petersen graph has 10 vertices and is $3$-regular.  There are two groups with 10 elements: $\mathbb{Z}_{10}$ and $D_{10}$, the dihedral group with 10 elements.  If we pick $|C|=3$ for either of these, we get a graph which contains cycles of order $4$, but the Petersen graph only contains cycles of order $5$, so there is no isomorphism.
\end{proof}


\section*{Edge-Transitive Graphs}

\definition{A graph $X$ is \textbf{edge-transitive} if its automorphism group acts transitively on its edge set $E(X)$.  That is, for any pair of edges $(u,v)$ and $(x,y)$, there is a group element which sends $(u,v)$ to $(x,y)$, i.e. it sends $u$ to $x$ and $v$ to $y$ or the other way around. An edge-transitive graph is vertex transitive.}

It's clear that the graphs $J(v,k,i)$ are edge-transitive, but the circulants are, in general, not.

\definition{Recall that an \textit{arc} is an ordered pair of adjacent vertices.  A graph $X$ is \textbf{arc transitive} if $Aut(X)$ acts transitively on the arcs.  That is, for any pair of arcs $(u,v)$ and $(x,y)$, there is a group element which sends $(u,v)$ to $(x,y)$, i.e. it sends $u$ to $x$ and $v$ to $y$ but \textit{not} the other way around.  It is often useful to view an undirected graph as a directed graph with arcs in both directions.  As such, an arc-transitive graph is necessarily edge-transitive (and vertex-transitive).}

The complete bipartite graphs $K_{m,n}$ are edge-, but not vertex-transitive (unless $m=n$) because there is no automorphism which maps a vertex with valency $m$ to one with valency $n$ (or vice versa).

\begin{lemma}
Let $X$ be an edge-transitive graph with no isolated vertices.  If $X$ is not vertex transitive, then $Aut(X)$ has exactly two orbits which form a bipartition of $X$.
\end{lemma}
\begin{proof}
Suppose that $X$ is edge- but not vertex-transitive and that $(x,y)$ is an edge of $X$ where $x$ and $y$ are vertices such that there exists no automorphism mapping $x$ to $y$.  If $w$ is a vertex of $X$, then $w$ lies on some edge and there is an element of $Aut(X)$ which maps this edge incident to $w$ to the one between $x$ and $y$, so any vertex either lies in the orbit of $x$ or the orbit of $y$.  These orbits are disjoint, as we know that $x$ and $y$ are in different orbits.  Thus there are exactly two orbits of $Aut(X)$.  An edge which connects two vertices in one orbit cannot be mapped by an automorphism to an edge which is incident to a vertex in the other orbit, so no such edge can exist.  Therefore, all edges in $X$ are incident to one vertex from the orbit of $x$ and one from the orbit of $y$, so $X$ is bipartite.


\end{proof}

\begin{lemma}
If $X$ is vertex- and edge-transitive but not arc-transitive, its valency is even.
\end{lemma}

\begin{proof}
Let $G=Aut(X)$ and suppose that $x$ and $y$ are adjacent vertices in $X$.  Let $\Omega$ be the orbit of $G$ on $V\times V$ which contains $(x,y)$.  Since $X$ is edge-transitive, there is an automorphism which maps any arc in $X$ to either $(x,y)$ or $(y,x)$. But since $X$ is not arc-transitive, we can choose $x$ and $y$ such that  $(y,x)$ is not in $\Omega$, so $\Omega$ is not symmetric.  Thus, $X$ is the graph with edges $\Omega\cup\Omega^T$.  Because the out-valency of $x$ is the same in $\Omega$ and $\Omega^T$, the valency of $X$ must be even.
\end{proof}

\corollary{A vertex- and edge-transitive graph of odd valency must be arc-transitive as well.}


\section*{Edge Connectivity}

\definition{An \textbf{edge cutset} in a graph $X$ is a collection of edges such that deleting these edges from $X$ separates $X$ into a strictly greater number of connected components.  For a connected graph, the \textbf{edge connectivity} is the minimum number of edges in any cutset.  That is, the size of the smallest set of edges which, if deleted, disconnects $X$.  We will denote this quantity $\kappa_1(X)$.  If a single edge $e$ is a cutset, then we call $e$ a \textbf{bridge} or \textbf{cut-edge}.}

The edge connectivity of a graph clearly cannot be greater than its minimum valency, so the edge connectivity of a vertex-transitive graph is at most its valency.  We're about to prove that the edge connectivity of a vertex-transitive graph is exactly equal to its valency.  If $A\subset V(X)$, we'll denote $\partial A$ to be the set of vertices with one end in $A$ and one end not in $A$.  If $A=\emptyset$ or $A=V(X)$, then $\partial A=\emptyset$.  The edge connectivity of $X$ is the minimum size of $\partial A$ as $A$ ranges over all possible proper subsets of $V(X)$.

\begin{lemma}
Let $X$ be a graph and $A$ and $B$ be subsets of $V(X)$.  Then $|\partial(A\cup B)|+|\partial(A\cap B)|\leq |\partial A|+|\partial B|$.  
\end{lemma} 
\begin{proof}
The right-hand side counts the number of edges leaving $A$ or $B$.  The left-hand side counts the number of edges leaving $A$ or $B$ except those between $A$ and $B$ plus the edges leaving those vertices in both $A$ and $B$.  Thus the difference between the right- and left-hand sides is twice the number of edges crossing the symmetric difference of $A$ and $B$.  Since this is at least zero, the inequality holds.
\end{proof}

\definition{An \textbf{edge atom} of a graph $X$ is a subset $S\subset V(X)$ such that $|\partial S|=\kappa_1(X)$ and, given that this holds, $|S|$ is minimal.  Since $\partial S = \partial(V\setminus S)$, if $S$ is an edge atom, then $2|S|\leq |V(X)|$.}

\corollary{Any two distinct edge atoms are vertex disjoint.}

\begin{proof}
Assume $\kappa=\kappa_19X)$ and let $A$ and $B$ be distinct edge atoms in $X$.  If $A\cup B=V(X)$, them since neither $A$ nor $B$ can contain more than half of the vertices, it must be that $|A|=|B|=\frac{1}{2}|V(X)|$, so $A\cap B=0$.  Thus $A\cup B \subsetneq V(X)$.  The previous lemma tells us that $|\partial (A\cup B)|+|\partial (A\cap B) \leq 2\kappa$.  But since $A\cup B\neq V(X)$ and $A\cap B\neq \emptyset$, we have that $|\partial(A\cup B)|=|\partial(A\cap B)|=\kappa$.  Since $A\cap B$ is a nonempty proper subset of $A$, this cannot happen, as $A$ is an edge atom.  Thus $A$ and $B$ must be disjoint.
\end{proof}

\begin{lemma}
If $X$ is a connected vertex-transitive graph, then its edge connectivity is equal to its valency.
\end{lemma}

\begin{proof}
Suppose that $X$ is vertex-transitive and has valency $k$.  Let $A$ be an edge atom of $X$.  If $A$ is a single vertex, then $|\partial A|=k$ and we are done.  Otherwise, suppose that $|A|\geq 2$.  If $g$ is an automorphism of $X$ and $B=A^g$ (the image of the vertices in $A$ under $g$), then $|B|=|A|$ and $|\partial B|=|\partial A|$.  From the previous lemma, we have that $A$ is either equal to or disjoint from $B$. Thus $A$ is a block of imprimitivity for $Aut(X)$, and by Exercise 2.13, it follows that the subgraph of $X$ induced by $A$ is regular, so let its valency be $\ell$.

Each vertex in $A$ has $k-\ell$ neighbors not in $A$, so $|\partial A|=|A|(k-\ell)$. Since $X$ is connected, $\ell<k$, so if $|A|\geq k$, then  $|\partial A|\geq k$.  So we assume $|A|<k$.  Since $\ell\leq |A|-1$, it follows that $|\partial A|\geq |A|(k+1-|A|)$.  The minimum value of the right-hand side occurs when $|A|=1$ or $|A|=k$.  Thus $|\partial A|\geq k$ for all cases.
\end{proof}

\section*{Vertex Connectivity}


\definition{A \textbf{vertex cutset} in a graph is a set of vertices whose deletion increases the number of connected components of $X$.  The \textbf{vertex connectivity} is the size of the smallest vertex cutset, which we denote $\kappa_0(X)$.  For any $k\leq \kappa_0(X)$, we say that $X$ is \textbf{$\boldsymbol{k}$-connected}.}

Complete graphs have no vertex cutsets, but it is conventional to let $\kappa_0(K_n)=n-1$.  The central result in this topic is Menger's theorem, which we are about to prove.

\definition{If $u$ and $v$ are distinct vertices of $X$, then two paths $P$ and $Q$ are \textbf{openly disjoint} if, aside from $u$ and $v$, the vertex sets of $P$ and $Q$ are disjoint.}

\begin{theorem}[Menger]

{Let $U$ and $v$ be distinct vertices in $X$.  Then the maximum number of openly disjoint paths from $u$ to $v$ is equal to the minimum size of a set of vertices $S\subset V(X)$ such that $u$ and $v$ lie in distinct connected components of $X\setminus S$.  That is, the maximum number of such paths is equal to the smallest vertex cutset which separates $u$ from $v$.}
\end{theorem}

\begin{proof}
If we have a collection of $m$ openly disjoint $u{-}v$ paths, then we must remove at least one vertex from each path in order to disconnect $u$ from $v$.
\end{proof}

This theorem tells us that two vertices that can't be separated by fewer than $m$ vertices must be joined by $m$ openly disjoint paths.  A basic corollary is that two vertices which cannot be separated by a single vertex must lie on a cycle.  We'll make use of the corollary that a pair of vertices that cannot be separated by a set of size two must be joined by three openly disjoint paths.  There are lots of variations of Menger's theorem.  In particular, two subsets $A$ and $B$ of $V(X)$ cannot be separated by fewer than $m$ vertices if and only if there are $m$ disjoint paths which start in $A$ and end in $B$.

We are about to prove a lower bound on the vertex connectivity of a vertex-transitive graph.

If $A$ is a set of vertices in $X$, let $N(A)$ denote the vertices in $V(X)\setminus A$ with a neighbor in $A$ and let $\overline{A}$ be the complement of $A\cup N(A)$ in $V(X)$.  That is, $A$ is a collection of vertices, $N(A)$ (the `neighborhood of $A$') is the collection of vertices just outside of $A$, and $\overline{A}$ is everything else.

\definition{A \textbf{fragment} of $X$ is a subset $A$ such that $\overline{A}$ is nonempty and $|N(A)|=\kappa_0(X)$.  We have $\overline{A}$ empty when every vertex in $V(X)$ is either in $A$ or adjacent to a vertex in $A$, and $|N(A)|=\kappa_0(X)$ when $N(A)$ is a minimum vertex cutset. }  

An \textit{atom} of $X$ is a fragment which contains the minimum possible number of vertices.  An atom must be connected and if $X$ is $k$-regular with an atom consisting of a single vertex, then $\kappa_0(X)=k$.  We can also see that if $A$ is a fragment, then $N(A)=N(\overline{A})$ and $\overline{\overline{A}}=A$.  The following lemma gives us some useful properties of fragments.

\begin{lemma}
Let $A$ and $B$ be fragments in $X$.  Then:
\begin{enumerate}
\item[a)] $N(A\cap B)\subset (A\cap N(B))\cup (N(A)\cap B) \cup (N(A)\cap N(B))$
\item[b)] $N(A\cup B) = (\overline{A}\cap N(B))\cup (N(A) \cap \overline{B}) \cup (N(A)\cap N(B))$
\item[c)] $\overline{A}\cup\overline{B} \subset \overline{A\cap B}$
\item[d)] $\overline{A\cup B}=\overline{A}\cap\overline{B}$
\end{enumerate}
\end{lemma}

\begin{proof}
	
	Suppose first that $x\in N(A\cap B)$.  Since $A\cap B$ and $N(A\cap B)$ are disjoint, if $x\in A$ then $x\notin B$, so $x\in N(A)$ (or vice versa).  If $x$ isn't in either $A$ or $B$, then it is in $N(A)\cap N(B)$, and we have proved (a).
	
	Similarly, we can show  $N(A\cup B) \subset (\overline{A}\cap N(B))\cup (N(A) \cap \overline{B}) \cup (N(A)\cap N(B))$.  To show inclusion in the other direction (and therefore equality), note that if $x\in \overline{A}\cap N(B)$, then $x$ is in neither $A$ nor $B$.  Since $x\in N(B)$ and $x\notin A$, $x\in N(A\cup B)$.  Similarly, if $x\in N(A)\cap\overline{B}$ or $x\in N(A)\cap N(B)$, then  $x\in N(A\cup B)$, and we have proved (b).
	
	Next, if $x\in \overline{A}$, then $x$ is not in $A$ or $N(A)$, so it can't be in $A\cap B$ or $N(A\cap B)$, so $x\in \overline{A\cap B}$, which proves (c).
	
	Finally, if $x\in \overline{A\cup B}$, then $x$ is not in $A\cup B$ or $N(A\cup B)$.  Thus $x$ is not in $A$ or $B$ nor $N(A)$ or $N(B)$, thus it is in both $\overline{A}$ and $\overline{B}$, and we have proved (d).
	
	
	
	
\end{proof}

\begin{theorem}
	
	
	Let $X$ be a graph on $n$ vertices with connectivity $k$.  Suppose $A$ and $B$ are fragments of $X$ and $A\cap B$ is nonempty.  If $|A|\leq |\overline{B}|$, then $A\cap B$ is a fragment.
	
	
\end{theorem}
\begin{proof}
	
	
	Since $A$, $N(A)$, and $\overline{A}$ (symmetrically for $B$) partition the set $V(X)$ into three disjoint parts, the pairwise intersections of one chunk from $A$ with a chunk from $B$ gives us a partition into nine disjoint parts.  As some shorthand, we'll deonte $$a=|A\cap N(B)|,b=|N(A)\cap B|,c=|N(A)\cap N(B)|,d=|N(A)\cap\overline{B}|,e=|\overline{A}\cap N(B)|$$.  We proceed in steps.
	
	\subsubsection*{a)} $|A\cup B|<n-k$
	
	Since $|F|+|\overline{F}| = n-k$ for any fragment $F$, $|A|\leq |\overline{B}| = n-k-|B|$, as $A$ and $B$ are fragments.  Thus $|A|+|B|\leq n-k$, and since $A$ and $B$ share at least one element in common, the inequality is strict.
	
	\subsubsection*{b)} $|N(A\cup B)|\leq k$
	
	From the previous lemma, $|N(A\cap B)|\leq a+b+c$ and $|N(A\cup B)|=c+d+e$.  Thus, $2k=|N(A)|+|N(B)| = a+b+2c+d+e \geq |N(A\cap B)|+|N(A\cup B)|$.  Since $|N(A\cap B)|\geq k$, it must be that $|N(A\cup B)|\leq k$.
	
	\subsubsection*{c)} $\overline{A}\cap\overline{B}\neq \emptyset$.
	
	
	From (a) and (b), we have that $|A\cup B| + |N(A\cup B)| < n$, so $\overline{A\cup B}\neq \emptyset$, and the claim follows from part (d) of the previous lemma.
	
	\subsubsection*{d)} $|N(A\cup B)|=k$
	
	For any fragment $F$, $N(F)=N(\overline{F})$.  By part (a) of the previous lemma, and step (b) above, we get
	
	\begin{align*}
	N(\overline{A}\cap\overline{B})&\subset (\overline{A}\cap N(\overline{B}))\cup (\overline{B}\cap N(\overline{A}))\cup(N(\overline{A})\cap N(\overline{B}))\\
	&=(\overline{A}\cap N(B))\cup (\overline{B}\cap N(A))\cup (N(A)\cap N(B))\\
	&=N(A\cup B)
	\end{align*}
	
	Since $\overline{A}\cap\overline{B}$ is nonempty, $|N(\overline{A}\cap\overline{B})|\geq k$, so $|N(A\cup B)|\geq k$.  Combining this with step (b), the claim follows.
	
	\subsubsection*{e)}  $A\cap B$ is a fragment.
	
	From step (b), we have that $|N(A\cap B)|+|N(A\cup B)|\leq 2k$, and (d) tells us that $|N(A\cap B)\leq k$, so $N(A\cap B)$ is of size $k$, and we are done. 
	
\end{proof}


\corollary{If $A$ is an atom and $B$ a fragment of $X$, then $A$ is entirely contained in one of $B$, $N(B)$, or $\overline{B}$.}

\begin{proof}
	Since $A$ is an atom, $|A|\leq |B|$ and $|A|\leq |\overline{B}|$.  Thus the intersection of $A$ with $B$ or $\overline{B}$ is a fragment (if nonempty).  Since $A$ is an atom, no proper subset can be a fragment.
\end{proof}


Now we are ready to prove the theorem mysteriously referenced earlier.

\thrm{A vertex-transitive graph with valency $k$ has vertex connectivity at least $\frac{2}{3}(k+1)$.}

\begin{proof}
	Let $X$ be a vertex-transitive graph with valency $k$, and let $A$ be an atom in $X$.  If $A$ is a singe vetex, then $|N(A)|=k$ and we are done.  Suppose $|A|\geq 2$.  If $g\in Aut(X)$, then $A^g$ is an atom as well, so by the previous corollary, either $A=A^g$ or $A$ and $A^g$ are disjoint.  Then $A$ is a block of imprimitivity for $Aut(X)$, and its translates partition $V(X)$.  Then again by the corollary, we have that $N(A)$ is also partitioned by the translates of $A$, so $|N(A)|=t|A|$ for some positive integer $t$.
	
	Let $u$ be a vertex in $A$.  Then the valency of $u$ is at most $|A|-1 + |N(A)|=(t+1)|A|-1$.  Thus it follows that $k+1\leq (t+1)|A|$ and $\kappa_0(X)\geq\frac{t}{t+1}k$.  To complete the proof, we only need to show $t\geq 2$.  
	
	Suppose for the sake of contradiction that $t=1$.  By the corollary above, $N(A)$ is a union of atoms, so $N(A)$ is also an atom.  Since $Aut(X)$ acts transitively on the atoms of $X$, it follows that $|N(N(A))|=|A|$, so since $A\cap N(N(A))$ is nonempty, $A=N(N(A))$.  This implies $\overline{A}=\emptyset$, contradicting the assumption that $A$ is a fragment.
\end{proof}






\section*{Matchings}

\definition{A \textbf{matching} $M$ in a graph $X$ is a set of edges such that no two edges are incident to the same vertex.  Equivalently, a matching is a subset of the vertices of $X$ such that $M$ can be partitioned into disjoint sets of size $2$ such that there is an edge in $X$ connecting each pair.  A matching $M$ is called \textbf{perfect} or a \textbf{$\boldsymbol{1}$-factor} if every vertex in $X$ belongs to $M$. A \textbf{maximum matching} is a matching $M$ such that no other edges can be added to $M$ without violating the definition of a matching.}

Our treatment of matchings will largely be from the perspective of edge sets rather than vertex sets.  That is, we we talk about a matching $M$, we formally mean that $M$ is the set of edges in the matching, but we will be sloppy and talk about a vertex being `in' the matching when what we really mean is that the vertex is incident to some edge in $M$.

Obviously any graph which has a perfect matching has an even number of vertices.  We can also induce a partial ordering on matchings by inclusions.  A maximum matching is therefore an element of this poset which has nothing sitting above it.

The following result tells us that a connected vertex-transitive graph on an even number of vertices \textit{must} have a perfect matching, and that such a graph on an odd number of vertices has a maximum matching which misses exactly one vertex.  To prove this, we first need two lemmas and a few definitions.  Throughout, we will assume $X$ is connected and vertex-transitive.

\definition{If $M$ s a matching, in $X$ and $P$ is a path in $X$ such that every second edge of $P$ is in $M$, then we call $P$ an \textbf{alternating path} with respect to $M$.  Similarly, an \textit{alternating cycle} is a cycle with every second edge in $M$.}

Suppose that $M$ and $N$ are matchings in $X$, and consider their symmetric difference $(M\cup N)\setminus (M\cap N)$, which we will write $M\oplus N$, for ease of notation.  Since $M$ and $N$ are regular subgraphs with valency $1$, $M\oplus N$ is  a  subgraph with valency at most $2$.  Thus each component of it must either be a path or a cycle.  Since no vertex of $M\oplus N$ has two incident edges in either $M$ or $N$, these paths or cycles are alternating with respect to both $M$ and $N$, and each cycle must have even length. Suppose $P$ is a path in $M\oplus N$ with odd length.  Without loss of generality, suppose that $P$ contains more edges from $M$ than $N$, so $N\oplus P$ is also a matching which contains more edges than $N$. Thus $P$ must contain an equal number of edges from $M$ and $N$, so its length is even.


\begin{lemma}
	Let $u$ and $v$ be vertices in $X$ such that no maximum matching misses both of them.  Suppose then that $M_u$ is a maximim matching which misses $u$ but not $v$ and $M_v$ a maximum matching which misses $v$ but not $u$.  Then there is a path of even length in $M_u\oplus M_v$ with $u$ and $v$ as its endpoints.
	
\end{lemma}




\begin{proof}
	Since $M_u$ and $M_v$ miss $u$ and $v$, respectively, their valencies in $M_u\oplus M_v$ must be $1$, so both are end vertices of some path.  We need to show that they are in the same connected component of $M_u\oplus M_v$.  As $M_u$ and $M_v$ have maximum size, all paths (including those with endpoints $u$ and $v$) have even length.  Suppose, for the sake of contradiction, that $u$ and $v$ lie on distinct paths.  Let $P$ be the path on $u$.  Then $P$ is alternating with respect to $M_v$,  has even length, and $M_v\oplus P$ is a matching in $X$ which misses $u$ and $v$ and has the same size as $M_v$, which contradicts how we chose $u$ and $v$.
\end{proof}

We have to prove one more lemma before our theorem.

\definition{We call a vertex $u$ in $X$ \textbf{critical} if it is in every maximum matching.  If $X$ is vertex transitive and one vertex is critical, then every vertex is critical, so $X$ has a perfect matching.}

\begin{lemma}
	Let $u$ and $v$ be distinct vertices in $X$, and let $P$ be a path from $u$ to $v$.  If no vertex of $V(P)\setminus\{u,v\}$ is critical, then no maximum matching misses both $u$ and $v$.
\end{lemma}
\begin{proof}
	
	We proceed by induction on the length of $P$.  If $u$ and $v$ are adjacent, then no maximum matching can miss both $u$ and $v$, as we can always add the edge $(u,v)$ to some matching which misses both to increase the size.
	
	Suppose $P$ has length at least $2$, and let $x$ be some vertex on $P$ distinct from $u$ and $v$.  Then $u$ and $x$ are joined by a path which has no critical vertices, and this path is shorter than $P$, so by induction, no maximum matching misses both $u$ and $x$ and no maximum matching misses both $v$ and $x$.  Since $x$ is not critical, there is a maximum matching $M_x$ which misses $x$.  Assume, for the sake of contradiction, that $N$ is a maximum matching which misses both $u$ and $v$.  Then by the previous lemma, there is a path from $u$ to $x$ in $M_x\oplus N$ and similarly there is a path from $x$ to $v$, so there is a $u{-}v$ path, which implies that $u=v$, contradicting the assumption that they are distinct.
	
	
	
\end{proof}

We can  now wrap this up into a proof of our big theorem:

\begin{theorem}
	Let $X$ be a connected vertex-transitive graph.  Then $X$ has a matching which misses at most one vertex, and for any edge there exists a maximum matching containing that edge.  
\end{theorem}
\begin{proof}
	We noted that a vertex-transitive graph which contains a critical vertex must contain a perfect matching, and by the previous lemma, if $X$ is vertex-transitive and does not contain a critical vertex, then no two vertices are both missed by any maximum matching, so a maximum matching covers all but one vertex of the graph.
	
	We now only need to show that any edge is contained in some maximum matching.  We proceed inductively, supposing it holds for vertex-transitive graphs smaller than $X$ (base cases of graphs on one, two, or three vertices are trivial).  If $X$ is edge-transitive, the claim is trivial, so we assume that $X$ is not edge-transitive.  Suppose, for the sake of contradiction, that $e$ is an edge not in any maximum matching. Let $Y$ be the subgraph of $X$ induced by the edge set consisting of the orbit of $e$ under $Aut(X)$.  Since $X$ is not edge-transitive, $Y$ is a strict subgraph of $X$ on the same vertex set.  We will show that $X$ has  a matching containing an edge of $Y$ which misses at most one vertex.  Thus under some $g\in Aut(X)$, this matching maps to one containing $e$ missing at most one vertex.
	
	If $Y$ is connected, then by induction each edge lies in a matching which misses at most one vertex, and we are done.  Suppose then that $Y$ is not connected.  The components of $Y$ form a system of imprimitivity for $Aut(X)$ and are pairwise isomorphic vertex-transitive graphs.  If the number of vertices in each component is even, then by induction we can find a perfect matching on each component whose union is a perfect matching in $Y$.  Assume then that there is some component of $Y$ which has an odd number of vertices.  Let $Y_1,Y_2,\dots,Y_r$ be the components of $Y$.  Consider the graph $Z$ which has a vertex for each $Y_i$ and an edge between $Y_i$ and $Y_j$ if and only if there is an edge in the original graph $X$ joining some vertex of $Y_i$ to $Y_j$.  Then $Z$ is vertex-transitive, so by induction contains a matching $N$ which misses at most one vertex.  Suppose $(Y_i,Y_j)\in N$ is an edge, and since $Y_i$ is adjacent to $Y_j$ in $Z$, there are vertices $y_i,y_j$ in $X$ which are adjacent.  Since $Y_i$ and $Y_j$ are vertex-transitive and have an odd number of vertices, there is a matching in $Y_i$ missing only $y_i$ and similarly for $y_j$, but then we can include the edge $(y_i,y_j)$ to get a matching in $X$ which misses nothing in either component.  If the number of components $Y_i$ is even, this construction gets us a perfect matching.  Otherwise, we have a matching which is perfect on all but one component, and then a matching within that last component which misses exactly one vertex.  This concludes the proof.
\end{proof}




\section*{Hamiltonian Paths and Cycles}

\definition{A \textbf{Hamilton path} in a graph $X$ is a path which meets every vertex.  A \textbf{Hamilton cycle} is a path which meets every vertex and starts and ends at the same vertex.  A graph is called \textbf{hamiltonian} if it contains a Hamilton cycle.  All known vertex-transitive graphs have Hamilton paths and only five are known which do not contain Hamilton cycles.  Let's take a look at these.}

Clearly $K_2$ is vertex transitive and has a Hamilton path but no Hamilton cycle (no nontrivial cycles at all!).  More interestingly, the Petersen graph doesn't have a Hamilton cycle.  The graph has enough symmetry that a case argument is tedious, rather than unmanageable, but we'll see an algebraic proof in a later chapter.  The Coxeter graph, which is arc-transitive and on 28 vertices, is also not hamiltonian.  the other two graphs are realized by replacing the vertices of the Petersen and Coxeter graphs with triangles.

\definition{The \textbf{subdivision graph} $S(X)$ of a graph $X$ is obtained by placing a new vertex in the middle of each edge of $X$.  That is, the vertex set of $S(X)$ is $V(X)\cup E(X)$, and two vertices $v,e$ in $S(X)$ are adjacent if and only if $v$ corresponds to a vertex in $V(X)$ and $e$ to an edge in $E(X)$ such that $e$ is incident to $v$.  This graph is bipartite, with the vertices from $V(X)$ and $E(X)$ forming the bipartition.  The vertices in the `edge class' all have valency $2$.  If $X$ is regular with valency $k$, then the vertices in the `vertex class' are also all of valency $k$.  In this case, $S(X)$ is semiregular bipartite.}

\begin{lemma}
	Let $X$ be a cubic graph.  Then $L(S(X))$ has a Hamilton cycle if and only if $X$ does.
\end{lemma}

\begin{proof}
	 
	
	A Hamilton cycle in a line graph $L(X)$ corresponds to an ordered enumeration of the edges of $X$ such that each edge in the enumeration is incident to a vertex in common with the preceding and succeeding edge, and the first and last edge in the enumeration is the same.  Since $S(X)$ for a cubic graph is a semiregular bipartite graph, a Hamilton cycle in $L(S(X))$ uniquely corresponds to an ordering of the valency $2$ vertices of $S(X)$ such that any two successive vertices in the ordering are at distance two from each other.  But this induces an ordering of the vertices of valency $3$, which corresponds to a sequence of vertices in $X$ itself.  It is clear that if there is a Hamilton cycle in $L(S(X))$, the induced ordering on the valency $3$ vertices corresponds to a Hamilton cycle in $X$.  But the same goes the other way.  If we have a Hamilton cycle in $X$, this corresponds to an ordering of the valency $3$ vertices such that successive vertices are at distance $2$, which means we have to visit each vertex of valency $2$ once, hence we use every edge (one to enter, one to leave).
\end{proof}


If $X$ is arc-transitive and cubic, then $L(S(X))$ is vertex-transitive.  Thus we get the last two of the known vertex-transitive graphs which are not hamiltonian.  Of these five graphs, only $K_2$ is a Cayley graph (for the group $\mathbb{Z}_2$, of course), and it is conjectured that all other Cayley graphs are hamiltonian and, even more strongly, that all other vertex-transitive graphs are hamiltonian.  This conjecture is essentially a totally open problem, but it is known to be false for directed graphs.

A natural question is to find a lower bound on the length of a longest cycle in a vertex-transitive graph $X$.  The best known bound is $O(\sqrt{|V(X)|})$, which isn't great, but we'll derive it anyway.  
\begin{lemma}
	Let $G$ be a transitive permutation group acting on a set $V$, let $S$ be a subset of $V$, and set $c$ equal to the minimum value of $|S\cap S^g|$ as $g$ ranges over $G$.  Then $|S|\geq \sqrt{c|V|}$.
		
		
\end{lemma}
\begin{proof}
	We'll count pairs $(g,x)$ where $g\in G$ and $x\in S\cap S^g$.  For each $g\in G$, there are at least $c$ such points in $S$, so there are at least $c|G|$ such pairs.  On the other hand, the elements of $G$ which maps $x$ to $y$ form a coset of $G_x$, so there are exactly $|S||G_x|$ elements $g^{-1}\in G$ such that $x^{g^{-1}}\in S$, (equivalently, $x\in S^g$).  Thus $c|G|\leq |S|^2|G_x|$ and since $G$ is transitive, $\frac{|G|}{|G_x|}=|V|$ by the Orbit-Stabilizer theorem.  The claim follows from basic algebraic manipulation.
\end{proof}



The next theorem depends on the fact that in a $3$-connected graph, any two cycles of maximum length have at least three vertices in common, which follows from Menger's theorem.

\begin{theorem}
	A connected vertex-transitive graph on $n$ vertices contains a cycle of length at least $\sqrt{3n}$.
	
	
\end{theorem}


\begin{proof}
	Let $X$ be a graph and $G=Aut(X)$.  First, a connected vertex transitive graph with valency at least $3$ is $3$-connected (the theorem is trivial for graphs with valency $2$), so let $C$ be a maximum-length cycle in $X$.  Then, $|C\cap C^g|\geq 3$ for any automorphism of $X$ (there are at least three vertices in common between any two maximum-length cycles), so the result follows from the bound in the previous lemma.
\end{proof}

In fact, the Petersen graph has cycles which pass through nine of the ten vertices.


\section*{Cayley Graphs}

An important class of objects in algebraic graph theory, we are now ready to develop some theory about Cayley graphs.

\definition{A permutation group $G$ acting on a set $V$ is \textbf{semiregular} if no nonidentity element of $G$ fixes a point of $V$.  From the Orbit-Stabilizer theorem, it follows that every orbit of a semiregular group has length $|G|$.  A group $G$ is \textbf{regular} if it is semiregular and transitive.  If $G$ is regular on $V$, then $|G|=|V|$.}

Any group $G$ acts regularly on itself.  Recall that $\rho_g$ for $g\in G$ is the permutation of the elements of $g$ such that $x\mapsto xg$.  The mapping $g\mapsto \rho_g$ is called the \textit{right regular representation} of $g$.  This group is isomorphic to $G$, hence $G$ acts transitively (and regularly) on it.

\begin{lemma}
	Let $G$ be a group and $C$ an inverse-closed subset of $G$ which does not include $e$.  Then $Aut(X(G,C))$ contains a regular subgroup isomorphic to $G$.
\end{lemma}
\begin{proof}
	This follows immediately from the proof of the earlier theorem that the Cayley graph $X(G,C)$ is vertex-transitive.
\end{proof}

There is a converse of this lemma, which we will prove.

\begin{lemma}
	If a group $G$ acts regularly on the vertices of the graph $X$, then $X$ is a Cayley graph relative to some inverse-closed set $C\subset G\setminus\{e\}$.
\end{lemma}
\begin{proof}
	Fix a vertex $u$ of $X$.  If $v$ is any vertex of $X$, there is a unique group element, say $g_v$, such that $u^{g_v} = v$, since $G$ acts regularly on $V(X)$.  Let $C=\{g_v|(u,v)\in E(X)\}$.  If $x$ and $y$ are vertices of $X$, since $g_x\in Aut(X)$, $x$ is adjacent to $y$ if and only if $x^{g_x^{-1}}$ is adjacent to $y^{g_x^{-1}}$.  But $x^{g_x^{-1}}=u$ and $y^{g_x^{-1}}=u^{g_yg_x^{-1}}$ so $x$ and $y$ are adjacent if and only if $g_yg_x^{-1}\in C$.  But this looks like the construction of a Cayley graph.  If we identify each vertex $x$ with the group element $g_x$, then $X$ is isomorphic to $X(G,C)$.  Since $X$ is undirected and has no self-loops, the set $C$ must be an inverse-closed subset of $G\setminus\{e\}$.
\end{proof}

One thing we've been sweeping under the rug is that there are many Cayley graphs for any given group.  It's natural to ask under what conditions two Cayley graphs for the same group are isomorphic, and the next lemma gets towards an answer to this question.  An \textit{automorphism of a group} is a bijection $\theta: G\rightarrow G$ such that $\theta(gh)=\theta(g)\theta(h)$ for all $g,h\in G$.  That is, an isomorphism from a group to itself.

\begin{lemma}
	If $\theta$ is an automorphism of the group $G$, then $X(G,C)$ and $X(G,\theta(C)$ are isomorphic as graphs.
	

\end{lemma}
\begin{proof}
	For any two vertices $x$ and $y$ in $X(G,C)$, it must be that $\theta(y)\theta(x)^{-1}=\theta(yx^{-1}$ (thinking of $x$ and $y$ as group elements).  Thus $\theta(y)\theta(x)^{-1}\in \theta(C)$ if and only if $yx^{-1}\in C$.  Thus $\theta$ preserves adjacency and non-adjacency between $X(G,C)$ and $X(G,\theta(C))$.
	
\end{proof}

The converse of this is not true; two Cayley graphs for a group can be isomorphic even if there is no automorphism relating their connection sets.

\definition{A \textbf{generating set} $C$ of a group $G$ is a subset such that any element of $G$ can be written as the product of elements of $C$.  Equivalently, the only subgroup of $G$ which contains $C$ is $G$ itself.}

\begin{lemma}
	The Cayley graph $X(G,C)$ is connected if and only if $C$ is a generating set for $G$.
\end{lemma}

\begin{proof}
	It is clear that if $C$ generates $G$, the Cayley graph is connected, as $x$ and $y$ are adjacent if and only if there is a $g\in C$ such that $xg=y$.
	
	For the other direction, suppose that $X(G,C)$ is connected.  Since two vertices are adjacent if and only if they belong to the subgroup generated by $C$, it follows that $C$ generates $G$.
\end{proof}



\section*{Directed Cayley Graphs With No Hamiltonian Cycles}

It turns out that it's pretty simple to find vertex-transitive directed graphs which are not hamiltonian, and the examples will even be directed Cayley graphs.

\begin{theorem}
	Suppose that distinct group elements $\alpha$ and $\beta$ generate a finite group $G$, and that the graph $X$ is the directed Cayley graph $X(G,\{\alpha,\beta\})$ with connection set $\{\alpha,\beta\}$.  Furthermore, assume that, in their actions by left mulitplication on $G$ that $\alpha$ and $\beta$ have $k$ and $\ell$ cycles, respectively.  If the element $\beta^{-1}\alpha$ has odd order and $V(X)$ has a partition into $r$ disjoint directed cycles, then $r$, $k$, and $\ell$ all have the same parity.
\end{theorem}
\begin{proof}
	Suppose $V(X)$ has a partition into $r$ directed cycles, and define the permutation $\pi$ of $G$ to be $x^\pi =y$ if the arc $(x,y)$ is in one of the directed cycles, that is, $\pi$ `pushes' every vertex forward along its directed cycle.  If we let $P=\{x\in V(X)|x^\pi = \alpha x\}$ and $Q=\{ x\in V(X)|x^\pi = \beta x \}$, then $P$ and $Q$ partition $V(X)$, because $\alpha$ and $\beta$ were chosen to be distinct.
	
	Let $\tau$ be the permutation in $G$ such that $x^\tau=\beta^{-1}x^\pi$.  Clearly $\tau$ fixes every element of $Q$, and it maps elements of $P$ to other elements of $P$, and for any $x\in P$, $x^\tau = \beta^{-1}\alpha x$, and since $\beta^{-1}\alpha$ has odd order, $\tau$ must as well.  This is because odd permutations have even order, as an element of odd order is the square of some element in the cyclic group it generates, thus it is even.  
	
	It is a fact about symmetric groups that the parity of a permutation on $n$ elements with $r$ disjoint cycles is the parity of $r+n$.  Since left multiplication by $\pi\beta^{-1}$ is an even permutation, $\ell+r$ is even.  A symmetric argument for $\pi\alpha^{-1}$ tells us that $k+r$ is even.  Thus $\ell,k,r$ all have the same parity.
	
	
\end{proof}


The symmetric group on $n$ elements can be generated by two permutations: $(12)$ and $(123\dots n)$.  For example, $Sym(4)$ is generated by $\alpha=(12)$ and $\beta=(1234)$.  The Cayley graph $X=X(Sym(4),\{(12),(1234)\})$ is shown below (maybe later) %add figure%
.  Now, $\beta^{-1}\alpha=(143)$, which has odd order (order $3$), and since $|Sym(4)|=24$, $\alpha$ and $\beta$ have $12$ and $6$ cycles in $Sym(4)$, respectively, under the action of left multiplication.  Thus $V(X)$ can be partitioned into an even number of directed cycles, so, in particular, does not have a directed Hamilton cycle.


This can generalize to an infinite family of directed Cayley graphs $X(n)=X(Sym(n),\{(12),(123\dots n)\})$.

\begin{corollary}
	{If $n\geq 4$ is even, then the directed Cayley graph $X(n)$ is not hamiltonian.}
	
\end{corollary}
\begin{proof}
	Performing the same construction above shows us that $\alpha$ has $\frac{n!}{2}$ cycles in its action by left multiplication and $(123\dots n)$ has $(n-1!)$, but $(123\dots n)^{-1}(12)$ has order $n-1$, so since we can only partition $V(X)$ into an even number of directed cycles, we are done.
\end{proof}


We know that $X(3)$ and $X(5)$ are hamiltonian, but we don't know about odd $n\geq 7$.


\section*{Retracts}
Recall that a \textit{retract} is a subgraph $Y$ of $X$ such that there exists a homomorphism $f$ from $X$ to $Y$ such that the restriction  $f\upharpoonright Y$ of $f$ to $Y$ is the identity map on the vertices in $Y$.  It's even enough to only require that $f\upharpoonright Y$ is a bijection, i.e. an automorphism of $Y$.  We're about to prove that every vertex-transitive graph is the retract of some Cayley graph.  If $G$ is our group acting transitively on $V(X)$ and $x$ and $y$ vertices of $X$, then by a lemma in Chapter 2, the set of group elements in $G$ which map $x$ to $y$ is a right coset of $G_x$.  Thus there is a bijection from $V(X)$ to the right cosets of $G_x$.  The action of $G$ on $V(X)$ coincides with the action of right multiplication on the cosets of $G_x$.

\begin{theorem}
	Any connected vertex-transitive graph is a retract of some Cayley graph.
\end{theorem}
\begin{proof}
	Let $X$ be a connected vertex-transitive graph and let $x\in V(X)$.  Define the set $C=\{ g\in G|(x,x^g)\in E(X) \}$, the set of group elements which send $x$ to one of its neighbors.  We have that $C$ is the union of right cosets of $G_x$, and since $x$ is not adjacent to itself, $C\cap G_x$ is empty.  Furthermore, since $x^a$ is adjacent to $x^b$ if and only if $x$ is adjacent to $x^{ba^{-1}}$, this is true if and only if $ba^{-1}\in C$.
	
	If $g\in C$ and $h,h'\in G_x$, then $x=x^h$, $x^h$ is adjacent to $x^{gh}$, and $x^{gh}=x^{h'gh}$, so $h'gh\in C$.  Thus $G_x CG_x\subset C$, and since $e\in G_x$, $C\subset G_x C G_x$, so $C=G_x CG_x$.
	
	Let $H$ be the subgroup of $Aut(X)$ generated by $C$.  By induction on the diameter of $X$, we can see that $H$ acts transitively on the vertices of $X$.  Now let $Y$ be the Cayley graph $X(H,C)$.  The right cosets of $H_x$ partition $V(Y)$, so we can express any group element of $H$ as $ga$ for some $g\in H_x$.  If $g$ and $h$ are both in $H_x$, then $ga$ and $hb$ are adjacent if and only if $hb(ga)^{-1}=hba^{-1}g^{-1}\in C$, which happens if and only if $ba^{-1}\in C$.  Thus any two distinct right cosets have no edges between them or are completely connected, and since $e\notin C$, the subgraph of $Y$ induced by each right coset is empty.
	
	Therefore, the subgraph of $Y$ induced by any complete set of coset representatives of $H_x$ is isomorphic to $X$.  The map sending the vertices of $Y$ in some right coset of $H_x$ to the corresponding right coset, viewed as a vertex of $X$, is a homomorphism from $Y$ to $X$, and its restriction to a complete set of coset representatives is a bijection, thus $X$ is a retraction of the Cayley graph $Y$.
\end{proof}


Some dissection of the proof tells us that, given $X$, we can get the Cayley graph $Y$ by replacing each vertex of $X$ with an independent set of size $|G_x|$.  The graph induced by a pair of these independent sets is empty when the vertices in $X$ are not adjacent, or is a complete bipartite subgraph if they are adjacent. Then
$$\frac{|V(X)|}{\alpha(X)}=\frac{|V(Y)|}{\alpha(Y)}$$
where $\alpha(\cdot)$ is the size of the largest independent set in the respective graph.  This will come back later.


\section*{Transpositions}

We will look at some special Cayley graphs for the symmetric groups.  A set of transpositions ($2$-cycles) from $Sym(n)$ can be thought of as the edge set of a graph on $n$ vertices with $(ij)$ as a transposition corresponding to the edge $(i,j)$.  It is (also) a fact about symmetric groups that they are generated by the complete set of $2$-cycles.

\definition{Call a generating set $C$ for a group \textbf{minimal} $G$ if for any $g\in C$, $C\setminus \{g\}$ is not a generating set.}

\begin{lemma}
	Let $\mathcal{T}$ be a set of transpositions in $Sym(n)$.  Then $\mathcal{T}$ is a generating set for $Sym(n)$ if and only if its graph is connected.
\end{lemma}
\begin{proof}
	Let $T$ be the graph of $\mathcal{T}$.  The vertex set of this graph is $\{1,2,3,\dots,n\}$.  Let $G$ be the group generated by $\mathcal{T}$.  If $(1i)$ and $(ij)$ are elements of $\mathcal{T}$, then $(1j)$ is in $G$, as $(1j)=(ij)(1i)(ij)$.  By induction, if there is a path from $1$ to $i$ in $T$, then $(1i)\in G$.  Thus if $k$ and $\ell$ are in the same connected component, then $(k\ell)\in G$, by the same argument using $k$ instead of $1$.  Thus the transpositions belonging to some connected component generate the symmetric group on the vertices of that component.  Since no transposition can map between elements in different connected components, the entire graph must be connected if and only if our set of transpositions generate all of $Sym(n)$.
\end{proof}

\begin{lemma}
	Let $\mathcal{T}$ be a set of transpositions in $Sym(n)$.  Then the following are equivalent:
	\begin{enumerate}
		\item[a)] $\mathcal{T}$ is a minimal generating set for $Sym(n)$.
		\item[b)] The graph of $\mathcal{T}$ is a tree.
		\item[c)] The product of the elements of $\mathcal{T}$ in any order is an $n$-cycle in $Sym(n)$.
	\end{enumerate}
\end{lemma}
\begin{proof}
	A connected graph on $n$ vertices must have at least $n-1$ edges with equality if and only if it is a tree.  Thus (a) and (b) are equivalent, as removal of an element of $\mathcal{T}$ disconnects the graph, so by the previous lemma does not generate all of $Sym(n)$.
	
	To see that (b) and (c) are equivalent, observe that an $n$-cycle can be written as the product of no fewer than $n-1$ transpositions, and since all of the $n$-cycles are conjugate to each other, we can see that if \textit{any} ordering of the transpositions in $\mathcal{T}$ has a product which is not an $n$-cycle, all of them do, and $\mathcal{T}$ cannot generate the $n$-cycles, contradicting the equivalence of (a) and (b).
\end{proof}

There are $(n-1)!$ possible products of $n-1$ transpositions, and if (c) in the previous lemma holds, each of these will be distinct, i.e. we see every $n$-cycle exactly once, written as the product of a unique ordering of elements of $\mathcal{T}$.

If $\mathcal{T}$ is a transposition, then the Cayley graph $X(Sym(n),\mathcal{T})$ has no triangles or any odd cycles, as the product of any odd number of transpositions cannot be another transposition.  This graph is bipartite, with classes corresponding to the parity of the elements of $Sym(n)$.

From (b) in the previous lemma, we can see that each tree on $n$ vertices determines a Cayley graph of $Sym(n)$.


\begin{lemma}
	Let $\mathcal{T}$ be a set of transpositions in $Sym(n)$, and let $g,h\in \mathcal{T}$.  If the graph of $\mathcal{T}$ contains no triangles, then $g$ and $h$ have exactly one common neighbor in the Cayley graph $X(Sym(n),\mathcal{T})$ if $gh\neq hg$ and exactly two common neighbors otherwise.
\end{lemma}
\begin{lemma}
	The neighbors of a vertex $g$ in $X(Sym(n),\mathcal{T})$ are those of the form $xg$, where $x\in \mathcal{T}$.  If $xg=yh$ is a common neighbor of $g$ and $h$, then $yx=hg$ and any solution to this yields a common neighbor.  If $h$ and $g$ commute, then $yx=hg$ has two solutions, one of which is the identity and the other is $hg$.
	
	If $h$ and $g$ do not commute, then they have overlapping support, but there are three ways to factor $hg$ into transpositions: as $hg$, as $ah$ and as $gb$, for specific $a$ and $b$.  But because there are no triangles in the graph of $\mathcal{T}$, $a$ and $b$ cannot be in $\mathcal{T}$, hence the identity is the only common neighbor of $g$ and $h$.
\end{lemma}



\begin{theorem}
	Let $\mathcal{T}$ be a minimal generating set of transpositions for $Sym(n)$.  If the graph of $\mathcal{T}$ is asymmetric, the groups $Aut(X(Sym(n),\mathcal{T}))$ and  $Sym(n)$ are isomorphic.
\end{theorem}

\begin{proof}
	Let $T$ be the graph of $\mathcal{T}$.  Since $\mathcal{T}$ is a minimal generating set, $T$ is a tree and is thus acyclic.  Then by the previous lemma, we can determine the set of non-commuting transpositions in $\mathcal{T}$ from the graph $X(Sym(n),\mathcal{T})$, or equivalently those transpositions in $\mathcal{T}$ which have overlapping support.  Thus $X(Sym(n),\mathcal{T})$ determines the line graph of $T$.  Since $T$ is a tree, it is determined by its line graph.
	
	Any (non-identity) element $g\in Aut(X(Sym(n),\mathcal{T}))$ induces a permutation of $\mathcal{T}$.  Since automorphisms preserve paths of length two, the restriction of $g$ to $\mathcal{T}$ is an automorphism of $T$, which, by the assumption on $T$, is trivial.
	
	Suppose now that $g\in Aut(X(Sym(n),\mathcal{T}))$ fixes at least one vertex.  We want to show that $g$ is the identity, and thus that this automorphism group acts regularly.  Suppose, for the sake of contradiction, that $g$ is not the identity.  Then since $X(Sym(n),\mathcal{T})$ is connected, there is a vertex $v$ fixed by $g$ adjacent to a vertex $w$ which is not fixed.  Then $\rho_vg\rho_v^{-1}$ fixes the vertex $e$ corresponding to the identity and moves the adjacent vertex $wv^{-1}$, which is impossible.  Thus $g$ must be the identity, so the automorphism group acts regularly.  Since the group acts regularly and every automorphism of $T$ is trivial, the automorphism group of $X(Sym(n),\mathcal{T})$ must be exactly $Sym(n)$.
\end{proof}

It is often difficult to determine the full automorphism group of a Cayley graph, so this theorem is actually kind of interesting.














\ifdraft

\input{../../zach_private_repo/alggraphth_exc/ex3}
\fi

	\end{document}
	
