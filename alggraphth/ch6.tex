\classheader{: Homomorphisms}

\section*{The Basics}

Recall that a graph homomorphism from $X$ to $Y$ is a map $X\rightarrow Y$ which preserves adjacency.  Homomorphisms also compose.  If $f$ is a homomorphism $X\rightarrow Y$ and $g$ a homomorphism $Y\rightarrow Z$, then $g\circ f$ is a homomorphism $X\rightarrow Z$.

Let $\rightarrow$ be a relation on the class of graphs, and read $X\rightarrow Y$ as `there exists a homomorphism from $X$ to $Y$.  Since composition works the way we expect it to, and the identity map is a homomorphism $X\rightarrow X$, the relation $\rightarrow$ is reflexive and transitive.  It is not, however, a partial order, as we don't have the case that $X\rightarrow Y$ and $Y\rightarrow X$ implies $X=Y$.  For a counterexample, let $X$ be a bipartite graph and $Y$ a graph with two vertices and one edge.

\definition{If $X$ and $Y$ are such that $X\rightarrow Y$ and $Y\rightarrow X$, we call them \textbf{homomorphically equivalent}.  A homomorphism $X\rightarrow Y$ is \textbf{surjective} if every vertex of $Y$ is the image of some vertex in $X$.  }
	
	If there is a surjective homomorphism $X\rightarrow Y$ and $Y\rightarrow X$, then $X$ and $Y$ are isomorphic (assuming both are finite, implicitly).
	
\definition{If $f$ is a homomorphism $X\rightarrow Y$, then the preimages of the vertices $y\in V(Y)$, $f^{-1}(y)\subset V(X)$, are called the \textbf{fibers} of $f$ (one in particular is the fiber above $y$).
The fibers of $f$ determine a partition of $V(X)$ called the \textbf{kernel} of $f$.  If $X$ has no self-loops, this partition forms a collection of independent sets in $X$.}

Given a graph $X$ and a partition $\pi$ of $V(X)$, we can construct a graph $X/\pi$ where each vertex corresponds to a chunk of the partition and there is an edge between vertices if and only if there is an edge with an endpoint in each corresponding chunk, with self-loops allowed.  This is a natural example of a graph such that a homomorphism $X\rightarrow X/\pi$ has kernel $\pi$.

In general, it's difficult to show that there does not exist any homomorphism from one graph to another.  We do have a few tools which can help.  Recall that we showed way back in Chapter 1 that a graph $Y$ can be properly $r$-colored if and only if there exists a homomorphism $Y\rightarrow K_r$.  Thus if there is a pair of homomorphisms $X\rightarrow Y\rightarrow K_r$, we know that the chromatic numbers of $X$ and $Y$ satisfy $\chi(X)\leq \chi(Y)$.  Thus if $\chi(X)>\chi(Y)$, there cannot be a homomorphism $X\rightarrow Y$.  Furthermore, since the homomorphic image of an odd cycle must be an odd cycle of no greater length, if the graph $X$ has an odd cycle of length $\ell$ and $Y$ has no odd cycle of length less than or equal to $\ell$, there cannot be a homomorphism $X\rightarrow Y$.  We call the length of the longest odd cycle of $X$ the \textit{odd girth} of $X$, and the odd girth of $X$ is an upper bound on the odd girth of any graph $Y$ such that there is a homomorphism $X\rightarrow Y$.


\section*{Cores}

\definition{A \textbf{core} is a graph $X$ such that any homomorphism from $X$ to itself is a bijection.  Equivalently, its endomorphism monoid is equal to its automorphism group.}

The simplest example of cores is the set of complete graphs, which we know have automorphism group $Sym(n)$.  A subgraph $Y$ of a graph $X$ is a core of $X$ if it is itself a core and there exists a homomorphism $X\rightarrow Y$.

Every graph has a core (this seems kind of obvious, just pick a proper coloring and use that as the kernel).  Interestingly, every core of a graph $X$ is isomorphic.  We can thus talk about \textit{the} core of a graph $X$, which we denote $X^\bullet$.  If $Y$ is a core of $X$ and $f$ a homomorphism from $X$ to $Y$, then $f\upharpoonright Y$ must be an automorphism of $Y$.  The composition of this homomorphism with the inverse of the restriction must be the identity on $Y$, so any core is also a retract.

\definition{A graph $X$ is \textbf{$\boldsymbol{\chi}$-critical} if the chromatic number of any subgraph is less than $\chi(X)$.}

A $\chi$-critical graph cannot have a homomorphism to any proper subgraph, so it must be its own core.  Some $\chi$-critical graphs are the complete graphs and the odd cycles.

\begin{lemma}
	Let $X$ and $Y$ be cores.  Then there exists homomorphisms $X\rightarrow Y$ and $Y\rightarrow X$ (i.e. they are homomorphically equivalent) if and only if $X$ and $Y$ are isomorphic.  That is, our relation $\rightarrow$ is a partial order over the set of cores.
\end{lemma}
\begin{proof}
	Since isomorphic is a stronger condition than homomorphically equivalent, we need only prove one direction.  Suppose that $X$ and $Y$ are cores and let $f:X\rightarrow Y$ and $g:Y\rightarrow X$ be the homomorphisms between them.  Then since $f\circ g$ and $g\circ f$ are both surjective, $f$ and $g$ themselves must be surjective, and therefore the graphs are isomorphic.
\end{proof}

\begin{lemma}
	Every graph has a core, the core is an induced subgraph, and it is unique up to isomorphism.
\end{lemma}
\begin{proof}
	Since $X$ is a finite graph and the identity automorphism is a homomorphism $X\rightarrow X$, the set of graphs homomorphically equivalent to $X$ is non-empty and has at a minimal element with respect to ordering by inclusion.  
	
	Since a core is a retract, it is an induced subgraph.
	
	Finally, suppose $Y_1$ and $Y_2$ are cores of $X$ and let $f_i$ be a homomorphism $X\rightarrow Y_i$.  Then $f_2\upharpoonright Y_1$ is a homomorphism $Y_2\rightarrow Y_1$ and $f_1\upharpoonright Y_2$ is a homomorphism $Y_1\rightarrow Y_2$.  By the previous lemma, $Y_1$ and $Y_2$ must be isomorphic.
\end{proof}

\begin{lemma}
	Two graphs are homomorphically equivalent if and only if their cores are isomorphic.
\end{lemma}

\begin{proof}
	First, if there is a homomorphism $f:X\rightarrow Y$, then there is a sequence of homomorphisms $X^\bullet\rightarrow X\rightarrow Y \rightarrow Y^\bullet$, which defines a homomorphism $X^\bullet\rightarrow Y^\bullet$.  Symmetrically, if there is a homomorphism $Y\rightarrow X$, we have one $Y^\bullet \rightarrow X^\bullet$.  Hence if $X$ and $Y$ are homomorphically equivalent, their cores are as well.
	
	Conversely, if $g:X^\bullet\rightarrow Y^\bullet$ is a homomorphism of the cores, then there is a sequence of homomorphisms $X\rightarrow X^\bullet \rightarrow Y^\bullet\rightarrow Y$, so by the same argument, if $X^\bullet$ and $Y^\bullet$ are homomorphically equivalent then $X$ and $Y$ are as well.
	
	Hence $X$ and $Y$ are homomorphically equivalent if and only if their cores are, and two cores are homomorphically equivalent if and only if they are isomorphic, by the previous lemma.
\end{proof}
\begin{corollary}
	The relation $\rightarrow$ is a partial order on the set of cores.
\end{corollary}
\begin{proof}
	We know already that $\rightarrow$ is reflexive and transitive.  By the previous lemmas, if $X$ and $Y$ are cores and $X\rightarrow Y$ and $Y\rightarrow X$, then $X$ and $Y$ are isomorphic demonstrates that $\rightarrow$ is antisymmetric on the set of cores.
\end{proof}


\section*{Graph Products}

\definition{If $X$ and $Y$ are graphs, then their (\textbf{Cartesian}) \textbf{product} is the graph with vertex set $V(X)\times V(Y)$ where vertices $(x,y)$ and $(x',y')$ are adjacent in $X\times Y$ if and only if the pairs $x,x'$ and $y,y'$ are adjacent in $X$ and $Y$, respectively.}

The map which sends $(x,y)$ to $(y,x)$ is a natural isomorphism between $X\times Y$ and $Y\times X$, and if we can easily describe an isomorphism between $X\times (Y\times Z)$ and $(X\times Y)\times Z$, so the product is sensibly commutative and associative.  However, if $X\times Y_1$ is isomorphic to $X\times Y_2$, it does not necessarily follow that $Y_1$ and $Y_2$ are isomorphic.  For example, $K_2\times 2K_3$ and $K_2\times C_6$ are both isomorphic to $2C_6$, but $2K_3$ and $C_6$ are not isomorphic.

For a fixed vertex $x\in V(X)$, the vertices $(x,y)$ for all $y\in V(Y)$ form an independent set.  Thus, the mapping $p_X:(x,y)\mapsto x$ is a natural homomorphism from $X\times Y$ to $X$.  Similarly, there is a projection $p_Y:X\times Y\rightarrow Y$.

\begin{theorem}
	If $X$, $Y$, and $Z$ are graphs and $f:Z\rightarrow X$ and $g:Z\rightarrow Y$ are homomorphisms, then there exists a unique homomorphism $\phi:Z\rightarrow X\times Y$ such that $f=p_X\circ \phi$ and $g=p_Y\circ \phi$.
\end{theorem}

\begin{proof}
	Let such $f$ and $g$ be given.  Then we claim $\phi:z\mapsto(f(z),g(z))$ does the trick.  It is clearly a homomorphism $Z\rightarrow X\times Y$, satisfies the requirement of composition with the respective projections, and furthermore it is uniquely determined by $f$ and $g$.
\end{proof}

We use $Hom(X,Y)$ to denote the set of homomorphisms from $X$ to $Y$.

\begin{corollary}
	For any graphs $X,Y,Z$, we have that $$|\mathrm{Hom}(Z,X\times Y)|=|\mathrm{Hom}(Z,X)||\mathrm{Hom}(Z,Y)|$$
\end{corollary}
\begin{proof}
	Since any pair of homomorphisms from the right hand side corresponds to a unique $\phi$ from the left hand side, the result follows combinatorially.
\end{proof}

\definition{A partially ordered set forms a \textbf{lattice} if every pair of elements has a greatest lower bound and a least upper bound.}

\begin{lemma}
	The set of cores with the partial order induced by $\rightarrow$ forms a lattice.
\end{lemma}
\begin{proof}
	Let $X$ and $Y$ be cores.  For any core $Z$, if $X\rightarrow Z$ and $Y\rightarrow Z$, then $X\cup Y\rightarrow Z$.  So $(X\cup Y)^\bullet$ is the least upper bound of $X$ and $Y$.
	
	Similarly, if $Z\rightarrow X$ and $Z\rightarrow Y$, then by the previous theorem we have $Z\rightarrow X\times Y$.  Hence $(X\times Y)^\bullet$ is the greatest lower bound of $X$ and $Y$. 
\end{proof}

Somewhat counterintuitively, the greatest lower bound often has more vertices than the least upper bound.\footnote{`Life can be surprising.' Thanks, ssassy math book.}

\definition{If $X$ is a graph then the vertices of $X\times X$ of the form $(x,x)$, where $x\in V(X)$ induce a subgraph of $X\times X$ isomorphic to $X$, called the \textbf{diagonal} of the product.}

In general, $X\times Y$ does not necessarily contain a copy of $X$ (or $Y$).  Consider the product $K_2\times K_3$, which is isomorphic to $C_6$, which has no copy of $K_3$.

We finish the discussion on graph products with another construction related to the Cartesian product.

\definition{Let $X$ and $Y$ be graphs and $f$ and $g$ be homomorphisms from $X$ and $Y$, respectively, to some graph $F$.  The \textbf{subdirect product} of $(X,f)$ and $(Y,g)$ is the subgraph of $X\times Y$ induced by the vertices $(x,y)\in V(X\times Y)$ such that $f(x)=g(y)$.}

If $X$ is a connected bipartite graph, then it has two homomorphisms $f_1$ and $f_2$ to $K_2$ (corresponding to the two choices of colorings).  

\begin{claim}
	
Suppose $Y$ is connected and $g$ is some homomorphism $Y\rightarrow K_2$.  Then the two subdirect products of $(X,f_i)$ with $(Y,g)$ form the components of $X\times Y$.
\end{claim}
\begin{proof}
	(Left as an exercise to Future Zach)
\end{proof}

\section*{The Map Graph}
\definition{Let $F$ and $X$ be graphs.  The \textbf{map graph} $F^X$ has as its vertices the set of functions from $V(X)$ to $V(F)$, and two such functions are adjacent in $F^X$ if and only if whenever $u$ and $v$ are adjacent in $X$, the vertices $f(u)$ and $g(v)$ are adjacent in $F$.  A vertex in the map graph has a self-loop if and only if the corresponding function $h$ is a homomorphism.}

Suppose $\psi$ is a homomorphism from $X$ to $Y$.   If $f$ is a function from $V(Y)$ to $V(F)$, then the composition $f\circ \psi$ is a function from $V(X)$ to $V(F)$, so, following the result from before, $\psi$ determines a map from the vertices of $F^Y$ to those of $F^X$.

\definition{This map $\psi$ is called the \textbf{adjoint map} to $\psi$.}

\begin{theorem}
	If $F$ is a graph and $\psi$ a homomorphism $X\rightarrow Y$, then the adjoint of $\psi$ is a homomorphism $F^Y\rightarrow F^X$.
\end{theorem}
\begin{proof}
	Suppose $f$ and $g$ are adjacent in $F^Y$ and that $x_1$ and $x_2$ are adjacent in $X$.  Then $\psi(x_1)$ is adjacent to $\psi(x_2)$ in $Y$, so $f(\psi(x_1))$ is adjacent to $g(\psi(x_2))$ in $F^Y$.  Hence $f\circ\psi$ and $g\circ\psi$ are adjacent in $F^X$, which demonstrates the homomorphism.
\end{proof}

\begin{theorem}
	For any $F,X,Y$, the graphs $(F^X)^Y$ and $F^{X\times Y}$ are isomorphic.
\end{theorem}
\begin{proof}
	It's clear that these two graphs have the same number of vertices.  We'll describe a natural bijection between the vertex sets and show that this extends to an isomorphism of the graphs.
	
	Let $g$ be a map $g:V(X\times Y)\rightarrow F$.  For any $y\in V(Y)$, the map $g_y:x\mapsto(x,y)$ is an element of $F^X$.  Thus the map $\Phi_g:y\mapsto g_y$ is an element of $(F^X)^Y$, and $g\mapsto \Phi_g$ is the bijection on the vertex sets.  We now need to show that this is in fact an isomorphism.
	
	Let $f$ and $g$ be adjacent vertices in $F^{X\times Y}$.  We'll show that $\Phi_f$ and $\Phi_g$ are adjacent in $(F^X)Y$.  Let $y_1,y_2$ be adjacent vertices in $Y$.  For any two adjacent $x_1,x_2$ in $X$, we have that $(x_1,y_1)$ is adjacent to $(x_2,y_2)$ in $X\times Y$.  Since $f$ and $g$ are adjacent in $F^{X\times Y}$, we have that $f(x_1,y_1)$ is adjacent to $g(x_2,y_2)$ in $F^{X\times Y}$.  Thus $\Phi_f(y_1)$ is adjacent to $\Phi_g(y_2)$ in $(F^X)^Y$.  A symmetric argument starting from $f$ and $g$ being non-adjacent implying $\Phi_f$ and $\Phi_g$ not being adjacent completes the proof.
\end{proof}

\begin{corollary}
	For any graphs $F,X,Y$, $|\mathrm{Hom}(X\times Y,F)|=|\mathrm{Hom}(Y,F^X)|$.
\end{corollary}
\begin{proof}
	Since $(F^X)^Y$ and $F^{X\times Y}$ are isomorphic, they have the same number of loops, which also counts the number of homomorphisms.
\end{proof}
Since there is a homomorphism $X\times F\rightarrow F$ (by projection), this corollary implies the existence of a homomorphism $F\rightarrow F^X$.  More specifically:

\begin{lemma}
	If $X$ has at least one edge, then the constant functions from $V(X)$ to $V(F)$ induce a subgraph of $F^X$ isomorphic to $F$.
\end{lemma}
\begin{proof}
	Let $f$ and $g$ be constant functions $V(X)\rightarrow F(X)$ and let $x_1,x_2$ be vertices in $X$.  Let $f(x)=z_1$ and $g(x)=z_2$.
	
	By the definition of the map graph, $f$ and $g$ are adjacent in $F^X$ if and only if for adjacent $x_1$ and $x_2$, $z_1$ and $z_2$ are adjacent, so $f$ and $g$ are adjacent if and only if $z_1$ and $z_2$ are adjacent, so the constant functions form an induced subgraph isomorphic to $F$.
\end{proof}


\section*{Counting Homomorphisms}

\begin{lemma}
	Let $X,Y$ be graphs.  Suppose that for any graph $Z$, we have $|\mathrm{Hom}(Z,X)|=|\mathrm{Hom}(Z,Y)|$.  Then $X$ and $Y$ are isomorphic.
\end{lemma}

\begin{proof}
	Let $Inj(A,B)$ denote the set of injective homomorphisms from $A$ to $B$.  We start by showing that $|\mathrm{Inj}(Z,X)|=|\mathrm{Inj}(Z,Y)|$.  By letting $Z$ equal $X$ and then $Y$, we can see that the existence of injective homomorphisms $X\rightarrow Y$ and $Y\rightarrow X$ means that $X$ and $Y$ have the same number of vertices, so an injective homomorphism must also be surjective, thus demonstrating isomorphism.
	
	We proceed by induction on the number of vertices in $Z$.  If $Z$ has a single vertex, the claim must be true, as any homomorphism from a graph on one vertex is trivially injective.  Next, we partition the homomorphisms from $Z$ into any graph $W$ according to the kernel, so we get that $|\mathrm{Hom}(Z,W)|=\sum\limits_\pi |\mathrm{Inj}(Z/\pi,W)|$ as we let $\pi$ range over all possible partitions.  A homomorphism is an injection if and only if its kernel is the discrete partition (the one which puts each vertex into a singleton cell), which we call $\delta$.  Thus, $|\mathrm{Inj}(Z,W)|=|\mathrm{Hom}(Z,W)|-\sum\limits_{\pi\neq \delta}|\mathrm{Inj}(Z/\pi,W)|$.  By the induction hypothesis, the terms on the right hand side are the same for $W=X$ and $W=Y$, so we have that $|\mathrm{Inj}(Z,X)|=|\mathrm{Inj}(Z,Y)|$, and we are done.
\end{proof}

\begin{lemma}
	For any graphs $X,Y,F$, we have that $F^{X\cup Y}$ is isomorphic to $F^X\times F^Y$.
\end{lemma}

\begin{proof}
	For any graph $Z$, we have:
	\begin{align*}
	|\mathrm{Hom}(Z,F^{X\cup Y})|&=|\mathrm{Hom}(Z\times(X\cup Y),F)|\\
	                    &=|\mathrm{Hom}((Z\times X)\cup (Z\times Y),F)|\\
	                    &=|\mathrm{Hom}(Z\times X,F)||\mathrm{Hom}(Z\times Y,F)|\\
	                    &=|\mathrm{Hom}(Z,F^X)||\mathrm{Hom}(Z,F^Y)|
	\end{align*}
	Since $|\mathrm{Hom}(Z,X\times Y)|=|\mathrm{Hom}(Z,X)||\mathrm{Hom}(Z,Y)|$, the right hand side of the last line is the number of homomorphisms from $Z$ to $F^X\times F^Y$, and the result follows from the previous lemma.
\end{proof}

\section*{Products and Colorings}

Recall that we know if $X\rightarrow Y$, then $\chi(X)\leq \chi(Y)$.  Since $X\times Y\rightarrow X$ and $X\times Y\rightarrow Y$, we have that $\chi(X\times Y)\leq\min\left\{\chi(X),\chi(Y)\right\}$.  A conjecture of Hedetniemi states that for all $X$ and $Y$, this holds with equality, so $\chi(X\times Y)=\min\left\{\chi(X),\chi(Y)\right\}$.  

Equivalently, if $X$ and $Y$ are not $n$-colorable, then neither is $X\times Y$.  For $n=2$, we can show that the product of two odd cycles contains an odd cycle, hence the product of two non-bipartite graphs is not bipartite.  For $n=3$, the proof is due to El-Zahar and Sauer\footnote{In 1985}.  The remaining cases ($n\geq 4$) are still open.

Here we prove a statement which simplifies Hedetniemi's conjecture using the map graph.

\begin{theorem}
	Let $\chi(X)>n$.  Then $K_n^X$ is $n$-colorable if and only if $\chi(X\times Y)>n$ for all graphs $Y$ such that $\chi(Y)>n$.
\end{theorem}
\begin{proof}
	{By a previous corollary , we have that $$|\mathrm{Hom}(X\times K_n^X,K_n)|=|\mathrm{Hom}(K_n^X,K_n^X)|>0$$ as the identity homomorphism is certainly in the second set. So $X\times X_n^X$ is $n$-colorable.  Therefore, if $\chi(X)>n$ and $\chi(X\times Y)>n$ whenever $\chi(Y)>n$, then $K_n^X$ must be $n$-colorable.
	
Assume then that $\chi(K_n^X)\leq n$ and let $Y$ be such that $\chi(Y)>n$.  Then there are no homomorphisms from $Y$ into any $n$-colorable graph, so $$0=|\mathrm{Hom}(Y,K_n^X)|=|\mathrm{Hom}(X\times Y,K_n)|$$ Therefore $\chi(X\times Y)>n$. }
\end{proof}

A consequence of this theorem is that we can prove Hedetnemi's conjecture by proving that $\chi(K_n^X)\leq n$ if $\chi(X)>n$.  The next few results show what we know about the few cases where we know this to be true.

\begin{theorem}
	The graph $K_n^{K_{n+1}}$ is $n$-colorable.
\end{theorem}
\begin{proof}
	We will construct a proper $n$ coloring of this graph.  For any $f\in K_n^{K_{n+1}}$, there are two distinct vertices in $K_{n+1}$ such that $f(i)=f(j)$.  Define $\phi(f)$ to be the least value in the range of $f$ that is the image of at least two vertices.  If $\phi(f)=\phi(g)$, then for some distinct $i',j'$ we have $f(i)=f(j)=g(i')=g(j')$.  Since $i$ is not equal to both $i'$ and $j'$, we have that $f$ and $g$ are not adjacent.  Therefore, $\phi$ can assign the same color to both $f$ and $g$ and in fact this procedure generates a proper $n$-coloring of $K_n^{K_{n+1}}$.
\end{proof}

\begin{corollary}
	Suppose $X$ contains an $(n+1)$-clique (and not one larger).  Then $K_n^X$ is $n$-colorable.
\end{corollary}
\begin{proof}
	Since there exists a homomorphism $K_{n+1}\rightarrow X$, there is a homomorphism (via the adjoint) $K_n^X\rightarrow K_{n+1}^{K_{n+1}}$.  By the previous theorem, this graph is $n$-colorable, so $K_n^X$ is as well.
\end{proof}

\begin{theorem}
	All of the loops in $K_n^{K_{n}}$ are isolated vertices.  The subgraph induced by the set of vertices without loops is $n$-colorable.
\end{theorem}
\begin{proof}
	Suppose $f\in K_n^{K_n}$ and $f$ is a proper $n$-coloring of $K_n$.  If $g$ is adjacent to $f$, then $g(i)\neq f(j)$ for $i,j\in V(K_n)$, $i\neq j$.  This implies that $g(i)=f(i)$, so $g$ and $f$ are identical, hence a loop.
	For any $f$ in the part of the graph with no loops, there are at least two distinct vertices $i,j$ such that $f(i)=f(j)$, and we can define a proper $n$-coloring of this part by the above theorem.
\end{proof}

\begin{theorem}
	If $X$ is connected and not $n$-colorable, then $K_n^X$ contains a unique $n$-clique, corresponding to the constant functions.
\end{theorem}

\begin{proof}
	The subgraph of $K_n^X$ induced by the constant functions is isomorphic to $K_n$, and is therefore an $n$-clique.  We need only to show that it is the only one.
	
	If $\chi(X)>n$ and $f$ is a homomorphism from $X$ to $K_n^{K_n}$, then, by the previous theorem, $f$ must send each vertex of $X$ onto the same loop of $K_n^{K_n}$.  Since $K_n$ has exactly $n!$ proper $n$-colorings, $K_n^{K_n}$ has exactly $n!$ loops, so, combinatorially,
	\begin{align*}
	n!&=|\mathrm{Hom}(X,K_n^{K_n})|\\
	  &=|\mathrm{Hom}(K_n\times X,K_n)|\\
	  &=|\mathrm{Hom}(X\times K_n,K_n)|\\
	  &=|\mathrm{Hom}(K_n,K_n^X)|
	\end{align*} 
	
	Since there are $n!$ homomorphisms from $K_n$ into $K_n^X$ and therefore $K_n^X$ must contain a unique $n$-clique, as there are $n!$ distinct colorings of an $n$-clique.
\end{proof}

This proof also tells us that if $\chi(X)>n$, there are exactly $n!$ homomorphisms from $X\times K_n$ into $K_n$.  Therefore $X\times K_n$ is uniquely $n$-colorable.

\begin{theorem}
	Let $n\geq 2$ and let $X$ and $Y$ be connected graphs which each contain an $n$-clique.  If $X$ and $Y$ are not $n$-colorable, neither is $X\times Y$.
	
\end{theorem}
\begin{proof}
	Let $x_1,x_2\dots, x_n$ and $y_1,y_2,\dots,y_n$ be the $n$-cliques in $X$ and $Y$, respectively.  Suppose, for the sake of contradiction, that there is some homomorphism $f$ from $X\times Y$ into $K_n$.  If we consider the induced homomorphism from $Y$ to $K_n^X$, the previous theorem tells us that the image of $y_1,\dots,y_n$ are the constant maps, as these form the unique $n$-clique, and symmetrically the image of $x_1,\dots,x_n$ are the constant maps in $K_n^Y$.  That is, $f(x,y_j)$ is constant for each $y_j$ and $f(x_i,y)$ is constant for each $x_i$.  Therefore, $f(x_1,y_1)=f(x_1,y_2)=f(x_2,y_2)$, so the vertices $(x_1,y_2)$ and $(x_2,y_2)$ are adjacent in $X\times Y$ but are mapped to the same vertex under our homomorphism into $K_n$, contradicting the assumption that this is a proper homomorphism.
\end{proof}

\begin{corollary}
	If $X$ and $Y$ are not bipartite, then neither is $X\times Y$.
\end{corollary}

\begin{corollary}
	Let $X$ be a graph such that every vertex lies in some $n$-clique and $\chi(X)>n$.  If $Y$ is a connected graph with $\chi(Y)>n$, then $\chi(X\times Y)>n$ as well.
\end{corollary}

\begin{proof}
	Again, suppose for the sake of contradiction that there is a homomorphism $f$ from $X\times Y$ into $K_n$.  Consider the induced mapping $\Phi_f$ from $X$ into $K_n^Y$.  Because $K_n^Y$ has no loops, each $n$-clique in $X$ is mapped injectively onto the unique $n$-clique in $K_n^Y$.  Each vertex of $X$ lies in some $n$-clique so every vertex of $X$ is mapped onto this $n$-clique.  However, we assumed that $\chi(X)>n$, so there cannot exist a homomorphism from $X$ onto an $n$-clique.
\end{proof}



\section*{Uniquely Colorable Graphs}

A graph $X$ with chromatic number $n$ is partitioned into $n$ independent sets by any $n$-coloring, and conversely a partition into $n$ independent sets yields $n!$ proper $n$-colorings.  

\begin{definition}
	A graph is \textbf{uniquely $\boldsymbol{n}$-colorable if its chromatic number is $n$ and it admits a unique partition into $n$ independent sets.}
\end{definition}

It's straightforward to see that a graph on at least $n$ vertices is uniquely $n$-colorable if and only if it admits $n!$ homomorphisms onto $K_n$.  Connected bipartite graphs are such a class, being uniquely 2-colorable.

\begin{theorem}
	{If $X$ is a connected graph with $\chi(X)>n$, then $X\times K_n$ is uniquely $n$-colorable.}  The proof of this is immediate from the proof that $K^X_n$ contains a unique $n$-clique in the previous section.
\end{theorem}

\begin{theorem}
	If $X$ is uniquely $n$-colorable, then each $n$-coloring, representing a homomorphism to $K_n$, is an isolated vertex in $K_n^X$.
\end{theorem}
\begin{proof}
	Let $f$ be such a coloring and let $x\in V(X)$.  Since the coloring is unique, each of the other $n-1$ colors must appear on a neighbor of $x$.  Then, if $g$ is adjacent to $f$, it must be that $g(x)=f(x)$, so $f$ can only be adjacent to itself.
\end{proof}

Let $\lambda(K_n^X)$ denote the induced subgraph of $K_n^X$ on only the vertices without self-loops.  Then we state the following three conjectures:

\begin{enumerate}
	\item[($B_n$)] If $X$ is uniquely $n$-colorable, and $Y$ is a connected graph which is not $n$-colorable, then $X\times Y$ is uniquely $n$-colorable.
	\item[($D_n$)] If $X$ is uniquely $n$-colorable, then so is $\lambda(K^X_n)$.
	\item[($H_n$)] If $\chi(X)=\chi(Y)=n+1$, then $\chi(X\times Y)=n+1$  
\end{enumerate}

\begin{claim}
The statement that ($H_n$) holds for all positive integers $n$ is equivalent to Hedetniemi's conjecture.  We will show that ($B_n$) is equivalent to ($D_n$) and either of these imply ($H_n$).
\end{claim}

We first need a lemma which will be used to show that ($D_n$) implies ($H_n$).

\begin{lemma}
	If $\chi(X)>n$ then there exists a homomorphism from $K^X_n$ into the subgraph of $K_n^{X\times K_n}$ induced by the vertices without self-loops.
\end{lemma}
\begin{proof}
	Let $p_X$ be the projection map from $X\times K_n$ to $X$ and let $\phi$ be the induced mapping from $K^X_n$ to $K_n^{X\times K_n}$ (constructed from the adjoint of $p_X$).  If $g\in K^X_n$, then $g$ is not a proper coloring of $X$, so there are adjacent vertices $u,v$ in $X$ such that $g(u)=g(v)$.  But $\phi(g) = g\circ p_X$ and for any vertices $i,j$ in $K_n$, $\phi(g)$ maps $(u,i)$ and $(v,j)$ to $g(u)$, so $\phi(g)$ isn't a proper coloring of $X\times K_n$, so it doesn't appear as a loop in the map graph.
\end{proof}

We now prove the claim:

\begin{proof}
	Suppose ($B_n$) holds and let $X$ be a uniquely $n$-colorable graph and let $Y$ be any subgraph of $\lambda(K_n^X)$.  Then, there are more than $n!$ homomorphisms from $Y$ into $K_n^X$ (at the very least, one for each of the $n!$ self-loops and the identity), so $$|\mathrm{Hom}(X\times Y,K_n)|=|\mathrm{Hom}(Y,K^X_n)|\geq n!+1$$ so $X\times Y$ is not uniquely $n$-colorable, but ($B_n$) implies $\chi(Y)\leq n$, so ($B_n$) implies ($D_n$).
	
	The converse is true as well.  If $Y$ is connected and $\chi(Y)>n$ and $\lambda(K^X_n)$ is $n$-colorable, then the only homomorphisms from $Y$ into $K^X_n$ are the ones onto the self-loops, so $$n!=|\mathrm{Hom}(Y,K^X_n)|=|\mathrm{Hom}(X\times Y,K_n)|$$ which implies that $X\times Y$ is uniquely $n$-colorable.
	
	Now, suppose that ($D_n$) is true and let $X$ be such that $\chi(X)>n$.  By the earlier theorem, $X\times K_n$ is uniqely $n$-colorable, so ($D_n$) implies that $\lambda(K_n^{X\times K_n})$ is $n$-colorable, so by the lemma, $K^X_n$ is $n$-colorable, implying Hedetniemi's conjecture.
\end{proof}


\section*{Foldings and Covers}



A homomorphism from $X$ to $Y$ is a \textbf{simple folding} if all its fibers are singletons except for one which consists of a pair of vertices in $X$ at distance 2.  For example, either of the homomorphisms from the 3-path to the 2-path ($K_2$) are simple foldings.  A homomorphism is a \textbf{folding} if it can be written as a composition of simple foldings.

\begin{lemma}
	A retraction $f$ from $X$ to a proper subgraph $Y$ is a folding.  
\end{lemma}

\begin{proof}
We induct on the number of vertices in $X$.  Recall that a retraction is the identity map when restricted to $Y$, so if $X=Y$, $f$ is the empty composition of foldings.  Otherwise, there is a vertex in $y$ in $Y$ adjacent to a vertex $x$ not in $Y$.  Since $f$ fixes $y$, it must map $x$ to some neighbor of $y$ in $Y$, $z$.

Let $\pi$ be the partition of the vertices of $X$ with $\{x,z\}$ as one cell and all other vertices as singleton cells.  Then there is a homomorphism $f_1$ from $X$ to a graph $X_1$ with $\pi$ as its kernel.  Since the kernel of $f_1$ is a refinement of the kernel of $f$, there is a homomorphism $f_2$ from $X_1$ to $Y$ such that $f_2\circ f_1=f$.  Since $f_1$ is the identity on $Y$, it must be that $Y$ is a subgraph of $X_1$ and thus $f_2$ is a retraction from $X_1$ to $Y$.  Finally, we observe that $f_1$ is a simple folding, and the lemma follows inductively.
\end{proof}


A \textit{local injection} is a homomorphism for which the minimum distance between any pair of vertices in the same fiber is at least three.  Any injection, including automorphisms, are clearly local injections.


\begin{lemma}
If $h$ is a homomorphism, then $h$ can be written as the composition $f\circ g=h$ where $g$ is a folding and $f$ is a local injection.
\end{lemma}
\begin{proof}
	Let $\pi$ be the kernel of $h$ and let $u,v$ be vertices in $X$.  We write $u\approx v$ if $u$ and $v$ are in the same cell of $\pi$ and are either the same vertex or vertices at distance two in $X$.  This is symmetric and reflexive, so the transitive closure forms an equivalence relation, which induces a partition $\pi'$ on the vertices of $X$.  Thus we can define a homomorphism $g$ from $X$ to $X{/}\pi'$ with kernel $\pi'$ and a homomorphism $f$ from $X{/}\pi'$ to $Y$ such that $f\circ g=h$.
	
	The homomorphism $g$ is a folding by construction, and it remains to be shown that $f$ is a local injection.  Suppose, for the sake of contradiction, that it is not, and that there are vertices $\alpha,\beta$ in $X{/}\pi'$ at distance two such that $f(\alpha)=f(\beta)$.  Let $\gamma$ be a common neighbor of $\alpha$ and $\beta$ in $X{/}\pi'$  Then, there are vertices $u,u',v,v'$ in $X$ such that $u\in g^{-1}(\alpha)$, $v\in g^{-1}(\beta)$, $u',v'\in g^{-1}(\gamma)$, $u$ and $u'$ are adjacent, and $v$ and $v'$ are adjacent.  But $u'$ and $v'$ are the ends of  even path in $X$, so $u$ and $v$ are as well.  This implies that $u$ and $v$ lie in the same cell in $\pi'$, which is our contradiction.
\end{proof}

A homomorphism $f:X\rightarrow Y$ is a \textbf{local isomorphism} if, for each vertex $y$ in $Y$, the induced mapping from a the neighbors of a vertex in $f^{-1}(y)$ to the neighbors of $y$ is a bijection.  The map $f$ is a \textbf{covering map} if it is a surjective local isomorphism, and we sometimes say \textit{$X$ covers $Y$} in this case.  If $f$ is a local isomorphism, then each fiber is an independent set in $X$, and between any two fibers, there is either a perfect matching or no edges.  If the image of $X$ under a covering is connected, then each fiber must have the same size.  The size of the fiber is called the \textbf{index} of the map, and is usually denoted $r$, and we say that $X$ is an \textit{$r$-fold cover} of $Y$.  

There may not be a unique covering map from $X$ to $Y$, so we can define the \textbf{covering graph} $X$ of $Y$ to be a pair $(X,f)$ where $f$ is a local isomorphism from $X$ to $Y$.  If $(X,f)$ is a cover of $Y$ and $Y_1$ is an induced subgraph of $Y$, then $f^{-1}(Y_1)$ must cover $Y_1$.  This tells us that we can talk about covers of a graph by examining covers of its components. If $Y$ is connected and $(X,f)$ is a cover of $Y$, then each component of $X$ must cover $Y$.

The following is a fundamental property of covering maps.

\begin{lemma}
	If $X$ covers $Y$ and $Y$ is a tree, then $X$ is the disjoint union of a bunch of copies of $Y$.
\end{lemma}
\begin{proof}
	Let $f$ be a covering map from $X$ to $Y$.  Since $f$ is a local isomorphism, if $x$ is a vertex in $X$, then the valency of $f(x)$ in $Y$ is equal to the valency of $x$ in $X$.  Inductively, this shows that the image of a cycle in $X$ is a cycle in $Y$, and since $Y$ is a tree, $X$ must be acyclic as well.
	
	Since a local isomorphism is a local surjection, if $f(x)=y$, each edge incident to $y$ is the image under $f$ of some edge incident to $x$.  Inductively, this shows that any path starting at $y$ is the image of a path starting at $x$ (and similarly for walks).  Therefore, $X$ contains a tree $T$ such that $f$ is an isomorphism from $T$ to $Y$.  Therefore $Y$ is a retract of $X$, and since each component of $X$ covers $Y$, we have that $f$ is a folding from each component of $X$ to $Y$, so $X$ must be a set of disjoint copies of $Y$.
\end{proof}



We call a cover $(X,f)$ \textit{trivial} if $X$ is isomorphic to $r$ disjoint copies of $Y$ and the restriction of $f$ to any copy is an isomorphism.  The previous lemma can be restated as `any cover of a tree is trivial.'

There are lots of interesting covers.  Recall that the cube $Q$ has the property that each vertex $x$ has a unique vertex $x'$ at distance three, so we can naturally partition the vertex set into four pairs and let each pair be a fiber of a cover of $K_4$ to construct a 2-fold cover.

Similarly, the dodecahedron forms a 2-fold cover of the Petersen graph, and the line graph of the Petersen graph covers $K_5$.  The 42 vertices at distance two from any fixed vertex in the Hoffman-Singleton graph define a 6-fold cover of $K_7$.  For any graph $X$, there is a natural 2-fold cover $X\times K_2$ of $X$.

IF $(X,f)$ and $(Y,g)$ are covers of $F$, them so is their \textit{subdirect product}.




\section*{Cores with no Triangles}

We saw earlier that every graph has a core, but it's not easy to find good examples of cores.  A \textbf{critical graph} is one where deletion of any vertex decreases its chromatic number, and these are a class of graphs where it's straightforward to describe the core. The kinds of critical graphs we know about are the complete graphs and the odd cycles.  There are a lot of more interesting ones, but we're not going to talk about them because we don't have anything algebraic to say about them.

Since any homomorphism must map triangles to triangles, it should be easy to construct cores with lots of triangles.  Here, we try to construct cores \textit{without} triangles.

We start with a simple sufficient condition for a graph to be a core:

\begin{lemma}
	If $X$ is a connected, non-bipartite graph and every 2-arc in $X$ lies in some shortest odd cycle, then $X$ is a core.
\end{lemma}
\begin{proof}
	Let $f$ be a homomorphism from $X$ onto itself.  This will necessarily send a shortest odd cycle in $X$ to  one of the same length, so any two vertices on that cycle will be mapped to different vertices under $f$.  Since we assume that every 2-arc lies on a shortest odd cycle, we have that $f$ is a local injection and it is not a folding, so by a lemma in the previous section, $f$ cannot map $X$ to a proper subgraph of itself.
\end{proof}

If two vertices $u,v$ in $X$ have identical neighborhoods, then $X$ isn't a core, since there is a retraction from $X$ to $X{\setminus}u$.  Thus we can define a \textbf{reduced graph} to be one with no isolated vertices and all pairs of vertices have distinct neighborhoods.

Supposing that $X$ is triangle-free, then if $u,v$ are vertices at distance at least three, the graph obtained by adding an edge between $u$ and $v$ is also triangle-free.  Inductively, we can see that any triangle-free graph $X$ is a spanning subgraph of a triangle-free graph with diameter two.

These graphs have a useful property:

\begin{lemma}
	Let $X$ be reduced, triangle-free, and of diameter two.  For any pair of distinct vertices $u$ and $v$, there is a vertex adjacent to $u$ but not $v$.
\end{lemma}
\begin{proof}
	Suppose, for the sake of contradiction, that every neighbor of $u$ is a neighbor of $v$.  Since $X$ is reduced, there is a vertex $w$ adjacent to $v$ but not $u$, so the containment is proper.  Since $X$ has no triangles, $w$ can't be adjacent to any neighbor of $u$, which means the distance between $u$ and $w$ is at least three, contradicting the assumption on the diameter of $X$.
\end{proof}


We can now characterize a class of cores:

\begin{lemma}
	Let $X$ be triangle-free and of diameter two.  Then $X$ is a core if and only if it is reduced.
\end{lemma}
\begin{proof}
	We have seen already that if a graph is not reduced then it is not a core.  We now assume that $X$ is reduced and show that each 2-arc lies in a 5-cycle, which demonstrates that it is a core by the earlier lemma.
	
	Let $(u,v,w)$ be a 2-arc, so $w$ is at distance two from $u$.  Since the neighborhood of $w$ is not contained in the neighborhood of $u$, there are neighbors $v',w'$ of $w$ such that $v'$ is a neighbor of $u$ and $w'$ is not (and therefore at distance two).  Since $X$ is triangle-free, we know that $v\neq v'$ so there is a 5-cycle $(u,v,w,w',v')$.
\end{proof}

This also shows that any reduced triangle-free graph of diameter two must be a core.  The graph obtained by deleting a vertex from the Petersen graph is triangle-free with diameter three, but any homomorphic image of it must contain a triangle, so the condition in the lemma is not necessary.

\section*{Coloring Andr\'asfai Graphs}

The \textbf{Andr\'asfai graphs}, denoted $\mathrm{And}(k)$, are a family of Cayley graphs which are reduced, triangle-free, and of diameter two.  For any integer $k\geq 1$, take $G$ to be the group $\mathbb{Z}_{3k-1}$ and let $C=\{g\in G \mid g\mod 3 \equiv 1\}$ be the generating set.  Then the graph $X(G,C)$ is $\mathrm{And}(k)$.  $\mathrm{And}(2)$ is isomorphic to the 5-cycle, $\mathrm{And}(3)$ is called the \textit{M\"obius ladder}, and $\mathrm{And}(4)$ is pictured:

 \textbf{[figure here]}

\begin{lemma}
	For $k\geq 2$, $\mathrm{And}(k)$ is reduced, triangle-free, and of diameter two.
\end{lemma}

\begin{proof}
	We start by showing that $\mathrm{And}(k)$ has diameter two, which is a simple arithmetic exercise.  Given any $0\leq m < 3k-1$, there is a path of length at most two from zero to $m$, since any such number is either congruent to $3k-1$ modulo 3 and is therefore at distance one (or zero), or congruent to $3k$ or $3k-2$, which are both at distance one from $3k-1$.  The result follows by symmetry.
	
	Next, we show that $\mathrm{And}(k)$ is reduced.  If $\mathrm{And}(k)$ has two distinct vertices with the same neighbors, then there is some $g\in G$ such that $g\neq 0$ and $g+C=C$, but then $g+1$ and $g-1$ must both be in $C$, which is not possible by the construction of the generating set.
	
	Finally, we show that $\mathrm{And}(k)$ is triangle-free by arguing that there is no triangle containing 0.  Take $g$ and $h$ to be two neighbors of 0.  Then $g,h\in C$, so $g-h$ must be congruent to 0 modulo 3, so $g$ and $h$ cannot be adjacent, and again by symmetry the whole graph must be triangle-free.
	
\end{proof}


If $X$ is reduced, triangle-free, and of diameter two and $S$ is an independent set in $X$ which is maximal under the inclusion partial-ordering.  Then, every vertex in $X$ is adjacent to at least one vertex in $S$, and the graph created by making a new vertex adjacent to every vertex in $S$ is also triangle-free and of diameter two.  If $S$ is not the entire neighborhood of some vertex, then this new graph is also reduced.  This gives us a procedure for inductively constructing such graphs.  We can make the following observation about $\mathrm{And}(k)$ which shows that this procedure cannot be performed on this class of graphs.

\begin{lemma}
	{Each independent set of vertices in $\mathrm{And}(k)$ is contained in the neighborhood of some vertex.}
\end{lemma}

\begin{proof}
	Take the $k-1$ pairs of adjacent vertices of the form $\{3i,3i-1\}$ where $1\leq i <k$.  Any vertex $g\in C$ is of the form $3q+1$ for some $q$.  If $q\geq i$, then $g$ is adjacent to $3i$ and not $3i-1$.  If $q<i$, then $g$ is adjacent to $3i-1$ and not $3i$.  Thus each vertex in $C$ is adjacent to exactly one vertex in each pair.
	
	Now, suppose for the sake of contradiction that there is an independent set in $\mathrm{And}(k)$ which is not contained in the neighborhood of some vertex.  Then we can take an independent set $S$ and a vertex $x$ such that $S\cup x$ is also independent, $S$ is contained in the neighborhood of a vertex, and $S\cup x$ is not.  By transitivity, we can assume without loss that $S$ is in the neighborhood of 0.  Then $x$ is not, so $x$ must be $3i$ or $3i-1$ for some $i$.  If $x$ is $3i$ then, by independence, every vertex of $S$ is not adjacent to $x$, and so is adjacent to $x-1$, which means that $S\cup x$ is in the neighborhood of $x-1$, which is a contradiction.  A symmetric argument holds for $x=3i-1$ making $S\cup x$ a subset of the neighborhood of $x+1$.
\end{proof}


\section*{Coloring Andr\'asfai Graphs}


We can provide a characterization of Andr\'asfai graphs using some observations about their colorings.  

\begin{lemma}
	If $k\geq 2$, then the number of 3-colorings of $\mathrm{And}(k)$ is $6(3k-1)$ and they are all equivalent under the automorphism group of the graph.
\end{lemma}

\begin{proof}
	In any coloring, there are an average of $k-\tfrac{1}{3}$ vertices in each color class.  Since the maximum size of a class is $k$, it must be that two classes are of size $k$ and one is of size $k-1$.  By the previous lemma, the two larger color classes must be neighborhoods of vertices, since a color class is an independent set, and these are maximal.
	
	Suppose we have a 3-coloring of $\mathrm{And}(k)$ and one of the big color classes is the neighborhood of 0.  This can be any of the three colors, and we make this assumption without loss.  Next, take the set of all vertices not adjacent to 0.  We can partition this into two sets: $A=\{2,5,\dots 3k-4  \}$ and $B=\{ 3,6,\dots, 3k-3 \}$.  These two sets are independent, and the $i$th vertex in $A$ is adjacent to the last $k-i$ vertices in $B$.  Thus $A\cup B$ is connected and bipartite, so it can be colored with two colors in exactly two ways.  We now have two choices of colors to assign to 0, so when the neighborhood of 0 is a big color class, there are 12 distinct colorings thus far -- corresponding to a choice of three colors for the neighborhood of 0, followed by two choices for the induced bipartite graph, followed by two choices for 0 itself.
	
	Since $\mathrm{And}(k)$ is transitive with $3k-1$ vertices, and each 3-coloring has two big color classes, we have the equivalence under transitivity of all 3-colorings of $\mathrm{And}(k)$.
	
	To finish off the combinatorial claim, observe that the permutation which exchanges $i$ with $-i$ modulo $3k-1$ is an automorphism of $\mathrm{And}(k)$ which exchanges the sets $A$ and $B$, so there are $(3k-1)/2$ additional choices of colorings.
\end{proof}

We note an additional property of $\mathrm{And}(k)$, that the subgraph induced by $\{ 0,1,2,\dots, 3(k-1)-2  \}$ is exactly $\mathrm{And}(k-1)$, meaning we can turn $\mathrm{And}(k)$ into $\mathrm{And}(k-1)$ by deleting the path $(3k-4,3k-2,3k-2)$.

\begin{lemma}
	Let $X$ be regular of valency $k\geq 2$ and triangle-free, and suppose $P$ is a path of length two in $X$.  If $X{\setminus}P \iso\mathrm{And}(k-1)$, then $X\iso \mathrm{And}(k)$.
\end{lemma}

\begin{proof}
	Let $P$ be the path $(u,v,w)$ and let $Y$ denote $X{\setminus}P$.  Since $X$ is triangle-free and regular, the neighbors of $u$, $v$, and $w$ in $Y$ form independent sets of size $k-1$, $k-2$, and $k-1$, respectively.  Since $Y$ is regular of valency $k-1$, each vertex of $Y$ is adjacent to exactly one vertex in $P$.  Therefore, these independent sets form a 3-coloring of $Y$, and since all 3-colorings are equivalent under automorphisms of $Y$, the claim follows.
\end{proof}


\section*{A Characterization}

It turns out that any reduced triangle-free graph where every independent set lies in the neighborhood of a vertex is one of the Cayley graphs $\mathrm{And}(k)$, which is cool because it is a simple combinatorial condition which yields a large amount of symmetry,

\begin{theorem}
	If $X$ is reduced, triangle-free, and has the property that every independent set is contained in the neighborhood of a vertex, then $X$ is isomomorphic to $\mathrm{And}(k)$ for the correct choice of $k$.
\end{theorem}
\begin{proof}
	We proceed in steps:
	
	\textbf{Step 1:} If $u$ and $v$ are distinct and non-adjacent, then there must be a unique vertex $\sigma_u(v)$ adjacent to $v$ but not $u$ such that 
\end{proof}
\section*{Cores of Vertex-Transitive Graphs}

\section*{Cores of Cubic Vertex-Transitive Graphs}

