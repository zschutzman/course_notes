
\chno{13}{Wrapping up $NP$ and Beginning Space Complexity}{Sampath Kannan}{Zach Schutzman}
%\footnotetext{These notes are partially based on those of Nigel Mansell.}

% **** YOUR NOTES GO HERE:

% Some general latex examples and examples making use of the
% macros follow.  
%**** IN GENERAL, BE BRIEF. LONG SCRIBE NOTES, NO MATTER HOW WELL WRITTEN,
%**** ARE NEVER READ BY ANYBODY.


\section*{More $NP\mhyphen Complete$ness}

Other problems are $NP\mhyphen Complete$, the reduction for $3COLOR$ creates gadgets from each clause and each variable (see any Algorithms book).

There is an obvious reduction from $3COLOR$ to $4COLOR$ (and beyond).  Add a new vertex, connect it to every vertex in your original graph.

$SUBSETSUM$, the question of whether a set of numbers contains a subset that sums to a particular value $k$ is $NP\mhyphen Complete$.  The reduction is from $3SAT$.  Given a formula $\phi$ with $m$ clauses and $n$ variables, we create $2n$ numbers with $n+m$ base $7$ digits.  Each variable gets a $1$ in positions indicating the index of the variable and the clauses it appears in.  $\phi$ is satisfiable if and only if there is a subset that sums to $11\dots111$ in the first $n$ positions, and $4\dots44$ in the remaining $m$ positions, by introducing some dummy numbers which are all zeros in the first $n$ positions and $1\ or \ 2$ in the corresponding clause position.


We know that $P\subseteq NP$, $NP\mhyphen Complete \subset NP$

\definition{A language $L$ is in $Co\mhyphen NP$ if its complement is in $NP$.  Equivalently, given $x\notin L$, a verifier can check non-membership given the right certificate (easy to check that something is not a solution).}

\textbf{Example:} $TAUT = \{ \phi | \phi \ is  \ satistfied \ by \ every \ assignment\}$ is in $Co\mhyphen NP$.  If a $\phi$ is not in $L$, then any non-satisfying assignment can be quickly checked.


We don't know if $NP=Co\mhyphen NP$.

If $P=NP$, then $NP = Co\mhyphen NP$.

If $NP\neq Co\mhyphen NP$, then $P\neq NP$.


\section*{Space Complexity}

\definition{\textbf{Space complexity} refers to the number of cells of tape scanned by the head of a Turing machine in running an input.}

\definition{$DSPACE(s(n))$ is the set of languages recongized by a deterministic TM in $O(s(n))$ space.}

\definition{$NSPACE(s(n))$ is the set of languages recognized by a non-deterministic TM in $O(s(n))$ space.}

Let's look at $SAT\in NP$.  What is its space complexity?  Let's say $n$ is the length of the input and $k$ is the number of variables.  If we don't care about time, we can represent the assignment as a $k$-bit number and try one assignment at a time.  We only need $n+k+c$ (where $c$ is some small fixed workspace).  Therefore, $SAT\in DSPACE(n)$.

Let $L_{NANFA} = \{ \langle M \rangle | M \ is \ an \ NFA, \ L(M)\neq \Sigma^* \}$.  We don't know if this is in $NP$ because the naive approach for a verifier does not necessarily have a polynomial length certificate.  We can show that it is in $NSPACE(n)$.









