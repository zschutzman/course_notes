
	%FILL IN THE RIGHT INFO.
	%\lecture{**LECTURE-NUMBER**}{**DATE**}{**LECTURER**}{**SCRIBE**}
	\chno{15}{PSPACE Completeness}{Sampath Kannan}{Zach Schutzman}



\section*{The Class $PSPACE\mhyphen Complete$}

\definition{A language $L$ is $PSPACE\mhyphen Complete$ if $L$ is in $PSPACE$ and for all $A\in PSPACE$, there exists a mapping reduction from $A$ to $L$ in polynomial time.}

\definition{A \textbf{quantified boolean formula} is a formula of the form $\forall x_1 \exists x_2 \forall x_3 \dots \phi(x_1,x_2,\dots)$, where $\phi$ is a boolean formula over the $x_i$.}

\claim{The language of True Quantified Boolean Formulas ($TQBF$) is $PSPACE\mhyphen Complete$.}

\begin{proof}
	
	We'll think of a machine that decides this language as taking a quantified boolean formula as input and accepting if it is true, rejecting if it is false.
	
	This language is obviously in $PSPACE$, we can simply check each allowable assignment and accept if the formula is true.
	
	To show that it is $PSPACE$-Complete, we need to show that every other $PSPACE$ language can be reduced to it in polynomial time.  Let $L$ be a language in $PSPACE$.  Without loss of generality, we can say that a decider for $L$ takes $O(2^{n^k})$ time. 
	
	We saw in the Cook-Levin Theorem how to think of states of computation as boolean formulas, and we will do something similar here.  Let $C_{init}$ the initial configuration and $C_{accept}$ as the canonical accepting configuration.  We write $\Phi(C_a,C_b,t)$ to denote the quantified boolean formula corresponding to `there exists a sequence of computation steps of length at most $t$ such that the machine moves from configuration $C_a$ to $C_b$.  We want to find formula equivalent to $\Phi(C_{init},C_{accept},2^{n^k})$ which is true if and only if the input $x$ to the decider for $L$ is in $L$.
	
	We start by observing that if $x\in L$, then there must be some intermediate configuration $C_m$ which the machine passes through in its computation.  We can then write \[\Phi(C_{init},C_{accept},2^{n^k}) \iff \Phi(C_{init},C_{m},\frac{2^{n^k}}{2})\land \Phi(C_{m},C_{accept},\frac{2^{n^k}}{2})\]
	
	We can also consider quarter-way and eighth-way points, and so on and do the same thing, however, we get an exponentially large formula by the time we get down to step sizes of $t=1$.
	
	We get around this by introducing two new variables $C_3,C_4$ and universally quantifying them over our configuration pairs $(C_1,m),(m,C_2)$ such that we have 

	\[ \Phi(C_1,C_2,t)=\exists m \forall(C_3,C_4)\in \{ (C_1,m),(m,C_2) \} (\Phi(C_3,C_4,\frac{t}{2})) \]
	
	We only need polynomial space for this, hence $L$ is reducible to $TQBF$, and $TQBF$ is $PSPACE$-Complete.
\end{proof}



\textbf{Example:} The language Generalized Geography ($GG$) is $PSPACE$-Complete.  $GG$ is a two player game on a directed graph.  One player picks a node, then the other picks a node for which there exists an arc from the previous player's choice into that node.  A player loses if there are no nodes for her to legally choose (i.e. the previous player picked a node with out degree zero).  Player One's first choice is a designated start node.  The language $GG$ is the decision problem of whether, given a directed graph with a start node, Player One can always win given perfect play.

We can think of this as a $TQBF$ in the following way:  Player One has a winning strategy if there exists a node with out-degree zero such that for any previous choice made by Player Two there exists a previous choice by Player One, and so on.


