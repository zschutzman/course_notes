
%\lecture{**LECTURE-NUMBER**}{**DATE**}{**LECTURER**}{**SCRIBE**}
\chno{16}{Space Complexity, Continued}{Sampath Kannan}{Zach Schutzman}
%\footnotetext{These notes are partially based on those of Nigel Mansell.}

% **** YOUR NOTES GO HERE:

% Some general latex examples and examples making use of the
% macros follow.  
%**** IN GENERAL, BE BRIEF. LONG SCRIBE NOTES, NO MATTER HOW WELL WRITTEN,
%**** ARE NEVER READ BY ANYBODY.
\section*{Logarithmic Space}

Recall, $L$ and $NL$ are the classes of problems recognizable in logarithmic space by a deterministic and non-deterministic Turing machine, respectively.  These machines are modeled as having a read-only input tape and a read/write work tape.  The space complexity is determined by the number of cells used on the work tape.

The language $PATH = \{\langle G,s,t\rangle | G \ is \ a \ digraph \ with \ a \ path \ from \ s \ to \ t\}$.
	
\clm{$PATH$ is in $NL$}.

\begin{proof}
	Nondeterministically guess a sequence of neighboring vertices, for at most $n$ steps. This needs only $O(log(n))$ space to track the current vertex and a counter.
\end{proof}


It turns out, $PATH$ is one of the `hardest' languages in $NL$, that is, it is $NL\mhyphen Complete$.





	 We need to show that $PATH$ is in $NL$ (done) and that any $NL$ language can be reduced to it.  What is our notion of reduction?  Before, we had polynomial time reductions, but $NL\subset P$, so a polynomial time (and hence polynomial space) could be powerful enough to solve our language.  We therefore restrict our reductions to the class $L$, that is, a log-space reduction.
	 
	 
	 \defn{A \textbf{log-space transducer} is a machine with a read-only input, a read-write work tape, and a write-only output tape.  The work tape has size $O(log(n))$ and the write tape can have polynomial length in $n$.}
	 
	 \defn{A language $A$ \textbf{log-space reduces} to a language $B$ (written $A\preceq_m^L B$) if there is a log-space transducer computing a function $f$ such that $x\in A \iff f(x)\in B$.}
	 
	\clm{$PATH$ is $NL\mhyphen Complete$}. 
\begin{proof}	 
	
	Let $A$ be any language in $NL$.  We show that there exists a log-space transducer reducing $A$ to $PATH$.
	
	We have some input $w\in\Sigma^*$ to $A$ and we want to construct a log-space transducer to convert it into a $\langle G,s,t\rangle$.  The high level idea is to construct a graph with vertices corresponding to configurations of a machine recognizing $A$, $s$ and $t$ corresponding with initial and accept configurations, and edges for configurations which are one computation step apart.
	
	A configuration of the $A$ machine is specified by the location of the head in the input tape, the current state, and the contents of the work tape.  This requires only $O(log(n))$ bits to specify, as we need $O(log(n))$ to keep track of the position in the input, plus $O(log(n)))$ to represent the work tape, plus some constant to track the current state.  The number of possible configurations is $2^{O(log(n))}$, which is polynomial in $n$.
	
	The graph $G$ has one node for each possible configuration, the node $s$ will be the initial configuration, and the node $t$ will be the accept configuration in canonical form.  There will be an edge $(c_i,c_j)$ if the machine for $A$ can go from configuration $c_i$ to $C_j$ in one step.  There may be multiple edges from each node, corresponding to nondeterministic choice.  There exists an $s\mhyphen t$ path in this graph if and only if $w\in A$.
	
	We now only need to show that a log-space transducer can compute this reduction.  The transducer will first construct the vertices of $G$, then the edges.
	
	First, the transducer knows the length of each configuration, so it uses its $O(log(n))$ space to iterate through the possible configurations and if it's a valid configuration $c$, writes $c$ to its output tape.  After this, all of the nodes in the graph are written to the output.  Then, for the edges, check each pair of configurations to determine if the second is reachable from the first in 1 step.
	
	This procedure only requires $O(log(n))$ work space and produces the appropriate $G, s,t$, hence is is a log-space reduction and the proof that $PATH$ is $NL\mhyphen Complete$.
	
	
	 
\end{proof}


Note that any function computable by a log-space transducer can be computed in polynomial time, a transducer can never repeat a configuration (else it would not be a decider), and there are only polynomially many configurations.

\clm{$NL\subseteq P$}

\begin{proof}
	We know that $PATH \in P$ (we can do breadth-first search). Therefore, given any $A\in NL$ and input $w$, we can, in polynomial time, log-space reduce $A$ to an instance of $PATH$ with input $f(w)$ and use a polynomial time algorithm for $PATH$.  This gives a polynomial time algorithm for $A$.
\end{proof}
	



\section*{The Class $Co\mhyphen NL$}

\defn{A language $A$ is in $Co\mhyphen NL$ if its complement is in $NL$.}  

From before, a language $B$ is in $NL$ if there is a nondeterministic log-space Turing machine $M$ such that given $w$ in $B$, $M(w)$ accepts on some path.  If $w\notin B$, all computation paths reject.  $Co\mhyphen NL$ is the opposite.  If $w\in C\in Co\mhyphen NL$, then all paths accept.  If $w\notin C$, there exists at least one rejecting path.


\clm{$NL = Co\mhyphen NL$}

\begin{proof}
	
	We have $PATH$ as an $NL\mhyphen Complete$.  Let's define the language $\overline{PATH} = \{ \langle G,s,t\rangle | G \ is \ a \ digraph \ with \ no \ s\mhyphen t \ path    \}$.  The same reduction as before shows that $\overline{PATH}$ is $Co\mhyphen NL\mhyphen Complete$.  We prove this theorem by showing $\overline{PATH}\in NL$.  This implies $NL\subseteq Co\mhyphen NL$, which by symmetry of complements gives us $NL = Co\mhyphen NL$.
	
	We want a nondeterministic log-space Turing machine $M$ which, given $\langle G,s,t \rangle$ and a number $c$ equal to the number of nodes reachable from $s$, accepts if there is no path from $s$ to $t$ in $G$.  This machine will guess whether there is or is not a path from $s$ to $v_1$.  If there is one, using a log-space $PATH$ subroutine, continue guessing paths from $s$ to $v_2$.  If every path in this side rejects, we know there is no path from $s$ to $v_1$, so we continue down the $no$ branch.  When we find a path from $s$ to a vertex, we increment a counter.  We continue checking for each vertex except $t$. If there is no $s\mhyphen t$ path, the counter will reach $c$ before we finish exhausting vertices.
	
	We now need to think about how to get $c$.  Let $c_i$ be the number of vertices reachable from $s$ by paths of length at most $i$.  We can, given $c_i$, compute $c_{i+1}$.  We have $c_0=1$, so if we have a procedure to compute the increments, we can inductively find $c_{n-1}$.  The idea will be to note that vertices reachable in at most $i+1$ steps have an edge from something reachable in at most $i$ steps.  
	
	
	
\end{proof}



