%\lecture{**LECTURE-NUMBER**}{**DATE**}{**LECTURER**}{**SCRIBE**}
\chno{8}{More on Reducibility}{Sampath Kannan}{Zach Schutzman}
%\footnotetext{These notes are partially based on those of Nigel Mansell.}

% **** YOUR NOTES GO HERE:

% Some general latex examples and examples making use of the
% macros follow.  
%**** IN GENERAL, BE BRIEF. LONG SCRIBE NOTES, NO MATTER HOW WELL WRITTEN,
%**** ARE NEVER READ BY ANYBODY.


\section*{Mapping Reducibility}

Consider languages $A\subseteq\Sigma^*$, and $B\subseteq\Sigma^*$.  

Recall the idea of a reduction is $A$ reduces to $B$ means that we can use $B$ to solve $A$.  If $A$ is hard, then $B$ must be (at least as) hard.  If $B$ is easy, then $A$ must be easier.





\definition{A Turing machine \textbf{computes} a function $f:\Sigma^*\rightarrow\Sigma^*$ if on all inputs $x$ it halts with exactly $f(x)$ on its output tape.}

\definition{A function $f$ is \textbf{Turing computable} if there exists a Turing machine which computes it.}


\definition{A \textbf{mapping reduction} is a Turing computable function $f:\Sigma^*\xrightarrow{TC}\Sigma^*$ such that if $x\in A$, then $f(x)\in B$.  Similarly, if $y\notin A$, then $f(y) \notin B$.}

Recall last time we constructed a Turing reduction from $A_{TM} = \{\langle M,w\rangle | M \ accepts \ w\}$ to $E_{TM} \{\langle M \rangle|L(M)=\emptyset\}$.  Note that this was not a mapping reduction as we had to take the complement of the solver for $E_{TM}$ at the end to properly complete the reduction.


\claim{If we have a mapping reduction from $A$ to $B$ and $A$ is not Turing recognizable, then $B$ is not Turing recognizable.}

\begin{proof}
	If $B$ is Turing recognizable, let $B=L(M)$.  Then we construct a recognizer for $A$ by using the mapping function on the input for $A$ and running $M$ on it.
	
\end{proof}




\textbf{Example:} let $L_{AE} = \{\langle M \rangle | \epsilon\in L(M)\}$.  We want to find a mapping reduction from $A_{TM}$ to $L_{AE}$.  

\begin{proof} 
	A typical input to $A_{TM}$ is of the form $\langle M,w\rangle$.  Our function $f$ should map $\langle M,w\rangle$ to $\langle M' \rangle$.  $M'$, on any input, prints $w$ on its input tape and runs $M$.

We can see that $M'$ either accepts everything or nothing, but it accepts if and only if $\langle M,w\rangle\in A_{TM}$.  Because $A_{TM}$ is undecidable, we must have $L_{AE}$ undecidable as well.
\end{proof}
\claim{If $A \preceq_m B$, then $\bar{A} \preceq_m \bar{B}$.}


\begin{proof}
	
	We can use the same reduction as the original for the complement, because the function $f$ preserves membership in sets.
	
	
\end{proof}

\textbf{Example:} let $EQ_{TM}=\{\langle M_1,M_2\rangle|L(M_1)=L(M_2)  \}$.  We will show this is unrecognizable by a mapping reduction.


\begin{proof}
We will reduce from $\overline{A_{TM}}$.  $\overline{A_{TM}}$ takes input of the form $\langle M,w\rangle$.  We want to make a construction that builds two equivalent TMs if $M$ accepts $w$ and two inequivalent TMs if $M$ does not accept $w$.

Let $L(M_1) = \emptyset$ and $M_2$ be a machine that on any input, runs $M$ on $w$.  This machine either accepts every string or no strings, depending on whether $M$ accepts $w$.  

$M_1\sim M_2$ if and only if $M$ rejects $w$, which is exactly what we wanted to show.  Therefore, $EQ_{TM}$ is unrecognizable.
\end{proof}

\textbf{Example:} let's also prove that $\overline{EQ_{TM}}$ is also not recognizable, via mapping reduction.  

\begin{proof}
	
If we can show a mapping reduction from $A_{TM}$ to $EQ_{TM}$ then we know that $\overline{A_{TM}}$ has a mapping reduction to $\overline{EQ_{TM}}$, and $EQ_{TM}$ is therefore unrecognizable.

Let's make $L(M_1) = \Sigma^*$ and $M_2$ works exactly as before.  The same process shows that $M_1\sim M_2$ if and only if $M$ accepts $w$.  Therefore, by examining the complements of these languages, we have a reduction from $\overline{A_{TM}}$ to $\overline{EQ_{TM}}$, so neither $EQ_{TM}$ nor its complement are recognizable.
\end{proof}

\definition{A \textbf{property} of Turing machines is a function from TM descriptions to $\{0,1\}$}



\definition{A property of Turing machines is a \textbf{language property} if only depends on the language recognized by the TM and not on the description of the TM itself.  That is, the property is true for any machine $M$ recognizing a language $L$.}



Let $L_P = \{ \langle M \rangle | M \ has\ property \ P \}$

\definition{A property is \textbf{non-trivial} if there is some Turing machine that has the property and some that does not.  That is, $L_P$ is not the set of all TMs nor is it empty.}

\claim{(Rice's Theorem) $L_P$ for any non-trivial language property of Turing machines is undecidable.}

\begin{proof}
	
	Let $P$ be a non-trivial language property and $L_P$ be the language of the property.  We will show, by constructing a mapping reduction, that $L_P$ is undecidable.
	
	Consider $A_{TM}$.  Assume that a Turing machine that accepts the empty language does not have property $P$ (this is without loss of generality, because we can always consider $\overline{P}$ instead).  On input $\langle M,w\rangle$, create a machine $M'$ which, on any input $x$, first runs $M$ on $w$.  If $M$ rejects, then $M'$ rejects.  Otherwise, then, runs some Turing machine $T$ with property $P$ on input $x$.
	
	If $M$ does not accept $w$, it may be because it rejects, or runs forever.  In both cases, $M'$ rejects.  In the first, it explicitly does so.  In the second, it never gets to the second step, and therefore rejects.  So $L(M') = \emptyset$, which does not have property $P$.
	
	If $M$ accepts $w$, then $L(M') = L(T)$.  $T$ has property $P$, but since $P$ is a language property, $M'$ has the property as well.  Therefore, the reduction results in $M'$ having property $P$ if and only if $M$ accepts $w$.  Since $A_{TM}$ is undecidable, $L_P$ must be as well.
	
	
	
\end{proof}








