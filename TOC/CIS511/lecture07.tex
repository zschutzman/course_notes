\chno{7}{Undecidability, Continued}{Sampath Kannan}{Zach Schutzman}
%\footnotetext{These notes are partially based on those of Nigel Mansell.}

% **** YOUR NOTES GO HERE:

% Some general latex examples and examples making use of the
% macros follow.  
%**** IN GENERAL, BE BRIEF. LONG SCRIBE NOTES, NO MATTER HOW WELL WRITTEN,
%**** ARE NEVER READ BY ANYBODY.

\section*{An Undecidable Language}

Recall, we showed last time that, by a cardinality argument, that there must exist some language that is not Turing decidable, or even recognizable.

\textbf{Example:}  Let the language $A_{TM} = \{ \langle M,w\rangle | M(w) \ accepts  \}$, the language of machines and words such that $M$ reaches its $q_a$ on input $w$.

\clm{$A_{TM}$ is not decidable.}

\begin{proof}
	Assume for the sake of contradiction that $A_{TM}$ is decidable, and let $H$ be the TM that decides it.  We will show that $H$ cannot exist.
	
	How does $H$ work?  $H$ takes a pair $\langle M,w\rangle$ and accepts if $M$ accepts $w$ and rejects if $M$ does not accept (reject or loop forever).  Note that because of this `loop forever' case, we can't just simulate $M$ on $w$.
	
	Create a new TM $D$ that takes as input the description of a Turing machine $\langle M \rangle$.  $D$ calls $H$ on $\langle M,\langle M \rangle \rangle$.  Since $H$ decides a language, it always terminates.  Define the behavior of $D$ to return the complement of the answer that $H$ provides.  That is, if $H$ accepts on $\langle M,\langle M \rangle \rangle$, then $D$ rejects.  I.e., if $H$ says that Machine \#17 accepts on the input `17', then $D$ rejects.
	
	Now, how should $D$ behave when given input $\langle D\rangle$?  First, $D$ calls $H$ on $\langle D,\langle D \rangle \rangle$.  Next, $H$ decides if $D$ accepts $D$. In the former case, it accepts, the latter, rejects.  So, if $H$ says that $D$ accepts $\langle D \rangle$, then $D$ rejects.  Conversely, if $H$ says that $D$ does not accept $\langle D \rangle$, then $D$ accepts.  Both of these cases are paradoxical, hence no such decider $H$ for $A_{TM}$ exists, so $A_{TM}$ is not a decidable language.
\end{proof}

This proof is equivalent to Cantor's diagonal argument for uncountablility.  Consider the rows indexed by Turing machines and the columns by string descriptions of Turing machines.  The table values are either `accept' or `reject' when $H$ receives input $\langle r,\langle c \rangle \rangle$.  $D$ then works by going down the diagonal and taking the complement of the entry at that point.  If we think about `accept' and `reject' as being `1' and `0', we have exactly the same case as Cantor's diagonal argument that $\mathbb{R}$ is uncountable.

What about an unrecognizable language?  

\section*{An Unrecognizable Language}

\clm{A language $L$ is decidable if and only if $L$ and $L^C$ are both recognizable.}

\begin{proof}
	
	If $L$ is decidable, then clearly it is recognizable.  Additionally, $L^C$ must be decidable, as it is decided by a machine that simulates a decider for $L$ and outputs the complement of the result.
	\corly{The class of decidable languages is closed under complement.}
	Hence, $L^C$ is recognizable.
	
	Now, assume that $L$ and $L^C$ are both recognizable.  We will construct a decider for $L$.  Let $M$ and $M_C$ be recognizers for $L$ and $L^C$, respectively.  Let $M_D$ be a TM that works by alternating simulating $M$ and $M_C$ one step at a time.  Because $L$ and $L^C$ are both recognizable, if a string $x$ is in $L$, then $M$ halts and accepts on $x$.  If $x$ is not in $L$, then $M_C$ halts and accepts on $x$.  Since $M_D$ alternates steps of $M$ and $M_C$, it halts in a finite amount of time, and outputs an answer corresponding to the submachine that accepted.  $M_D$ accepts if $M$ accepts or $M_C$ rejects.  $M_D$ rejects when $M$ rejects or $M_C$ accepts.
	
	
	
	
\end{proof}


Is $A_{TM}$ recognizable?  Yes!  The simple approach of simulating $M$ on $w$ will halt if $M$ accepts.  The issue came from us wanting a machine to halt in the case that $w$ is not in $L(M)$.

Let's define $\overline{A_{TM}}$ as the language where $M$ does not accept $w$.  Is this language Turing recognizable?

\clm{$\overline{A_{TM}}$ is not recognizable.}

\begin{proof}
	
	If $\overline{A_{TM}}$ is recognizable, then together with the proof that $A_{TM}$ is recognizable, we know that $A_{TM}$ would have to be decidable.  We know this is false, so $\overline{A_{TM}}$ is not recognizable.
\end{proof}


We found a language that is not recognizable!


\section*{Reductions}

The idea of reductions is ubiquitous in mathematics and computer science.  Fundamentally, it is the idea of solving a new problem by converting it to one you already know how to solve.

Suppose we have two languages $A$ and $B$.

\defn{A reduction from $A$ to $B$ is a method for using a procedure for solving/deciding/recognizing $B$ to solve/decide/recognize $A$.}

We can think about a subroutine that can solve $B$ being used to solve $A$.

The existence of a reduction from $A$ to $B$ in a sense means that $A$ is no harder than $B$ ($B$ is at least as hard as $A$).

If we can reduce an undecidable language $A$ to some language $B$, then $B$ must be undecidable.  We will denote ``$A$ reduces to $B$'' as $A\preceq B$.


A reduction from $A$ to $B$: given a TM $S$ for solving $B$, construct a TM $R$ that uses $S$ as a subroutine to solve $A$.

\textbf{Example:}  Define $HALT_{TM} = \{ \langle M,w \rangle \ | M \ halts \ on \ w  \}$.  This is the set of pairs $\langle M,w\rangle$ such that $M$ halts on $w$.

\clm{$H_{TM}$ is undecidable}

\begin{proof}
	We prove this by reduction.  We will show that, if we assume $HALT_{TM}$ is decidable, we can use it to decide $HALT_{TM}$.  This is a reduction from $A_{TM}$ to $HALT_{TM}$.
	
	Suppose $S$ is a TM that decides $HALT_{TM}$. $S$ takes as input $\langle M,w \rangle$ and accepts if $M$ halts on $w$, rejects if $M$ loops forever.  Then a TM for $A_{TM}$ works by taking input $\langle M,w \rangle$, asks $S$ whether $M$ halts on $w$, if $S$ rejects, then reject, otherwise, simulate $M$ on $w$ and output the result.
	
	This decides $A_{TM}$, which is a contradiction.  Hence such an $S$ cannot exist and $HALT_{TM}$ is not decidable.
	
	
	
\end{proof}


\textbf{Example:}  Define $E_{TM} = \{ \langle M \rangle  \ | M \ recognizes \ \emptyset\}$ be the language of TMs that accept no strings.

\clm{$E_{TM}$ is undecidable}

\begin{proof}
	We proceed again by reduction from $A_{TM}$ to $E_{TM}$.
	
	Let $S$ be a decider for $E_{TM}$.  $S$ takes as input $\langle M \langle$ and accepts if $M$ accepts no strings and rejects if $M$ accepts at least one string.  A machine for $A_{TM}$ takes as input $\langle M,w\rangle$, a machine and a string.  We will show that $S$ can be used to build a decider for $A_{TM}$
	
	Define a new machine $M'$ such that $M'$ on input $x\neq w$ rejects and on input $x=w$ simulates $M$ on $w$.  $M'$ recognizes a language that is either $\emptyset$ or $\{w\}$.  But $M'$ accepts $w$ only when $M$ accepts $w$.  We can then feed $\langle M' \rangle$ to $S$.  If $S$ accepts, then $M$ does not accept $w$.  Conversely, if $S$ rejects, then $M$ accepts $w$.
	
	We can now decide $A_{TM}$ as follows: on input $\langle M,w\rangle$, construct $M'$ and pass it to $S$.  Since $S$ always halts, $A_{TM}$ halts on all inputs, and is therefore decidable.
	
	Because $A_{TM}$ is not decidable, no such $S$ can exist and $E_{TM}$ must be undecidable.
	
	
	
\end{proof}








