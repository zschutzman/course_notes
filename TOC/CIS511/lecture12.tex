
\chno{12}{$NP\mhyphen Complete$ness}{Sampath Kannan}{Zach Schutzman}
%\footnotetext{These notes are partially based on those of Nigel Mansell.}

% **** YOUR NOTES GO HERE:

% Some general latex examples and examples making use of the
% macros follow.  
%**** IN GENERAL, BE BRIEF. LONG SCRIBE NOTES, NO MATTER HOW WELL WRITTEN,
%**** ARE NEVER READ BY ANYBODY.

\section*{Cook-Levin and $NP\mhyphen Complete$ Languages}

From last time, we sketched a proof of the Cook-Levin Theorem - any problem in $NP$ can be converted into an instance of $SAT$ in time and size polynomial in the original input.

Let's show that $3SAT = \{ \phi | \phi \ is  \ a \ satisfiable \  Boolean \ formula \ in \ 3\mhyphen CNF\}$ is $NP\mhyphen Complete$.

\begin{proof}
	We will show that $SAT$ reduces to $3SAT$ and $3SAT$ is in $NP$ to show $3SAT$ is $NP\mhyphen Complete$, because Cook-Levin gives us the reduction from any $NP$ language to $SAT$, by the transitivity of composition of polynomial reductions.
	
	The reduction is as follows:
	
	Take some instance $\phi$ of $SAT$.  If we want $y \Leftrightarrow x_1\land x_2$, then we can say $(\lnot x_1 \lor \lnot x_2 \lor y)$.  We don't need to worry about the $OR$s.  Taking the $AND$ of all of these clauses gets us a logically equivalent $\phi'\in 3SAT$.
	
	For the same reason $SAT$ is in $NP$, $3SAT$ is as well, as it is easily verifiable.
	
	
	
\end{proof}

Karp took Cook's proof and showed a number of problems are $NP\mhyphen Complete$.  Some of these are:

\begin{itemize}
	\item Independent Set - a subset of vertices, no two of which are adjacent (inputs are $\langle G,k\rangle$, for IS of size $k$).  The reduction from $3SAT$ involves creating triangle for each clause and connecting negations of corresponding variables.  Set $k$ to the number of clauses.
	
	\item $CLIQUE$ is $NP\mhyphen Complete$.  Take a $\langle G,k\rangle$ instance of $IS$.  Create $G^C$ where two vertices are adjacent in $G^C$ if and only if they are not adjacent in $G$.  Now, the $IS$ in $G$ is exactly a $k$-clique in $G^C$.
	
	\item $VERTEX\ COVER$ is a vertex set of size $k$ such that every edge is incident to some element in the set.  The complement of an independent set is a vertex cover, by definition, so a $|V|-k$ independent set is a $k$ vertex cover.
	
	
	
\end{itemize}




