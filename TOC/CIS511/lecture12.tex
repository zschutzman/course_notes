%
% This is the LaTeX template file for lecture notes for CS294-8,
% Computational Biology for Computer Scientists.  When preparing 
% LaTeX notes for this class, please use this template.
%
% To familiarize yourself with this template, the body contains
% some examples of its use.  Look them over.  Then you can
% run LaTeX on this file.  After you have LaTeXed this file then
% you can look over the result either by printing it out with
% dvips or using xdvi.
%
% This template is based on the template for Prof. Sinclair's CS 270.

\documentclass[twoside]{article}
\usepackage{graphics}
\usepackage{amsfonts}
\usepackage{tikz}
\usepackage{amsmath}
\usepackage{IEEEtrantools}
\usepackage{amsmath}

\mathchardef\mhyphen="2D % Define a "math hyphen"

\usetikzlibrary{chains,fit,shapes}
\setlength{\oddsidemargin}{0.25 in}
\setlength{\evensidemargin}{-0.25 in}
\setlength{\topmargin}{-0.6 in}
\setlength{\textwidth}{6.5 in}
\setlength{\textheight}{8.5 in}
\setlength{\headsep}{0.75 in}
\setlength{\parindent}{0 in}
\setlength{\parskip}{0.1 in}

%
% The following commands set up the lecnum (lecture number)
% counter and make various numbering schemes work relative
% to the lecture number.
%
\newcounter{lecnum}
\renewcommand{\thepage}{\thelecnum-\arabic{page}}
\renewcommand{\thesection}{\thelecnum.\arabic{section}}
\renewcommand{\theequation}{\thelecnum.\arabic{equation}}
\renewcommand{\thefigure}{\thelecnum.\arabic{figure}}
\renewcommand{\thetable}{\thelecnum.\arabic{table}}

%
% The following macro is used to generate the header.
%
\newcommand{\chno}[4]{
   \pagestyle{headings}
   \thispagestyle{plain}
   \newpage
   \setcounter{lecnum}{#1}
   \setcounter{page}{1}
   \noindent
   \begin{center}
   \framebox{
      \vbox{\vspace{2mm}
    \hbox to 6.28in { {\bf CIS 511: Theory of Computation
                        \hfill Feb 21, 2017} }
       \vspace{4mm}
       \hbox to 6.28in { {\Large \hfill Lecture #1: #2  \hfill} }
       \vspace{2mm}
       \hbox to 6.28in { {\it Professor #3 \hfill #4} }
      \vspace{2mm}}
   }
   \end{center}
   \markboth{Lecture #1: #2}{Lecture #1: #2}
   {\bf NB}: {\it These notes are from CIS511 at Penn. The course followed Michael Sipser's \textit{Introduction to the Theory of Computation (3ed)} text.}
   \vspace*{4mm}
}

%
% Convention for citations is authors' initials followed by the year.
% For example, to cite a paper by Leighton and Maggs you would type
% \cite{LM89}, and to cite a paper by Strassen you would type \cite{S69}.
% (To avoid bibliography problems, for now we redefine the \cite command.)
% Also commands that create a suitable format for the reference list.
\renewcommand{\cite}[1]{[#1]}
\def\beginrefs{\begin{list}%
        {[\arabic{equation}]}{\usecounter{equation}
         \setlength{\leftmargin}{2.0truecm}\setlength{\labelsep}{0.4truecm}%
         \setlength{\labelwidth}{1.6truecm}}}
\def\endrefs{\end{list}}
\def\bibentry#1{\item[\hbox{[#1]}]}

%Use this command for a figure; it puts a figure in wherever you want it.
%usage: \fig{NUMBER}{SPACE-IN-INCHES}{CAPTION}
\newcommand{\fig}[3]{
			\vspace{#2}
			\begin{center}
			Figure \thelecnum.#1:~#3
			\end{center}
	}
% Use these for theorems, lemmas, proofs, etc.
\newtheorem{theorem}{Theorem}[lecnum]
\newtheorem{lemma}[theorem]{Lemma}
\newtheorem{proposition}[theorem]{Proposition}
\newtheorem{claim}[theorem]{Claim}
\newtheorem{corollary}[theorem]{Corollary}
\newtheorem{definition}[theorem]{Definition}
\newenvironment{proof}{{\bf Proof:}}{\hfill\rule{2mm}{2mm}}

% **** IF YOU WANT TO DEFINE ADDITIONAL MACROS FOR YOURSELF, PUT THEM HERE:

\begin{document}
%FILL IN THE RIGHT INFO.
%\lecture{**LECTURE-NUMBER**}{**DATE**}{**LECTURER**}{**SCRIBE**}
\chno{12}{$NP\mhyphen Complete$ness}{Sampath Kannan}{Zach Schutzman}
%\footnotetext{These notes are partially based on those of Nigel Mansell.}

% **** YOUR NOTES GO HERE:

% Some general latex examples and examples making use of the
% macros follow.  
%**** IN GENERAL, BE BRIEF. LONG SCRIBE NOTES, NO MATTER HOW WELL WRITTEN,
%**** ARE NEVER READ BY ANYBODY.

\section*{Cook-Levin and $NP\mhyphen Complete$ Languages}

From last time, we sketched a proof of the Cook-Levin Theorem - any problem in $NP$ can be converted into an instance of $SAT$ in time and size polynomial in the original input.

Let's show that $3SAT = \{ \phi | \phi \ is  \ a \ satisfiable \  Boolean \ formula \ in \ 3\mhyphen CNF\}$ is $NP\mhyphen Complete$.

\begin{proof}
	We will show that $SAT$ reduces to $3SAT$ and $3SAT$ is in $NP$ to show $3SAT$ is $NP\mhyphen Complete$, because Cook-Levin gives us the reduction from any $NP$ language to $SAT$, by the transitivity of composition of polynomial reductions.
	
	The reduction is as follows:
	
	Take some instance $\phi$ of $SAT$.  If we want $y \Leftrightarrow x_1\land x_2$, then we can say $(\lnot x_1 \lor \lnot x_2 \lor y)$.  We don't need to worry about the $OR$s.  Taking the $AND$ of all of these clauses gets us a logically equivalent $\phi'\in 3SAT$.
	
	For the same reason $SAT$ is in $NP$, $3SAT$ is as well, as it is easily verifiable.
	
	
	
\end{proof}

Karp took Cook's proof and showed a number of problems are $NP\mhyphen Complete$.  Some of these are:

\begin{itemize}
	\item Independent Set - a subset of vertices, no two of which are adjacent (inputs are $\langle G,k\rangle$, for IS of size $k$).  The reduction from $3SAT$ involves creating triangle for each clause and connecting negations of corresponding variables.  Set $k$ to the number of clauses.
	
	\item $CLIQUE$ is $NP\mhyphen Complete$.  Take a $\langle G,k\rangle$ instance of $IS$.  Create $G^C$ where two vertices are adjacent in $G^C$ if and only if they are not adjacent in $G$.  Now, the $IS$ in $G$ is exactly a $k$-clique in $G^C$.
	
	\item $VERTEX\ COVER$ is a vertex set of size $k$ such that every edge is incident to some element in the set.  The complement of an independent set is a vertex cover, by definition, so a $|V|-k$ independent set is a $k$ vertex cover.
	
	
	
\end{itemize}


\end{document}







