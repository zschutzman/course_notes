\classheader{09-11-2017}


\section{Topological Bases}

\definition{Let $X$ be a set.  A collection $\mathcal{B}$ of subsets of $X$ is called a \textbf{base} (or \textbf{basis}) of $X$ if:
	
	\begin{enumerate}
		\item[1] If $x\in X$ then there is a $B\in \mathcal{B}$ such that $x\in B$.  Equivalently, $\mathcal{B}$ covers $X$.
		\item[2] If $B_1,B_2 \in \mathcal{B}$ and $x\in B_1\cap B_2$, then there is a $B_3\in\mathcal{B}$ such that $B_3\subset B_1\cap B_2$ and $x\in B_3$.
	\end{enumerate}
}

This is a weaker concept than a topology; we don't require that the union of base elements is a base element and we only require that the intersection of base elements contains another base element.

\example{Consider $\mathbb{R}^2$ with the standard topology.  Define $\mathcal{B} = \{ B_x(r) |x\in\mathbb{R}^2,r>0 \}$ as the set of open balls in $\mathbb{R}^2$. This is a base.  It is easy to see the first criterion is satisfied.  To see the second, consider two balls which both contain some point $x$.  Then there is a small ball centered at $x$ which is fully contained in the intersection of the two balls.  This smaller ball is also a base element, so we are done.}

\lemma{If$\mathcal{B}$ is a base and  $B_1,B_2,\dots,B_n \in \mathcal{B}$, and $x\in B_1\cap B_2\cap\dots\cap B_n$ then there exists a base element $B'\subset B_1\cap\dots\cap B_n$ which contains $x$.}

\begin{proof}
	We proceed by induction.  Since $x\in B_1\cap B_2$, there exists some $D_1\in\mathcal{B}$ such that $x\in D_1\subset B_1\cap B_2$.  Then $x\in D_1\cap B_3$, so there exists some $D_2\in\mathcal{B}$ with $x\in D_2$.  We proceed iteratively like this to find there is some $D_{n-1}\in \mathcal{B}$ with $x\in D_{n-1}$, and we set $B'=D_{n-1}$.
	
	
	
\end{proof}
	

\definition{A \textbf{topology generated by a base} is the collection of sets which are unions of base elements.}

If $X$ is a set with a base $\mathcal{B}$, then there is a smallest (coarsest) topology on $X$ containing $\mathcal{B}$, which is the topology generated by $\mathcal{B}$.  Open sets are the base elements, arbitrary unions of base elements, and $\emptyset$ and $X$ by definition.  Do we get the intersection property as well?

\claim{Yes.}

\begin{proof}
	If $B_1,\dots,B_n$ are base elements, then we can write the intersection $\bigcap\limits_{i\in [n]}B_i$ as the union of base elements just by taking neighborhoods of each point in the intersection.  If we have $U_1,\dots,U_n$ open in $X$ and $x\in \bigcap\limits_{i\in[n]}U_i$, then there is some base element in the intersection containing $x$.  If we do this for all points in the intersection, we can write the intersection as an arbitrary union of base elements, and we are done.
\end{proof}

\definition{Let $(X,\mathcal{A}$ be a topological space.  Take $\mathcal{B}\subset \mathcal{A}$ a collection of sets such that $\emptyset,X\in \mathcal{B}$ and if $x\in U\in \mathcal{A}$, then there is some $B\in\mathcal{B}$ such that $x\in B\subset U$.  We call $\mathcal{B}$ a \textbf{base for the topology $\boldsymbol{\mathcal{A}}$}.}

\theorem{If $\mathcal{A}$ is a topology on $X$ and $\mathcal{B}$ is a base for $\mathcal{A}$, then $\mathcal{B}$ is a base.