\classheader{09-18-2017}

\definition{A topological space $X$ is \textbf{first-countable} if there exists a countable base at every point $x\in X$.}

\definition{A topological space $X$ is \textbf{second-countable} if it has a countable base.}

\example{Consider $\mathbb{R}\sqcup\mathbb{R}/\sim$, where $x\sim y$ if $x=y$ and $x,y<1$.  This looks like the line but with two copies of the point $\{1\}$ and a second copy of every point greater than $1$.  This space is not Hausdorff.  If we take an open set at the first copy of $1$, it must intersect any open neighborhood of the other copy of $1$.}

\example{Consider $\mathbb{R}\sqcup\mathbb{R}/\sim$, where $x\sim y$ if $x=y$ and $x,y\leq1$. This looks like the line with one branch extending towards $-\infty$  and two branches extending towards $+\infty$ from $1$.  This space is Hausdorff.  If we take an open neighborhood around $1$, we can see that the inverse image is open only if that neighborhood contains pieces of all three branches, so it isn't possible to have an open set that contains $1+\epsilon_1$ for all $\epsilon_1>0$ on one branch without also having some small neighborhood which also contains $1+\epsilon_2$ for some $\epsilon_2>0$ on the second.}


\section*{Back to Subbases}

Recall that a subbase $\mathcal{S}$ of a set $X$ is a collection of subsets such that $\bigcup\mathcal{S}=X$.

\definition{If $\mathcal{S}$ is a subbase, then \textbf{the topology generated by $\boldsymbol{\mathcal{S}}$} is the set of all arbitrary unions and finite intersections of elements of $\mathcal{S}$.}

The proof that this is in fact a topology is trivial.

\definition{If $\mathcal{S}$ is a subbase, then a base $\mathcal{B}$ formed by the set of all finite intersections of elements of $\mathcal{S}$ plus the set $X$ itself is \textbf{the base generated by $\boldsymbol{\mathcal{S}}$}.}

Again, the proof that this is a proper base is trivial.

\definition{Let $(X,\mathcal{A}_1)$ and $(Y,\mathcal{A}_2)$ be topological spaces with respective bases $\mathcal{B}_1\subset X$ and $\mathcal{B}_2\subset Y$.  Then $X\times Y$ has  a topology called the \textbf{product topology} which is generated by a base $\mathcal{B}_3$ where $W\in \mathcal{B}_3$ if and only if $W=U\times V$ for some $U\in\mathcal{B}_1$ and $V\in\mathcal{B}_2$.  That is, base elements in the product topology are products of the base elements in the factor topologies.}


\example{Consider $\mathbb{R}\times\mathbb{R}$ and let $\mathcal{B}_1=\mathcal{B}_2=\{ (a,b)|-\infty\leq a<b\leq +\infty  \}$ be bases for $\mathbb{R}$ which consist of all open intervals.  Then the base for $\mathbb{R}^2$, $\mathcal{B}_3=\{(a,b)\times(c,d) |(a,b)\in\mathcal{B}_1,(c,d)\in\mathcal{B}_2  \}$ is the set of open rectangles in $\mathbb{R}^2$.  We showed last time (in the general case of prisms in $\mathbb{R}^n$) that this is indeed a base.}

\definition{Let $(X,\mathcal{A}_1)$ and $(Y,\mathcal{A}_2)$ be topological spaces with respective bases $\mathcal{B}_1\subset X$ and $\mathcal{B}_2\subset Y$.  The the subbase $\mathcal{S} = \{ U\times Y | U\in\mathcal{B}_1 \}\cup \{ X\times V|V\in\mathcal{B}_2 \}$ is the \textbf{standard subbase on $\boldsymbol{X\times Y}$}.}

The proof that the standard subbase generates the product topology extends naturally to all product topologies generated by a finite number of factor spaces.

We can think of the product of $k$ copies of $X$ as the set of functions from a set of size $k$ to $X$, $X^k = \{ f:[k]\rightarrow X\}$.

\example{We can think of $\mathbb{R}^3$ as $\{  f: \{1,2,3\}\rightarrow \mathbb{R} \}$.  $(f(1),f(2),f(3))$ is an ordered triple in $\mathbb{R}^3$, and it also completely specifies a function.  We can think of $\mathbb{R}^n$ as $\bigtimes\limits_{[n]}\mathbb{R}=\{ f:[n]\rightarrow\mathbb{R}\}$.  Similarly, the set of all functions from $\mathbb{R}$ to $\mathbb{R}$ is $\mathbb{R}^\mathbb{R}=\bigtimes\limits_\mathbb{R}\mathbb{R}$.}

