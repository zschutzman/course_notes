\classheader{10-02-2017}

\section*{Topological Groups}

We will look at some topological properties of matrix groups.  Throughout, we will assume everything happens with respect to some fixed basis.

\definition{The group $\boldsymbol{O(n)}$ is the set of rigid transformations of $n$-dimensional Euclidean space which leaves the origin fixed.  That is, $A\in O(n)$ if and only if for any $\vec{x},\vec{y}\in\mathbb{R}^n$, $\innprod{\vec{x}}{\vec{y}} = \innprod{A\vec{x}}{A\vec{y}}$, where $\innprod{\cdot}{\cdot}$ is the standard Euclidean inner product.}

These matrices all have determinant $\pm 1$, and this holds with respect to every basis.

\definition{The group $\boldsymbol{SO(n)}$ is the subset of $O(n)$ of $n\times n$ matrices with determinant $+1$.}

We'll observe that $O(1) = \{-1,1\} = C_2$ and  $SO(1) = \{1\}$ is the trivial group.

$SO(2)$ is infinite, and can be thought of as the set of $2\times 2$ rotation matrices
$SO(2)= \left\{  \begin{pmatrix}
\cos\theta & -\sin\theta \\
\sin\theta & \cos\theta
\end{pmatrix}  | \theta\in\R   \right\}$

\definition{A \textbf{topological group}  $X$ is a set $X$ with a topology and a group structure such that the group operation $\circ :X\times X\rightarrow X$ is a continuous function.}

Topologically, we can think of $SO(2)$ as the unit circle in $\mathbb{R}^4$.

The group $SO(3)$ is harder to write down, but it is the set of $3\times 3$ matrices with determinant $1$ (equivalently, the $3x3$ matrices $A$ such that $AA^T=I$).

$SO(3)$ is generated by three one-parameter subgroups, which are copies of $SO(2)$ each fixing a dimension: 

$X_\theta = 
\begin{pmatrix} 
1 & 0 & 0 \\
0 & \cos\theta & -\sin\theta \\
0 & \sin\theta & \cos\theta
\end{pmatrix}
$ corresponds to rotation about the $x$-axis

$Y_\theta = 
\begin{pmatrix} 
\cos\theta & 0 & -\sin\theta \\
0 & 1 & 0 \\
\sin\theta & 0 &  \cos\theta
\end{pmatrix}
$ corresponds to rotation about the $y$-axis


$Z_\theta = 
\begin{pmatrix} 
\cos\theta  & -\sin\theta & 0 \\
\sin\theta &  \cos\theta & 0\\
0 & 0 & 1
\end{pmatrix}
$ corresponds to rotation about the $z$-axis

thus $SO(3) = \left\{  X_{\theta_1}\cdot Y_{\theta_2}\cdot Z_{\theta_3} | \theta_1,\theta_2,\theta_3\in \R  \right\}\subset \R^9$.

Since multiplication in $\R^9$ is a continuous function, matrix multiplication in $SO(3)$ is as well, so it is a topological group as well.  There is a homeomorphism between $SO(3)$ and $\R P^3$, the real projective plane of dimension $3$.  Thus, $SO(3)\simeq S^3/{\text{antipodal map}}$

\section*{Back to Sequences}

\lemma{The Sequence Lemma: If $A\subset X$ and $(x_1,x_2,x_3\dots)$ is a sequence in $A$ with $x$ as a limit point, then $x\in\overline{A}$ (we proved this last time).  Additionally, if $X$ is a first-countable space and $A\subset X$, then $x\in\overline{A}$ implies there exists a sequence in $A$ with $x$ as a limit point.}

\begin{proof}
	
	
	Let $\{B_i\}$ be a countable basis at $x$.  Because $x\in A$, we have that $B_i\cap A\neq \emptyset$, so there is at least one point in $B_i\cap A$ for all $B_i$.  For each, take one such point.  This defines a sequence $(x_1,x_2,x_3,\dots)$ in $A$.  By the definition of convergence, a sequence converges to $x$ of $x$ is in the closure of every subsequence.
	
    Take $U$ to be an open set containing $x$.  Then there exists some base element $B_N \subset U$ but by construction, $x_i\in B_N$ for all $i\geq N$, so $B_N$ contains every infinite subsequence, so $U$ also does.  Since $U$ is an open set around $x$ which has infinite intersection with any subsequence, $x$ is a limit point of the sequence.
	
	
	
\end{proof}

\definition{A function $f:X\rightarrow Y$ is \textbf{sequentially continuous} if whenever $S=(x_1,x_2,x_3,\dots)$ is a sequence in $X$ and $x$ is a limit point of $S$, then the sequence $T=(f(x_1),f(x_2),f(x_3),\dots)$ has (at least) $f(x)$ as a limit point.}

\thrm{If $f:X\rightarrow Y$ is continuous, then $f$ is sequentially continuous.}

\begin{proof}
	
	Take $S=(x_1,x_2,x_3,\dots)$ to be a sequence in $X$ with limit point $x$ and let $y=f(x)$.  Take $U$ to be an open neighborhood of $y$ in $Y$.
	
	By continuity, $f^{-1}(U)$ is an open neighborhood of $x$, and since $x$ is a limit point of $S$, the tails of all subsequences of $S$ are in $f^{-1}(U)$, so the image of the tails of all subsequences is in the image $U$.  Therefore, $y$ is a limit point of the sequence $(f(x_1),f(x_2),f(x_3),\dots)$, as desired.
	
	
	
\end{proof}
\corollary{By this theorem and the Sequence Lemma, we have that if $X$ is first-countable, then sequential continuity implies continuity.}

\corollary{In a metric space, continuity, sequential continuity, and $\epsilon-\delta$ continuity are all equivalent.}