\classheader{09-13-2017}



\section*{Bases, Separability, Hausdorff}

\definition{If $(X,\mathcal{A})$ is a topological space and $p$ a point in $X$, then a \textbf{base at the point $\boldsymbol{p}$} is a collection $\mathcal{U}$ of open sets such that whenever $p$ is in some $V\in \mathcal{A}$, there exists a $U\in\mathcal{U}$ with $p\in U\in\mathcal{U}$.}

\example{The set of open balls centered at $p$ form a base at $p$ in $\mathbb{R}^n$ with the standard topology.}

\example{The set of open balls forms a base for $\mathbb{R}^n$.  So does the set of all rectangular prisms.  So does the set of all cubes.  The set of cubes is obviously a subset of the set of prisms, but neither of these is a subset or a superset of the set of balls.  We can also have a base where we have balls/prisms/cubes with rational centers and rational radii/side lengths.  The cardinality of these bases is the same as the cardinality of $\mathbb{Q}$.}

\definition{If a set has a base which has cardinality in bijection with some subset of $\mathbb{N}$ or $\mathbb{Q}$, then it has a \textbf{countable base}.}

Which topologies have countable bases?  We have seen that $\mathbb{R}^n$ with the standard topology does.  How about $\mathbb{R}^n$ with the discrete topology?  The answer is no.  Since every singleton set is open in the discrete topology, any base must contain every singleton.  Since the number of singleton subsets of $\mathbb{R}$ is uncountable, there cannot be a countable base.

\definition{A subset is \textbf{dense} if every open set in the space contains some element of the subset.}

\definition{A space is \textbf{separable} if it has a countable, dense subset.}

If a set has a countable base, it is  separable, one countable, dense subset is just a single element from each base element.

Observe that any topology on a finite or countable set is separable, as the set of singletons is countable.

\definition{A topological space is \textbf{Hausdorff} if whenever we have two points $p,q\in X$ with $p\neq q$, there exist disjoint open sets such that $p$ belongs to one and $q$ belongs to the other.}

\example{$\mathbb{R}^n$ with the standard topology is separable.  If $p\neq q$, we can take small open balls around $p$ and $q$ with radius less than half the distance between them.  These balls are disjoint and open, so we are done.}


\example{The set $[-1,1]$ with the either-or topology is not Hausdorff.  If we take $p$ to be any non-zero point in $(-1,1)$ and $q=0$, then any open set containing zero must also contain $p$.}

\example{The Zariski topology is not Hausdorff (this is on the homework).}


\example{The line with a double point, defined as $\mathbb{R}\sqcup\mathbb{R}/\sim$ with $x\sim y$ if $x=y$ and $x,y\neq 0$ looks like the real line with two zeros, call them $0_1$ and $0_2$.  The open sets in this topology are the empty set, the whole space, and anything that kind of looks like a standard open set.  This space is not Hausdorff.  Any open set containing $0_1$ necessarily contains a neighborhood of $0_2$, and vice versa.  This space is separable, however.  Rational balls will form a countable base, for example.}



\theorem{If $\mathcal{A}$ is a topology on $X$ and $\mathcal{B}$ is a base for $\mathcal{A}$, then $\mathcal{B}$ is a base.}
	
	\begin{proof}
		
		Clearly we have $\emptyset, X\in \mathcal{B}$.  We only need to show that any point in the intersection of two base elements $B_1,B_2$ is in a third base element $B_3\subset B_1\cap B_2$.  We have $B_1\cap B_2$ open because $\mathcal{B}\subset\mathcal{A}$.  So by the definition of a topology, there must e a $B_3\in \mathcal{B}$ with $x\in B_3\subset B_1\cap B_2$, and we're done.
		
	\end{proof}
	

\theorem{If $\mathcal{B}$ is a base for a set $X$ and $\mathcal{A}$ is the topology generated by $\mathcal{B}$, then $\mathcal{B}$ is a base for the topology $\mathcal{A}$.}

\begin{proof}
	This proof is also straightforward.  If $B_1,B_2\in \mathcal{B}$ are basis elements, and we take an $x\in B_1\cap B_2$ then there is some $B_3\in\mathcal{B}$ with $x\in B_3\subset B_1\cap B_2$.  Since $\mathcal{B}$ generates $\mathcal{A}$, if $x$ is in some open set $U\in\mathcal{A}$, then it is in the union of some collection $\{B_i\}\subset \mathcal{B}$, so $x\in B_j$ for some $B_j\in\mathcal{B}$ with $B_j\subset U$, and we are done.
\end{proof}





\definition{A \textbf{subbase} (or subbasis) for a set $X$ is a collection $\mathcal{S}$ of sets such that $\bigcup\limits_{S\in\mathcal{S}} S=X$.}

