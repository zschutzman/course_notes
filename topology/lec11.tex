\classheader{09-25-2017}

\section*{Metrics, Continued}

Recall that $\ell^p$ is a proper subset of $\mathbb{R}^\omega$.

\definition{The \textbf{$\boldsymbol{\ell^p}$ norm} on $\mathbb{R}^n$ is $d(x,y) = \left(\sum|x_i|^p\right)^{\frac{1}{p}}$.}

\definition{The \textbf{$\boldsymbol{\ell^\infty}$ norm} on $\mathbb{R}^n$ is $d(x,y) = \sup\{|x_i|\}$.}

If $X$ is a vector space, then a function $||\cdot ||: X\rightarrow \mathbb{R}$ is a norm if:

\begin{enumerate}
	\item $||\vec{x}||\geq 0$ for all $x\in X$ with equality if and only if $x=\vec{0}$
	\item $||c\vec{x}|| = |c|\cdot||\vec{x}||$ for all scalars $c$ and vectors $\vec{x}$
	\item $||\vec{x}+\vec{y}|| \leq ||\vec{x}||+||\vec{y}||$ for all vectors $\vec{x},\vec{y}$
\end{enumerate}

\definition{We call such an $X$ a \textbf{normed vector space}.}

Metrics do not need to come from norms, but norms induce metrics.  A normed vector space is a metric space, but a metric space need not be a normed vector space.

Consider the function $f:\mathbb{R}\rightarrow \mathbb{R}$ with a norm 
$$||f||_p = \left(\int_{-\infty}^{\infty}|f(x)|^p dx\right)^\frac{1}{p}$$

\definition{This is called the \textbf{$\boldsymbol{L^p}$ pseudonorm}.}

Why is this a pseudonorm?  There are functions which are non-zero which have zero integral, such as functions which are zero everywhere except on a set of measure zero.  We can resolve this by defining an equivalence relation $\sim$ where $f\sim g$ if and only if $||f-g||_p=0$.  These equivalence classes form the vector space of $L^p$ functions on $\mathbb{R}$, denoted $L^p(\mathbb{R})$.  On these classes, the $L^p$ pseudonorm $||\cdot ||_p$ is a proper norm.  As it turns out, these are actually \textit{complete} normed vector spaces, i.e. Banach spaces.

\definition{Given a metric $d(x,y)$, we can create a \textbf{bounded metric} $\bar{d}(x,y)$ by defining $\bar{d}(x,y)=\min(d(x,y),1)$.  It is easy to see that this forms a proper metric.}

When do two metrics generate the same topology?  Metric balls form a base.  If we can show that balls in one metric contain balls in the other, and vice versa, then the metrics generate the same topology.

\definition{The \textbf{Hilbert cube} is the set $I^\omega \subset \mathbb{R}^\omega$.  That is, it is the set of all sequences with entries from the interval $[0,1]$.}

The Hilbert cube has finite volume, but the diagonal has infinite length.  That is, the point $(0,0,0,\dots)$ is infinitely far from the point $(1,1,1,\dots)$ with respect to any $\ell^p$ metric (for finite $p$).  This space is homeomorphic to the ball of radius $1$ in the $\ell^\infty$ metric.

A subbase for the Hilbert cube is the set $\{ \pi_i^{-1}(U) | U \ open \ in \ [0,1] \}$.

\definition{A topological space is \textbf{metrizable} if there exists a metric on $X$ which generates that topology.}

As an example, the standard topology on $\mathbb{R}^n$ is metrizable as it is generated by the $\ell^2$ metric, for example.

Since metric spaces are first-countable, first-countability is a necessary condition for a space to be metrizable.

\example{$\mathbb{R}^\mathbb{R}$ with the product topology is not first-countable, therefore there does not exist a metric on $\mathbb{R}^\mathbb{R}$ which generates the product topology.}

\theorem{A countable product of metric spaces (with the product topology) is metrizable.}

\begin{proof}
	
	Let $X$ be a metric space with metric $d$ and consider the product $X^\mathbb{N}$.  Let $x=(x_1,x_2,\dots)$ and $y=(y_1,y_2,\dots)$ be points in $X^\mathbb{N}$.  Next, define $D(x,y) = \sup\left\{\frac{\bar{d}(x_i,y_i)}{i}\right\}$ where $\bar{d}$ is the bounded version of $d$.  It's easy to see that $D$ is a proper metric.
	
	Under $D$, the ball of radius $r$ centered at $x$ is 
	
\begin{align*}
B_r(x) &= \{ y\in X^\mathbb{N} | D(x,y)<r   \}\\
\ \\
&= \left\{  y\in X^\mathbb{N} | \bar{d}(x_1,y_1) < r ,\ \bar{d}(x_2,y_2) < 2r, \ \bar{d}(x_3,y_3)<3r,\dots              \right\}
\end{align*}
	
Since eventually $nr$ becomes larger than $1$, everything beyond the $n$th point must be inside the ball.  Thus, only finitely many $(x_i,y_i)$ need to be checked, so these balls look like the product of finitely many open balls in $X$ and infinitely many copies of $X$ itself, which is a base for the product topology.
	
\end{proof}