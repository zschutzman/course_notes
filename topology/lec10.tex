\classheader{09-25-2017}

\section*{Metric Topologies}


\definition{A \textbf{metric} on a space $X$ is a function $d:X\times X\rightarrow R_+$ which satisfies:
	
	\begin{enumerate}
		\item $d(x,y)=0$ if and only if $x=y$ (positivity)
		\item $d(x,y) = d(y,x)$ for all $x,y$ (symmetry)
		\item $d(x,z)\leq d(x,y)+d(y,z)$ (triangle inequality)
	\end{enumerate}
	
}

\definition{If $X$ is a space with metric $d$, and $x\in X$, $r>0$, \textbf{the ball of radius $\boldsymbol{r}$ centered at $\boldsymbol{x}$} is the set $\{y|d(x,y)<r\}$.  This is denoted $B_r(x)$ (or $B_x(r)$, $B(x,r)$, $B(r,x)\dots$).}

\definition{The \textbf{topology generated by a metric} is the topology on a metric space generated by the base of all balls of finite radius centered at all points.}



If $x$ is an element of $X$ and $x$ is in some set $U$ which is open in the metric topology, then there is a sufficiently small $r$ such that $B_r(x)$ is entirely contained in $U$.  Hence all points of $U$ are interior.

Let's note that while such a base does indeed generate the topology we want, it's not always the best or smallest base that does so.  For example, the set of all balls of rational radius with rational center generates the standard topology on $\mathbb{R}^n$.

In fact, every metric space is at least first-countable, as we can take a base at a point consisting of the balls centered at that point of rational radius.

Metric spaces are also Hausdorff.  To  see this, consider two distinct points $x,y$.  Since they are not identical, the distance between them is positive, say $3\epsilon$.  Then $B_\epsilon(x)$ and $B_\epsilon(y)$ are disjoint open sets separating $x$ and $y$.

\definition{The \textbf{discrete metric} is the metric $d(x,y) = 0$ if $x=y$ and $1$ if $x\neq y$.}

\claim{The discrete metric generates the discrete topology.  }

\begin{proof}
	
To see this, consider the ball $B_{\frac{1}{2}}(x)$.  This is just the set $\{x\}$, and since the singletons are open, the corresponding topology must be the discrete one.
\end{proof}

\claim{The trivial topology is not generated by any metric.}

\begin{proof}
	The trivial topology is not Hausdorff, so it cannot be a metric topology.
\end{proof}


The Euclidean metric on $\mathbb{R}^n$, $d(x,y)=\sqrt{\sum(x_i-y_i)^2}$ generates the standard topology.

\definition{The \textbf{$\boldsymbol{\ell^p}$ metric} on $\mathbb{R}^n$ is $d(x,y) = \left(\sum|x_i-y_i|^p\right)^{\frac{1}{p}}$.}

\definition{The \textbf{$\boldsymbol{\ell^\infty}$ metric} on $\mathbb{R}^n$ is $d(x,y) = \sup\{|x_i-y_i|\}$.}

\claim{All of the $\ell^p$ metrics generate the same topology on $\mathbb{R}^n$.}

\begin{proof}
	Pick two metrics $\ell^p$ and $\ell^q$ with $p<q$.  We'll argue that base elements (balls) in one topology contain base elements of the other.  Containment one way is trivial.  For a fixed radius $r$, the $\ell^p$ ball of radius $r$ sits inside of the $\ell^q$ ball of radius $r$.  To see containment the other way, consider the $\ell^q$ ball of radius $r_1$ and think of it as an open set in the $\ell^p$ topology.  There is some point of maximum distance from the center, and if we pick some radius $r_2$ less than this, the ball will sit inside of the $\ell^q$ ball, and we're done.
\end{proof}


We call $\ell^p(\mathbb{R}^\omega)$ the set of sequences whose $\ell^p$ norm is finite.  This forms a vector space.