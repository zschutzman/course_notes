\classheader{09-22-2017}

\section*{Back To Bases}

How can we determine if two bases generate the same topology?

\begin{theorem}
	
	Let $\mathcal{A}$ and $\mathcal{A}'$ be two topologies on the same underlying set $X$ generated by bases $\mathcal{B}$ and $\mathcal{B}'$, respectively.  Then the following are equivalent:
	
	\begin{enumerate}
		\item $\mathcal{A}'$ is finer than $\mathcal{A}$
		\item If $x$ is in $X$ and $x$ is in some base element $B\in\mathcal{B}$, then there exists a $B'\in\mathcal{B}'$ such that $x\in B'$ and $B'\subset B$.
	\end{enumerate}
	
	
	
\end{theorem}

\begin{proof}
	
	Assume that $\mathcal{A}'$ is finer than $\mathcal{A}$ and take some $x\in X$ and some $B\in\mathcal{B}$ such that $x\in B$.  Since $B$ is an open set in $\mathcal{A}$, $B$ is also an open set in $\mathcal{A}'$, so $B$ can be written as the union $\bigcup B'_i$ for some $\{B'_i\}\subset \mathcal{B}'$.  Since $x\in B$, $x$ is in the union of these $B'_i$, so $x$ must be in at least one of the $B'_i$ which by construction is a subset of $B$.
	
	Now, assume that $x\in B\in\mathcal{B}$ implies the existence of some $B'\in\mathcal{B}'$ with $x\in B'\subset B$.  By taking intersections and small neighborhoods, we can necessarily write any such $B$ as a union of some collection of $B'_i$.  But since we can do this, any open set built from elements of $\mathcal{B}$ can be built from elements of $\mathcal{B}'$, so any open set in $\mathcal{A}$ is also open in $\mathcal{A}'$, hence $\mathcal{A}'$ is finer than $\mathcal{A}$.
	
	
	
\end{proof}

\section*{The Product Topology}

\remark{The book uses $\mathbb{R}^\omega$ to denote the set of all sequences in $\mathbb{R}$ indexed by the natural numbers and $\mathbb{R}^\infty$ to denote those sequences which are eventually all zeros.  We'll do our best to be consistent with this.}


\definition{Given a product of topological spaces $\prod X_\alpha$, the \textbf{box topology} is the one with open sets that can be written as a product $\prod U_\alpha$ where $U_alpha$ is open in $X_\alpha$.}

Under infinite products, this topology is too fine.  Consider the function $f:\mathbb{R}\rightarrow \mathbb{R}^\omega$, where $f(t)=(t,t,t,\dots)$.  This function is not continuous from the standard topology to the box topology. To see this, consider the set $(-1,1)\times (-\frac{1}{2},\frac{1}{2})\times (-\frac{-1}{3})\times \dots$ in $\mathbb{R}^\omega$.  This is open in the box topology.  Under $f$, the inverse image of this set is $\{0\}$, which is not open.

\definition{Let $\pi_\alpha(x)$ be the function which sends $x$ in the product to its $\alpha$ coordinate.  This extends to sets by saying that $\pi_\alpha(U)$ is the set of elements $y\in X_\alpha$ such that there exists some $x\in U$ where $\pi_\alpha(x) = y$.  The \textbf{product topology} is defined by the base consisting of the sets $\pi^{-1}_\alpha (U_\alpha)$ where $U_\alpha$ is a base element (or any open set) in $X_\alpha$.  Equivalently, an open set in the product topology is one which is a product of open sets in the factor spaces where all but finitely many are equal to the whole space.}