%---------------------------------------------------------------------%
%  LaTeX Course Notes Template                                        %
%                                                                     %
%  Copyright (C) 2012 Zev Chonoles                                    %
%  zevchonoles@gmail.com                                              %
%  http://math.uchicago.edu/~chonoles/                                %
%                                                                     %
%  Please leave this information in the source code as                %
%  attribution if you choose to edit or redistribute this file.       %
%                                                                     %
%  This work is licensed under the Creative Commons Attribution-      %
%  ShareAlike 3.0 Unported License. To view a copy of this license,   %
%  visit http://creativecommons.org/licenses/by-sa/3.0/.              %
%                                                                     %
%---------------------------------------------------------------------%

\documentclass[11pt]{article}






%----------%
%  Basics  %
%----------%


%  Specfies basic information.
%  In the metadata section of the preamble, you can specify the subject and a list of keywords for the PDF.
%
\newcommand{\coursetitle}{Math 500 - Topology and Geometry}
\newcommand{\lecturer}{Brian Weber}
\newcommand{\notetaker}{Zach Schutzman}
\newcommand{\notetakersemail}{ianzach+notes@seas.upenn.edu}
\newcommand{\courseterm}{Fall 2017}
\newcommand{\institution}{University of Pennsylvania}


%  array provides more column styles for the tabular and array environments.
%  (http://ctan.org/pkg/array)
%
%  parskip sets block paragraphs as the default, instead of indentation.
%  (http://www.ctan.org/pkg/parskip)
%
\usepackage[margin=1in]{geometry}
\usepackage{amsmath,amssymb,amsthm,amsfonts,array,parskip}


%  Allows equation, align, gather, etc. environments to split across pages.
\allowdisplaybreaks


%  Sets date formatting to the ISO 8601 standard, YYYY-MM-DD.
\usepackage{datetime} \renewcommand{\dateseparator}{-} \yyyymmdddate






%---------%
%  Fonts  %
%---------%


%  Defines \cal for standard calligraphy, \eucal for Euler calligraphy, and \frak for Fraktur.
\usepackage{eucal}  \let\eucal\mathcal  \let\cal\CMcal  \renewcommand{\frak}{\mathfrak}


%  Removes ligatures (e.g. the connection ordinarily made between the two f's in "differentiable").
\usepackage{microtype} \DisableLigatures{encoding=*,family=*}


%  Removes extra space after periods.




%-------------------------------%
%  Environments and Sectioning  %
%-------------------------------%


%  Defines some standard theorem environments, in both numbered and non-numbered versions. The numbering of each enviroment will be reset for each lecture.
\newcounter{lecture}       \setcounter{lecture}{0}
\newcounter{tN}[lecture]   \newcounter{dN}[lecture]
\newcounter{lN}[lecture]   \newcounter{rN}[lecture]
\newcounter{cN}[lecture]   \newcounter{eN}[lecture]
\newcounter{pN}[lecture]
\newcounter{clN}[lecture]

\newtheorem*{theorem}{Theorem}          \newtheorem{theorem-N}[tN]{Theorem}
\newtheorem*{lemma}{Lemma}              \newtheorem{lemma-N}[lN]{Lemma}
\newtheorem*{corollary}{Corollary}      \newtheorem{corollary-N}[cN]{Corollary}
\newtheorem*{proposition}{Proposition}  \newtheorem{proposition-N}[pN]{Proposition}

\theoremstyle{definition}
\newtheorem*{definition}{Definition}    \newtheorem{definition-N}[dN]{Definition}
\newtheorem*{remark}{Remark}            \newtheorem{remark-N}[rN]{Remark}
\newtheorem*{example}{Example}          \newtheorem{example-N}[eN]{Example}


\newtheorem*{claim}{Claim}    \newtheorem{claim-N}[clN]{Claim}


%  Modifies the spacing above theorem environments, which is messed up when using the parskip package.
%  (http://tex.stackexchange.com/questions/22119)
%
\makeatletter \def\thm@space@setup{\thm@preskip=\parskip \thm@postskip=0pt} \makeatother


%  Modifies the spacing above the proof environment.
%  (http://tex.stackexchange.com/questions/49801)
%
\makeatletter \renewenvironment{proof}[1][\proofname]{\pushQED{\qed}\normalfont
	\partopsep=\z@skip \topsep=\z@skip \trivlist \item[\hskip\labelsep\itshape #1\@addpunct{.}]
	\ignorespaces}{\popQED\endtrivlist\@endpefalse} \makeatother


%  Removes extra space before and after section headings.
\usepackage[compact]{titlesec}






%-------------------------%
%  Pictures and Diagrams  %
%-------------------------%


%  Allows for the use of colors.
%  (http://www.ctan.org/pkg/xcolor)
%
\usepackage[usenames,dvipsnames]{xcolor}
\definecolor{myred}{rgb}{0.9,0.2,0.2}
\definecolor{mygreen}{rgb}{0.2,0.6,0.2}
\definecolor{myblue}{rgb}{0.2,0.2,0.8}


%  graphicx provides advanced graphics options.
%  (http://ctan.org/pkg/graphicx)
%
\usepackage{graphicx}


%  tikz is for drawing all sorts of pictures and diagrams.
%  tikz-cd makes creating commutative diagrams in tikz a bit easier.
%  (http://www.ctan.org/pkg/pgf)
%  (http://www.ctan.org/pkg/tikz-cd)
%
\usepackage{tikz}
\usepackage{tikz-cd}
\usepackage{pgf,pgfplots}
\usetikzlibrary{arrows,calc,decorations,decorations.markings,fadings,positioning,patterns,shapes}
\tikzset{>=latex}
\tikzstyle{mypoint}=[inner sep=0pt,outer sep=0pt,minimum size=5pt,fill,circle]


\definecolor{ttqqqq}{rgb}{0.2,0.,0.}
\definecolor{ffffff}{rgb}{1.,1.,1.}
%\usetikzlibrary{external}
%\tikzexternalize



%------------------------%
%  Commands and Symbols  %
%------------------------%


%  Creates commands by running over a comma-separated list. For example,
%
%     \forcsvlist{\define{\newcommand}{\textbf}{bold}}{A,B}
%
%  would create
%
%     \newcommand{\boldA}{\textbf{A}}    \newcommand{\boldB}{\textbf{B}}
%
%  (http://tex.stackexchange.com/a/5776/20882)
%
\usepackage{etoolbox}
\newcommand{\define}[4]{\expandafter#1\csname#3#4\endcsname{#2{#4}}}
\forcsvlist{\define{\DeclareMathOperator}{}{}}{im,coker,rad,nil,Ann,Ass,codim,Spec,mSpec,diam,ord,Supp,supp,disc,Ob,vol,rank,Sym,Alt,Ind}
\forcsvlist{\define{\newcommand}{\mathrm}{}}{Hom,Mor,id,GL,SL,SO,SU,U,M,Mat,Ext,Tor,Res,Cor,Inf,End,Irr,Aut,Gal,lcm,tr,sign,triv,diag,Map,op,ev,act,alg,sep,unr,nr,ab}

%  Creates commands for some names of categories in the sans-serif font.
\forcsvlist{\define{\newcommand}{\mathsf}{}}{Set,Grp,Ab,CRing,Mod,Vect,Cat,Top,PreSh,Sh,Sch,Nat,Fun,Diff}

%  Creates commands for some blackboard bold letters.
\forcsvlist{\define{\newcommand}{\mathbb}{}}{N,Z,Q,R,C,F,G,T,A,B,D}


%  Saves the section symbol, paragraph symbol, Hungarian accent, and Scandanavian O in the macros \SS, \PP, \HH, and \OO, then redefines \S, \P, \H, and \O to be the corresponding blackboard bold letters.
%
\let\SS\S  \let\PP\P  \let\HH\H  \let\OO\O
\forcsvlist{\define{\renewcommand}{\mathbb}{}}{S,P,H,O}


%  latexsym defines some alternative versions of amssymb symbols.
%  (http://www.bakoma-tex.com/doc/latex/base/latexsym.pdf)
%
\usepackage{latexsym}


%  Defines a copyright symbol that is a bit nicer than the built-in one.
\newcommand{\mycopyrightsymbol}{\raisebox{-0.3ex}{\tikz{\node[inner sep=0pt,outer sep=0pt] at (0,0) {\textsc{c}};\draw (0,0) circle (0.18);}}}


%  Defines commands for real and complex projective space.
\newcommand{\RP}{\mathbb{R}\mathrm{P}}  \newcommand{\CP}{\mathbb{C}\mathrm{P}}


%  Defines a bordered matrix with square bracket delimiters instead of parentheses.
%  (http://tex.stackexchange.com/questions/55054)
%
\let\bbordermatrix\bordermatrix
\patchcmd{\bbordermatrix}{8.75}{4.75}{}{}
\patchcmd{\bbordermatrix}{\left(}{\left[}{}{}
\patchcmd{\bbordermatrix}{\right)}{\right]}{}{}


%  Calls one of the mathabx font families so that it is possible to use its symbols without making a global change.
%  (http://www.ctan.org/pkg/mathabx)
%  (http://tex.stackexchange.com/questions/14386)
%
\DeclareFontFamily{U}{mathb}{\hyphenchar\font45}
\DeclareFontShape{U}{mathb}{m}{n}{<5> <6> <7> <8> <9> <10> gen * mathb
	<10.95> mathb10 <12> <14.4> <17.28> <20.74> <24.88> mathb12}{}
\DeclareSymbolFont{mathb}{U}{mathb}{m}{n}


%  Defines circular arrows.
\DeclareMathSymbol{\lcirclearrow}{0}{mathb}{'366}
\DeclareMathSymbol{\rcirclearrow}{0}{mathb}{'367}
\newcommand{\leftcirclearrow}{\mathrel{\ensuremath{\raisebox{0.1ex}{\scalebox{0.9}{\rotatebox[origin=c]{90}{$\lcirclearrow$}}}}}}
\newcommand{\rightcirclearrow}{\mathrel{\ensuremath{\raisebox{0.1ex}{\scalebox{0.9}{\rotatebox[origin=c]{270}{$\rcirclearrow$}}}}}}


%  Gives semantic names for some common math symbols.
\newcommand{\iso}{\cong}
\newcommand{\htop}{\sim}
\newcommand{\htopequiv}{\simeq}
\newcommand{\cupprod}{\mathbin{\smallsmile}}
\newcommand{\capprod}{\mathbin{\smallfrown}}
\newcommand{\wedgesum}{\mathbin{\vee}}
\newcommand{\boundary}{\partial}
\renewcommand{\emptyset}{\varnothing}
\newcommand{\characteristic}{\mathrm{char}}
\newcommand{\symdiff}{\mathbin{\vartriangle}}
\newcommand{\convolute}{\mathbin{\ast}}
\newcommand{\actson}{\rightcirclearrow}
\newcommand{\actedonby}{\leftcirclearrow}
\newcommand{\directsum}{\oplus}
\newcommand{\bigdirectsum}{\bigoplus}
\newcommand{\tensor}{\otimes}
\newcommand{\bigtensor}{\bigotimes}
\newcommand{\free}{\mathbin{\ast}}
\newcommand{\bigfree}{\mathop{\ensuremath{\raisebox{-0.7ex}{\scalebox{2.3}{$\ast$}}}}}
\renewcommand{\complement}[1]{{#1}^{\mathsf{c}}}
\newcommand{\transpose}[1]{{#1}^{\textsf{T}}}
\newcommand{\union}{\cup}
\newcommand{\intersect}{\cap}
\newcommand{\transverse}{\mathrel{\raisebox{1.1ex}{$-$}\mathllap{\pitchfork\hspace{0.22mm}}}}



\def\multiset#1#2{\ensuremath{\left(\kern-.3em\left(\genfrac{}{}{0pt}{}{#1}{#2}\right)\kern-.3em\right)}}



%-----------------------------------%
%  Things Specific to Course Notes  %
%-----------------------------------%


%  Formatting for the table of contents. The first line allows for multi-column environments, the second line removes the heading "Contents".
\usepackage{multicol} \setlength{\columnsep}{3cm}
\makeatletter \renewcommand\tableofcontents{\@starttoc{toc}} \makeatother


%  Sets the page style.
\usepackage{fancyhdr}
\pagestyle{fancy}
\renewcommand{\headrulewidth}{0pt}
\renewcommand{\footrulewidth}{0.5pt}
\setlength{\headheight}{14pt}
\lfoot{\parbox[t]{1in}{\centering Last edited\\ \today}}
\cfoot{\parbox[t]{3in}{\centering \coursetitle}}
\rfoot{\parbox[t]{0.9in}{\centering Page \thepage\\ Lecture \arabic{lecture}}}


%  Sets the inputs for \maketitle.
\author{%Lectures by \lecturer\\ 
	Notes by \notetaker}
\title{\coursetitle}
\date{\institution, \courseterm}


%  Defines headings for each day's notes.
\newcommand{\classheader}[1]{\stepcounter{lecture}\newpage\section*{Lecture \arabic{lecture} (#1)}
	\phantomsection \addcontentsline{toc}{section}{Lecture \arabic{lecture} (#1)}}


%---------------------------------------%
%  Miscellaneous Additions to Template  %
%---------------------------------------%

% http://tex.stackexchange.com/questions/18359
\pgfplotsset{compat=newest}

\newcommand{\Cinfty}{\ensuremath{C^{\infty}}}
\newcommand{\Crit}{\mathrm{Crit}}
\usepackage{mathtools}
\newcommand{\Or}{\mathrm{Or}}
\renewcommand{\Re}{\mathrm{Re}}
\renewcommand{\Im}{\mathrm{Im}}
\usepackage{mathrsfs}
\newtheorem*{examples}{Examples}
\newtheorem*{exercise}{Exercise}
\usepackage{pdfpages}
\newcommand{\Lie}{\mathrm{Lie}}
\newcommand{\Diffeo}{\mathrm{Diffeo}}

\newcommand{\connection}{\nabla}
\newcommand{\new}{\mathrm{new}}


\newcommand{\review}{{\huge\color{myred}{$\star$}}}


%---------------------------%
%  Hyperlinks and Metadata  %
%---------------------------%
%
% (this section must come last!)


%  hyperref enables for the creation of hyperlinks, and also specifies the metadata of the PDF file.
%  hyperxmp allows more metadata to be specified.
%  (http://www.ctan.org/pkg/hyperref)
%  (http://www.ctan.org/pkg/hyperxmp)
%  (http://tex.stackexchange.com/questions/41461)
%
\usepackage{hyperref}
\usepackage{hyperxmp}
\hypersetup{
	pdfauthor={\notetaker},
	pdftitle={\coursetitle},
	pdfproducer={LaTeX},
	%pdfcopyright={Copyright (C) \the\year\ \notetaker. This work is licensed under a Creative Commons Attribution-ShareAlike 3.0 Unported License. All attribution should be to \lecturer\ as the lecturer, and to \notetaker\ as the person taking these notes.},
	pdfsubject={differential topology},
	pdfkeywords={},
	%pdflicenseurl={http://creativecommons.org/licenses/by-sa/3.0/},
	colorlinks=true,
	linkcolor=myred,
	citecolor=mygreen,
	urlcolor=myblue,
	linktoc=page,
	pdfstartview=FitH
}






%------------%
%  Document  %
%------------%


\begin{document}
	
	
	%  The command
	%
	%  \thispagepdflabel{text}
	%
	%  sets the PDF page number (*not* the internal LaTeX page number) to be "text". This does not have to be a numeral; it could be a word, e.g. "Title". This lets one avoid the issue of having the PDF's page numbering not aligning with the page numbering LaTeX used in the document.
	%
	%  (http://tex.stackexchange.com/questions/85558)
	
	
	%  Title
	%
	\maketitle
	\thispdfpagelabel{Title}
	\thispagestyle{empty}
	\setcounter{page}{-1}
	\vspace{0.3in}
	
	
	
	%  Table of Contents
	%
	\begin{center}
		\begin{minipage}[t]{0.9\textwidth}
			\begin{multicols}{2}
				\tableofcontents
			\end{multicols}
		\end{minipage}
	\end{center}
	
	
	
	\newpage
	\thispdfpagelabel{-}
	\thispagestyle{empty}
	
	
	
	%  Introduction
	%
	\section*{Introduction}
	Math 500 is a Masters-level first-course in Topology and Geometry.  The course follows James Munkres' \textit{Topology, 2ed.} and this set of notes is based on the Fall 2017 offering.
	
	These notes are being live-TeXed, though I edit for typos and add diagrams requiring the Ti\textit{k}Z package separately. I am using the editor TeXstudio.  The template for these notes was created by Zev Chonoles and is made available (and being used here) under a Creative Commons License. 
	
	I am responsible for all faults in this document, mathematical or otherwise; any merits of the material here should be credited to the lecturer, not to me.
	
	Please email any corrections or suggestions to \expandafter\href{mailto:\notetakersemail}{\texttt{\notetakersemail}}.
	
	%\medskip
	%
	%\section*{Acknowledgments}
	%
	%Thank you to all of my fellow students who sent me suggestions and corrections, and who lent me their own notes from days I was absent. My notes are much improved due to your help.
	
	
	%%  Copyright
	%%
	%\section*{Copyright}
	%Copyright \mycopyrightsymbol\ 2012 \notetaker.
	%
	%This work is licensed under a Creative Commons Attribution-ShareAlike 3.0 Unported License. This means you are welcome to do essentially anything with this work, including editing, %adapting, distributing, and making commercial use of it, as long as you
	%\begin{itemize}
	%\item include an attribution of \lecturer\ as the lecturer of the course these notes are based on, and \notetaker\ as the person taking the notes,
	%\item do so in a way that does not suggest either of us endorses you or your use of this work, and
	%\item if you alter, transform, or build upon this work, you must apply to your work the same, or similar, license to this one.
	%\end{itemize}
	%More details are available at \href{https://creativecommons.org/licenses/by-sa/3.0/deed.en\_US}{\texttt{https://creativecommons.org/licenses/by-sa/3.0/deed.en\_US}}.
	
	\newpage
	
	
	%  Make a separate file for each lecture, for example, using a naming scheme like this:
	%
	%  lecture1.tex, lecture2.tex, ...
	%
	%  and keep them in the same folder as this main file. By doing it this way (instead of keeping all the notes in the main file), if you're only working on the notes for one lecture, you can easily comment out the lines corresponding to the other lectures.
	%
	
	\classheader{2017-08-30}

\section*{What is Topology?}

\definition{A \textbf{topology} is a set $X$ with a collection of subsets $\mathcal{A}\subset \mathcal{P}(X)$ such that:
	
	\begin{itemize}
		\item[1] $\emptyset,X\in \mathcal{A}$
		\item[2] $\mathcal{A}$ is closed under finite intersection (the intersection of a finite subset of $\mathcal{A}$ is in $\mathcal{A}$)
		\item[3] $\mathcal{A}$ is closed under arbitrary union (the union of any (possibly infinite) subset of $\mathcal{A}$ is in $\mathcal{A}$)
	\end{itemize}
	
}

The phrases ``$\mathcal{A}$ is a topology on $X$", ``$X$ is a topological space with topology $\mathcal{A}$, and the notation $(X,\mathcal{A})$ all refer to the same concept.

\definition{The \textbf{standard topology} (also called the euclidean topology or metric topology) on $\mathbb{R}^n$ is the set of subsets $U\subset\mathbb{R}^n$ such that for every $U$, every point $x\in U$ is interior, meaning that there exists some radius $r>0$ such that the ball of radius $r$ centered at $x$ is entirely contained in $U$.}

\definition{A set is \textbf{open} in a topological space $X$ if it belongs to the topology on $X$.}

\example{The standard topology is a topology over $\mathbb{R}^n$:
	
	\begin{itemize}
		\item[1] Every point in the empty set is vacuously interior, and every point of $\mathbb{R}^n$ is trivially interior
		\item[2] If we take two open sets and intersect them, any point in the intersection must be an interior point in both constituent sets.  The smaller of the two balls witnessing this must lie entirely within both constituent sets, and therefore entirely within the intersection.  By induction, we have the finite intersection of open sets being open.
		\item[3] Intuitively, taking any union of open sets only creates a bigger set.  The ball witnessing any point as interior to some open set clearly lies in any union including that open set.
	\end{itemize}
	
}

We can see from this example why it's important to specify closure under \textit{finite} intersection.  Singleton sets are not open in the standard topology on $\mathbb{R}^n$, but the Nested Interval Theorem gives us a way to construct a singleton set from the countable intersection of open intervals.

\example{If $X$ is our topological space, $\{\emptyset , X\}$ is a topology, called the \textbf{trivial topology}.}

\example{Similarly, all of $\mathcal{P}(X)$ is a topology, called the \textbf{discrete topology}.}

\example{The \textbf{Zariski topology} on $\mathbb{R}^n$ is a little more interesting.  A set is open in the Zariski topology if it is the complement of the root set of some polynomial.  Open sets in the one-dimensional case look like the real line minus a finite number of points.  It gets a little more complicated in higher dimensions, as we can have zeroes along entire dimensions of a euclidean space.  Let's verify that this is a topology:
	
	\begin{itemize}
		\item[1] The empty set is the complement of the root set of the zero function, and the entire space $\mathbb{R}^n$ is the complement of the root set of a polynomial which has no real roots, such as $f(\vec{x})=6$.
		\item[2]  The intersection of two open sets, corresponding to polynomials $P$ and $Q$ is, by DeMorgan's Laws, $\mathbb{R}^n \setminus \{x\ | \ x \ is \ a \ root \ of \ P \ or \ Q\}$.  Something is a root of $P$ or $Q$, it must be a root of the product $PQ$.  Since the finite product of polynomials is a polynomial, this set is still the complement of the root set of some polynomial, and is therefore open, and we have closure under finite intersection.
		\item[3] Again by DeMorgan's Laws, the union of two open sets corresponding to polynomials $P$ and $Q$ is the set $\mathbb{R}^n \setminus\{ x \ | \ x  \ is \ a \ root \ of \ P \ and \ Q\}$.  The set of points which are roots of $P$ and $Q$ are the roots of the greatest common polynomial divisor of $P$ and $Q$.  Since this is also a polynomial, our set is the complement of the root set of a polynomial and is therefore open. Since the greatest common polynomial divisor of any set of polynomials has root set no greater than any of the constituent polynomials, we properly have closure under arbitrary union.
	
	\end{itemize}
}

The Zariski topology is an object of importance in the area of algebraic geometry.

\definition{If $X$ is a topological space with topology $\mathcal{A}$ and $Y\subset X$, then $\mathcal{B}$ is a topology on $Y$ where a subset $V\subset Y$ is open in $\mathcal{B}$ if and only if there is a $U$ open in $\mathcal{A}$ such that $V=U\cap Y$.  This is called the \textbf{subset} or \textbf{subspace topology}.}

\example{Let $H^2$ denote the closed upper-half plane in  $\mathbb{R}^2$.  That is, the set of points $(x,y)\in \mathbb{R}^2$ such that $y\geq 0$.  Any set which was open in $\mathbb{R}^2$ and does not intersect the $x$-axis is still open in $H^2$.  However, a set like an open half-disk against the $x$-axis together with the line segment where it rests up against the $x$-axis was not an open set in $\mathbb{R}^2$, as the boundary points are not interior, but it is open in $H^2$ with the subspace topology, as it is the intersection of an open disk in $\mathbb{R}^2$ with the upper half-plane.}

	\classheader{2019-01-30}{((I missed this))}
	\classheader{2019-02-06}{tdb}

\section{Square Roots (modulo $\boldsymbol{p}$)}
	
	Consider a prime $p=3\mod 4$, and the equation $x^2-a=0\mod p$, i.e. $a$ is a \textit{square root} of $p$.
	
	By Fermat's Little Theorem, $a^{p-1}=1$, so $a^{p}=a$, and $a^{p+1} = a^2$, so $a^{\frac{p+1}{2}} = \pm a$.  We know it's $+a$ because $-a = -1 \times a$, and $-1$ is not a quadratic residue in $\Z_p^*$ when $p=3\mod 4$ because $(x)^\frac{p-1}{2}$ will be $1$ if and only if $x$ is a quadratic residue (i.e. is an even power of the generator $g$).  This is not the case for $-1$ because $\frac{p-1}{2}$ is odd since $p-1 = 2\mod 4$. Therefore, $-a$ is not a quadratic residue. Therefore $a^\frac{p+1}{4}$ is a square root of $a$.
	
	Furthermore, the set of quadratic residues in $\Z_p^*$ when $p=3\mod 4$ forms a group $Q_p$, since for quadratic residues $a,b$, since the product of the square roots of $a$ and $b$ is the square root of $ab$, and demonstrating closure suffices to prove that this is a group, since it's finite.  
	
	However, if $p=1\mod 4$, then we don't have a good formula for finding square roots.  Take such a $p$ and consider $\Z_p^*$ and an element $a$.  We have $a^{p-1}=1$ so $a^\frac{p-1}{2} = 1$ and $a^{p+1} = a^2$, but $p+1$ is not divisible by 4.
	
	Let's write $p=2^s\times m$ where $m$ is odd (this is easy).  We want to solve $x^2-a=0$ and we know $a^\frac{p-1}{2}=1$.  So we can write $a^{m(2^{s-1})} = 1$.  Let $a^m=u_0$.  We know that the order of $u_0$ divides $2^{s-1}$ and is therefore a power of 2.  If the order is 1, i.e. $u_0=1$, then $a^{m+1} = u_0a = a$.  Since $m+1$ is even, $a^\frac{m+1}{2}$ is a square root of $a$.  Otherwise, $a^\frac{m+1}{2}$ is a square root of $u_0a$, which we will call $v_0$.
	
	Construct two sequences, $u_0,u_1,\dots$ and $v_0,v_1,\dots$, such that $v_i^2 = u_ia$.  The $u_i$ will have orders which are powers of 2 and decreasing, so we eventually get a $u_j=1$, which will imply $v_j$ is a square root of $a$.
	
	We'll need some randomness to construct the $u_i$, since we'll need to find an element of $\Z_p^*$ which is not a residue.  We can do this by randomly testing, since half of the elements are non-residues.  Let $b$ be a randomly chosen non-residue in $\Z_p^*$, and let $c=b^m$.  By Fermat's Little Theorem, $b^{p-1}=b^{m2^s}=1$.  Furthermore, $c^{2^{s-1}} = b^\frac{p-1}{2} = -1$ because $b$ is a non-residue.
	
	Let $r_i$ denote the order of $u_i$.  Then $u_0^{2^{r_0}} = 1$ and $u_0^{2^{r_0-1}} = -1$, since it's the square root and not equal to 1.  Given this, we'll construct $u_1$ by taking the product $(c^{2^{s-1}})(u_0^{2^{r_0-1}}) = (u_0c^{2^{s-r_0}})^{2^{r_0-1}} = 1$ and we'll let the base $(u_0c^{2^{s-r_0}}) = u_1$.  Now, we want a $v_1$ such that $v_1^2 = u_1 a$.  
	
	We have $v_0^2=u_0a$, so let's try $v_1=v_0 c^{2^{s-r_0-1}}$, and so $$v_1^2 = v_0^2 c^{2^{s-r_0}} = u_0c^{2^{s-r_0}} a = u_1a$$
	
	which works.  We repeat this process until we find a $u_j$ of order 1.
	
	
\begin{definition}
	
	The \textbf{Legendre symbol} of $a$ in $\Z_p^*$ is written as $\left(\frac{a}{p}\right)$ and is equal to $1$ if $a\in Q_p$ (i.e. is a quadratic residue) and $-1$ otherwise, and 0 if $a=0$.
	
\end{definition}

What if we have $N=pq$ for primes $p$ and $q$.  Given $a\in \Z_N^*$, can we find a square root of $a$? If $a$ is a quadratic residue, then $a$ has exactly four square roots, by the Chinese Remainder Theorem, since we can write it as a pair which is a square modulo $p$ and a square modulo $q$, so if $(b_1,b_2)$ is a square root of $a$, then the four square roots are given by $(\pm b_1,\pm b_2)$.

It turns out that finding a square root in $\Z_N^*$ is equivalent to factoring in the ring, given that a randomized polynomial time algorithm for finding square roots can be used to factor. [[see exercise 6.1 from Michael's CIS625 problem set]]

This gives us a candidate \textit{one-way function}. Take $N=pq$ and $f(x)=x^2 \mod n$ for $x\in\Z_n^*$.  This gives an encryption scheme for messages over $\Z_n^*$.  Alice can encode $x$ with $f(x)$.  Bob can decode it using the factors of $N$ using the Chinese Remainder Theorem and finds the square roots with respect to $p$ and $q$ (quickly).  This is actually a \textit{trapdoor function} since it is easy to compute the forward direction and easy to compute the backward direction only if you know a certain secret.

As a cryptosystem, this is called the \textit{Rabin} cryptosystem. It's relatively simple, but not widely used.  One drawback is that it gives four possible decryptions (assuming $N$ is the product of two primes).  We can restrict the message space to numbers less than $N/2$.

Another cryptosystem, which is very well-known and used a lot in practice is the \textit{Rivest-Shamir-Adelman} (RSA) cryptosystem.  In this system, Bob picks $N=pq$ and an encryption exponent $e$.  He then publishes $N$ and $e$.  The secret will be knowledge of $p$ and $q$ as well as a decryption exponent $d$.  When Alice sends a message $x$, she sends $f(x) = m^e \mod N$ (observe that the Rabin cryptosystem has $e=2$).  Bob needs to choose $d$ such that $m^{ed}=m$, so $m^{ed-1}=1$, so the order of $m$ divides $ed-1$.  It also divides $(p-1)(q-1) = \phi(N)$ by Lagrange's theorem. Then $m^{\phi(N)} = 1$, so we might want to pick $d$ such that $ed-1=k\phi(N)$.  We know the GCD of $e$ and $\phi(N)$ must be 1, so Bob should pick $e$ according to that.  He should pick $d$ by using the extended GCD on $e,\phi(N)$ which gives an $xe+y\phi(N) = 1$, and let $d=x$.  This will guarantee the uniqueness of the decryption.

Interestingly, there is no result equating the difficulty of RSA decryption to the difficulty of factoring.  It may be easier, though it is certainly not harder.  If Eve can factor $N$, she can use the same process that Bob used to pick $d$ and use the same process as Bob to decrypt the message.  Even knowing $\phi(N)$ is sufficient to decrypt.  

\section{Primality Testing}

The first randomized algorithms for primality testing were developed in the 1970s: Solovay-Strassen and Miller-Rabin.  Both therefore show that $\mathrm{COMPOSITES}\in\mathrm{RP}$ ($\mathrm{PRIMES}\in\mathrm{Co-RP}$, equivalently).  A paper in the early 2000s shows that $\mathrm{PRIMES}\in P$, but it's a much more complicated algorithm than either of these.

How does it work?  We discussed earlier that we can use Fermat's Little Theorem as an almost-test, because for all $a<p$, $a^{p-1}=1\mod p$.  For any $N$, the set of numbers $a<N$, the set of things which satisfy $a^{N-1}=1\mod N$ form a group.  For primes, this group is all of the numbers less than $N$.  For a composite $N$, we'll consider $a\in\Z_N^*$.  The group $G=\{  a\vert a^{N-1}=1\mod N   \} < \Z_N^*$.  If this subgroup is proper, it's at most half the size.

This suggests an algorithm.  Given $N$, pick $a\in_R\{1,2,\dots,N-1\}$ and check if $a\in G$.  If no, say that $a$ is composite and if yes, say that $a$ is prime.  If $G$ is a proper subgroup, this has probability of success at least half.  However, there are composite numbers $N$ (the Carmichael numbers) for which $G$ is not a proper subgroup.  This means the algorithm is incorrect, since it is always wrong about Carmichael numbers (unless you pick a number not in $\Z_N^*$).

The Miller-Rabin algorithm works as follows, leveraging the property of primes that the sequence (modulo $N$) $x,x^2,x^4,x^8,\dots,1$ has as its second-last term -1 if $N$ is prime and sometimes $-1$ and sometimes something else if $N$ is composite.  Given an input $N$, write $N-1 = m(2^s)$ where $m$ is odd.  Then, pick $a\in_R[N-1]$  If $a^{N-1}\neq 1 \mod N$, output composite.  If $a^m=1\mod N$, output prime.  Otherwise, let $b=a^m$ and compute the sequence $b,b^2,\dots,b^{2^s} $ and look at the term before the first one.  If this is not $-1$, output composite. Otherwise, output prime.

If the number is composite, it will return composite with probability at least one half.  If the number is prime, it will return prime always.



\begin{definition}
	A \textbf{strongly one-way (length-preserving) function} $f$ is a function such for all $x$, $|x|=|f(x)|$ and for any probabilistic polynomial-time adversaries $A$, there is a sufficiently large $n$ such that $\Pr\limits_{x\in U_n}\left[ f(A(f(x)))  = f(x)       \right]$ is negligible.  I.e. the probability of an adversary successfully inverting $f$ is exponentially small.
	
	A \textbf{weakly one-way (length-preserving) function} if there exists a polynomial $p(n)$ such that for any probabilistic polynomial-time adversary $A$, $\Pr\limits_{x\in U_n}\left[f(A(f(x))) \neq f(x)\right] \geq \frac{1}{p(n)}$.  I.e. any adversary which finds a sort of inverse of $f$  fails sufficiently often.
\end{definition}


\begin{lemma}
	
	The existence of a weak one-way function implies the existence of a strong one.
	
	
\end{lemma}


\begin{proof}
	Suppose that $f$ is weakly one-way with polynomial $p(n)$.  Define $g(x_1,x_2,\dots,x_{np(n)}) = (f(x_1),f(x_2),\dots f(x_{np(n)}))$ where each $x_i$ has $n$ bits.  We'll show that $g$ is strongly one-way.
	
	Assume, for the sake of contradiction, that $g$ is not strongly one-way.  Let $B$ be an algorithm that inverts $g$ with non-negligible probability, at least $\frac{1}{q(n)}$ for some polynomial $q(n)$.  We'll use $B$ to invert $f$.
	
	This algorithm $I$ takes a string $y$ and finds a string $x$ such that $f(x) = y$.  Given a $y$, $I$ chooses an $x_i$ to be equal to $y$ and sets the rest of the $x_j$ to be uniformly random, will try $y$ in each position $i$, and calls $B$ on $(f(x_1),\dots, y, \dots,f(x_{np(n)})$.
	
	Let $S_n$ be the set of $n$-bit strings such that $I$ succeeds in inverting $f(x)$ with probability at least $\frac{n}{p(n)}$.  For any $x$ in this set, we can boost the success probability to be exponentially close to one by repeating $I$ on it enough times.  In order for $f$ to be weakly one-way, $S_n$ can't have more than $(1-\frac{1}{p(n_)})2^n$ strings in it.  We'll show that $|S_n| \geq (1-\frac{1}{2p(n)})2^n$, which is our contradiction.
	
	Consider the success probability of $I$ on two cases.  First, when all inputs are in $S_n$ and second when at least one input is outside of $S_n$. The probability of all the inputs being in $S_n$ is exponentially small.  The probability that $B$ succeeds on an input where at least one block is outside of $S_n$ is also exponentially small, because otherwise the whole thing would have been in $S_n$.  But the success probability of $B$ is like the sum of these two probabilities, which is exponentially small, contradicting the assumption that $B$ succeeds often enough.
\end{proof}
	\classheader{09-08-2017}


\section*{Mapping Cylinders and Tori}


\definition{Let $I$ denote the closed unit interval $[0,1]$.  The \textbf{mapping cylinder} of a continuous map $:fX\rightarrow Y$ is the quotient space defined by $X\times I/\sim$, where $(x,0)\sim (f(x),1)$.}


\definition{The \textbf{mapping torus} of a map $f:X\rightarrow X$ is similar, except that we require that the map be from a space to a copy of itself and we define the equivalence relation as $(x,0)\sim(f(x),0)$.}

\example{The mapping cylinder of $f:\mathbb{S}^1\rightarrow \mathbb{S}^1$ where $f(x)=x$ is a regular old cylinder.  The mapping torus is a regular old torus.}

\example{We can equivalently think of $\mathbb{S}^1$ as $\{ (x,y)|x^2+y^2=1 \}$ in Euclidean space or as $\{(r,\theta)|r=1  \}$ in polar coordinates.  Using this second formulation, consider the map $f:\mathbb{S}^1\rightarrow \mathbb{S}^1$ where $f(\theta)=2\theta$.  This is a two-to-one map which maps antipodal points to each other.  The mapping cylinder of $f$ is a Moebius strip.}

\example{What about the three-to-one map $f(\theta)=3\theta$?  The mapping cylinder of this looks like a three-bladed wing with a one-third twist and the ends glued together.}

\section*{Boundaries and Exteriors}

Let $(X,\mathcal{A})$ be a topological space and let $K\subseteq X$.

\definition{The \textbf{interior} of $K$ is the largest open subset contained in $K$.  That is, it is the union of all $U\subset K$ such that $U\in \mathcal{A}$.}

\definition{The \textbf{closure} of $K$ is the smallest closed subset containing $K$.  That is, it is the intersection of all $V\supset K$ such that $(X-V)\in \mathcal{A}$.}

\definition{The \textbf{boundary} of $K$ is the intersection of the closure of $K$ with the closure of the complement of $K$, that is $Bd(K) = \overline{K}\cap \overline{X-K}$.  If $K$ is open, then $Bd(K) = \overline{K}-K$.  If $K$ is closed, then $Bd(K)=\emptyset$.}


\example{Take $K = \mathbb{Q}\cap [0,1]\subset \mathbb{R}$ with the standard topology on $\mathbb{R}$.  The interior of this set is empty, as there is no open interval which doesn't contain an irrational number, so $\emptyset$ is the largest open subset in $K$.  The closure of $K$ is the entire interval $[0,1]$, as there is no smaller closed set which contains all of the rationals in that interval.  We also have that the boundary $Bd(K) = [0,1]$.}

\example{Take $K=\mathbb{R}-\{0\}$ with the Zariski topology on $\mathbb{R}$.  The interior of $K$ is $K$, as $K$ is open.  The closure of $K$ is all of $\mathbb{R}$, and the boundary is $\{0\}$.}

\definition{A point $x$ is a \textbf{limit point} of $K\subset X$ if every open set containing $x$ has non-empty intersection with $K$.  Equivalently, $x$ is a limit point of $K$ if $x\in \overline{K-\{x\}}$.}



\example{Take $\mathbb{R}$ with the Zariski topology.  If $U$ is an open set, then every $x\in\mathbb{R}$ is a limit point of $U$.  In fact, for any infinite subset of $\mathbb{R}$, every point in $\mathbb{R}$ is a limit point.}

\definition{A \textbf{neighborhood} of a point $x\in X$ is an open set containing $x$.}

\example{In the either-or topology on $[-1,1]\subset\mathbb{R}$ has open sets $\emptyset$, $[-1,1]$, a set is open if and only if it does not contain $0$ or it contains $(-1,1)$.  If $U$ is an open set and $0\in U$, then $U=(-1,1),\ [-1,1),\ (-1,1],\ or \ [-1,1]$.  Closed sets are subsets of $\{-1,1\}$, all of $[-1,1]$, $\emptyset$, and any set that contains $0$.
	
	What are the continuous functions?  From $EO$ to $std$, constant functions are continuous.  Anything else?  From $std$ to $EO$, continuous functions look like $f(x)=\frac{1}{2}sgn(x)$.}

\example{Take $\frac{1}{2}\in [-1,1]$ with the either-or topology. Is $\frac{1}{2}$ a limit point of $[-1,1]-\{\frac{1}{2} \}$?  No!  That set is already closed, so it contains all of its limit points.
	
	Is $0$ a limit point of $(\frac{1}{2},\frac{3}{4})$? Yes!  Every open set containing $0$ contains $(-1,1)$, so it obviously also contains $(\frac{1}{2},\frac{3}{4})$.
	
}



	
	\classheader{09-11-2017}


\section*{Topological Bases}

\definition{Let $X$ be a set.  A collection $\mathcal{B}$ of subsets of $X$ is called a \textbf{base} (or \textbf{basis}) of $X$ if:
	
	\begin{enumerate}
		\item[1] If $x\in X$ then there is a $B\in \mathcal{B}$ such that $x\in B$.  Equivalently, $\mathcal{B}$ covers $X$.
		\item[2] If $B_1,B_2 \in \mathcal{B}$ and $x\in B_1\cap B_2$, then there is a $B_3\in\mathcal{B}$ such that $B_3\subset B_1\cap B_2$ and $x\in B_3$.
	\end{enumerate}
}

This is a weaker concept than a topology; we don't require that the union of base elements is a base element and we only require that the intersection of base elements contains another base element.

\example{Consider $\mathbb{R}^2$ with the standard topology.  Define $\mathcal{B} = \{ B_x(r) |x\in\mathbb{R}^2,r>0 \}$ as the set of open balls in $\mathbb{R}^2$. This is a base.  It is easy to see the first criterion is satisfied.  To see the second, consider two balls which both contain some point $x$.  Then there is a small ball centered at $x$ which is fully contained in the intersection of the two balls.  This smaller ball is also a base element, so we are done.}

\lemma{If$\mathcal{B}$ is a base and  $B_1,B_2,\dots,B_n \in \mathcal{B}$, and $x\in B_1\cap B_2\cap\dots\cap B_n$ then there exists a base element $B'\subset B_1\cap\dots\cap B_n$ which contains $x$.}

\begin{proof}
	We proceed by induction.  Since $x\in B_1\cap B_2$, there exists some $D_1\in\mathcal{B}$ such that $x\in D_1\subset B_1\cap B_2$.  Then $x\in D_1\cap B_3$, so there exists some $D_2\in\mathcal{B}$ with $x\in D_2$.  We proceed iteratively like this to find there is some $D_{n-1}\in \mathcal{B}$ with $x\in D_{n-1}$, and we set $B'=D_{n-1}$.
	
	
	
\end{proof}
	

\definition{A \textbf{topology generated by a base} is the collection of sets which are unions of base elements.}

If $X$ is a set with a base $\mathcal{B}$, then there is a smallest (coarsest) topology on $X$ containing $\mathcal{B}$, which is the topology generated by $\mathcal{B}$.  Open sets are the base elements, arbitrary unions of base elements, and $\emptyset$ and $X$ by definition.  Do we get the intersection property as well?

\claim{Yes.}

\begin{proof}
	If $B_1,\dots,B_n$ are base elements, then we can write the intersection $\bigcap\limits_{i\in [n]}B_i$ as the union of base elements just by taking neighborhoods of each point in the intersection.  If we have $U_1,\dots,U_n$ open in $X$ and $x\in \bigcap\limits_{i\in[n]}U_i$, then there is some base element in the intersection containing $x$.  If we do this for all points in the intersection, we can write the intersection as an arbitrary union of base elements, and we are done.
\end{proof}

\definition{Let $(X,\mathcal{A}$ be a topological space.  Take $\mathcal{B}\subset \mathcal{A}$ a collection of sets such that $\emptyset,X\in \mathcal{B}$ and if $x\in U\in \mathcal{A}$, then there is some $B\in\mathcal{B}$ such that $x\in B\subset U$.  We call $\mathcal{B}$ a \textbf{base for the topology $\boldsymbol{\mathcal{A}}$}.}


	\classheader{09-13-2017}



\section*{Bases, Separability, Hausdorff}

\definition{If $(X,\mathcal{A})$ is a topological space and $p$ a point in $X$, then a \textbf{base at the point $\boldsymbol{p}$} is a collection $\mathcal{U}$ of open sets such that whenever $p$ is in some $V\in \mathcal{A}$, there exists a $U\in\mathcal{U}$ with $p\in U\in\mathcal{U}$.}

\example{The set of open balls centered at $p$ form a base at $p$ in $\mathbb{R}^n$ with the standard topology.}

\example{The set of open balls forms a base for $\mathbb{R}^n$.  So does the set of all rectangular prisms.  So does the set of all cubes.  The set of cubes is obviously a subset of the set of prisms, but neither of these is a subset or a superset of the set of balls.  We can also have a base where we have balls/prisms/cubes with rational centers and rational radii/side lengths.  The cardinality of these bases is the same as the cardinality of $\mathbb{Q}$.}

\definition{If a set has a base which has cardinality in bijection with some subset of $\mathbb{N}$ or $\mathbb{Q}$, then it has a \textbf{countable base}.}

Which topologies have countable bases?  We have seen that $\mathbb{R}^n$ with the standard topology does.  How about $\mathbb{R}^n$ with the discrete topology?  The answer is no.  Since every singleton set is open in the discrete topology, any base must contain every singleton.  Since the number of singleton subsets of $\mathbb{R}$ is uncountable, there cannot be a countable base.

\definition{A subset is \textbf{dense} if every open set in the space contains some element of the subset.}

\definition{A space is \textbf{separable} if it has a countable, dense subset.}

If a set has a countable base, it is  separable, one countable, dense subset is just a single element from each base element.

Observe that any topology on a finite or countable set is separable, as the set of singletons is countable.

\definition{A topological space is \textbf{Hausdorff} if whenever we have two points $p,q\in X$ with $p\neq q$, there exist disjoint open sets such that $p$ belongs to one and $q$ belongs to the other.}

\example{$\mathbb{R}^n$ with the standard topology is separable.  If $p\neq q$, we can take small open balls around $p$ and $q$ with radius less than half the distance between them.  These balls are disjoint and open, so we are done.}


\example{The set $[-1,1]$ with the either-or topology is not Hausdorff.  If we take $p$ to be any non-zero point in $(-1,1)$ and $q=0$, then any open set containing zero must also contain $p$.}

\example{The Zariski topology is not Hausdorff (this is on the homework).}


\example{The line with a double point, defined as $\mathbb{R}\sqcup\mathbb{R}/\sim$ with $x\sim y$ if $x=y$ and $x,y\neq 0$ looks like the real line with two zeros, call them $0_1$ and $0_2$.  The open sets in this topology are the empty set, the whole space, and anything that kind of looks like a standard open set.  This space is not Hausdorff.  Any open set containing $0_1$ necessarily contains a neighborhood of $0_2$, and vice versa.  This space is separable, however.  Rational balls will form a countable base, for example.}



\theorem{If $\mathcal{A}$ is a topology on $X$ and $\mathcal{B}$ is a base for $\mathcal{A}$, then $\mathcal{B}$ is a base.}
	
	\begin{proof}
		
		Clearly we have $\emptyset, X\in \mathcal{B}$.  We only need to show that any point in the intersection of two base elements $B_1,B_2$ is in a third base element $B_3\subset B_1\cap B_2$.  We have $B_1\cap B_2$ open because $\mathcal{B}\subset\mathcal{A}$.  So by the definition of a topology, there must e a $B_3\in \mathcal{B}$ with $x\in B_3\subset B_1\cap B_2$, and we're done.
		
	\end{proof}
	

\theorem{If $\mathcal{B}$ is a base for a set $X$ and $\mathcal{A}$ is the topology generated by $\mathcal{B}$, then $\mathcal{B}$ is a base for the topology $\mathcal{A}$.}

\begin{proof}
	This proof is also straightforward.  If $B_1,B_2\in \mathcal{B}$ are basis elements, and we take an $x\in B_1\cap B_2$ then there is some $B_3\in\mathcal{B}$ with $x\in B_3\subset B_1\cap B_2$.  Since $\mathcal{B}$ generates $\mathcal{A}$, if $x$ is in some open set $U\in\mathcal{A}$, then it is in the union of some collection $\{B_i\}\subset \mathcal{B}$, so $x\in B_j$ for some $B_j\in\mathcal{B}$ with $B_j\subset U$, and we are done.
\end{proof}





\definition{A \textbf{subbase} (or subbasis) for a set $X$ is a collection $\mathcal{S}$ of sets such that $\bigcup\limits_{S\in\mathcal{S}} S=X$.}


	\classheader{09-18-2017}

\definition{A topological space $X$ is \textbf{first-countable} if there exists a countable base at every point $x\in X$.}

\definition{A topological space $X$ is \textbf{second-countable} if it has a countable base.}

\example{Consider $\mathbb{R}\sqcup\mathbb{R}/\sim$, where $x\sim y$ if $x=y$ and $x,y<1$.  This looks like the line but with two copies of the point $\{1\}$ and a second copy of every point greater than $1$.  This space is not Hausdorff.  If we take an open set at the first copy of $1$, it must intersect any open neighborhood of the other copy of $1$.}

\example{Consider $\mathbb{R}\sqcup\mathbb{R}/\sim$, where $x\sim y$ if $x=y$ and $x,y\leq1$. This looks like the line with one branch extending towards $-\infty$  and two branches extending towards $+\infty$ from $1$.  This space is Hausdorff.  If we take an open neighborhood around $1$, we can see that the inverse image is open only if that neighborhood contains pieces of all three branches, so it isn't possible to have an open set that contains $1+\epsilon_1$ for all $\epsilon_1>0$ on one branch without also having some small neighborhood which also contains $1+\epsilon_2$ for some $\epsilon_2>0$ on the second.}


\section*{Back to Subbases}

Recall that a subbase $\mathcal{S}$ of a set $X$ is a collection of subsets such that $\bigcup\mathcal{S}=X$.

\definition{If $\mathcal{S}$ is a subbase, then \textbf{the topology generated by $\boldsymbol{\mathcal{S}}$} is the set of all arbitrary unions and finite intersections of elements of $\mathcal{S}$.}

The proof that this is in fact a topology is trivial.

\definition{If $\mathcal{S}$ is a subbase, then a base $\mathcal{B}$ formed by the set of all finite intersections of elements of $\mathcal{S}$ plus the set $X$ itself is \textbf{the base generated by $\boldsymbol{\mathcal{S}}$}.}

Again, the proof that this is a proper base is trivial.

\definition{Let $(X,\mathcal{A}_1)$ and $(Y,\mathcal{A}_2)$ be topological spaces with respective bases $\mathcal{B}_1\subset X$ and $\mathcal{B}_2\subset Y$.  Then $X\times Y$ has  a topology called the \textbf{product topology} which is generated by a base $\mathcal{B}_3$ where $W\in \mathcal{B}_3$ if and only if $W=U\times V$ for some $U\in\mathcal{B}_1$ and $V\in\mathcal{B}_2$.  That is, base elements in the product topology are products of the base elements in the factor topologies.}


\example{Consider $\mathbb{R}\times\mathbb{R}$ and let $\mathcal{B}_1=\mathcal{B}_2=\{ (a,b)|-\infty\leq a<b\leq +\infty  \}$ be bases for $\mathbb{R}$ which consist of all open intervals.  Then the base for $\mathbb{R}^2$, $\mathcal{B}_3=\{(a,b)\times(c,d) |(a,b)\in\mathcal{B}_1,(c,d)\in\mathcal{B}_2  \}$ is the set of open rectangles in $\mathbb{R}^2$.  We showed last time (in the general case of prisms in $\mathbb{R}^n$) that this is indeed a base.}

\definition{Let $(X,\mathcal{A}_1)$ and $(Y,\mathcal{A}_2)$ be topological spaces with respective bases $\mathcal{B}_1\subset X$ and $\mathcal{B}_2\subset Y$.  The the subbase $\mathcal{S} = \{ U\times Y | U\in\mathcal{B}_1 \}\cup \{ X\times V|V\in\mathcal{B}_2 \}$ is the \textbf{standard subbase on $\boldsymbol{X\times Y}$}.}

The proof that the standard subbase generates the product topology extends naturally to all product topologies generated by a finite number of factor spaces.

We can think of the product of $k$ copies of $X$ as the set of functions from a set of size $k$ to $X$, $X^k = \{ f:[k]\rightarrow X\}$.

\example{We can think of $\mathbb{R}^3$ as $\{  f: \{1,2,3\}\rightarrow \mathbb{R} \}$.  $(f(1),f(2),f(3))$ is an ordered triple in $\mathbb{R}^3$, and it also completely specifies a function.  We can think of $\mathbb{R}^n$ as $\bigtimes\limits_{[n]}\mathbb{R}=\{ f:[n]\rightarrow\mathbb{R}\}$.  Similarly, the set of all functions from $\mathbb{R}$ to $\mathbb{R}$ is $\mathbb{R}^\mathbb{R}=\bigtimes\limits_\mathbb{R}\mathbb{R}$.}


	\classheader{09-20-2017}

\section*{The Cantor set}

The Cantor set is one of those pathological examples in mathematics.  Consider the interval $C_0=[0,1]\subset\mathbb{R}$.  Given $C_k$, define $C_{k+1}$ as $C_k$ with the middle third of each constituent interval removed.  So $$C_1=[0,\frac{1}{3}]\cup[\frac{2}{3},1]$$ $$C_2=[0,\frac{1}{9}]\cup[\frac{2}{9},\frac{1}{3}]\cup[\frac{2}{3},\frac{7}{9}]\cup[\frac{8}{9},1]$$ and so on.

While we don't have a precise definition for what it means to take a limit to $C_\infty$, we can define $C_\infty=\bigcap C_i$ and this is totally fine from a topological point of view.



\definition{The \textbf{Cantor set} is the name given to $C_\infty$.}

The Cantor set is not empty, it's actually uncountable.  To see this, consider all reals in $[0,1]$ expressed in their ternary expansion.  The Cantor set contains all numbers which do not have any $1$s in this representation.  This is obviously an uncountable set.  We also note that the Cantor set doesn't contain any intervals.  

The Cantor set inherits a subspace topology from $\mathbb{R}$.  It is an exercise on the homework to show that the map $f:C_\infty \rightarrow [0,1]$ where we replace all of the $2$s in the ternary representation with $1$s and interpret it as the binary representation of a real number is a continuous function.

The Cantor set has no interior points, so the complement of the Cantor set is dense and open.

\definition{Denote the \textbf{measure} of the set $C_k$ as $mC_k$. Here we'll use measure as `total length', although in a proper, measure-theoretic sense, this definition isn't quite correct.}

What is the measure of the Cantor set?  The measure of $C_0=[0,1]$ is $1$.  We can define a recurrence, where $mC_k=1-\frac{1}{2}\sum\limits_{i=1}^{k}(\frac{2}{3})^i$.  Then, $mC_\infty = 1-\frac{1}{2}\sum\limits_{i=1}^{\infty}(\frac{2}{3})^i = 1-\frac{1}{2}\frac{\frac{2}{3}}{1-\frac{2}{3}}=0$.

What if at each stage we remove a little less than $\frac{1}{3}$ of each interval?  Say we remove $\frac{\alpha}{3}$, where $0<\alpha<1$.  We still get a Cantor-like set, but here we get a recurrence which looks like $mC_k=mC_{k-1}-2^{k-1}\alpha(\frac{1}{3})^k$.  Here, the measure of $C_\infty$ is $1-\alpha$, but its interior is still empty, so its complement is open and dense.

We can even make a Cantor-like set of full measure by putting a smaller copy of the Cantor set into each gap created by removing an interval.  This set is uncountable, has measure $1$, and is nowhere dense.
	\classheader{09-22-2017}

\section*{Back To Bases}

How can we determine if two bases generate the same topology?

\begin{theorem}
	
	Let $\mathcal{A}$ and $\mathcal{A}'$ be two topologies on the same underlying set $X$ generated by bases $\mathcal{B}$ and $\mathcal{B}'$, respectively.  Then the following are equivalent:
	
	\begin{enumerate}
		\item $\mathcal{A}'$ is finer than $\mathcal{A}$
		\item If $x$ is in $X$ and $x$ is in some base element $B\in\mathcal{B}$, then there exists a $B'\in\mathcal{B}'$ such that $x\in B'$ and $B'\subset B$.
	\end{enumerate}
	
	
	
\end{theorem}

\begin{proof}
	
	Assume that $\mathcal{A}'$ is finer than $\mathcal{A}$ and take some $x\in X$ and some $B\in\mathcal{B}$ such that $x\in B$.  Since $B$ is an open set in $\mathcal{A}$, $B$ is also an open set in $\mathcal{A}'$, so $B$ can be written as the union $\bigcup B'_i$ for some $\{B'_i\}\subset \mathcal{B}'$.  Since $x\in B$, $x$ is in the union of these $B'_i$, so $x$ must be in at least one of the $B'_i$ which by construction is a subset of $B$.
	
	Now, assume that $x\in B\in\mathcal{B}$ implies the existence of some $B'\in\mathcal{B}'$ with $x\in B'\subset B$.  By taking intersections and small neighborhoods, we can necessarily write any such $B$ as a union of some collection of $B'_i$.  But since we can do this, any open set built from elements of $\mathcal{B}$ can be built from elements of $\mathcal{B}'$, so any open set in $\mathcal{A}$ is also open in $\mathcal{A}'$, hence $\mathcal{A}'$ is finer than $\mathcal{A}$.
	
	
	
\end{proof}

\section*{The Product Topology}

\remark{The book uses $\mathbb{R}^\omega$ to denote the set of all sequences in $\mathbb{R}$ indexed by the natural numbers and $\mathbb{R}^\infty$ to denote those sequences which are eventually all zeros.  We'll do our best to be consistent with this.}


\definition{Given a product of topological spaces $\prod X_\alpha$, the \textbf{box topology} is the one with open sets that can be written as a product $\prod U_\alpha$ where $U_alpha$ is open in $X_\alpha$.}

Under infinite products, this topology is too fine.  Consider the function $f:\mathbb{R}\rightarrow \mathbb{R}^\omega$, where $f(t)=(t,t,t,\dots)$.  This function is not continuous from the standard topology to the box topology. To see this, consider the set $(-1,1)\times (-\frac{1}{2},\frac{1}{2})\times (-\frac{-1}{3})\times \dots$ in $\mathbb{R}^\omega$.  This is open in the box topology.  Under $f$, the inverse image of this set is $\{0\}$, which is not open.

\definition{Let $\pi_\alpha(x)$ be the function which sends $x$ in the product to its $\alpha$ coordinate.  This extends to sets by saying that $\pi_\alpha(U)$ is the set of elements $y\in X_\alpha$ such that there exists some $x\in U$ where $\pi_\alpha(x) = y$.  The \textbf{product topology} is defined by the base consisting of the sets $\pi^{-1}_\alpha (U_\alpha)$ where $U_\alpha$ is a base element (or any open set) in $X_\alpha$.  Equivalently, an open set in the product topology is one which is a product of open sets in the factor spaces where all but finitely many are equal to the whole space.}
	\classheader{09-25-2017}

\section*{Metric Topologies}


\definition{A \textbf{metric} on a space $X$ is a function $d:X\times X\rightarrow R_+$ which satisfies:
	
	\begin{enumerate}
		\item $d(x,y)=0$ if and only if $x=y$ (positivity)
		\item $d(x,y) = d(y,x)$ for all $x,y$ (symmetry)
		\item $d(x,z)\leq d(x,y)+d(y,z)$ (triangle inequality)
	\end{enumerate}
	
}

\definition{If $X$ is a space with metric $d$, and $x\in X$, $r>0$, \textbf{the ball of radius $\boldsymbol{r}$ centered at $\boldsymbol{x}$} is the set $\{y|d(x,y)<r\}$.  This is denoted $B_r(x)$ (or $B_x(r)$, $B(x,r)$, $B(r,x)\dots$).}

\definition{The \textbf{topology generated by a metric} is the topology on a metric space generated by the base of all balls of finite radius centered at all points.}



If $x$ is an element of $X$ and $x$ is in some set $U$ which is open in the metric topology, then there is a sufficiently small $r$ such that $B_r(x)$ is entirely contained in $U$.  Hence all points of $U$ are interior.

Let's note that while such a base does indeed generate the topology we want, it's not always the best or smallest base that does so.  For example, the set of all balls of rational radius with rational center generates the standard topology on $\mathbb{R}^n$.

In fact, every metric space is at least first-countable, as we can take a base at a point consisting of the balls centered at that point of rational radius.

Metric spaces are also Hausdorff.  To  see this, consider two distinct points $x,y$.  Since they are not identical, the distance between them is positive, say $3\epsilon$.  Then $B_\epsilon(x)$ and $B_\epsilon(y)$ are disjoint open sets separating $x$ and $y$.

\definition{The \textbf{discrete metric} is the metric $d(x,y) = 0$ if $x=y$ and $1$ if $x\neq y$.}

\claim{The discrete metric generates the discrete topology.  }

\begin{proof}
	
To see this, consider the ball $B_{\frac{1}{2}}(x)$.  This is just the set $\{x\}$, and since the singletons are open, the corresponding topology must be the discrete one.
\end{proof}

\claim{The trivial topology is not generated by any metric.}

\begin{proof}
	The trivial topology is not Hausdorff, so it cannot be a metric topology.
\end{proof}


The Euclidean metric on $\mathbb{R}^n$, $d(x,y)=\sqrt{\sum(x_i-y_i)^2}$ generates the standard topology.

\definition{The \textbf{$\boldsymbol{\ell^p}$ metric} on $\mathbb{R}^n$ is $d(x,y) = \left(\sum|x_i-y_i|^p\right)^{\frac{1}{p}}$.}

\definition{The \textbf{$\boldsymbol{\ell^\infty}$ metric} on $\mathbb{R}^n$ is $d(x,y) = \sup\{|x_i-y_i|\}$.}

\claim{All of the $\ell^p$ metrics generate the same topology on $\mathbb{R}^n$.}

\begin{proof}
	Pick two metrics $\ell^p$ and $\ell^q$ with $p<q$.  We'll argue that base elements (balls) in one topology contain base elements of the other.  Containment one way is trivial.  For a fixed radius $r$, the $\ell^p$ ball of radius $r$ sits inside of the $\ell^q$ ball of radius $r$.  To see containment the other way, consider the $\ell^q$ ball of radius $r_1$ and think of it as an open set in the $\ell^p$ topology.  There is some point of maximum distance from the center, and if we pick some radius $r_2$ less than this, the ball will sit inside of the $\ell^q$ ball, and we're done.
\end{proof}


We call $\ell^p(\mathbb{R}^\omega)$ the set of sequences whose $\ell^p$ norm is finite.  This forms a vector space.
	
	\end{document}
	
