\classheader{11-17-2017}

Homotopy is a more general quality than homeomorphism.  Whereas a homeomorphism requires the existence of a continuous bijection with continuous inverse, we say that $X$ and $Y$ are homotopy equivalent if there exist continuous functions $f:X\rightarrow Y$ and $g:Y\rightarrow X$ such that $f\circ g$ is homotopic to the identity on $Y$ and $g\circ f$ is homotopic to the identity on $X$.

For example, the closed disk and a single point are homotopic, but obviously not homeomorphic.  Similarly, $\R^n$ is homotopic to a single point.  Also, $\R^n{-}\{0\}$ is homotopic to $\S^{n-1}$ by the maps $f(x)=x/||x||$ and $g(y)=y$.

Homotopy equivalence is an equivalence relation.  We've seen how it's obviously reflexive and symmetric, but it is transitive as well, via composition of functions.

\begin{theorem}
	The fundamental group $\pi_1$ is a homotopy invariant.  That is, if $X$ and $Y$ are each path connected and homotopy equivalent to each other, their fundamental groups are isomorphic.
\end{theorem}
\begin{proof}
	We proceed in two steps:
	
	Step I:
	
\end{proof}
