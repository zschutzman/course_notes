\phantomsection \addcontentsline{toc}{section}{Monderer and Samet (1989)}
\chno{4}{\textit{Approximating CK with Common Beliefs}, Monderer and Samet (1989) }{Annie Liang}{Zach Schutzman}{January 10, 2018}

\section*{Overview}
\begin{enumerate}
	\item What should we mean when we say \textit{``almost common knowledge"}?
	\item Rubenstein (1989) looks at `knowledge which is almost-common'
	\item This paper looks at `common almost-knowledge'
\end{enumerate}

\section*{Main Contribution}
\begin{enumerate}
	\item Weakens common knowledge to a notion of \textit{common $\mathit{p}$-beliefs}
	\item Shows that equilibrium predictions under common $p$-belief are close to those under common knowledge when $p$ is large
\end{enumerate}

\section*{Quick Review of the Standard Model}
\begin{enumerate}
	\item[-] $I$ - the set of all agents
	\item[-] $(\Omega,\Sigma,\mu)$ - a probability space
	\item[-] $\pi_i$ - the partition of $\Omega$ into measurable sets with positive measure of Player $i$
	\item[-] $\mathsf{F}_i$ - a $\sigma$-field generated by the collection of $\pi_i$
	
\end{enumerate}
The `knowledge' of Player $i$ at the event $E$ means $K_i(E)=\{\omega : \pi_i(\omega)\subset E\}$

We can define \textit{common knowledge} in the following two equivalent ways:


\begin{enumerate}
	\item[A)] Let $\mathsf{C}^1$ be the set of states at which every agent knows $E$
	
	 Recursively let $\mathsf{C}^n$ be the set of states at which every agent knows $\mathsf{C}^{n-1}$.
	
	Let $\mathbb{C} (E)=\bigcap\limits_{n\geq 1}\mathsf{C}^n$.  Then $E$ is \textit{common knowledge at $\mathit{\omega}$} if and only if $\omega\in \mathbb{C}(E)$.\\
	
	\item[B)] Call $E$ \textit{evident knowledge} if $E\in K_i(E)$ for all $i$.  In other words, $E$ is evident knowledge if whenever $E$ occurs, every agent knows that it has.  An event $F$ is \textit{common knowledge at $\mathit{\omega}$} if there exists an evident knowledge event $E$ such that $\omega\in E$ and for all agents $i$, $E\subset K_i(F)$.
\end{enumerate}

\section*{New Framework}

In the new framework, we'll let $I$, $(\Omega,\Sigma,\mu)$, and $\pi_i$ be as before.

\definition{Player $i$ \textbf{$\mathbf{p}$-believes an event $\mathbf{E}$ at $\mathbf{\omega}$} if $\mu(E|\pi_i(\omega))\geq p$. }

As a special case, if we take $p=1$, this is almost, but not exactly the same as, knowledge.

