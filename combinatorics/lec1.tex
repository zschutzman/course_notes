	
	\classheader{}
	
	\section*{What is Combinatorics?}
	
	Combinatorics is the mathematics of counting.  Suppose we have some $n\in\mathbb{N}$ and a set $S$ of objects which somehow depend on $n$.  Combinatorics addresses the question "How many objects are in $S$?"  More formally, this is a function $f: \mathbb{N}\rightarrow \mathbb{N}$ which counts the number of objects in $S$ as a function of $n$.  What do we know about $f$?
	
	\example{Let $S_1$ be the set of binary sequences of length $n$.  Then $f(n)=2^n$.}
	\example{Let $S_2$ be the symmetric group on $n$ elements.  Then $f(n)=n!$.}
	
	\definition{A \textbf{derangement} is a permutation in the symmetric group which has no fixed points.}  
	\example{The set of derangements on $n$ elements, $D_n = \{\sigma\in S_n |\  \sigma(k)\neq k \  \forall k\leq n\}$, has size  $\#D_n = n!\sum\limits_{i=0}^{n} \frac{(-i)^i}{n!}$. }
	
	\example{Suppose we have a $2\times n$ board, which we want to tile with $2\times 1$ dominoes.  How many different ways are there to do this?  If $S = \{proper \ domino \ tilings \ of \ a \ 2\times n \ board\}$, then we have $\#S=F_n$, the $n$th Fibonacci number.}
	
	
\begin{center}	
\begin{tikzpicture}[line cap=round,line join=round,>=triangle 45,x=1.0cm,y=1.0cm]
\clip(-6.08,-1.94) rectangle (9.52,7.2);
\fill[line width=2.pt,color=ffffff,fill=ffffff,fill opacity=1.0] (-3.2,5.42) -- (1.8,5.42) -- (1.8,3.68) -- (-3.2,3.68) -- cycle;
\draw [line width=2.pt] (-3.2,5.42)-- (1.8,5.42);
\draw [line width=2.pt,color=ttqqqq] (-3.2,5.42)-- (1.8,5.42);
\draw [line width=2.pt,color=ttqqqq] (1.8,5.42)-- (1.8,3.68);
\draw [line width=2.pt,color=ttqqqq] (1.8,3.68)-- (-3.2,3.68);
\draw [line width=2.pt,color=ttqqqq] (-3.2,3.68)-- (-3.2,5.42);
\draw [line width=2.pt] (-2.34,5.42)-- (-2.34,3.68);
\draw [line width=2.pt] (-1.5,5.42)-- (-1.5,3.68);
\draw [line width=2.pt] (0.96,5.42)-- (0.96,3.68);
\draw [line width=2.pt] (-3.2,4.54)-- (1.8,4.54);
\draw [line width=2.pt] (-3.2,5.42)-- (-2.34,5.42);
\draw (-0.66,5.1) node[anchor=north west] {$\dots$};
\draw (-0.66,4.2) node[anchor=north west] {$\dots$};
\draw (-3.62,6.32) node[anchor=north west] {$\overbrace{\ \ \ \ \ \ \ \ \ \ \ \ \ \ \ \ \ \ \ \ \ \ \ \ \ \ \ \ \ \ \ \ \ \ \ \ \ \ \ \ \ \ }$};
\draw (-3.94,5.36) node[anchor=north west] {$\begin{cases} \ \ \\\ \\ \ \\  \end{cases}$};
\draw (-1.,6.74) node[anchor=north west] {$n$};
\draw (-4.32,4.42) node[anchor=north west] {$2$};
\end{tikzpicture}
\end{center}
	
\section*{Generating Functions}

\definition{A \textbf{generating function} corresponding to some counting function $f$ is an element of the ring of formal power series $\mathbb{C}[[x]]$ where the coefficient of the $x^n$ term is $f(n)$.}


 If $F$ and $G$ are two generating functions, than we have $F(x) = G(x) \iff f(n)=g(n)$ for all $n\in \mathbb{N}$.  We can do addition with generating functions, where we add the corresponding coefficients.  $F(x)+G(x) = f(0)+g(0) + (f(1)+g(1))x + (f(2)+g(2))x^2 \dots $.  We define multiplication as $F(x)\cdot G(x) = \sum\limits_{n=0}^\infty(\sum\limits_{m=0+^n}f(m)g(n-m))x^n$.  
 
 Generating functions obey many of the properties of series that we learned in Calculus, except that we don't worry about these things converging.  If $f(n)=1$ for all $n$, then the generating function is $F(x)=1+x+x^2+\dots$, which equals $\frac{1}{1-x}$.  Similarly, if $f(n)=\alpha^n$ for all $n$, then the generating function is $F(x)=1+\alpha x + \alpha^2 x^2 +\dots$, which equals $\frac{1}{1-\alpha x}$.  These look like geometric series from Calculus.
 
 \example{Let $F(x)$ be the generating function for the Fibonacci numbers.  By the Fibonacci recurrence, we can rewrite this as $F(x) = F_0+F_1x+(F_0+F_1)x^2 + \dots + (F_{n-2}+F_{n-1})x^n + \dots = F_0 + F_1x+F_0x^2+F_1x^2 + \dots$.   Factoring out, we can rewrite $F(x) = 1+xF(x) + x^2F(x) = \frac{1}{1-x-x^2}$.}
 
 \section*{Sets and Multisets}
 
 Let $S = \{x_1,x_2,\dots,x_n\}$ be a set of $n$ objects.  The number of distinct subsets of $S$ of size $k$ is $\binom{n}{k}$, and is called the binomial coefficient.
 
 \theorem{$ \binom{n}{k} = \frac{n(n-1)(n-2)\dots (n-k+1)}{k!}$}
 \begin{proof}
 	
 	We can think of $\binom{n}{k}$ as the number of subsets of size $k$ on a set of size $n$, and $k!$ as the number of ways of ordering $k$ objects.  Then, $\binom{n}{k}\cdot k!$ represents the number of ordered sequences (without repetition of elements) of length $k$.  We can also think about this as choosing each of the $k$ elements in order.  There are $n$ choices for the first, $n-1$ for the second, $n-2$ for the third, and so on, down to $n-k+1$ for the $k$th.  We therefore have $\binom{n}{k}\cdot k! = n(n-1)(n-2)\dots (n-k+1)$.  Moving the $k!$ to the denominator of the righthand side completes the proof.
 	
 \end{proof}
 
 \definition{A \textbf{multiset} is a collection of objects, like a set, which allows objects to occur with some multiplicity greater than one.}
 
 If we denote the natural numbers $1,2,\dots, n$ as $[n]$, then the number of multisets of size $k$ is denoted $\multiset{n}{k}$.
 
 \theorem{$\multiset{n}{k}=\binom{n+k-1}{k}$}
 \begin{proof}
 	
 	Observe that if we have a multiset on $[n]$, we can, without loss of generality, arrange it in increasing order.  The set looks like $\{a_1\leq a_2 \leq \dots \leq a_k  \}$.  We can map each such multiset to a unique set by adding 0 to the first element, 1 to the second, 2 to the third, and so on, up to adding $k-1$ to the last element.  To see that this is a unique mapping, we can look at the inverse, where we take a set on $[n+k-1]$ and sort it in increasing order $\{ b_1 < b_2 < \dots < b_k\}$, then subtract 0 from the first element, 1 from the second, and so on, up to subtracting $k-1$ from the last element.  Since this creates a bijective mapping between multisets of size $k$ on $[n]$ and sets of size $k$ on $[n+k-1]$, we have $\multiset{n}{k}=\binom{n+k-1}{k}$ as desired.
 	
 	
 	
 \end{proof}
 
 \section*{Compositions}
 
 \definition{A \textbf{composition} $\alpha = a_1,a_2,a_3,\dots$ of a natural number $n$ is an ordered multiset of natural numbers such that $\sum\alpha_i = n$.}
 
 \definition{A \textbf{k-composition} of a natural number $n$ is a composition of $n$ into $k$ parts.}
 
 \example{The compositions of $4$ are $(4),(3,1),(1,3),(2,2),(2,1,1),(1,1,2),(1,2,1),(1,1,1,1)$.}
 
 How many $k$-compositions of $n$ are there?
 
 \theorem{There are $\binom{n-1}{k-1}$ $k$-compositions of $n$.}
\begin{proof}
	We proceed combinatorially.  Imagine a string of $n$ 1's.  Between each, we can place a plus, indicating we should add those two (or more) adjacent 1's together to make a larger piece, or a comma, indicating that we should separate these two adjacent 1's into separate components. There are $n-1$ spots between the 1's, and we need to place $k-1$ commas to create a composition into $k$ parts.  There are clearly $\binom{n-1}{k-1}$ ways to do this, and we are done.
\end{proof} 


