
%\lecture{**LECTURE-NUMBER**}{**DATE**}{**LECTURER**}{**SCRIBE**}
\chno{1}{Situating Redistricting}{Moon Duchin (Tufts)}{Zach Schutzman}{August 7th, 2017}
%\footnotetext{These notes are partially based on those of Nigel Mansell.}

% **** YOUR NOTES GO HERE:

% Some general latex examples and examples making use of the
% macros follow.  
%**** IN GENERAL, BE BRIEF. LONG SCRIBE NOTES, NO MATTER HOW WELL WRITTEN,
%**** ARE NEVER READ BY ANYBODY.




\section*{Congressional Representation}
Constitutionally mandated, allocations according to decennial Census.


There are issues:

\begin{enumerate}
\item[] Census-taking isn't straightforward or perfect
\item[] Apportionment isn't straightforward or perfect - how many reps should we have for each state?
\item[] Drawing districts isn't straightforward or perfect - this is the topic of the week

\end{enumerate}









Mathematically, we are interested in partitioning a population with attributes

We have a population of nodes, each with attributes

We want to partition the sets into blocks (``districts") and think about how the attribute patterns at the district level compare to that at the population model.

In practice, we also have geographic features to think about ($S^2$ embedding).

\subsection*{What are the goals?}

We can think about proportionality - can we get the districts to ``look like" the population at large?

We can gerrymander! - can we maximize/minimize the incidence of some attribute at the district level?

\subsection*{What are the constraints?}

Districts must be (very nearly) equipopulous

Districts should be contiguous and non-punctured

Districts shouldn't be weirdly shaped (!)

\subsection*{Math v Politics}

Any goal or constraint represents a mathematization of a normative ideal of politics

\begin{enumerate}
	\item[] Equal population - representational equality (one person - one vote)
	\item[] Geographical division - bare majorities shouldn't dominate (appeal to the Law of Large Numbers - if people are assigned to a district randomly, a scant majority should make scant majorities in each district)
	\item[] Shape - may indicate gerrymandering or some other extreme agenda
	\item[] Proportionality - government should reflect the populace
	\item[] Competitiveness - elections should be ``fair"
	\item[] Partisan favor - prevent government deadlock
\end{enumerate}

The latter three of these are not encoded in the law.

\section*{How to Gerrymander}

\subsection*{Packing and Cracking}

\defn{\textbf{Packing} is the act of creating a few districts with a strong majority of individuals with a certain attribute.}
\defn{\textbf{Cracking} is the act of spreading out individuals with a certain attribute across several districts so as to make them a minority in those districts}.

Together, Packing and Cracking looks like taking members of one group and making a few districts where they have a strong majority and many where they are a scant minority so as to minimize that group's representation at the district level.

\subsection*{Evaluating Shapes}

Intuitively, any weird agenda should make weird looking districts.

At the legal level, districts are usually only stipulated to be ``reasonably compact".  What does that mean?

Mathematically, there are numerous definitions for compactness.
\subsection*{Compactness}

\subsubsection*{Isoperimetry}
\defn{\textbf{Isoperimetry} is a measure of how close to being circular a region is. The  Poslby-Popper score is $0\leq 400\frac{\pi A}{P^2} \leq 100$ and is one way to measure this.}

  This is weak because ``perimeter" isn't really a thing.  We have a Coastline Paradox effect at play.

\subsubsection*{Convexity}

We can look at the convex hull of a district and see how far the district deviates from this.  Also not great, because there are some very good reasons for nonconvexity.

\subsubsection*{Dispersion}

Look at things like moment of inertia or how spread the district is.  The failings of this are a little more subtle, well-detailed in the literature.
	
	
All of this is based on old mathematics.

Courts have discarded maps based on ``weird" shapes, but there is no standard.  This is the ``Eyeball Test".

\subsection*{Race as Issue}

People of Color tend to vote for Democrats. We have to think about the issue of race proxying for partisan allegiance.

 Large cities tend to vote for Democrats.  Since cities are populous and dense, we need to think carefully about how we divide cities.  There are strong correlations between method of commuting and Presidential vote in 2016.  Three of the top 40 largest cities voted for Trump in 2016 (OKC, Mesa, AZ, Colorado Springs).  Even cities in red states go blue.
 
 The IL-4 has two neighborhoods joined by a highway (zero population, of course).  The northern chunk is a Puerto Rican neighborhood and the southern chunk is a Mexican neighborhood.  This might look like an instance of packing, but it was actually done in order to give these two Latino populations the ability to pick their own representative.
 
\subsection*{Density and Splittability}

Density of population creates ``natural gerrymandering" (Chen and Rodden).  How you draw the lines in and around cities decides how much packing and cracking you end up doing.  In this sense, density and shape both contribute to how easy or hard it is to draw nice lines.


\subsection*{Thinking about District ``Guts"}

What abstraction should we be thinking about to capture the information we care about?

First, note that our data is discrete - we get block-level data from the census, and individual people are obviously discrete units.

We can think about Census units like vertices in a graph - but what are the edges?

\begin{enumerate}
	\item[] Adjacency - the spatially obvious thing to do
	\item[] Distance or travel time
	\item[] Commonalities - edge between blocks that ``look similar"
\end{enumerate}

\subsection*{Curvature as an Approach to Compactness}

Graphs have shape, which reveals something about both isoperimetry and dispersion.  What if you try to build you district out of a sheet of paper, with a face for each block.  This shape will have (discrete) \textbf{curvature} which tells us something about the geometry of the distric.