
%\lecture{**LECTURE-NUMBER**}{**DATE**}{**LECTURER**}{**SCRIBE**}
\chno{5}{Race and Redistricting}{Ellen Katz (UMich)}{Zach Schutzman}{August 8th, 2017}
%\footnotetext{These notes are partially based on those of Nigel Mansell.}

% **** YOUR NOTES GO HERE:

% Some general latex examples and examples making use of the
% macros follow.  
%**** IN GENERAL, BE BRIEF. LONG SCRIBE NOTES, NO MATTER HOW WELL WRITTEN,
%**** ARE NEVER READ BY ANYBODY.


\section*{Modern History of Race and Voting}

In Tuskegee, AL (1957), city lines were redrawn by the State Legislature to excise the Tuskegee Institute, removing many Black voters from the city.  Lower courts ruled that legislatures have authority to redraw lines, but SCOTUS reversed the decision, finding that the purpose of drawing lines was explicitly racial in a way that disenfranchised Black voters, violating the 15th Amendment.  In this case (and \textit{Carr} two years later) established authority and precedent for courts to intervene in the redistricting process.

There is a tension in modern political discourse over majority-minority districts, as there is the question of whether they give minorities the ability to elect a candidate of choice or they are a form of packing minority voters.

In the early 1960s, voter registration in Alabama was heavily White-dominated.  Related protests and marches led to Johnson passing civil rights legislation, including the VRA.  Certain criteria subjected a jurisdiction to the Sec. 5 federal pre-clearance provisions discussed by Kristen Clarke (see Talk 1.3).  This was an ``intrusive remedy", as it flipped the presumption of legislative action being ``good until proven otherwise" to one of ``bad until proven otherwise" in the areas covered by Sec. 5.  Following the VRA, Black registration and participation shot up almost immediately.

This led to a discussion about how political participation isn't just free voting access, and we needed to think about how district lines are drawn.  SCOTUS heard cases about racial vote dilution. They found that just because you \textit{can} draw a majority-minority district doesn't mean that you have to, that multimember districts in Dallas County, TX violated the Equal Protection Clause, and that evidence of intentional discrimination is necessary to make a claim of disenfranchisement.  Congress responded by amending the VRA to include a results-based provision for claiming discrimination.

In 1982, Justice Brennan in \textit{Gingles} argued that a numerous minority group living in a sufficiently compact area which voted in a cohesive bloc and the majority group consistently voted against this minority constituted a sufficient test for violation of the new Sec. 2 provisions.  Within a few decades, the number of African-Americans in elected positions rose to historic levels.  In \textit{Holder}, a county which had a single elected leader and a 20\% Black minority discussed changing their government to a five-member group.  Thomas and Scalia decide that racial vote dilution isn't a thing and dissent in this case and all other racial voting cases.  In a case arguing that Florida violated Sec. 2 by not drawing enough majority-minority districts.  Souter argued that since the number of majority-minority districts was roughly equal to the proportion of minorities in the population, there were enough. He argued that majority-minority districts are a ``second-best'' solution, as in an ideal world, voters could form coalitions and a person being a member of a minority group doesn't preclude their ability to get elected.

\section*{Recent Events}

In the \textit{Shaw} cases in the 1990s, voters argued that the North Carolina districting was so irregular that there was no sufficient justification for it.  The courts find in \textit{Shaw} that the gerrymander was not much different than the Tuskegee case and that it was not acceptable.  The courts said they would apply strict scrutiny to districting, where a strong cohesive argument is necessary to justify a districting.  

The debate turned to the balance between forming majority-minority districts and racial gerrymandering in the 2000s.  District-drawers decided they would use party, rather than race, to gerrymander.  This cut both ways.  In \textit{Georgia v Ashcroft}, Democrats ``unpacked" majority-minority districts to do a Democrat-favored partisan gerrymander.  SCOTUS found that this was acceptable, as forming coalitions with White Democrats would still be a possible route to electing a candidate of choice.  Congress overturned this when it reauthorized the VRA.

The court found in a Texas redistricting effort which, mid-decade, redrew lines around Laredo, dismantling a majority Latino district, that the state violated the VRA.  The Roberts court takes a much narrower view of racial discrimination in voting, culminating in \textit{Shelby County v Holder}, which struck down Sec. 5 of the VRA and opened the floodgates for some of the modern techniques of disenfranchisement. A new challenge to Sec. 2 may result in further dismantling of the VRA.
