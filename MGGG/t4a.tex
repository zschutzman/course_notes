
%\lecture{**LECTURE-NUMBER**}{**DATE**}{**LECTURER**}{**SCRIBE**}
\chno{4}{Mathematics - Compactness and Curvature}{Moon Duchin (Tufts)}{Zach Schutzman}{August 8th, 2017}
%\footnotetext{These notes are partially based on those of Nigel Mansell.}

% **** YOUR NOTES GO HERE:

% Some general latex examples and examples making use of the
% macros follow.  
%**** IN GENERAL, BE BRIEF. LONG SCRIBE NOTES, NO MATTER HOW WELL WRITTEN,
%**** ARE NEVER READ BY ANYBODY.





\section*{Discrete Curvature}

\defn{$\delta$-hyperbolicity is a large-scale, metric version of sectional curvature.  Formally, a space is $\delta$-hyperbolic if the sides of every geodesic triangle are within a $\delta$-neighborhood of each other at all points along the sides.}  

We can think of this as a measurement-based analysis of how ``thin" triangles are.  This has a nice correspondence with the concept of negative curvature.  $\delta$ is a measure of how "thin" the triangles are.  This comes from geometric group theory, and we'd like to figure out how to apply this kind of thing to finite graphs.

\defn{\textbf{Ricci curvature} asks whether the distance between the neighbors of $x$ and the neighbors of $y$ is greater than the distance between $x$ and $y$ for two points $x,y$ in a metric space.}

In the discrete case, we lose the nice pairwise correspondence between points.  Instead, we think about an $L_1$ transportation distance from neighbors of $x$ to neighbors of $y$ (the cost to move one unit of weight one unit of distance is one).

\defn{The \textbf{transportation distance} is the minimum over all transport plans.} 

\subsection*{Inspirational Theorems}

\thrm{Gromov: (For infinite groups) A region is $\delta$-hyperbolic if and only if its area and perimeter grow at the same rate.  Otherwize, area grows at least proportionally to the square of the perimeter.}

\thrm{Duchin:(For infinite groups) If a ball is $\delta$-hyperbolic, then the average distance between two points is proportional to the maximum distance between any two points.}



We'd like to prove large-scale (but finite) theorems about the discrete setting.


Bottlenecks in graphs could create negative curvature, meaning that negative curvature edges are good ones to cut when we're drawing our districts.


\section*{Optimal Partitioning}

Two Morals:
\begin{itemize}
	\item Eigenvalues tell you a lot of things about geometry
	\item Optimal partitioning is computationally hard and not something that produces good districts
\end{itemize}

What if we try to minimize the first eigenvalue $\lambda_1$ of the laplacian over the region?  This does have a nice connection to curvature (bounds on curvature imply bounds on the eigenvalue).

\thrm{Faber-Krahn: A ball minimizes $\lambda_1$.}

Large $\lambda_1$ corresponds to high eccentricity (in the elliptical sense).

There is also a notion for graphs, where a graph gradient looks like a discrete difference between adjacent points (we'll take squares).  Boundaries of a subset correspond to those nodes outside the subset adjacent to ones in the subset.

\defn{A collection of regions $\Omega_1\dots \Omega_n$ is an optimal partition of $\Omega$ if the $\Omega_i$ is a proper partition, each is of equal size, and  $\sum\lambda_i$ is minimal across all proper equal size partitions.}

In the redistricing setting, population plays the role of ``area".

\thrm{Ramos-Tavres-Terracini: The minimizer always exists and we get nice regularity properties, but we don't get the property that our districts are connected.}

\thrm{Quantitative Stability: If your domain $\Omega$ is the same size as the unit ball, then $\lambda_1$ of $\Omega$ tells you how close to being a ball $\Omega$ is.}
