
%\lecture{**LECTURE-NUMBER**}{**DATE**}{**LECTURER**}{**SCRIBE**}
\chno{1.3}{Voting Rights Litigation}{Kristen Clarke}{Zach Schutzman}
%\footnotetext{These notes are partially based on those of Nigel Mansell.}

% **** YOUR NOTES GO HERE:

% Some general latex examples and examples making use of the
% macros follow.  
%**** IN GENERAL, BE BRIEF. LONG SCRIBE NOTES, NO MATTER HOW WELL WRITTEN,
%**** ARE NEVER READ BY ANYBODY.

\section*{The Voting Rights Act}

Yesterday was the 52nd anniversary of the VRA (1965).  The VRA is an important part of the context and history of the issues we'll talk about this week.

The VRA is one of the most important piece of civil rights legislation.  It followed the incidents at the march from Selma to Montgomery and directly targeted racial discrimination in voting, banning literacy tests, grandfather clauses, and other tools of disenfranchisement.  Certain states (mostly the South, but also NY and CA) were also required to get federal pre-clearance for passing voting legislation.

Section 2 of the VRA protects minority voters' equal opportunity to elect a representative of their choice, i.e. create "minority-majority" districts.

In 2006, Congress reauthorized the VRA for 25 years, including Sec. 5, by a wide majority.  Things like packing and cracking of minorities and canceling a community election to prevent African-Americans from running for town council and mayor are examples of things struck down under the authority of Sec. 5.

Groups opposed to policies which attempt to correct historical wrongs, such as the VRA or race-conscious admissions policies, challenged the VRA.  In 2009, SCOTUS (\textit{Austin Municipal Util. Dist. No. 1 v Holder}), questioned, but did not strike down, Sec. 5.  In \textit{Shelby County v Holder} (2013) made a direct Constitutional challenge to Sec. 5.  SCOTUS found that the coverage formula, used to determine which states were subject to Sec. 5, was unconstitutional, striking it down.  This verdict opened the floodgates for a lot of the voter suppression we are seeing today, such as ID requirements.  

Experts found that 600,000 people were disenfranchised the day this verdict passed, largely poor (disproportionately minorities).  Costs of getting an ID are \$20+, which is significant for people living below the poverty line.  North Carolina cut early voting, eliminated pre-registration for teenagers, eliminated same-day registration, and made absentee voting more difficult, and this law would likely have been blocked by Sec. 5.  A Court of Appeals found that this law discriminated against minorities with near "surgical precision" after examining how the State used data like Black voters huge participation in early voting to inform how it made its restrictions.

Today's Congress is fairly unproductive, and it does not look to be a fruitful avenue for protecting voting rights.  Progress is being made case-by-case in court challenges.  Unfortunately, in order to make a court case, you need evidence that the discrimination is occurring, which entails having to live under these repressive laws.

In Texas, 25\% of Blacks, compared to 8\% of Whites, do not have an ID valid for voting.  As an example, concealed carry permits are valid (held disproportionately by White men, but student IDs do not).

This doesn't just affect Congressional elections, but also state- and municipal-level elections which also use redistricting procedures.  Over 8,000 jurisdictions will engage in redistricting for the first time without Sec. 5.

Fears of vote fraud are being used to justify implementing restrictive laws.  Claims that undocumented Americans are voting illegally are an unfounded but powerful tool used to create support for these laws.  Recently, the Election Integrity Commission (chaired by Kansas Secy. of State Kobach) is being used to promote voter suppression laws.  CrossCheck, a process which checks if people are voting in more than one place, has been a big thing with Kobach, but the program has been found to have a 99\% error rate in matching people registered in more than one place.


\section*{Redistricting}

There is no cookie-cutter approach to redistricting that will solve all of the issues with discrimination simultaneously, and solutions must involve careful analysis of data, laws, and voting patterns.

Typically, redistricting occurs decennially, following the Census.  Recently, some states have been doing "mid-decade redistricting".  A suit recently has been brought against Georgia for a 2015 redistricting plan which carves out minorities in places where there have been demographic shifts.  As an example, State District 105 had a 550 vote margin in 2012, with the White incumbent barely edging out the Black challenger.  The state redrew the lines to widen the margin protecting the incumbent.  

Having representation which reflects the populace is important for social justice issues, particularly at the local level.  School boards and city councils dictate important factors regarding education and policing, both areas where issues of discrimination pop up.

We also need to think about trade-offs created by constructing majority-minority districts.  Doing so does provide important political power to these minority voters, but it comes at the cost of possibly creating a plan which packs minority voters and/or packs Democrats (given current political trends).

Another important question is how to handle incarcerated citizens.  Where do you count them as residents for apportionment, whether or not they can vote, and the racial makeup of the prison population are all important things to think about when drawing district lines and figuring out apportionment.