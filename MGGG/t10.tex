%\lecture{**LECTURE-NUMBER**}{**DATE**}{**LECTURER**}{**SCRIBE**}
\chno{10}{DistrictBuilder}{Bob Cheetham (Azavea)}{Zach Schutzman}{August 9th, 2017}



Azavea is a small software company in Philly which uses geospatial data for civic and social impact.  Their tool DistrictBuilder is available as an open source piece of software.  Their Summer of Maps program matches companies with geospatial projects to students with those skills.



\section*{The Software}
DistrictBuilder came about kind of accidentally.  Azavea's Cicero database of elected officials naturally has a spatial component, as it needs to track legislative districts and how they change.  A research project on evaluating how bad Philly-area gerrymandering actually was led to producing a white paper and development of this tool.

Researchers interested in automated redistricting worked with Azavea to create an open source, web-based, easy-to-use, non-partisan, map generating tool.  The software allows users to create, edit, and save multiple district plans, use blank, existing, or template maps to design plans, import and merge plans from other systems such as GIS software, display data, existing political boundaries, and demographics, automatically calculates relevant statistics, and integrates with other mapping systems like GoogleMaps, and ArcGIS.

DistrictBuilder has been used at the state and local level for public mapping efforts, such as the Philadelpia-area competition Fix Philly Districts.  One of the teams in the Virginia competition  created a plan which was used as a basis for a legal challenge against a racial gerrymander. In Minneapolis, Latino and Somali community groups drew districts to serve their local communities, and commissioners adopted these districts into the plan.  Fix Philly Districts was an unofficial project born out of the government's reluctance to engage on the issue of unfair redistricting.  Azavea ran a competition, where plans were scored on how many city wards were split, and it got so much attention that the city reengaged in the process.

New Census APIs and improvements to distributed architectures present new opportunities for running software like this at-scale.