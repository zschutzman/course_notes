
%\lecture{**LECTURE-NUMBER**}{**DATE**}{**LECTURER**}{**SCRIBE**}
\chno{9}{A Legal and Conceptual Primer on Political Gerrymandering}{Guy-Uriel Charles (Duke)}{Zach Schutzman}{August 9th, 2017}
%\footnotetext{These notes are partially based on those of Nigel Mansell.}

% **** YOUR NOTES GO HERE:

% Some general latex examples and examples making use of the
% macros follow.  
%**** IN GENERAL, BE BRIEF. LONG SCRIBE NOTES, NO MATTER HOW WELL WRITTEN,
%**** ARE NEVER READ BY ANYBODY.


\section*{Why Does the Court Think About the Problem the Way It Does?}

One of the fundamental problems of American democracy is the structure of our elections and government

The big question is whether political gerrymandering cases are justiciable (can the courts hear them)? The Constitution doesn't give federal courts the authority to do so explicitly, and the devolution of rights to the states suggests that gerrymandering issues should be resolved at the state level rather than the federal level, and tradition and history has put few limitations on the ability of states to design electoral structures.

Additionally, the Constitution puts very few limits on the states.  Limits on state elections are pretty much limited to the 14th, 15th, 19th, and 26th Amendments, as well as precedent from \textit{Baker v Carr}, which upholds the principle of one-person-one-vote, derived from the 14th Amendment.  For federal elections, we have the 14th, 15th, 19th, and 26th Amendments, plus \textit{Baker} as before, plus the 24th Amendment preventing poll taxes, but there is no affirmative right to vote in federal elections.  Additionally, states have some control over federal elections.  Each state can regulate elections, and the 17th Amendment affects election of Senators.

There are a few federal laws imposing limitations, such as the VRA, most importantly Sec. 2 and Sec. 5, the National Voter Registration Act (1993), which requires voter registration be available when applying for a driver's license, and the Help America Vote Act (2002), which sets someelection administration standards.  However, the terrain is largely lacking in federal and Constitutional constraint on elections.  

The Court views redistricting as a vital state function, inherently political, and a domain where the federal government's role is limited.  

\section*{The Anatomy of \textit{Davis v Bandemer} (1986)}

Indiana Democrats received 51.9\% of the vote but only 43 of 100 seats in the lower house, and made a claim of vote dilution.  The majority of the Court held that political gerrymandering are justiciable, and a plurality held that the district court applied an incorrect and not sufficiently high standard.  The Court agrees the cases are justiciable, but they can't agree \textit{why}.

The big question is whether there are judicially manageable and discernible standards.  In other words, is there a Constitutional theory of harm which grants the federal courts the ability to address these cases?  In other other words, is this an issue of constitutionality or one of courts setting policy?

Initially, we should think about these standards as a legal and constitutional question about whether the Constitution gives the Court the tools and ability to tackle this problem.  In \textit{Bandemer}, the Court relies on reapportionment cases, seeing that malapportionment and political gerrymandering both infringe on fair representation, the Court devised the one-person-one-vote principle as a result of malappotionment cases, and the Equal Protection Clause protects the right to fair representation in state legislatures.  The majority opinion is based on racial gerrymandering precedent, as both cases are about vote dilution (cracking Black constituencies and cracking Democrat constituencies).

The majority falls apart in thinking about the standard to apply.  The plurality opinion is that plaintiffs must prove that discrimination was intentional and that the results of the actions had a discriminatory effect.  The Court says that intention of discrimination is a very low bar, stating that whoever is drawing the districts presumably know what the political breakdown of the representatives will be.  The plurality also accepts that discriminatory effect is part of the political process and does not sufficiently form a constitutional violation.  One single election is not a large enough sample size to prove a consistent degradation of voters' ability to influence the political process.

Justice O'Connor write an opinion concurring with the judgment and stated that partisan gerrymandering cases for major political parties are non-justiciable, and the logic of the plurality opinion leads down the slippery slope to proportional representation, which is clearly not required by the Constitution.  Since major political parties have the power to defend themselves in the political process, they are not analogous to racial groups.

Justice Powell concurred with the judgment but dissented from the plurality's standard.  He agrees that intentional discrimination and discriminatory effect are required, and that both standards were met in this case.  However, he states that discriminatory effects should be determined by the shape of the districts, deviation from traditional districting principles, and to what extent the plaintiffs were locked out of the legislative and political process, such as whether the minority party was allowed to participate in the districting process.

\section*{Political vs Racial Gerrymandering}

Political and racial discrimination are different issues.  First, racial identities are protected by the 14th Amendment, whereas there is no constitutional protection for political party identification, and the 14th Amendment applies to both benign and malicious classifications.  Additionally, racial minorities are discrete and insular, and we worry about the impact of majoritarian rule on these communities.  Conversely, the claims of political parties are about frustration of outcome.  Finally, we have federal legislation protecting racial minorities, such as the VRA, based on a long history of discrimination in this country.  There are some similarities, however.  Both are `transmitted' by birth, tend to be unchanging, there is geographical sorting, and there are strong correlations between racial and party identification.  Additionally, claims defending partisan gerrymandering look a lot like claims defending racial gerrymandering in terms of disguising intent.


\section*{\textit{Vieth}}
Plaintiffs challenged a PA gerrymander of federal House districts.  The court split 4-4-1, finding that gerrymandering is a state issue, and since no cases have been adjudicated under \textit{Bandemer}, the Court did not throw out the PA districting.  Kennedy wrote that the courts shouldn't intrude on the nation's political process by regulating lines drawn, and that the Court should not adopt a standard which favors one party over another.  The goal shouldn't be to find excessive or extreme gerrymanders, but rather the subtle and clever political gerrymanders.  This shifts the view from the 14th Amendment (Equal Protection) to the 1st Amendment (Freedom of Affiliation).  The question becomes whether the state has put an undue burden on people as a result of their party affiliation.  We need to be come up with good standards of what constitutes such a burden.