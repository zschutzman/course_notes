
%\lecture{**LECTURE-NUMBER**}{**DATE**}{**LECTURER**}{**SCRIBE**}
\chno{9}{A Legal and Conceptual Primer on Political Gerrymandering}{Guy-Uriel Charles}{Zach Schutzman}
%\footnotetext{These notes are partially based on those of Nigel Mansell.}

% **** YOUR NOTES GO HERE:

% Some general latex examples and examples making use of the
% macros follow.  
%**** IN GENERAL, BE BRIEF. LONG SCRIBE NOTES, NO MATTER HOW WELL WRITTEN,
%**** ARE NEVER READ BY ANYBODY.


\section*{Why Does the Court Think About the Problem the Way It Does?}

One of the fundamental problems of American democracy is the structure of our elections and government

The big question is whether political gerrymandering cases are justiciable (can the courts hear them)? The Constitution doesn't give federal courts the authority to do so explicitly, and the devolution of rights to the states suggests that gerrymandering issues should be resolved at the state level rather than the federal level, and tradition and history has put few limitations on the ability of states to design electoral structures.

Additionally, the Constitution puts very few limits on the states.  Limits on state elections are pretty much limited to the 14th, 15th, 19th, and 26th Amendments, as well as precedent from \textit{Baker v Carr}, which upholds the principle of one-person-one-vote, derived from the 14th Amendment.
