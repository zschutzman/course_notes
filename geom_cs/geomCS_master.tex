%---------------------------------------------------------------------%
%  LaTeX Course Notes Template                                        %
%                                                                     %
%  Copyright (C) 2012 Zev Chonoles                                    %
%  zevchonoles@gmail.com                                              %
%  http://math.uchicago.edu/~chonoles/                                %
%                                                                     %
%  Please leave this information in the source code as                %
%  attribution if you choose to edit or redistribute this file.       %
%                                                                     %
%  This work is licensed under the Creative Commons Attribution-      %
%  ShareAlike 3.0 Unported License. To view a copy of this license,   %
%  visit http://creativecommons.org/licenses/by-sa/3.0/.              %
%                                                                     %
%---------------------------------------------------------------------%

\documentclass[11pt]{article}






%----------%
%  Basics  %
%----------%


%  Specfies basic information.
%  In the metadata section of the preamble, you can specify the subject and a list of keywords for the PDF.
%
\newcommand{\coursetitle}{CIS 610 - Advanced Geometric Methods in Computer Science}
\newcommand{\lecturer}{Jean Gallier}
\newcommand{\notetaker}{Zach Schutzman}
\newcommand{\notetakersemail}{ianzach+notes@seas.upenn.edu}
\newcommand{\courseterm}{Spring 2018}
\newcommand{\institution}{University of Pennsylvania}


%  array provides more column styles for the tabular and array environments.
%  (http://ctan.org/pkg/array)
%
%  parskip sets block paragraphs as the default, instead of indentation.
%  (http://www.ctan.org/pkg/parskip)
%
\usepackage[margin=1in]{geometry}
\usepackage{amsmath,amssymb,amsthm,amsfonts,array,parskip}


%  Allows equation, align, gather, etc. environments to split across pages.
\allowdisplaybreaks


%  Sets date formatting to the ISO 8601 standard, YYYY-MM-DD.
\usepackage{datetime} \renewcommand{\dateseparator}{-} \yyyymmdddate






%---------%
%  Fonts  %
%---------%


%  Defines \cal for standard calligraphy, \eucal for Euler calligraphy, and \frak for Fraktur.
\usepackage{eucal}  \let\eucal\mathcal  \let\cal\CMcal  \renewcommand{\frak}{\mathfrak}


%  Removes ligatures (e.g. the connection ordinarily made between the two f's in "differentiable").
\usepackage{microtype} \DisableLigatures{encoding=*,family=*}


%  Removes extra space after periods.




%-------------------------------%
%  Environments and Sectioning  %
%-------------------------------%


%  Defines some standard theorem environments, in both numbered and non-numbered versions. The numbering of each enviroment will be reset for each lecture.
\newcounter{lecture}       \setcounter{lecture}{0}
\newcounter{tN}[lecture]   \newcounter{dN}[lecture]
\newcounter{lN}[lecture]   \newcounter{rN}[lecture]
\newcounter{cN}[lecture]   \newcounter{eN}[lecture]
\newcounter{pN}[lecture]
\newcounter{clN}[lecture]

\newtheorem*{theorem}{Theorem}          \newtheorem{theorem-N}[tN]{Theorem}
\newtheorem*{lemma}{Lemma}              \newtheorem{lemma-N}[lN]{Lemma}
\newtheorem*{corollary}{Corollary}      \newtheorem{corollary-N}[cN]{Corollary}
\newtheorem*{proposition}{Proposition}  \newtheorem{proposition-N}[pN]{Proposition}

\theoremstyle{definition}
\newtheorem*{definition}{Definition}    \newtheorem{definition-N}[dN]{Definition}
\newtheorem*{remark}{Remark}            \newtheorem{remark-N}[rN]{Remark}
\newtheorem*{example}{Example}          \newtheorem{example-N}[eN]{Example}


\newtheorem*{claim}{Claim}    \newtheorem{claim-N}[clN]{Claim}


%  Modifies the spacing above theorem environments, which is messed up when using the parskip package.
%  (http://tex.stackexchange.com/questions/22119)
%
\makeatletter \def\thm@space@setup{\thm@preskip=\parskip \thm@postskip=0pt} \makeatother


%  Modifies the spacing above the proof environment.
%  (http://tex.stackexchange.com/questions/49801)
%
\makeatletter \renewenvironment{proof}[1][\proofname]{\pushQED{\qed}\normalfont
	\partopsep=\z@skip \topsep=\z@skip \trivlist \item[\hskip\labelsep\itshape #1\@addpunct{.}]
	\ignorespaces}{\popQED\endtrivlist\@endpefalse} \makeatother


%  Removes extra space before and after section headings.
\usepackage[compact]{titlesec}






%-------------------------%
%  Pictures and Diagrams  %
%-------------------------%


%  Allows for the use of colors.
%  (http://www.ctan.org/pkg/xcolor)
%
\usepackage[usenames,dvipsnames]{xcolor}
\definecolor{myred}{rgb}{0.9,0.2,0.2}
\definecolor{mygreen}{rgb}{0.2,0.6,0.2}
\definecolor{myblue}{rgb}{0.2,0.2,0.8}


%  graphicx provides advanced graphics options.
%  (http://ctan.org/pkg/graphicx)
%
\usepackage{graphicx}


%  tikz is for drawing all sorts of pictures and diagrams.
%  tikz-cd makes creating commutative diagrams in tikz a bit easier.
%  (http://www.ctan.org/pkg/pgf)
%  (http://www.ctan.org/pkg/tikz-cd)
%
\usepackage{tikz}
\usepackage{tikz-cd}
\usepackage{pgf,pgfplots}
\usetikzlibrary{arrows,calc,decorations,decorations.markings,fadings,positioning,patterns,shapes}
\tikzset{>=latex}
\tikzstyle{mypoint}=[inner sep=0pt,outer sep=0pt,minimum size=5pt,fill,circle]


\definecolor{ttqqqq}{rgb}{0.2,0.,0.}
\definecolor{ffffff}{rgb}{1.,1.,1.}
%\usetikzlibrary{external}
%\tikzexternalize



%------------------------%
%  Commands and Symbols  %
%------------------------%


%  Creates commands by running over a comma-separated list. For example,
%
%     \forcsvlist{\define{\newcommand}{\textbf}{bold}}{A,B}
%
%  would create
%
%     \newcommand{\boldA}{\textbf{A}}    \newcommand{\boldB}{\textbf{B}}
%
%  (http://tex.stackexchange.com/a/5776/20882)
%
\usepackage{etoolbox}
\newcommand{\define}[4]{\expandafter#1\csname#3#4\endcsname{#2{#4}}}
\forcsvlist{\define{\DeclareMathOperator}{}{}}{im,coker,rad,nil,Ann,Ass,codim,Spec,mSpec,diam,ord,Supp,supp,disc,Ob,vol,rank,Sym,Alt,Ind}
\forcsvlist{\define{\newcommand}{\mathrm}{}}{Hom,Mor,id,GL,SL,SO,SU,U,M,Mat,Ext,Tor,Res,Cor,Inf,End,Irr,Aut,Gal,lcm,tr,sign,triv,diag,Map,op,ev,act,alg,sep,unr,nr,ab}

%  Creates commands for some names of categories in the sans-serif font.
\forcsvlist{\define{\newcommand}{\mathsf}{}}{Set,Grp,Ab,CRing,Mod,Vect,Cat,Top,PreSh,Sh,Sch,Nat,Fun,Diff}

%  Creates commands for some blackboard bold letters.
\forcsvlist{\define{\newcommand}{\mathbb}{}}{N,Z,Q,R,C,F,G,T,A,B,D}


%  Saves the section symbol, paragraph symbol, Hungarian accent, and Scandanavian O in the macros \SS, \PP, \HH, and \OO, then redefines \S, \P, \H, and \O to be the corresponding blackboard bold letters.
%
\let\SS\S  \let\PP\P  \let\HH\H  \let\OO\O
\forcsvlist{\define{\renewcommand}{\mathbb}{}}{S,P,H,O}


%  latexsym defines some alternative versions of amssymb symbols.
%  (http://www.bakoma-tex.com/doc/latex/base/latexsym.pdf)
%
\usepackage{latexsym}


%  Defines a copyright symbol that is a bit nicer than the built-in one.
\newcommand{\mycopyrightsymbol}{\raisebox{-0.3ex}{\tikz{\node[inner sep=0pt,outer sep=0pt] at (0,0) {\textsc{c}};\draw (0,0) circle (0.18);}}}


%  Defines commands for real and complex projective space.
\newcommand{\RP}{\mathbb{R}\mathrm{P}}  \newcommand{\CP}{\mathbb{C}\mathrm{P}}


%  Defines a bordered matrix with square bracket delimiters instead of parentheses.
%  (http://tex.stackexchange.com/questions/55054)
%
\let\bbordermatrix\bordermatrix
\patchcmd{\bbordermatrix}{8.75}{4.75}{}{}
\patchcmd{\bbordermatrix}{\left(}{\left[}{}{}
\patchcmd{\bbordermatrix}{\right)}{\right]}{}{}


%  Calls one of the mathabx font families so that it is possible to use its symbols without making a global change.
%  (http://www.ctan.org/pkg/mathabx)
%  (http://tex.stackexchange.com/questions/14386)
%
\DeclareFontFamily{U}{mathb}{\hyphenchar\font45}
\DeclareFontShape{U}{mathb}{m}{n}{<5> <6> <7> <8> <9> <10> gen * mathb
	<10.95> mathb10 <12> <14.4> <17.28> <20.74> <24.88> mathb12}{}
\DeclareSymbolFont{mathb}{U}{mathb}{m}{n}


%  Defines circular arrows.
\DeclareMathSymbol{\lcirclearrow}{0}{mathb}{'366}
\DeclareMathSymbol{\rcirclearrow}{0}{mathb}{'367}
\newcommand{\leftcirclearrow}{\mathrel{\ensuremath{\raisebox{0.1ex}{\scalebox{0.9}{\rotatebox[origin=c]{90}{$\lcirclearrow$}}}}}}
\newcommand{\rightcirclearrow}{\mathrel{\ensuremath{\raisebox{0.1ex}{\scalebox{0.9}{\rotatebox[origin=c]{270}{$\rcirclearrow$}}}}}}


%  Gives semantic names for some common math symbols.
\newcommand{\iso}{\cong}
\newcommand{\htop}{\sim}
\newcommand{\htopequiv}{\simeq}
\newcommand{\cupprod}{\mathbin{\smallsmile}}
\newcommand{\capprod}{\mathbin{\smallfrown}}
\newcommand{\wedgesum}{\mathbin{\vee}}
\newcommand{\boundary}{\partial}
\renewcommand{\emptyset}{\varnothing}
\newcommand{\characteristic}{\mathrm{char}}
\newcommand{\symdiff}{\mathbin{\vartriangle}}
\newcommand{\convolute}{\mathbin{\ast}}
\newcommand{\actson}{\rightcirclearrow}
\newcommand{\actedonby}{\leftcirclearrow}
\newcommand{\directsum}{\oplus}
\newcommand{\bigdirectsum}{\bigoplus}
\newcommand{\tensor}{\otimes}
\newcommand{\bigtensor}{\bigotimes}
\newcommand{\free}{\mathbin{\ast}}
\newcommand{\bigfree}{\mathop{\ensuremath{\raisebox{-0.7ex}{\scalebox{2.3}{$\ast$}}}}}
\renewcommand{\complement}[1]{{#1}^{\mathsf{c}}}
\newcommand{\transpose}[1]{{#1}^{\textsf{T}}}
\newcommand{\union}{\cup}
\newcommand{\intersect}{\cap}
\newcommand{\transverse}{\mathrel{\raisebox{1.1ex}{$-$}\mathllap{\pitchfork\hspace{0.22mm}}}}



\def\multiset#1#2{\ensuremath{\left(\kern-.3em\left(\genfrac{}{}{0pt}{}{#1}{#2}\right)\kern-.3em\right)}}



%-----------------------------------%
%  Things Specific to Course Notes  %
%-----------------------------------%


%  Formatting for the table of contents. The first line allows for multi-column environments, the second line removes the heading "Contents".
\usepackage{multicol} \setlength{\columnsep}{3cm}
\makeatletter \renewcommand\tableofcontents{\@starttoc{toc}} \makeatother


%  Sets the page style.
\usepackage{fancyhdr}
\pagestyle{fancy}
\renewcommand{\headrulewidth}{0pt}
\renewcommand{\footrulewidth}{0.5pt}
\setlength{\headheight}{14pt}
\lfoot{\parbox[t]{1in}{\centering Last edited\\ \today}}
\cfoot{\parbox[t]{3in}{\centering \coursetitle}}
\rfoot{\parbox[t]{0.9in}{\centering Page \thepage\\ Lecture \arabic{lecture}}}


%  Sets the inputs for \maketitle.
\author{Lectures by \lecturer\\ 
	Notes by \notetaker}
\title{\coursetitle}
\date{\institution, \courseterm}


%  Defines headings for each day's notes.
\newcommand{\classheader}[1]{\stepcounter{lecture}\newpage\section*{Lecture \arabic{lecture} (#1)}
	\phantomsection \addcontentsline{toc}{section}{Lecture \arabic{lecture} (#1)}}


%---------------------------------------%
%  Miscellaneous Additions to Template  %
%---------------------------------------%

% http://tex.stackexchange.com/questions/18359
\pgfplotsset{compat=newest}

\newcommand{\Cinfty}{\ensuremath{C^{\infty}}}
\newcommand{\Crit}{\mathrm{Crit}}
\usepackage{mathtools}
\newcommand{\Or}{\mathrm{Or}}
\renewcommand{\Re}{\mathrm{Re}}
\renewcommand{\Im}{\mathrm{Im}}
\usepackage{mathrsfs}
\newtheorem*{examples}{Examples}
\newtheorem*{exercise}{Exercise}
\usepackage{pdfpages}
\newcommand{\Lie}{\mathrm{Lie}}
\newcommand{\Diffeo}{\mathrm{Diffeo}}

\newcommand{\connection}{\nabla}
\newcommand{\new}{\mathrm{new}}


\newcommand{\review}{{\huge\color{myred}{$\star$}}}


%---------------------------%
%  Hyperlinks and Metadata  %
%---------------------------%
%
% (this section must come last!)


%  hyperref enables for the creation of hyperlinks, and also specifies the metadata of the PDF file.
%  hyperxmp allows more metadata to be specified.
%  (http://www.ctan.org/pkg/hyperref)
%  (http://www.ctan.org/pkg/hyperxmp)
%  (http://tex.stackexchange.com/questions/41461)
%
\usepackage{hyperref}
\usepackage{hyperxmp}
\hypersetup{
	pdfauthor={\notetaker},
	pdftitle={\coursetitle},
	pdfproducer={LaTeX},
	%pdfcopyright={Copyright (C) \the\year\ \notetaker. This work is licensed under a Creative Commons Attribution-ShareAlike 3.0 Unported License. All attribution should be to \lecturer\ as the lecturer, and to \notetaker\ as the person taking these notes.},
	pdfsubject={differential topology},
	pdfkeywords={},
	%pdflicenseurl={http://creativecommons.org/licenses/by-sa/3.0/},
	colorlinks=true,
	linkcolor=myred,
	citecolor=mygreen,
	urlcolor=myblue,
	linktoc=page,
	pdfstartview=FitH
}




\renewcommand{\R}{\mathbb{R}}
\newcommand{\thrm}[1]{\theorem{#1}}
\newcommand{\innprod}[2]{\left\langle #1,#1 \right\rangle}

%------------%
%  Document  %
%------------%


\begin{document}
	
	
	%  The command
	%
	%  \thispagepdflabel{text}
	%
	%  sets the PDF page number (*not* the internal LaTeX page number) to be "text". This does not have to be a numeral; it could be a word, e.g. "Title". This lets one avoid the issue of having the PDF's page numbering not aligning with the page numbering LaTeX used in the document.
	%
	%  (http://tex.stackexchange.com/questions/85558)
	
	
	%  Title
	%
	\maketitle
	\thispdfpagelabel{Title}
	\thispagestyle{empty}
	\setcounter{page}{-1}
	\vspace{0.3in}
	
	
	
	%  Table of Contents
	%
	\begin{center}
		\begin{minipage}[t]{0.9\textwidth}
			\begin{multicols}{2}
				\tableofcontents
			\end{multicols}
		\end{minipage}
	\end{center}
	
	
	
	\newpage
	\thispdfpagelabel{-}
	\thispagestyle{empty}
	
	
	
	%  Introduction
	%
	\section*{Introduction}
	This course is an advanced topics course in Geometry and its applications.  The current (Spring 2018) offering is focused on Riemannian manifolds, their differential geometry, and the Lie groups and Lie algebras associated with them.
	
	I am taking these notes as the class progresses and doing my best to transcribe them promptly. I am using the editor TeXstudio.  The template for these notes was created by Zev Chonoles and is made available (and being used here) under a Creative Commons License. 
	
	I am responsible for all faults in this document, mathematical or otherwise; any merits of the material here should be credited to the lecturer, not to me.
	
	Please email any corrections or suggestions to \expandafter\href{mailto:\notetakersemail}{\texttt{\notetakersemail}}.
	
	%\medskip
	%
	%\section*{Acknowledgments}
	%
	%Thank you to all of my fellow students who sent me suggestions and corrections, and who lent me their own notes from days I was absent. My notes are much improved due to your help.
	
	
	%%  Copyright
	%%
	%\section*{Copyright}
	%Copyright \mycopyrightsymbol\ 2012 \notetaker.
	%
	%This work is licensed under a Creative Commons Attribution-ShareAlike 3.0 Unported License. This means you are welcome to do essentially anything with this work, including editing, %adapting, distributing, and making commercial use of it, as long as you
	%\begin{itemize}
	%\item include an attribution of \lecturer\ as the lecturer of the course these notes are based on, and \notetaker\ as the person taking the notes,
	%\item do so in a way that does not suggest either of us endorses you or your use of this work, and
	%\item if you alter, transform, or build upon this work, you must apply to your work the same, or similar, license to this one.
	%\end{itemize}
	%More details are available at \href{https://creativecommons.org/licenses/by-sa/3.0/deed.en\_US}{\texttt{https://creativecommons.org/licenses/by-sa/3.0/deed.en\_US}}.
	
	\newpage
	
	
	%  Make a separate file for each lecture, for example, using a naming scheme like this:
	%
	%  lecture1.tex, lecture2.tex, ...
	%
	%  and keep them in the same folder as this main file. By doing it this way (instead of keeping all the notes in the main file), if you're only working on the notes for one lecture, you can easily comment out the lines corresponding to the other lectures.
	%

	\classheader{2018-01-11}

\section*{Intro}

The high-level overview of this course is that we are going to build up the mathematical machinery to talk about \textit{homogeneous manifolds} and their geometry.  In a sentence, these are surfaces for which there is some group $G$ which acts transitively on that space.

Suppose that we have a sequence of objects $B=B_0,B_1,B_2,\dots, B_m$ where the transformation $D_i$ sends $B$ to $B_i$.  We can think about the $D_i$ as coming from some group, so we know how to compose them and each is invertible.

The motion of a rigid body can be described by a curve in this group of transformations of a space $E$ (maybe this is $\R^n$).  Given $B\in E$, a \textbf{deformation} is a (reasonably smooth) curve $D:[0,1]\rightarrow G$.  We write $B_t=D(t)(B)$ to represent the position of $B$ at time $t$.

\begin{example}
	If $G=SO(3)$, then we are looking at rigid rotations in $\R^3$.  With respect to any orthonormal basis, any rotation has an associated matrix $R$ such that $RR^T=R^TR=I$.
	
	If $G=SE(3)$, we are looking at rigid motions in $\R^3$.  If $B$ is a matrix in this group, it can rotate and translate.  This is the group of affine maps $\rho\in SE(3)$ has the form $rho(x)=f(x)+u$, where $f(x)$ is a rotation and $u$ a translation.  We can write these as transformations in $\R^4$ by doing the following:
	
$$\begin{pmatrix}
		R&u\\
		\boldsymbol{0}&1
	\end{pmatrix}$$
	
	Where $R$ is an $n\times n$ rotation, $\boldsymbol{0}$ is a row of $n$ zeroes, $u$ is a column vector representing the translation, and the last entry is $1$.  The matrix representations of elements of $SE(3)$ are $4\times 4$ matrices.
	
	If $G=SIM(3)$, we are looking at simple deformations of a non-rigid body - we can grow and shrink as well as rotate and translate.  These have matrices which look like 
	
	$$\begin{pmatrix}
	\alpha R & u\\
	\boldsymbol{0} & 1
	\end{pmatrix}$$ where $R\in SO(3)$, $\alpha>0$, and $u\in \R^n$.
	
$\star$ all of these are \textit{Lie groups} and we will study lots of other things with Lie groups in this course.
\end{example}


\section*{The Interpolation Problem}

Suppose we have a sequence of deformations $g_0,g_1,\dots,g_m$, with each $g_i\in G$ and $g_0=1\in G$.  We would like to find a smooth curve in $G$ $c:[0,m]\rightarrow G$ such that $c(i)=g_i$ for all $i$.

The naive approach would be to take $(1-t)g_i + tg_{i+1}$, which is a linear/convex interpolation between adjacent points.  However, there is no guarantee that all of these intermediate points are actually in $G$!

What we can do is use Lie groups.  These are topological groups, so they come along with a nice manifold where we can do geometric things.  At every $g\in G$, there is a tangent space $T_gG$.  The tangent space at $1\in G$ is special, and it is called the \textbf{Lie algebra}.  We use Fraktur fonts to denote Lie algebras.  This Lie algebra is denoted $\mathfrak{g}$ and it comes with a multiplication $\left[\cdot,\cdot\right]$ called the \textbf{Lie bracket}.  When $G$ is a matrix, group, $\left[X,Y\right]=XY-YX$.

The Lie algebra $\mathfrak{so}(n)$ of $SO(n)$ is the set of skew-symmetric $n\times n$ matrices $B^T=-B$.

The Lie algebra $\mathfrak{se}(n)$ of $SE(n)$ is the set of matrices of the form
$$\begin{pmatrix}
B&u\\0&0
\end{pmatrix}$$ where $B\in \mathfrak{so}(n)$ and $u\in \R^n$.

The Lie algebra $\mathfrak{sim}(n)$ of $SIM(n)$ is the set of matrices of the form
$$\begin{pmatrix}
\lambda I B&u\\0&0
\end{pmatrix}$$ where $B\in \mathfrak{so}(n)$, $\lambda\in \R$, $I$ is the $n\times n$ identity matrix, and $u\in \R^n$.

We can think of $\mathfrak{g}$ as a linearization of the group $G$.  There is a map called the \textbf{exponential} $\exp: \mathfrak{g}\rightarrow G$ such that $$\exp(X) = e^X = 1+\frac{X}{1!}+\frac{X^2}{2!}+\dots$$

For the groups we talked about, $\exp$ is a surjective map.  There is a multivalued `function' called the \textbf{logarithm} $\log:G\rightarrow \mathfrak{g}$ such that $\exp(\log(A))=A\in G$.

We can use $\log$ and $\exp$ to do our interpolation.  First, let $x_0=\log(g_0)$, $x_1=\log(g_1)$, and so on.   Then find a curve $X:[0,m]\rightarrow\mathfrak{g}$ to interpolate the $x_i$ in $\mathfrak{g}$.  Finally, the curve in $G$ is given by $c(t)=\exp(X(t))$.

If $\mathfrak{g}$ is a vector space, we can do fancy things like use splines to interpolate.

We still need to worry about actually computing $\exp$ and $\log$.  There are formulas if we are in $\mathfrak{so}(n),\mathfrak{se}(n),$ and $\mathfrak{sim}(n)$.  For $\mathfrak{so}(n)$< this is the Rodrigues formula, and there is a variant for $\mathfrak{se}(n)$.  Logarithms can be computed for $SO(n),SE(n),$ and $SIM(n)$, but there is an issue when we have an eigenvalue equal to $-1$ and the logarithm is multivalued.

A real matrix doesn't always have a real logarithm (it does always have  a complex one).  Let $S(n)$ be the set of real matrices whose eigenvalues $\lambda + \mu i$ live in the horizontal strip of the complex plane $-\pi <\mu <\pi$.  Then $\exp: S(n)\rightarrow \exp(S(n))$ is a bijection onto the set of real matrices with no negative eigenvalues.

There are efficient algorithms to compute matrix logarithms, which we will discuss later.


\section*{Metrics on Lie Groups}

Metrics formally define a measurable sense of `closeness'.  How `close' are two given group elements?

We can give an inner product to $\mathfrak{g}=T_1G$, then propagate this to $T_gG$ at any $g$ to get a Riemannian metric.

For $G=SO(n)$, $\left\langle X,Y \right\rangle = -\frac{1}{2}Tr(XY)= \frac{1}{2}Tr(X^TY)$ is an inner product on $\mathfrak{so}(n)$.

A curve $\gamma:[0,1]\rightarrow G$ has length
$$L(\gamma)=\int\limits_0^1 \left\langle \gamma'(t),\gamma'(t) \right\rangle^{1/2}dt$$

A \textbf{geodesic} through $I$ is a curve $\gamma(t)$ such that $\gamma(0)=I$ and $\gamma''(t)$ is normal to the tangent space $T_{\gamma(t)}G$ for all $t$.  It turns out that for all $X\in\mathfrak{so}(n)$ there is a unique geodesic through $I$ such that $\gamma'(0)=X$, namely $\gamma(t)=\exp(tX)$.

If $G=SO(n)$, then we have for all $A\in G$ that there exists some geodesic from $I$ to $A$.  Define the distance $$d(I,A)=\inf\limits_\gamma\left\{ L(\gamma) : \gamma \text{ joins } I\text{ and } A    \right\}$$

The distance $d(A,B)$ in $SO(n)$ is 
$$d(A,B) = \sqrt{\theta_1^2 + \theta_2^2 + \dots + \theta_m^2}$$ where $e^{\pm i \theta_j}$ are the eigenvalues (not equal to 1) of $A^TB$ with $0<\theta_j\leq \pi$ for all $j$.

The same inner product $\frac{1}{2}Tr(X^TY)$ works in $\mathfrak{se}(n)$, but this metric is only left- and not both left- and right-invariant.  Consequently, not all geodesics in $\mathfrak{se}(n)$ are given by the exponential.  Related to this is that $SE(n)$ is not a compact or a semisimple group.
	\classheader{2018-01-16}

\section*{Manifolds Induced by Actions of $\boldsymbol{SO(n)}$}

What do we mean by the `distance' $d(V,W)$ between two subspaces $V,W$ of the Grassmannian $G(k,n)$?  We know something about distances in $SO(n)$, so what happens if we let $SO(n)$ act on $G(k,n)$?

We can specify a $k$-dimensional subspace $V$ with $k$ orthonormal (column) vectors in $\R^n$ and write this as an $n\times k$ matrix $A$ which satisfies $A^TA=I_k$.

A rotation $R\in SO(n)$ acts on $V$ by rotating each vector in $V$, that is $R$ applied to $A\in V$ can be described as $(R,A)\mapsto RA$, with respect to the ordinary matrix product.  Here, $RA$ will also have $k$ orthonormal columns (We're hiding an equivalence relation here).

The action $\cdot:SO(n)\times G(k,n)\rightarrow G(k,n)$ is transitive!  We're not going to prove this here, but it should be intuitively pretty clear: pick some $V$ and a $W$ we want to send it to.  Then we just need to show that the map which sends the $i$th column of $V$ to the $i$th column of $W$ is in $SO(n)$.

Since it is transitive, we can look at the \textbf{stabilizer} of a subspace $V$, which is the subgroup $K\subset SO(n)$ such that $R\cdot V=V$ for all $R\in K$.  It can be shown that $G(k,n)\iso SO(n)/K$, where we think of $G(k,n)$ as the cosets $RK$ with $R\in SO(n)$ modulo the equivalence relation $R_1\sim R_2$ if and only if $R_1^{-1}R_2\in K$, and then taking the canonical projection onto these equivalence classes.

The stabilizer of the first $k$ columns of $I_n$ is $K=S(O(k)\times O(n-k))$.  Here, we can write $K$ as the set $$k=\left\{  \begin{pmatrix} P&0\\0&Q\end{pmatrix} : P\in O(k),\ Q\in O(n-k),\ \det(P)\det(Q)=1   \right\}$$

This is a Lie group, and its associated Lie algebra is 
$$\mathfrak{k}=\left\{  \begin{pmatrix} S&0\\0&T\end{pmatrix} : S\in\mathfrak{so}(k),\ T\in\mathfrak{so}(n-k)   \right\}$$

We call the $X\in \mathfrak{k}$ the `vertical' tangent vectors and the $X\in \mathfrak{m}$ the `horizontal' tangent vectors.


The tangent space $T_ISO(n)=\mathfrak{so}(n)$ splits as the direct sum of $\mathfrak{k}\oplus \mathfrak{m}$ where 
$$\mathfrak{m}=\left\{  \begin{pmatrix} 0&-A^T\\A&0\end{pmatrix} : A\in M_{n-k,n}   \right\}$$

The tangent space $T_o(SO(n)/K)$ to $SO(n)/K$ at $o$ is isomorphic to $\mathfrak{m}$ (a coset of $K$).

Using the metric on $\mathfrak{so}(n)$, $\frac{1}{2}Tr(X^TY)$, $\mathfrak{k}$ and $\mathfrak{m}$ are orthogonal complements and $SO(n)/K$ is what is called a \textit{naturally reductive homogeneous space}.


Geodesics in $G(k,n)$ are projections of horizontal geodesics in $SO(n)$.

\begin{theorem}
	The distance between two subspaces $V,W\in G(k,n)$ specified by matrices $A$ and $B$, respectively, is $$d(V,W)=\sqrt{\theta_1^2+\theta_2^2+\dots+\theta_k^2}$$ where $\cos\theta_i$ are the singular values of $A^TB$ with $0\leq\theta_i\leq \frac{\pi}{2}$.  These $\theta_i$ are called the \textbf{principle angles}.
\end{theorem}


Another interesting group is $SPD(n)$, which are the symmetric positive definite $n\times n$ matrices.  We can write this as $SPD(n)\iso GL^+(n)/SO(n)$, where $GL^+(n)$ is the set of real $n\times n$ matrices with strictly positive determinant.  Here, geodesics can be computed, but this involves a big ugly integral.

Explicit computation of geodesics in $G(k,n)$ allows for the generalization of optimization methods like gradient descent.  Whether we can do it in $SE(n)$ and Grassmannians of affine spaces is an open problem.


\section*{The Matrix Exponential}


We can think of Lie groups (naively) as groups of symmetries of geometric and topological objects, and Lie algebras are kind of like the `infinitesimal' transformations of these objects.

We can look at $SO(n)$ as rotations of $\R^n$ and $\mathfrak{so}(n)$ as the set of real skew-symmetric matrices.  

The Lie algebra at the identity of the Lie group can be thought of as a `linearization' of the group.  The exponential is a way of `delinearizing'.

Recall we defined the matrix exponential as the map $$e^A\mapsto I_n+\sum\limits_{k\geq 1}\frac{A^k}{k!} = \sum\limits_{k\geq 0}\frac{A^k}{k!}$$ where matrix powers are repeated products and $A^0$ is the identity matrix.

\begin{lemma}
	If the $n\times n$ real or complex matrix $A$ has entries $A=(a_{ij})$ and we let $\mu$ denote the $a_{ij}$ of maximum absolute value (or modulus), then the absolute value (or modulus) of any element of $A^p$ (denoted $a_{ij}^{(p)}$) is no greater than $(n\mu)^p$. 
	
	Thus the series $\sum\limits_{p\geq 0}\frac{a_{ij}^{(p)}}{p!}$ converges absolutely, so the matrix exponential as an infinite sum is well-defined.
\end{lemma}


What is the exponential of the matrix $A=\begin{pmatrix}
0&-\theta\\\theta&0\end{pmatrix}$?
We can write $A=\theta J$ where $J$ is the matrix $J=\begin{pmatrix}0&-1\\1&0\end{pmatrix}$.

We know what the powers of $J$ look like.  $J^2=\begin{pmatrix}
-1&0\\0&-1\end{pmatrix} = I^2$.  It's easy to show that $J^3=-J$.


We can compute a few powers of $A$: $A=\theta J$, $A^2=\theta^2J^2=-\theta^2I$, $A^3=-\theta^3 J$, $A^4=\theta^4 I$.

From here, we can write $$e^A = I+\frac{\theta J}{1!} -\frac{\theta^2 I}{2!} - \frac{\theta^3 J}{3!} + \frac{\theta^4 I}{4!} + \dots$$

We can pull this apart into two sums, one with $I$s and the other with $J$s:

$$e^A = I\left(1-\frac{\theta^2}{2!} + \frac{\theta^4}{4!} - \frac{\theta^6}{6!} + \dots \right) + J\left(   \frac{\theta}{1!}   - \frac{\theta^3}{3!} + \frac{\theta^5}{5!} - \dots     \right)$$

which we recognize as $I\cos\theta + J\sin\theta$, so we get $$e^A=\begin{pmatrix}
\cos\theta&-\sin\theta\\\sin\theta&\cos\theta\end{pmatrix}$$

In fact, every rotation matrix looks like the exponential of some skew-symmetric matrix.

The matrix exponential is not always surjective.  Let $A=\begin{pmatrix} a&b\\c&-a\end{pmatrix}$.  Note that the trace of $A$ is $0$.  We have that $A^2=(a^2+bc)I=-\det(A)I$. 

 If $a^2+bc=0$, we have $e^A=I+A$.
 
 If $a^2+bc<0$, let $\omega>0$ be such that $\omega^2=-(a^2+bc)$. Then $e^A=\cos\omega I+\frac{\sin\omega}{\omega}$.
 
 If $a^2+bc>0$, let $\omega>0$ be such that $\omega^2=a^2+bc$.  Then $e^A=\cosh\omega I + \frac{\sinh\omega}{\omega}$.  In all cases, $\det(A)=1$ and $Tr(e^A)\geq -2$.  But $B=\begin{pmatrix}a&0\\0&a^{-1}\end{pmatrix}$ is not the exponential of any matrix with trace $0$, as $Tr(e^B)<-2$.
 
 A fundamental property of the matrix exponential is that if $\lambda_i$ are the eigenvalues of the matrix $A$, then $e^{\lambda_i}$ are the eigenvalues of $e^A$, and the eigenvector associated with $\lambda_i$ is the same as the one associated with $e^{\lambda_i}$.
 
 \begin{lemma}
 	Let $A$ and $U$ be real or complex matrices, with $U$ invertible.  Then $e^{UAU^{-1}}=Ue^AU^{-1}$.
 \end{lemma}
\begin{lemma}
	Given any $n\times n$ matrix $A$, there exists an invertible $P$ and upper triangular $T$ such that $A=PTP^{-1}$.  
\end{lemma}
\begin{lemma}[Schur]
	Given any $n\times n$ matrix $A$, there exists a unitary $U$ and upper triangular $T$ such that $A=UTU^*$, where $U^*$ is the conjugate-transpose of $U$.
\end{lemma}

This tells us that if $A$ is Hermitian, there exists a unitary $U$ and a real diagonal matrix $D$ such that $A=UDU^*$.

If $A=PTP^{-1}$ where $T$ is upper triangular, the diagonal entries of $T$ are the eigenvalues of $A$, and $A$ and $T$ have the same characteristic polynomial.

\begin{lemma}
	Given any complex $n\times n$ matrix $A$ with eigenvalues $\lambda_i$, the eigenvalues of $e^A$ are $e^{\lambda_i}$, and the eigenvectors are the same.  Thus, $\det(e^A)=e^{Tr(A)}$.
\end{lemma}
	\end{document}
	
